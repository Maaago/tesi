\documentclass[11pt, t, aspectratio=169]{beamer}

\usepackage[utf8]{inputenc}
\usepackage[T1]{fontenc}

\graphicspath{{../Tesi/imgs}}

\usepackage{booktabs}

\usetheme{uniud}
\usecolortheme{uniud}

\usefonttheme{structurebold}

\usepackage{mathptmx}
\usepackage[default]{lato}

\usepackage{circuitikz}
\usepackage{wrapfig}
\usepackage{caption}
\usepackage{subcaption}
\usepackage{comment}

\DeclareMathOperator{\arcsinh}{arcsinh}

\newcommand{\screenshot}[3][]
{
	\begin{figure}[H]
		\centering
		\includegraphics[#1]{#2}
		\vspace*{-25px}
		\caption{#3}
		\label{fig:#2}
	\end{figure}
}

\newcommand{\screenshotvst}[3][]
{
	\begin{figure}[H]
		\centering
		\includegraphics[#1]{#2}
		\vspace*{-8px}
		\caption{#3}
		\label{fig:#2}
	\end{figure}
}

\newcommand{\subscreenshot}[3][]
{
	\begin{subfigure}{0.4\paperwidth}
		\centering
		\includegraphics[width=0.4\paperwidth,#1]{#2}
		\caption{#3}
		\label{fig:#2}
	\end{subfigure}
}

\newcommand{\grafico}[2]
{
	\screenshot[width=\textwidth]{plots/#1}{#2}
}

\newcommand{\subgrafico}[2]
{
	\subscreenshot[]{plots/#1}{#2}
}

\title{Realizzazione a tempo reale di un clipper per il suono\\attraverso schemi numerici innovativi}
\author{Laureando: Francesco Magoga \qquad\qquad Relatore: prof. Federico Fontana}
\institute{Dipartimento di Scienze Matematiche, Informatiche e Fisiche\\
		Corso di laurea in informatica magistrale}
\date{2020-2021}

\begin{document}
	\begin{frame}
		\titlepage
	\end{frame}
	
	\begin{frame}
		\frametitle{Introduzione}
		
		{\small
		\begin{itemize}
			\item Spesso nella creazione di effetti audio digitali si parte da un modello analogico che viene opportunamente modificato per poter sottostare ai limiti imposti dalle caratteristiche hardware del computer.
			\item Problema: modellare \textit{delay-free loop}, circuiti che presentano un ciclo che collega l'uscita con l'entrata e non vi sono componenti che creano dei ritardi in tale ciclo. L'unico modo per calcolare il risultato è utilizzare dei metodi numerici.
			\item Complicazione: componenti non lineari che spesso caratterizzano le delay-free loop. Nel percorso del feedback sono presenti dei componenti elettronici che presentano caratteristiche non lineari.
			\item Per analizzare il metodo numerico proposto si è deciso di digitalizzare un circuito clipper analogico formato da 4 diodi, poiché tale circuito presenta entrambe le caratteristiche di non linearità e di assenza di ritardo nel feedback.
			\item Base di partenza: il lavoro di C. Christoffersen.
		\end{itemize}
		}
		
		%TODO aggiungere una slide introduttiva in cui si accenna alla difficoltà di realizzare effetti audio digitali a tempo reale in cui sono presenti nonlinearità onerose da calcolare numericamente. Si dovrebbe far cenno almeno a un paio di anteriorità esistenti in letteratura
	\end{frame}

	% Metodo numerico
	\begin{frame}
		\frametitle{Metodo numerico}
		
		Al variare del parametro $L$ il metodo numerico si comporta come un risolutore di punto fisso standard o come un risolutore Newton-Raphson.
		
		\begin{equation}
			\mathbf{x}^{(\lambda+1)} = \mathbf{x}^{(\lambda)}-\mathbf{K}(\mathbf{x}^{(\lambda)})(\mathbf{x}^{(\lambda)}-\mathbf{f}(\mathbf{x}^{(\lambda)}))
		\end{equation}
		
		\begin{equation}
			\mathbf{K}^{(L)}(\mathbf{x}) = \sum_{l=0}^{L} \left(\mathbf{J_{f}(x)}\right)^{l}
		\end{equation}
		
		Dato $\lVert\mathbf{J_{f}(x)}\rVert < 1$:
		\begin{itemize}
			\item se $L = 0$ allora $\mathbf{K}^{(0)}(\mathbf{x}) = \mathbf{I}$\qquad\qquad\qquad\qquad$\Rightarrow$ risolutore punto fisso standard
			\item se $L = \infty$ allora $\mathbf{K}^{(\infty)}(\mathbf{x}) = (\mathbf{I}-\mathbf{J_{f}(x)})^{-1}$\qquad\quad\hspace{-2px}$\Rightarrow$ risolutore Newton-Raphson
		\end{itemize}
	\end{frame}

	% Clipper audio
	\begin{frame}
		\frametitle{Clipper audio}
		
		\begin{figure}
				\begin{subfigure}{0.4\paperwidth}
					\centering
					\begin{circuitikz}[american voltages, scale=0.55, transform shape]
						\draw
						% Maglia esterna
						(-1,7) to[sinusoidal voltage source,l_=$V_{in}$] (-1,0)		% segnale d'ingresso
						(-1,7) to[resistor=$R_{in}$] (3,7)							% resistenza
						to[short, i^=$i_{in}$] (4,7)									% iin
						to[short, i^=$i_{out}$] (5,7)								% iout
						-- (8,7)														% filo in alto
						to[C, l_=$C$] (8,0)											% condensatore
						-- (-1,0)													% filo in basso
						(4,0) node[ground]{}											% messa a terra
						
						% Diodi
						(4,7) to[short, *-*] (4,6)			% connessione tra la maglia esterna e i diodi A
						(4,7) to[short, i^=$i_{D}$] (4,6)	% iD
						(5,6) -- (3,6)						% connessione in alto tra i diodi A
						to[empty diode, l=$D_{A}$] (3,4)		% diodo A di sinistra
						-- (5,4)								% connessione in basso tra i diodi A
						to[empty diode, l_=$D_{A}$] (5,6)	% diodo A di destra
						(4,4) to[short, *-*] (4,3)			% connessione i diodi A e i diodi B
						(5,3) -- (3,3)						% connessione in alto tra i diodi B
						to[full diode, l=$D_{B}$] (3,1)		% diodo B di sinistra
						-- (5,1)								% connessione in basso tra i diodi B
						to[full diode, l_=$D_{B}$] (5,3)		% diodo B di destra
						(4,1) to[short, *-*] (4,0)			% connessione i diodi B e la maglia esterna
						
						% Etichette
						(-1.25,2.5) to[open, v_<=$$] (-1.25,4.5)		% segno di Vin
						(2.3,6) to[open, v_=$V_{A}$] (2.3,4)			% tensione dei diodi A
						(2.3,3) to[open, v_=$V_{B}$] (2.3,1)			% tensione dei diodi B
						(9,7) to[open, v^=$V_{out}$] (9,0)			% tensione Vout
						;
					\end{circuitikz}
					\caption{Circuito elettronico del clipper audio utilizzato}
					\label{fig:clipper}
				\end{subfigure}
				\hfill
				\subscreenshot[width=0.4\paperwidth]{analog/example}{Esempio di input e output del clipper}
		\end{figure}
	\end{frame}
	
	% Soluzione analitica
	\begin{frame}
		\frametitle{Soluzione analitica}
		
		Passi svolti per ottenere la soluzione al problema:
		\begin{enumerate}
			\item dalle leggi di Kirchhoff e dalle leggi delle correnti sui vari componenti si è ottenuta un'equazione differenziale
			\item tutte le variabili presenti nell'equazione differenziale sono state espresse in funzione della tensione sulla coppia di diodi B
			\item è stata calcolata la derivata della tensione sulla coppia di diodi A
			\item alla differenziale è stato applicato il metodo di Eulero implicito, dal quale si è ottenuta la funzione discretizzata mostrata di seguito
			\item è stato applicato il metodo numerico
			\item è stata calcolata la derivata della funzione discretizzata
			\item tramite le leggi di Kirchhoff si è trovato il valore della tensione di output
		\end{enumerate}
	\end{frame}
	
	\begin{frame}
		\frametitle{Soluzione analitica}
		
		Equazione differenziale dopo l'applicazione del metodo di Eulero all'indietro (passo 4 della slide precedente)
		\begin{equation}
			\begin{split}
				&\widehat V_{B}[n] = \frac{T}{C}\cdot\frac{\sqrt{1+\left(\frac{\beta_{B}}{\beta_{A}}\sinh(\alpha_{B}V_{B}[n])\right)^{2}}}{\alpha_{B}\beta_{B}\cosh(\alpha_{B}V_{B}[n])+\alpha_{A}\beta_{A}\sqrt{1+\left(\frac{\beta_{B}}{\beta_{A}}\sinh(\alpha_{B}V_{B}[n])\right)^{2}}}\cdot\\&\cdot\left(\frac{1}{R_{in}}\left(V_{in}(n)-\frac{1}{\alpha_{A}} \arcsinh \left(\frac{\beta_{B}}{\beta_{A}} \sinh(\alpha_{B}\widehat V_{B}[n])\right)-\widehat V_{B}[n]\right)-2\beta_{B}\sinh(\alpha_{B}\widehat V_{B}[n])\right)+\\&+\widehat V_{B}[n-1]\\
			\end{split}
		\end{equation}
	\end{frame}
	
	% Simulazione
	\begin{frame}
		\frametitle{Simulazione Matlab}
		
		L'input utilizzato è un'onda sinusoidale definita dalla classica formula $y(t) = A\,\sin (2\pi\,f\,t+\phi)$ dove:
		\begin{itemize}
			\item $A$ è l'ampiezza
			\item $f$ è la frequenza
			\item $t$ è il tempo
			\item $\phi$ è la fase, sempre uguale a zero
		\end{itemize}
	\end{frame}
	
	\begin{frame}
		\frametitle{Simulazione Matlab}
		\framesubtitle{Grafici input-output per $L = \infty$}
		
		\begin{figure}
			\subgrafico{in-out/ampiezze/1000Hz/1,5V}{Grafico input-output con $f = 1000Hz$ e $A = 1,5V$}
			\subgrafico{in-out/ampiezze/1000Hz/2V}{Grafico input-output con $f = 1000Hz$ e $A = 2V$}
		\end{figure}
	\end{frame}
	\begin{frame}
		\frametitle{Simulazione Matlab}
		\framesubtitle{Grafici input-output per $L = \infty$}
		
		\begin{figure}\ContinuedFloat
			\subgrafico{in-out/ampiezze/1000Hz/3V}{Grafico input-output con $f = 1000Hz$ e $A = 3V$}
			\subgrafico{in-out/ampiezze/1000Hz/5V}{Grafico input-output con $f = 1000Hz$ e $A = 5V$}
		\end{figure}
	\end{frame}
	
	\begin{frame}
		\frametitle{Simulazione Matlab}
		\framesubtitle{Grafici iterazioni-tempo per $L \in [0;50]$}
		
		\begin{figure}
			\subgrafico{iterations/media/1,0V}{Grafico iterazioni-tempo con $f = 100Hz$ e $A = 1V$}
			\subgrafico{iterations/media/1,5V}{Grafico iterazioni-tempo con $f = 100Hz$ e $A = 1,5V$}
		\end{figure}
	\end{frame}
	
	\begin{frame}
		\frametitle{Simulazione Matlab}
		\framesubtitle{Grafico iterazioni-valore di L}
		
		\begin{figure}
			\subgrafico{L-iterations/1000Hz/1,0V}{Grafico iterazioni-valore di L di un segnale sinusoidale con $f = 1000Hz$ e $A = 1,0V$}
			\subgrafico{L-iterations/1000Hz/1,45V}{Grafico iterazioni-valore di L di un segnale sinusoidale con $f = 1000Hz$ e $A = 1,45V$}
		\end{figure}
	\end{frame}
		
	\begin{frame}
		\frametitle{Implementazione C++}
		
		\begin{columns}[t]
			\begin{column}{0.45\textwidth}
				\vspace{-25px}
				\screenshotvst[width=0.8\textwidth]{vst/gui/L_2_3}{Interfaccia grafica del VST con $L = 32$}
			\end{column}		
			\begin{column}{0.5\textwidth}
				Scritta in C++ utilizzando il framework JUCE, grazie a cui si è potuto creare un VST.

				\quad\\
				La percentuale di utilizzo si un singolo core della CPU varia tra il $9\%$ e il $16\%$, con una media di utilizzo del $13\%$.
			\end{column}
		\end{columns}		
	\end{frame}
			
	\begin{frame}
		\frametitle{Circuito analogico}
		
		\begin{figure}
			\subscreenshot{analog/circuito}{Circuito analogico del clipper}
			\subscreenshot{analog/banco}{Banco di lavoro}
		\end{figure}
	\end{frame}
		
	\begin{frame}
		\frametitle{Circuito analogico}
		
		\begin{figure}
			\subscreenshot{analog/screens/1,0V}{Input e output del clipper analogico con $A = 1V$}
			\subscreenshot{analog/screens/1,5V}{Input e output del clipper analogico con $A = 1,5V$}
		\end{figure}
	\end{frame}
		
	\begin{frame}
		\frametitle{Conclusioni}
		
		\begin{figure}
			\subgrafico{conclusioni/noise/noise}{Grafico del rumore bianco}
			\subgrafico{conclusioni/noise/2,5V}{Grafico delle iterazioni per il rumore bianco}
		\end{figure}
		
		%TODO dare un'idea della riduzione di carico dopo avere ottimizzato L
		%conclusioni e dei possibili sviluppi future.
	\end{frame}
	
	\begin{frame}
		\frametitle{Conclusioni}
		
		\begin{itemize}
			\item Il numero medio di iterazioni richieste dal metodo numerico per arrivare ad un risultato diminuisce con l'aumentare del parametro $L$, tendendo al numero medio di iterazioni richieste dal metodo Newton-Raphson.
			\item Se il segnale in ingresso ha un'ampiezza che supera una certa soglia, determinata anche in base alla tipologia dell'onda che rappresenta il segnale, l'algoritmo non converge più per alcuni o tutti i valori di $L$, a partire da un determinato campione, solitamente vicino ad un picco dell'onda.
			\item Per valori ottimizzati del parametro $L$ l'utilizzo della CPU, quando viene eseguito il VST, si aggira intorno al $10\%$.
		\end{itemize}
		
		\vspace{10px}
		\textbf{Sviluppi futuri:} permettere all'algoritmo di trattare segnali multidimensionali, poiché nell'implementazione per processare segnali stereo si usano due clipper mono.
	\end{frame}
\end{document}