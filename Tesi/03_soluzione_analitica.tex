\chapter{Soluzione analitica}
	Il primo passo per trovare il valore di $V_{out}$ ad un istante di tempo $t$ è stato di trovare l'equazione differenziale che rappresenta la correlazione tra il valore della tensione in entrata $V_{in}$ e il valore delle tensioni delle due coppie di diodi $V_{A}$ e $V_{B}$.
	
	Per fare ciò si è partiti dalla seconda legge di Kirchhoff applicata al circuito preso in considerazione (figura \ref{fig:clipper}).
	\[
	\begin{cases}
		V_{in}-V_{R} = V_{out}\\
		V_{in}-V_{R} = V_{A}+V_{B}\\
		V_{A}+V_{B} = V_{out}\\
	\end{cases}
	\]
	
	Inoltre applicando la prima legge di Kirchhoff abbiamo
	\[
		i_{in} = i_{out}+i_{D}
	\]
	
	Considerando la definizione di capacità elettrica
	\[
		C = \frac{Q}{\Delta V} \Rightarrow Q = C \cdot \Delta V
	\]
	e che nell'intervallo di tempo infinitesimo $\mathrm{d}t$ l'intensità di corrente del condensatore vale 
	\[
		I = \diff{Q}{t}
	\]
	otteniamo che
	\[
		I = \diff{}{t}(C \cdot \Delta V) = C\diff{\Delta V}{t}
	\]
	\pagebreak
		
	\section{Equazione differenziale}
		Da quanto appena osservato otteniamo
		\begin{equation}
			\label{eq:differenziale}
			\begin{split}
				V_{out}(t) &= V_{in}(t)-V_{R}(t)\\
				V_{out}(t) &= V_{in}(t)-R_{in}i_{in}\\
				V_{out}(t) &= V_{in}(t)-R_{in}\left[i_{out}+i_{D}\right]\\
				V_{out}(t) &= V_{in}(t)-R_{in}\left[C\diff{V_{out}(t)}{t}+i_{D}\right]\\
				V_{out}(t) &= V_{in}(t)-R_{in}\left[C\diff{V_{A}(t)}{t}+C\diff{V_{B}(t)}{t}+i_{D}\right]\\
				V_{A}(t)+V_{B}(t) &= V_{in}(t)-R_{in}\left[C\diff{V_{A}(t)}{t}+C\diff{V_{B}(t)}{t}+\beta_{B}\left(e^{\alpha_{B}V_{B}(t)}-e^{-\alpha_{B}V_{B}(t)}\right)\right]\\
				V_{A}(t)+V_{B}(t) &= V_{in}(t)-R_{in}\left[C\diff{V_{A}(t)}{t}+C\diff{V_{B}(t)}{t}+2\beta_{B}\sinh(\alpha_{B}V_{B}(t))\right]\\
				\frac{V_{A}(t)+V_{B}(t)}{R_{in}} &= \frac{V_{in}(t)}{R_{in}}-C\diff{V_{A}(t)}{t}-C\diff{V_{B}(t)}{t}-2\beta_{B}\sinh(\alpha_{B}V_{B}(t))\\
				C\diff{V_{A}(t)}{t}+C\diff{V_{B}(t)}{t} &= \frac{V_{in}(t)-V_{A}(t)-V_{B}(t)}{R_{in}}-2\beta_{B}\sinh(\alpha_{B}V_{B}(t))\\
				\diff{V_{A}(t)}{t}+\diff{V_{B}(t)}{t} &= \frac{1}{C}\left(\frac{V_{in}(t)-V_{A}(t)-V_{B}(t)}{R_{in}}-2\beta_{B}\sinh(\alpha_{B}V_{B}(t))\right)\\
			\end{split}
		\end{equation}
		che è l'equazione che mette in relazione le tensioni sui vari componenti del circuito.
		\pagebreak
		
		Per poter risolvere l'equazione tramite il metodo numerico dobbiamo esprimere l'equazione differenziale appena trovata (\ref{eq:differenziale}) in funzione di una delle due variabili $V_{A}$ o $V_{B}$, variabile che comparirà sia nel primo, che nel secondo termine della nostra equazione. In questo caso è stato scelto di esprimere tutto in funzione di $V_{B}$, perciò il passo successivo è stato di trovare una formula che esprimesse $V_{A}$ in funzione di $V_{B}$.
		\begin{equation}
			\label{eq:tensione_A_B}
			\begin{split}
				\beta_{A}\left(e^{\alpha_{A}V_{A}(t)}-1\right)-\beta_{A}\left(e^{-\alpha_{A}V_{A}(t)}-1\right) &= \beta_{B}\left(e^{\alpha_{B}V_{B}(t)}-1\right)-\beta_{B}\left(e^{-\alpha_{B}V_{B}(t)}-1\right)\\
				\beta_{A}\left(e^{\alpha_{A}V_{A}(t)}-e^{-\alpha_{A}V_{A}(t)}\right) &= \beta_{B}\left(e^{\alpha_{B}V_{B}(t)}-e^{-\alpha_{B}V_{B}(t)}\right)\\
				\cancel{2}\beta_{A}\left(\frac{e^{\alpha_{A}V_{A}(t)}-e^{-\alpha_{A}V_{A}(t)}}{2}\right) &= \cancel{2}\beta_{B}\left(\frac{e^{\alpha_{B}V_{B}(t)}-e^{-\alpha_{B}V_{B}(t)}}{2}\right)\\
				\beta_{A} \sinh(\alpha_{A}V_{A}(t)) &= \beta_{B} \sinh(\alpha_{B}V_{B}(t))\\
				\sinh(\alpha_{A}V_{A}(t)) &= \frac{\beta_{B}}{\beta_{A}} \sinh(\alpha_{B}V_{B}(t))\\
				\alpha_{A}V_{A}(t) &= \arcsinh \left(\frac{\beta_{B}}{\beta_{A}} \sinh(\alpha_{B}V_{B}(t))\right)\\
				V_{A}(t) &= \frac{1}{\alpha_{A}} \arcsinh \left(\frac{\beta_{B}}{\beta_{A}} \sinh(\alpha_{B}V_{B}(t))\right)
			\end{split}
		\end{equation}
		\pagebreak
		
		Prima di sostituire le occorrenze di $V_{A}$ nell'equazione differenziale (\ref{eq:differenziale}) con la formula appena trovata (\ref{eq:tensione_A_B}) notiamo che nella (\ref{eq:differenziale}) è richiesto di derivare $V_{A}$.
		\begin{equation}
			\label{eq:derivata_va}
			\begin{split}
				\diff{V_{a}(t)}{t} &= \diff{}{t}\left(\frac{1}{\alpha_{A}} \arcsinh \left(\frac{\beta_{B}}{\beta_{A}} \sinh(\alpha_{B}V_{B}(t))\right)\hspace{-5px}\right)\\
				&= \frac{1}{\alpha_{A}}\cdot\diff{}{t}\left(\arcsin\left(\frac{\beta_{B}}{\beta_{A}}\sinh(\alpha_{B}V_{B}(t))\right)\hspace{-5px}\right)\\
				&= \frac{\diff{}{t}\left(\frac{\beta_{B}}{\beta_{A}}\sinh(\alpha_{B}V_{B}(t))\right)}{\alpha_{A}\sqrt{1+\left(\frac{\beta_{B}}{\beta_{A}}\sinh(\alpha_{B}V_{B}(t))\right)^{2}}}\\
				&= \frac{\beta_{B}}{\beta_{A}}\cdot\frac{\diff{}{t}(\sinh(\alpha_{B}V_{B}(t)))}{\alpha_{A}\sqrt{1+\left(\sinh(\alpha_{B}V_{B}(t))\right)^{2}}}\\
				&= \frac{\beta_{B}\cosh(\alpha_{B}V_{B}(t))\diff{}{t}(\alpha_{B}V_{B}(t))}{\alpha_{A}\beta_{A}\sqrt{1+\left(\frac{\beta_{B}}{\beta_{A}}\sinh(\alpha_{B}V_{B}(t))\right)^{2}}}\\
				&= \frac{\alpha_{B}\beta_{B}\cosh(\alpha_{B}V_{B}(t))}{\alpha_{A}\beta_{A}\sqrt{1+\left(\frac{\beta_{B}}{\beta_{A}}\sinh(\alpha_{B}V_{B}(t))\right)^{2}}}\cdot\diff{V_{B}(t)}{t}
			\end{split}
		\end{equation}
		\pagebreak

		Procediamo quindi con la sostituzione di $V_{A}$ e $\diff{V_{A}}{t}$ nell'equazione differenziale (\ref{eq:differenziale}) con quanto nelle equazioni (\ref{eq:tensione_A_B}) e (\ref{eq:derivata_va}).
		\begin{equation}
			\label{eq:integrale}
			\begin{split}
				&\frac{\alpha_{B}\beta_{B}\cosh(\alpha_{B}V_{B}(t))}{\alpha_{A}\beta_{A}\sqrt{1+\left(\frac{\beta_{B}}{\beta_{A}}\sinh(\alpha_{B}V_{B}(t))\right)^{2}}}\cdot\diff{V_{B}(t)}{t}+\diff{V_{B}(t)}{t} =\\&= \frac{1}{C}\left(\frac{1}{R_{in}}\left(V_{in}(t)-\frac{1}{\alpha_{A}} \arcsinh \left(\frac{\beta_{B}}{\beta_{A}} \sinh(\alpha_{B}V_{B}(t))\right)-V_{B}(t)\right)-2\beta_{B}\sinh(\alpha_{B}V_{B}(t))\right)\Rightarrow\\
				&\Rightarrow \diff{V_{B}(t)}{t}\left(\frac{\alpha_{B}\beta_{B}\cosh(\alpha_{B}V_{B}(t))}{\alpha_{A}\beta_{A}\sqrt{1+\left(\frac{\beta_{B}}{\beta_{A}}\sinh(\alpha_{B}V_{B}(t))\right)^{2}}}+1\right) =\\&= \frac{1}{C}\left(\frac{1}{R_{in}}\left(V_{in}(t)-\frac{1}{\alpha_{A}} \arcsinh \left(\frac{\beta_{B}}{\beta_{A}} \sinh(\alpha_{B}V_{B}(t))\right)-V_{B}(t)\right)-2\beta_{B}\sinh(\alpha_{B}V_{B}(t))\right)\Rightarrow\\
				&\Rightarrow \diff{V_{B}(t)}{t} = \left(\frac{\alpha_{B}\beta_{B}\cosh(\alpha_{B}V_{B}(t))}{\alpha_{A}\beta_{A}\sqrt{1+\left(\frac{\beta_{B}}{\beta_{A}}\sinh(\alpha_{B}V_{B}(t))\right)^{2}}}+1\right)^{-1}\cdot\\&\cdot\frac{1}{C}\left(\frac{1}{R_{in}}\left(V_{in}(t)-\frac{1}{\alpha_{A}} \arcsinh \left(\frac{\beta_{B}}{\beta_{A}} \sinh(\alpha_{B}V_{B}(t))\right)-V_{B}(t)\right)-2\beta_{B}\sinh(\alpha_{B}V_{B}(t))\right)\Rightarrow\\
				&\Rightarrow V_{B}(t) = \Int{\Bigg(\left(\frac{\alpha_{B}\beta_{B}\cosh(\alpha_{B}V_{B}(t))}{\alpha_{A}\beta_{A}\sqrt{1+\left(\frac{\beta_{B}}{\beta_{A}}\sinh(\alpha_{B}V_{B}(t))\right)^{2}}}+1\right)^{-1}\cdot\\&\cdot\frac{1}{C}\left(\frac{1}{R_{in}}\left(V_{in}(t)-\frac{1}{\alpha_{A}} \arcsinh \left(\frac{\beta_{B}}{\beta_{A}} \sinh(\alpha_{B}V_{B}(t))\right)-V_{B}(t)\right)-2\beta_{B}\sinh(\alpha_{B}V_{B}(t))\right)\Bigg)}{t}\\
			\end{split}
		\end{equation}
		\pagebreak
		
	\section{Calcolo numerico dell'equazione differenziale}
		Da notare che la precedente equazione (\ref{eq:sostituzione}) è in forma implicita, poiché il termine $V_{B}(t)$ compare sia nel lhs che nel rhs dell'equazione. Per tale motivo l'unico modo per trovare il valore di questa incognita è utilizzare un metodo numerico.
		
		Arrivati a questo punto per risolvere l'equazione differenziale appena ottenuta (\ref{eq:sostituzione}) si è utilizzato il metodo di Eulero implicito (o all'indietro). Si è scelto questo metodo per discretizzare l'equazione da risolvere perché ha una buona stabilità ed è semplice da applicare.
		
		Il metodo di Eulero all'indietro viene ricavato dall'approssimazione della derivata con le differenze finite all'indietro
		\[
			\diff{y}{x} \approx \frac{y_{n}-y_{n-1}}{h}
		\]
		da cui si ottiene
		\[
			\frac{y_{n}-y_{n-1}}{h} = f(x_{n},y_{n}) \implies y_{n} = y_{n-1}+hf(x_{n},y_{n})
		\]
		
		Sappiamo che 
		\begin{equation}
			\label{eq:sostituzione}
			\begin{split}
				&\left(\frac{\alpha_{B}\beta_{B}\cosh(\alpha_{B}V_{B}(t))}{\alpha_{A}\beta_{A}\sqrt{1+\left(\frac{\beta_{B}}{\beta_{A}}\sinh(\alpha_{B}V_{B}(t))\right)^{2}}}+1\right)^{-1} =\\
				&= \left(\frac{\alpha_{B}\beta_{B}\cosh(\alpha_{B}V_{B}(t))+\alpha_{A}\beta_{A}\sqrt{1+\left(\frac{\beta_{B}}{\beta_{A}}\sinh(\alpha_{B}V_{B}(t))\right)^{2}}}{\alpha_{A}\beta_{A}\sqrt{1+\left(\frac{\beta_{B}}{\beta_{A}}\sinh(\alpha_{B}V_{B}(t))\right)^{2}}}\right)^{-1}\\
				&= \frac{\sqrt{1+\left(\frac{\beta_{B}}{\beta_{A}}\sinh(\alpha_{B}V_{B}(t))\right)^{2}}}{\alpha_{B}\beta_{B}\cosh(\alpha_{B}V_{B}(t))+\alpha_{A}\beta_{A}\sqrt{1+\left(\frac{\beta_{B}}{\beta_{A}}\sinh(\alpha_{B}V_{B}(t))\right)^{2}}}
			\end{split}
		\end{equation}
		
		Impostiamo il parametro $h$ in modo che sia equivalente al periodo di campionamento
		\begin{equation}
			\label{eq:valore_h}
			\begin{split}
				h = T = \frac{1}{Fs}
			\end{split}
		\end{equation}
		
		L'equazione discretizzata risulta quindi essere
		\begin{equation}
			\label{eq:discretizzazione}
			\begin{split}
				&\widehat V_{B}[n] = h \; \widehat V_{B}[n]+\widehat V_{B}[n-1] =\\
				&= \frac{T}{C}\cdot\frac{\sqrt{1+\left(\frac{\beta_{B}}{\beta_{A}}\sinh(\alpha_{B}V_{B}[n])\right)^{2}}}{\alpha_{B}\beta_{B}\cosh(\alpha_{B}V_{B}[n])+\alpha_{A}\beta_{A}\sqrt{1+\left(\frac{\beta_{B}}{\beta_{A}}\sinh(\alpha_{B}V_{B}[n])\right)^{2}}}\cdot\\&\cdot\left(\frac{1}{R_{in}}\left(V_{in}(n)-\frac{1}{\alpha_{A}} \arcsinh \left(\frac{\beta_{B}}{\beta_{A}} \sinh(\alpha_{B}\widehat V_{B}[n])\right)-\widehat V_{B}[n]\right)-2\beta_{B}\sinh(\alpha_{B}\widehat V_{B}[n])\right)+\\&+\widehat V_{B}[n-1]\\
			\end{split}
		\end{equation}
		\pagebreak
	
	\section{Applicazione del metodo numerico}
		% Risolvo la funzione ottenuta dalla risoluzione della differenziale
		Richiamiamo ora la formula del metodo numerico da applicare (\ref{eq:fixed-point})
		\begin{equation}
			\label{eq:punto_fisso1}
			\begin{split}
				\mathbf{x}^{(\lambda+1)} &= \mathbf{x}^{(\lambda)}-\mathbf{K}(\mathbf{x}^{(\lambda)})(\mathbf{x}^{(\lambda)}-\mathbf{f}(\mathbf{x}^{(\lambda)}))\\
				&= \mathbf{x}^{(\lambda)}-\sum_{l=0}^{L} \left(\mathbf{J_{f}(x)}\right)^{l}(\mathbf{x}^{(\lambda)}-\mathbf{f}(\mathbf{x}^{(\lambda)}))\\
			\end{split}
		\end{equation}
		poiché per (\ref{eq:fixed-point-sum}) abbiamo $\mathbf{K}^{(L)}(\mathbf{x}) = \sum_{l=0}^{L} \left(\mathbf{J_{f}(x)}\right)^{l}$\\
		
		%Applichiamo quindi il metodo numerico al nostro problema, ovvero trovare il valore numerico di $V_{B}$.
		Nel nostro caso la formula del metodo numerico verrà usata per calcolare il valore $V_{B}$, utilizzando la funzione discretizzata $f = \widehat V_{B}$. La (\ref{eq:punto_fisso1}) dunque diventa
		\begin{equation}
			\label{eq:punto_fisso2}
			\begin{split}
				V_{B}^{(\lambda+1)} = V_{B}^{(\lambda)}-\sum_{l=0}^{L} \left(J_{\widehat V_{B}}(V_{B}^{(\lambda)})\right)^{l}(V_{B}^{(\lambda)}-\widehat V_{B}[V_{B}^{(\lambda)}])
			\end{split}
		\end{equation}
		
		Il metodo richiede il calcolo del termine $J_{\widehat V_{B}}$, ovvero la matrice jacobiana della funzione $\widehat V_{B}$. La matrice jacobiana nel caso di vettori con un singolo elemento, come nel nostro caso, equivale alla derivata della funzione a cui è associata. In poche parole, in questo caso $J_{\widehat V_{B}}(n) = \diff{\widehat V_{B}[n]}{n}$.\\
		
		Per calcolare tale derivata bisognerà applicare la regola del prodotto che richiederà di scomporre la funzione $\widehat V_{B}[n] = \frac{h}{C}\cdot f[n]\cdot g[n]$.
		Le due funzioni $f[n]$ e $g[n]$ risulteranno essere
		\begin{equation}
			\begin{split}
				f[n] = \left(\frac{\alpha_{B}\beta_{B}\cosh(\alpha_{B}\widehat V_{B}[n])}{\alpha_{A}\beta_{A}\sqrt{1+\left(\frac{\beta_{B}}{\beta_{A}}\sinh(\alpha_{B}\widehat V_{B}[n])\right)^{2}}}+1\right)^{-1} = \frac{1}{\varphi[n]}
			\end{split}
		\end{equation}
		\begin{equation}
			\begin{split}
				g[n] = \frac{1}{R_{in}}\left(V_{in}(n)-V_{A}[n]-\widehat V_{B}[n]\right)-2\beta_{B}\sinh(\alpha_{B}\widehat V_{B}[n])
			\end{split}
		\end{equation}
		dove si è sostituito il termine della sottrazione con $V_{A}[n]$ seguendo la (\ref{eq:tensione_A_B}).

		La derivata sarà
		\begin{equation}
			\begin{split}
				\diff{\widehat V_{B}[n]}{n} &= \frac{h}{C}\left(\diff{f[n]}{n}\cdot g[n]+f[n]\cdot \diff{g[n]}{n}\right)\\
				&= \frac{h}{C}\left(-\diff{\varphi[n]}{n}\cdot\frac{1}{\varphi[n]^{2}}\cdot g[n]+\frac{1}{\varphi[n]}\cdot \diff{g[n]}{n}\right)
			\end{split}
		\end{equation}

		Iniziamo con la derivata della funzione $f[n]$ che abbiamo detto equivalere a $\diff{f[n]}{n} = \diff{\frac{1}{\varphi[n]}}{n} = -\diff{\varphi[n]}{n}\cdot\frac{1}{\varphi[n]^{2}}$. A questo punto bisogna calcolare la derivata della funzione $\varphi[n]$.
		\begin{equation}
			\begin{split}
				\diff{\varphi[n]}{n} &= \diff{}{n}\left(\frac{\alpha_{B}\beta_{B}\cosh(\alpha_{B}\widehat V_{B}[n])}{\alpha_{A}\beta_{A}\sqrt{1+\left(\frac{\beta_{B}}{\beta_{A}}\sinh(\alpha_{B}\widehat V_{B}[n])\right)^{2}}}+1\right)\\
				&= \frac{\alpha_{B}\beta_{B}}{\alpha_{A}\beta_{A}}\cdot\diff{}{n}\left(\frac{\cosh(\alpha_{B}\widehat V_{B}[n])}{\sqrt{1+\left(\frac{\beta_{B}}{\beta_{A}}\sinh(\alpha_{B}\widehat V_{B}[n])\right)^{2}}}\right)\\
				%&= \frac{\alpha_{B}\beta_{B}}{\alpha_{A}\beta_{A}}\cdot\frac{\psi}{\left(\sqrt{1+\left(\frac{\beta_{B}}{\beta_{A}}\sinh(\alpha_{B}\widehat V_{B}[n])\right)^{2}}\right)^{2}}
				&= \frac{\alpha_{B}\beta_{B}}{\alpha_{A}\beta_{A}}\cdot\frac{\psi}{1+\left(\frac{\beta_{B}}{\beta_{A}}\sinh(\alpha_{B}\widehat V_{B}[n])\right)^{2}}
			\end{split}
		\end{equation}
		
		dove $\psi[n]$ è definita come segue
		\begin{equation}
			\begin{split}
				\psi[n] = \alpha_{B}\sinh(\alpha_{B}\widehat V_{B}[n])\cdot\sqrt{1+\left(\frac{\beta_{B}}{\beta_{A}}\sinh(\alpha_{B}\widehat V_{B}[n])\right)^{2}}-\cosh(\alpha_{B}\widehat V_{B}[n])\,\cdot\\
				\cdot\,\diff{}{n}\left(\sqrt{1+\left(\frac{\beta_{B}}{\beta_{A}}\sinh(\alpha_{B}\widehat V_{B}[n])\right)^{2}}\right)
			\end{split}
		\end{equation}
		\pagebreak
		
		Calcoliamo infine l'ultima derivata
		\vspace{-10px}
		\begin{equation}
			\begin{split}
				\diff{}{n}\left(\sqrt{1+\left(\frac{\beta_{B}}{\beta_{A}}\sinh(\alpha_{B}\widehat V_{B}[n])\right)^{2}}\right) &= \frac{\diff{}{n}\left(1+\left(\frac{\beta_{B}}{\beta_{A}}\sinh(\alpha_{B}\widehat V_{B}[n])\right)^{2}\right)}{2\sqrt{1+\left(\frac{\beta_{B}}{\beta_{A}}\sinh(\alpha_{B}\widehat V_{B}[n])\right)^{2}}}\\
				&= \frac{2\left(\frac{\beta_{B}}{\beta_{A}}\sinh(\alpha_{B}\widehat V_{B}[n])\right)\frac{\beta_{B}}{\beta_{A}}\diff{}{n}\left(\sinh(\alpha_{B}\widehat V_{B}[n])\right)}{2\sqrt{1+\left(\frac{\beta_{B}}{\beta_{A}}\sinh(\alpha_{B}\widehat V_{B}[n])\right)^{2}}}\\
				&= \frac{\alpha_{B}\beta_{B}^{2}}{\beta_{A}^{2}}\cdot\frac{\sinh(\alpha_{B}\widehat V_{B}[n])\cosh(\alpha_{B}\widehat V_{B}[n])}{\sqrt{1+\left(\frac{\beta_{B}}{\beta_{A}}\sinh(\alpha_{B}\widehat V_{B}[n])\right)^{2}}}
			\end{split}
		\end{equation}

		Passiamo ora alla derivata della funzione $g[n]$, che equivale a
		\begin{equation}
			\begin{split}
				\diff{g[n]}{n} &= \frac{1}{R_{in}}\cdot\diff{\left(V_{in}(n)-V_{A}[n]-V_{B}[n]\right)}{n}-\diff{\left(2\beta_{B}\sinh(\alpha_{B}\widehat V_{B}[n])\right)}{n} =\\
				&= \frac{1}{R_{in}}\left(\diff{V_{in}(n)}{n}-\diff{V_{A}[n]}{n}-\diff{V_{B}[n]}{n}\right)-2\alpha_{B}\beta_{B}\cosh(\alpha_{B}\widehat V_{B}[n]) =\\
				&= \frac{1}{R_{in}}\left(-\frac{\alpha_{B}\beta_{B}\cosh(\alpha_{B}\widehat V_{B}[n])}{\alpha_{A}\beta_{A}\sqrt{1+\left(\frac{\beta_{B}}{\beta_{A}}\sinh(\alpha_{B}\widehat V_{B}[n])\right)^{2}}}-1\right)-2\alpha_{B}\beta_{B}\cosh(\alpha_{B}\widehat V_{B}[n])
			\end{split}
		\end{equation}
		\vspace{-8px}
		
		\noindent dove $\diff{V_{A}[n]}{n}$ è data dalla (\ref{eq:derivata_va}).\\
		
		Dopo aver calcolato numericamente il valore di $V_{B}$ tramite la formula (\ref{eq:punto_fisso2}), si può calcolare la corrente ai capi del condensatore, ovvero $V_{out}$, utilizzando sempre la seconda legge di Kirchhoff:
		\begin{equation}
			\label{eq:tensione_finale}
			\begin{split}
				V_{out}[n] &= V_{A}[n]+V_{B}[n]\\
				&= \frac{1}{\alpha_{A}} \arcsinh \left(\frac{\beta_{B}}{\beta_{A}} \sinh(\alpha_{B}V_{B}[n])\right)+V_{B}[n]
			\end{split}
		\end{equation}
%\flushbottom