\raggedbottom
\chapter{Soluzione analitica}
	%TODO: aggiungere riferimenti a Kirchhoff

	Il primo passo per trovare il valore di $V_{out}$ ad un istante di tempo $t$ è stato di trovare l'equazione differenziale che rappresenta la correlazione tra il valore della tensione in entrata $V_{in}$ e il valore delle tensioni delle due coppie di diodi $V_{A}$ e $V_{B}$.
	
	Per fare ciò si è partiti dalla seconda legge di Kirchhoff applicata al circuito preso in considerazione (figura \ref{fig:clipper}).
	\[
	\begin{cases}
		V_{in}-V_{R} = V_{out}\\
		V_{in}-V_{R} = V_{A}+V_{B}\\
		V_{A}+V_{B} = V_{out}\\
	\end{cases}
	\]
	
	considerando la definizione di capacità elettrica
	\[
		C = \frac{Q}{\Delta V} \Rightarrow Q = C \cdot \Delta V
	\]
	e che nell'intervallo di tempo infinitesimo $\mathrm{d}t$ l'intensità di corrente del condensatore vale 
	\[
		I = \diff{Q}{t}
	\]
	otteniamo che
	\[
		I = \diff{}{t}(C \cdot \Delta V) = C\diff{\Delta V}{t}
	\]
	\pagebreak
	
	Da quanto appena osservato otteniamo
	\begin{equation}
		\label{eq:differenziale}
		\begin{split}
			V_{out}(t) &= V_{in}(t)-V_{R}(t)\\
			V_{out}(t) &= V_{in}(t)-R_{in}\left[i_{out}+i_{D}\right]\\
			V_{out}(t) &= V_{in}(t)-R_{in}\left[C\diff{V_{out}(t)}{t}+i_{D}\right]\\
			V_{out}(t) &= V_{in}(t)-R_{in}\left[C\diff{V_{A}(t)}{t}+C\diff{V_{B}(t)}{t}+i_{D}\right]\\
			V_{A}(t)+V_{B}(t) &= V_{in}(t)-R_{in}\left[C\diff{V_{A}(t)}{t}+C\diff{V_{B}(t)}{t}+\beta_{B}\left(e^{\alpha_{B}V_{B}(t)}-e^{-\alpha_{B}V_{B}(t)}\right)\right]\\
			V_{A}(t)+V_{B}(t) &= V_{in}(t)-R_{in}\left[C\diff{V_{A}(t)}{t}+C\diff{V_{B}(t)}{t}+2\beta_{B}\sinh(\alpha_{B}V_{B}(t))\right]\\
			\frac{V_{A}(t)+V_{B}(t)}{R_{in}} &= \frac{V_{in}(t)}{R_{in}}-C\diff{V_{A}(t)}{t}-C\diff{V_{B}(t)}{t}-2\beta_{B}\sinh(\alpha_{B}V_{B}(t))\\
			C\diff{V_{A}(t)}{t}+C\diff{V_{B}(t)}{t} &= \frac{V_{in}(t)-V_{A}(t)-V_{B}(t)}{R_{in}}-2\beta_{B}\sinh(\alpha_{B}V_{B}(t))\\
			\diff{V_{A}(t)}{t}+\diff{V_{B}(t)}{t} &= \frac{1}{C}\left(\frac{V_{in}(t)-V_{A}(t)-V_{B}(t)}{R_{in}}-2\beta_{B}\sinh(\alpha_{B}V_{B}(t))\right)\\
		\end{split}
	\end{equation}
	\pagebreak
	
	Per poter risolvere l'equazione tramite il metodo numerico dobbiamo esprimere l'equazione differenziale (\ref{eq:differenziale}) in funzione di una delle due variabili $V_{A}$ o $V_{B}$, variabile che comparirà sia nel primo, che nel secondo termine della nostra equazione. In questo caso è stato scelto di esprimere tutto in funzione di $V_{B}$, perciò il passo successivo è stato di trovare una formula che mettesse in relazione $V_{A}$ con $V_{B}$.
	\begin{equation}
		\label{eq:tensione_A_B}
		\begin{split}
			\beta_{A}\left(e^{\alpha_{A}V_{A}(t)}-1\right)-\beta_{A}\left(e^{-\alpha_{A}V_{A}(t)}-1\right) &= \beta_{B}\left(e^{\alpha_{B}V_{B}(t)}-1\right)-\beta_{B}\left(e^{-\alpha_{B}V_{B}(t)}-1\right)\\
			\beta_{A}\left(e^{\alpha_{A}V_{A}(t)}-e^{-\alpha_{A}V_{A}(t)}\right) &= \beta_{B}\left(e^{\alpha_{B}V_{B}(t)}-e^{-\alpha_{B}V_{B}(t)}\right)\\
			\cancel{2}\beta_{A}\left(\frac{e^{\alpha_{A}V_{A}(t)}-e^{-\alpha_{A}V_{A}(t)}}{2}\right) &= \cancel{2}\beta_{B}\left(\frac{e^{\alpha_{B}V_{B}(t)}-e^{-\alpha_{B}V_{B}(t)}}{2}\right)\\
			\beta_{A} \sinh(\alpha_{A}V_{A}(t)) &= \beta_{B} \sinh(\alpha_{B}V_{B}(t))\\
			\sinh(\alpha_{A}V_{A}(t)) &= \frac{\beta_{B}}{\beta_{A}} \sinh(\alpha_{B}V_{B}(t))\\
			\alpha_{A}V_{A}(t) &= \arcsinh \left(\frac{\beta_{B}}{\beta_{A}} \sinh(\alpha_{B}V_{B}(t))\right)\\
			V_{A}(t) &= \frac{1}{\alpha_{A}} \arcsinh \left(\frac{\beta_{B}}{\beta_{A}} \sinh(\alpha_{B}V_{B}(t))\right)
		\end{split}
	\end{equation}
	\pagebreak
	
	Prima di sostituire le occorrenze di $V_{A}$ nell'equazione differenziale (\ref{eq:differenziale}) con la formula appena trovata (\ref{eq:tensione_A_B}) notiamo che nella (\ref{eq:differenziale}) è richiesto di derivare $V_{A}$.
	\begin{equation}
		\label{eq:derivata}
		\begin{split}
			\diff{V_{a}(t)}{t} &= \diff{}{t}\left(\frac{1}{\alpha_{A}} \arcsinh \left(\frac{\beta_{B}}{\beta_{A}} \sinh(\alpha_{B}V_{B}(t))\right)\hspace{-5px}\right)\\
			&= \frac{1}{\alpha_{A}}\cdot\diff{}{t}\left(\arcsin\left(\frac{\beta_{B}}{\beta_{A}}\sinh(\alpha_{B}V_{B}(t))\right)\hspace{-5px}\right)\\
			&= \frac{\diff{}{t}\left(\frac{\beta_{B}}{\beta_{A}}\sinh(\alpha_{B}V_{B}(t))\right)}{\alpha_{A}\sqrt{1+\left(\frac{\beta_{B}}{\beta_{A}}\sinh(\alpha_{B}V_{B}(t))\right)^{2}}}\\
			&= \frac{\beta_{B}}{\beta_{A}}\cdot\frac{\diff{}{t}(\sinh(\alpha_{B}V_{B}(t)))}{\alpha_{A}\sqrt{1+\left(\sinh(\alpha_{B}V_{B}(t))\right)^{2}}}\\
			&= \frac{\beta_{B}\cosh(\alpha_{B}V_{B}(t))\diff{}{t}(\alpha_{B}V_{B}(t))}{\alpha_{A}\beta_{A}\sqrt{1+\left(\frac{\beta_{B}}{\beta_{A}}\sinh(\alpha_{B}V_{B}(t))\right)^{2}}}\\
			&= \frac{\alpha_{B}\beta_{B}\cosh(\alpha_{B}V_{B}(t))}{\alpha_{A}\beta_{A}\sqrt{1+\left(\frac{\beta_{B}}{\beta_{A}}\sinh(\alpha_{B}V_{B}(t))\right)^{2}}}\cdot\diff{V_{B}(t)}{t}
		\end{split}
	\end{equation}
	
	\vspace{100px}
	
	Procediamo quindi con la sostituzione di $V_{A}$ e $\diff{V_{A}}{t}$ nell'equazione differenziale (\ref{eq:differenziale}) con quanto nelle equazioni (\ref{eq:tensione_A_B}) e (\ref{eq:derivata}).
	\pagebreak
	
	\begin{equation}
		\label{eq:sostituzione}
		\begin{split}
			&\frac{\alpha_{B}\beta_{B}\cosh(\alpha_{B}V_{B}(t))}{\alpha_{A}\beta_{A}\sqrt{1+\left(\frac{\beta_{B}}{\beta_{A}}\sinh(\alpha_{B}V_{B}(t))\right)^{2}}}\cdot\diff{V_{B}(t)}{t}+\diff{V_{B}(t)}{t} =\\&= \frac{1}{C}\left(\frac{1}{R_{in}}\left(V_{in}(t)-\frac{1}{\alpha_{A}} \arcsinh \left(\frac{\beta_{B}}{\beta_{A}} \sinh(\alpha_{B}V_{B}(t))\right)-V_{B}(t)\right)-2\beta_{B}\sinh(\alpha_{B}V_{B}(t))\right)\Rightarrow\\
			&\Rightarrow \diff{V_{B}(t)}{t}\left(\frac{\alpha_{B}\beta_{B}\cosh(\alpha_{B}V_{B}(t))}{\alpha_{A}\beta_{A}\sqrt{1+\left(\frac{\beta_{B}}{\beta_{A}}\sinh(\alpha_{B}V_{B}(t))\right)^{2}}}+1\right) =\\&= \frac{1}{C}\left(\frac{1}{R_{in}}\left(V_{in}(t)-\frac{1}{\alpha_{A}} \arcsinh \left(\frac{\beta_{B}}{\beta_{A}} \sinh(\alpha_{B}V_{B}(t))\right)-V_{B}(t)\right)-2\beta_{B}\sinh(\alpha_{B}V_{B}(t))\right)\Rightarrow\\
			&\Rightarrow \diff{V_{B}(t)}{t}\left(\frac{\alpha_{B}\beta_{B}\cosh(\alpha_{B}V_{B}(t))+\alpha_{A}\beta_{A}\sqrt{1+\left(\frac{\beta_{B}}{\beta_{A}}\sinh(\alpha_{B}V_{B}(t))\right)^{2}}}{\alpha_{A}\beta_{A}\sqrt{1+\left(\frac{\beta_{B}}{\beta_{A}}\sinh(\alpha_{B}V_{B}(t))\right)^{2}}}\right) =\\&= \frac{1}{C}\left(\frac{1}{R_{in}}\left(V_{in}(t)-\frac{1}{\alpha_{A}} \arcsinh \left(\frac{\beta_{B}}{\beta_{A}} \sinh(\alpha_{B}V_{B}(t))\right)-V_{B}(t)\right)-2\beta_{B}\sinh(\alpha_{B}V_{B}(t))\right)\Rightarrow\\
			&\Rightarrow \diff{V_{B}(t)}{t} = \frac{\alpha_{A}\beta_{A}\sqrt{1+\left(\frac{\beta_{B}}{\beta_{A}}\sinh(\alpha_{B}V_{B}(t))\right)^{2}}}{\alpha_{B}\beta_{B}\cosh(\alpha_{B}V_{B}(t))+\alpha_{A}\beta_{A}\sqrt{1+\left(\frac{\beta_{B}}{\beta_{A}}\sinh(\alpha_{B}V_{B}(t))\right)^{2}}}\cdot\\&\cdot\frac{1}{C}\left(\frac{1}{R_{in}}\left(V_{in}(t)-\frac{1}{\alpha_{A}} \arcsinh \left(\frac{\beta_{B}}{\beta_{A}} \sinh(\alpha_{B}V_{B}(t))\right)-V_{B}(t)\right)-2\beta_{B}\sinh(\alpha_{B}V_{B}(t))\right)\Rightarrow\\
			&\Rightarrow V_{B}(t) = \frac{\alpha_{A}\beta_{A}}{C}\Int{\frac{\sqrt{1+\left(\frac{\beta_{B}}{\beta_{A}}\sinh(\alpha_{B}V_{B}(t))\right)^{2}}}{\alpha_{B}\beta_{B}\cosh(\alpha_{B}V_{B}(t))+\alpha_{A}\beta_{A}\sqrt{1+\left(\frac{\beta_{B}}{\beta_{A}}\sinh(\alpha_{B}V_{B}(t))\right)^{2}}}\cdot\\&\cdot\left(\frac{1}{R_{in}}\left(V_{in}(t)-\frac{1}{\alpha_{A}} \arcsinh \left(\frac{\beta_{B}}{\beta_{A}} \sinh(\alpha_{B}V_{B}(t))\right)-V_{B}(t)\right)-2\beta_{B}\sinh(\alpha_{B}V_{B}(t))\right)}{t}\\
		\end{split}
	\end{equation}
	
	% Risolvo la differenziale
	Arrivati a questo punto per risolvere l'equazione differenziale appena ottenuta (\ref{eq:sostituzione}) si è utilizzato il metodo di Eulero implicito (o all'indietro). Si è scelto questo metodo per discretizzare l'equazione da risolvere perché ha una buona stabilità ed è semplice da applicare.
	
	Il metodo di Eulero all'indietro viene ricavato dall'approssimazione della derivata con le differenze finite all'indietro
	\[
		\diff{y}{x} \approx \frac{y_{n}-y_{n-1}}{h}
	\]
	da cui si ottiene
	\[
		\frac{y_{n}-y_{n-1}}{h} = f(x_{n},y_{n}) \implies y_{n} = y_{n-1}+hf(x_{n},y_{n})
	\]
	
	%TODO equazione implicita, devo usare un metodo numerico per risolverla
	
	Sia $T = h$ allora l'equazione discretizzata risulta essere
	\begin{equation}
		\label{eq:discretizzazione}
		\begin{split}
			&\widehat V_{B}[n] = \frac{T}{C}\cdot\frac{\alpha_{A}\beta_{A}\sqrt{1+\left(\frac{\beta_{B}}{\beta_{A}}\sinh(\alpha_{B}\widehat V_{B}[n])\right)^{2}}}{\alpha_{B}\beta_{B}\cosh(\alpha_{B}\widehat V_{B}[n])+\alpha_{A}\beta_{A}\sqrt{1+\left(\frac{\beta_{B}}{\beta_{A}}\sinh(\alpha_{B}\widehat V_{B}[n])\right)^{2}}}\cdot\\&\cdot\left(\frac{1}{R_{in}}\left(V_{in}(t)-\frac{1}{\alpha_{A}} \arcsinh \left(\frac{\beta_{B}}{\beta_{A}} \sinh(\alpha_{B}\widehat V_{B}[n])\right)-\widehat V_{B}[n]\right)-2\beta_{B}\sinh(\alpha_{B}\widehat V_{B}[n])\right)+\\&+\widehat V_{B}[n-1]\\
		\end{split}
	\end{equation}
	\pagebreak
	
	% Risolvo la funzione ottenuta dalla risoluzione della differenziale
	Richiamiamo la formula del metodo numerico da applicare (\ref{eq:fixed-point})
	\begin{equation}
		\label{eq:punto_fisso1}
		\begin{split}
			\mathbf{x}^{(\lambda+1)} &= \mathbf{x}^{(\lambda)}-\mathbf{K}(\mathbf{x}^{(\lambda)})(\mathbf{x}^{(\lambda)}-\mathbf{f}(\mathbf{x}^{(\lambda)}))\\
			&= \mathbf{x}^{(\lambda)}-\sum_{l=0}^{L} \left(\mathbf{J_{f}(x)}\right)^{l}(\mathbf{x}^{(\lambda)}-\mathbf{f}(\mathbf{x}^{(\lambda)}))\\
		\end{split}
	\end{equation}
	poiché per (\ref{eq:fixed-point-sum}) abbiamo $\mathbf{K}^{(L)}(\mathbf{x}) = \sum_{l=0}^{L} \left(\mathbf{J_{f}(x)}\right)^{l}$\\
	
	%Applichiamo quindi il metodo numerico al nostro problema, ovvero trovare il valore numerico di $V_{B}$.
	Nel nostro caso la formula del metodo numerico verrà usata per trovare $V_{B}$, quindi la (\ref{eq:punto_fisso1}) diventa
	\begin{equation}
		\label{eq:punto_fisso2}
		\begin{split}
			%\widetilde V_{B}^{(\lambda+1)} = \widetilde V_{B}^{(\lambda)}-\sum_{l=0}^{L} \left(J_{V_{B}}(\widetilde V_{B}^{(\lambda)})\right)^{l}(\widetilde V_{B}^{(\lambda)}-\widehat V_{B}[\widetilde V_{B}^{(\lambda)}])
			V_{B}^{(\lambda+1)} = V_{B}^{(\lambda)}-\sum_{l=0}^{L} \left(J_{V_{B}}(V_{B}^{(\lambda)})\right)^{l}(V_{B}^{(\lambda)}-\widehat V_{B}[V_{B}^{(\lambda)}])
		\end{split}
	\end{equation}
	
	Il metodo richiede il calcolo del termine $J_{V_{B}}$, ovvero la matrice jacobiana della funzione $V_{B}$. La matrice jacobiana nel caso di vettori con un singolo elemento, come nel nostro caso, si riduce alla derivata della funzione a cui è associata. In poche parole, in questo caso $J_{V_{B}}(t) = \diff{V_{B}(t)}{t}$.
	\begin{equation}
		\label{eq:jacobiana}
		\begin{split}
			&J_{V_{B}}(t) = \diff{V_{B}(t)}{t} = \frac{\alpha_{A}\beta_{A}}{C}\reallybig(\frac{\sqrt{1+\left(\frac{\beta_{B}}{\beta_{A}}\sinh(\alpha_{B}V_{B}(t))\right)^{2}}}{\alpha_{B}\beta_{B}\cosh(\alpha_{B}V_{B}(t))+\alpha_{A}\beta_{A}\sqrt{1+\left(\frac{\beta_{B}}{\beta_{A}}\sinh(\alpha_{B}V_{B}(t))\right)^{2}}}\cdot\\&\cdot\left(\frac{1}{R_{in}}\left(V_{in}(t)-\frac{1}{\alpha_{A}} \arcsinh \left(\frac{\beta_{B}}{\beta_{A}} \sinh(\alpha_{B}V_{B}(t))\right)-V_{B}(t)\right)-2\beta_{B}\sinh(\alpha_{B}V_{B}(t))\right)\hspace{-7px}\reallybig)
		\end{split}
	\end{equation}
	%\pagebreak
	
	Dopo aver calcolato numericamente il valore di $V_{B}$, si può calcolare la corrente ai capi del condensatore, ovvero $V_{out}$, utilizzando sempre la seconda legge di Kirchhoff:
	\begin{equation}
		\label{eq:punto_fisso2}
		\begin{split}
			V_{out}(t) &= V_{A}(t)+V_{B}(t)\\
			&= \frac{1}{\alpha_{A}} \arcsinh \left(\frac{\beta_{B}}{\beta_{A}} \sinh(\alpha_{B}V_{B}(t))\right)+V_{B}(t)
		\end{split}
	\end{equation}
\flushbottom