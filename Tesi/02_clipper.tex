\chapter{Clipper analogico}
	Per poter analizzare il metodo numerico appena descritto si è deciso di scegliere un circuito analogico da digitalizzare per creare una rete di filtri digitali e poter così eseguire le computazioni necessarie alla risoluzione di questa rete utilizzando proprio il metodo numerico proposto.
	
	\section{Descrizione di un clipper analogico}
		I clipper, detti anche limitatori, sono dei circuiti che tagliano la parte di un'onda che supera una certa ampiezza. Le onde sinusoidali che superano di molto questa soglia tenderanno ad assomigliare a delle onde quadre.
		%TODO aggiungere un esempio di grafico input-output di un clipper semplice
		
		\begin{wrapfigure}{r}{0.615\textwidth}
			\begin{circuitikz}[american voltages, scale=0.9, transform shape]
				\draw
					% Maglia esterna
					(0,3) to[sinusoidal voltage source,l_=$V_{in}$] (0,0)		% segnale d'ingresso
					(0,3) to[resistor] (3,3)										% resistenza
					-- (7,3)														% filo in alto
					to[C, l^=$V_{out}$] (7,0)									% condensatore
					-- (0,0)														% filo in basso
					(3,0) node[ground]{}											% messa a terra
			
					% Diodi
					(3, 0) to[full diode, *-*] (3,3)								% diodo B di sinistra
					(5, 3) to[full diode, *-*] (5,0)								% diodo B di destra
					
					% Etichette
					(0,0.75) to[open, v_<=$$, outer sep = 2mm] (0,2.25)			% segno di Vin
					(6.5,2.75) to[open, v^=$$, outer sep = 6mm] (6.5,0.25)		% tensione Vout
					;
			\end{circuitikz}
			\caption{Circuito elettronico di un clipper audio semplice}
			\label{fig:easy_clipper}
		\end{wrapfigure}
		
		Il clipper più semplice, illustrato in figura \ref{fig:easy_clipper}, è composto da un generatore di tensione i serie con una resistenza, un condensatore e due diodi, tutto in parallelo.
		
		La scelta di utilizzare un clipper analogico per verificare l'efficacia del metodo numerico è dovuta al fatto che il clipper è un circuito semplice e molto utilizzato per la ricerca in ambito di effetti audio.
	\pagebreak
	
	\section{Circuito del clipper analogico utilizzato}
		Il clipper analogico che è stato scelto per analizzare l'efficacia del metodo numerico proposto differisce dal clipper classico in quanto in questo nuovo circuito sono presenti due ulteriori diodi.
		
		\vspace{15px}
		\begin{figure}[H]
			\centering
			\begin{circuitikz}[american voltages]
				\draw
				% Maglia esterna
				(-1,7) to[sinusoidal voltage source,l_=$V_{in}$] (-1,0)		% segnale d'ingresso
				(-1,7) to[resistor=$R_{in}$] (3,7)							% resistenza
				to[short, i^=$i_{in}$] (4,7)									% iin
				to[short, i^=$i_{out}$] (5,7)								% iout
				-- (8,7)														% filo in alto
				to[C, l_=$C$] (8,0)											% condensatore
				-- (-1,0)													% filo in basso
				(4,0) node[ground]{}											% messa a terra
				
				% Diodi
				(4,7) to[short, *-*] (4,6)			% connessione tra la maglia esterna e i diodi A
				(4,7) to[short, i^=$i_{D}$] (4,6)	% iD
				(5,6) -- (3,6)						% connessione in alto tra i diodi A
				to[empty diode, l=$D_{A}$] (3,4)		% diodo A di sinistra
				-- (5,4)								% connessione in basso tra i diodi A
				to[empty diode, l_=$D_{A}$] (5,6)	% diodo A di destra
				(4,4) to[short, *-*] (4,3)			% connessione i diodi A e i diodi B
				(5,3) -- (3,3)						% connessione in alto tra i diodi B
				to[full diode, l=$D_{B}$] (3,1)		% diodo B di sinistra
				-- (5,1)								% connessione in basso tra i diodi B
				to[full diode, l_=$D_{B}$] (5,3)		% diodo B di destra
				(4,1) to[short, *-*] (4,0)			% connessione i diodi B e la maglia esterna
				
				% Etichette
				(-1,2.5) to[open, v_<=$$, outer sep = 2mm] (-1,4.5)			% segno di Vin
				(3,6) to[open, v_=$V_{A}$, outer sep = 5mm] (3,4)			% tensione dei diodi A
				(3,3) to[open, v_=$V_{B}$, outer sep = 5mm] (3,1)			% tensione dei diodi B
				(8,7) to[open, v^=$V_{out}$, outer sep = 6mm] (8,0)			% tensione Vout
				;
			\end{circuitikz}
			\caption{Circuito elettronico del clipper audio utilizzato}
			\label{fig:clipper}
		\end{figure}
		\vspace{10px}
		
		Come possiamo notare dallo schema i diodi sono tra loro a due a due in parallelo e queste due coppie di diodi sono poi collegate in serie tra loro. In particolare la prima coppia di diodi ha caratteristiche diverse dalla seconda coppia. Le intensità dei vari diodi sono descritte dall'equazione di Shockley come segue:
		\[
			i_{D_{A}} = \beta_{A}\left(e^{\alpha_{A}V_{A}(t)}-1\right)
		\]
		\[
			i_{D_{B}} = \beta_{B}\left(e^{\alpha_{B}V_{B}(t)}-1\right)
		\]
		
		Le intensità delle due coppie sono quindi
		\[
			i_{D_{A'}} = \beta_{A}\left(e^{\alpha_{A}V_{A}(t)}-1\right)-\beta_{A}\left(e^{-\alpha_{A}V_{A}(t)}-1\right)
		\]
		per la prima coppia e
		\[
			i_{D_{B'}} = \beta_{B}\left(e^{\alpha_{B}V_{B}(t)}-1\right)-\beta_{B}\left(e^{-\alpha_{B}V_{B}(t)}-1\right)
		\]
		per la seconda.
		
		Essendo le coppie in serie sappiamo che $i_{D} = i_{D_{A'}} = i_{D_{B'}}$ dunque
		\[
			i_{D} = \beta_{A}\left(e^{\alpha_{A}V_{A}(t)}-1\right)-\beta_{A}\left(e^{-\alpha_{A}V_{A}(t)}-1\right) = \beta_{B}\left(e^{\alpha_{B}V_{B}(t)}-1\right)-\beta_{B}\left(e^{-\alpha_{B}V_{B}(t)}-1\right)
		\]
		
		Siamo quindi interessati a trovare il valore di $V_{out}$ fissati i vari parametri e dato il valore di $V_{in}$.