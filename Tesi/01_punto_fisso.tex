\chapter{Metodo mumerico}
	Definiamo un vettore di $N$ funzioni non lineari $\mathbf{f(x)} = [f_{1}(\mathbf{x}), \dots , f_{N}(\mathbf{x})]^{T}$ nel vettore sconosciuto $\mathbf{x} = [\mathbf{x}_{1}, \dots , \mathbf{x}_{N}]^{T}$, dove $^T$ indica la trasposizione. Christoffersen \cite{christoffersen} propone una famiglia di risolutori numerici per la computazione a punto fisso di una soluzione $\mathbf{x}^{*} = \mathbf{f(x^{*})}$ del vettore sconosciuto $\mathbf{x}$ attraverso lo schema seguente:
	\begin{equation}
		\label{eq:fixed-point}
		\mathbf{x}^{(\lambda+1)} = \mathbf{x}^{(\lambda)}-\mathbf{K}(\mathbf{x}^{(\lambda)})(\mathbf{x}^{(\lambda)}-\mathbf{f}(\mathbf{x}^{(\lambda)}))
	\end{equation}
	dove $\mathbf{x}^{(\lambda)}$ è la $\lambda$-esima iterazione dello schema e $\mathbf{K(x)}$ è una matrice quadrata di dimensione $N \times N$, i quali elementi dipendono da $\mathbf{x}$. Se $\mathbf{K}(\mathbf{x}) = \mathbf{I}$, la matrice identità, allora il metodo si riduce ad un risolutore di punto fisso standard \cite{atkinson}. Se $\mathbf{K}(\mathbf{x}) = (\mathbf{I}-\mathbf{J_{f}(x)})^{-1}$, dove $\mathbf{J}_{f}$ è la matrice Jacobiana di $f$, allora il metodo in particolare diventa un risolutore Newton-Raphson \cite{atkinson}.
		
	Una caratteristica notevole è che il metodo di punto fisso standard e il metodo Newton-Raphson possono essere visti come due casi limite che racchiudono una sotto-famiglia di (\ref{eq:fixed-point}). Consideriamo la somma
	\begin{equation}
		\label{eq:fixed-point-sum}
		\mathbf{K}^{(L)}(\mathbf{x}) = \sum_{l=0}^{L} \left(\mathbf{J_{f}(x)}\right)^{l}
	\end{equation}
	allora $\mathbf{K}^{(0)}(\mathbf{x}) = \mathbf{I}$, per definizione di potenza di zero di una matrice non-nulla; d'altro canto dato $\lVert\mathbf{J_{f}(x)}\rVert < 1$, dove l'operatore $\lVert\,\cdot\,\rVert$ indica la norma euclidea, allora (\ref{eq:fixed-point-sum}) converge a $\mathbf{K}^{(\infty)}(\mathbf{x}) = (\mathbf{I}-\mathbf{J_{f}(x)})^{-1}$. Questo risultato generalizza la convergenza della somma geometrica scalare quando il loro rapporto comune ha una norma più piccola di uno. In altre parole, (\ref{eq:fixed-point}) definisce un risolutore di punto fisso standard se la sommatoria si ferma immediatamente (i.e. a $L = 0$), al contrario essa definisce un risolutore Newton-Raphson se la stessa sommatoria non si ferma mai. Successivamente considereremo i casi $0 < L < \infty$ sotto il vincolo $\lVert\mathbf{J_{f}(x)}\rVert < 1$: chiameremo la corrispondente famiglia risolutori di punto fisso, così come chiameremo il parametro $L$ l'ordine dello specifico risolutore.