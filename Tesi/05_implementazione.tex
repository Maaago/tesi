\chapter{Implementazione}
	Partendo dal codice della simulazione scritto in Matlab è stata implementata una versione dell'algoritmo in C++. L'implementazione è stata scritta come plugin \textit{VST3}, uno fra gli standard più diffusi per la creazione i plugin audio.
	
	Il codice utilizza il framework JUCE ed è disponibile nella appendice (\ref{code:cpp}). Rispetto alla simulazione non è stata cambiata alcuna logica di funzionamento: il codice è solo stato riadattato per essere utilizzato col paradigma di programmazione orientato agli oggetti. L'unica cosa che il VST presenta in più è una semplicissima interfaccia grafica.
	\pagebreak
	
	\section{Grafica}
		L'interfaccia grafica del plugin VST è composta da due widget: una checkbox, chiamata toggle button in JUCE e uno slider orizzontale.
		
		\screenshot[scale=0.8]{vst/gui/L_0}{Interfaccia grafica del VST}
		
		Il primo widget serve per attivare o meno il plugin: quando è selezionato l'audio in entrata non viene modificato, quindi il plugin viene bypassato e non viene eseguito alcun calcolo, mentre se è selezionato l'effetto viene attivato e il segnale in ingresso viene processato come mostrato nei grafici input-output (\ref{graphs:input-output}) della simulazione.
		
		\screenshot[scale=0.8]{vst/gui/L_1_3}{Interfaccia grafica del VST con parametro $L$ a circa 1/3}
		\screenshot[scale=0.8]{vst/gui/L_2_3}{Interfaccia grafica del VST con parametro $L$ a circa 2/3}
		
		Lo slider invece serve ad aggiustare il parametro L: dal valore $0$ (default) si può arrivare fino al valore $500$. Il valore corrente viene mostrato in una tendina che si apre quando si clicca sul cursore dello slider e si chiude quando si rilascia il tasto del mouse.
		
		La scritta visualizzata può risultare non molto intuitiva, ma JUCE permette di appendere del testo solo dopo il valore numerico del parametro.
	
	\section{Prestazioni}
		Mentre la simulazione si presta molto a valutazioni quantitative l'implementazione si presta molto bene a misurazioni qualitative, sia sulla qualità dell'audio, sia sul tempo di risposta del metodo numerico. Se per esempio i calcoli eseguiti sul segnale audio in ingresso dovessero essere troppo pesanti, e quindi richiedere troppo tempo, l'audio in uscita ne risentirebbe, risultando frammentato oppure presentando dei click.
		
		Questa situazione si potrebbe presentare particolarmente in contesti real-time, dove il tempo di calcolo deve stare entro determinati limiti per eliminare difetti nel segnale in uscita. Inoltre più è breve il tempo di calcolo, più si può ridurre la latenza audio, ovvero il tempo trascorso dall'acquisizione del segnale e il tempo in cui il segnale viene riprodotto.
		
		È importante segnalare che per ridurre i tempi di calcolo è necessaria una buona ottimizzazione del codice. Nel codice sorgente presente nelle appendici (\ref{code:matlab}) e (\ref{code:cpp}) sono state apportate alcune semplici ottimizzazioni che però hanno influito notevolmente sulle prestazioni finali. Inizialmente infatti già per valori del parametro $L$ che si avvicinavano a $100$ il singolo core utilizzato dal VST veniva utilizzato al massimo, ma adesso è possibile utilizzare anche valori di $L \le 500$, richiedendo circa il $75\%$ di utilizzo del core singolo nel caso in cui $L = 500$, con un i7-4770S a $3,10\,$GHz.
		
		Le seguenti schermate possono dare un'idea qualitativa delle risorse richieste dall'implementazione in C++ del clipper. Dalle schermate si può riconoscere la Digital Audio Workstation (DAW) Live di Ableton, nel quale è stato utilizzato il VST.
		
		\screenshot[scale=0.75]{vst/performances/zero}{Risorse richieste dal clipper quando $L = 0$}
		\pagebreak
		\vspace*{-35px}
		\screenshot[scale=0.75]{vst/performances/half}{Risorse richieste dal clipper quando $L = 250$}
		\vspace*{-10px}
		\screenshot[scale=0.75]{vst/performances/full}{Risorse richieste dal clipper quando $L = 500$}
		\pagebreak
		
		Un'ulteriore misura qualitativa delle prestazioni consiste nel capire quante istanze del VST possono essere eseguite contemporaneamente, richieste per esempio in un contesto di produzione musicale o di missaggio del suono. Nella seguente schermata si può vedere che 4 istanze contemporanee del VST richiedono poco meno del $20\%$ di CPU. Difficilmente si supererà tale numero di istanze in un progetto musicale.
		
		\screenshot[width=\textwidth]{vst/four_instances}{Istanze multiple del VST}
		
		In questo caso è stato impostato il parametro $L = 0$, ma da come si può dedurre da quanto visto precedentemente più il parametro $L$ è alto, più risorse saranno richieste.
		
		