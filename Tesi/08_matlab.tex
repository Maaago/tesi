\chapter{Codice Matlab}
	\label{code:matlab}
	Il codice Matlab si compone di 4 file:
	
	\noindent\hspace{5px}\begin{tabularx}{\textwidth}{r X}
		\textbf{clipper.m}			&dove viene generata l'onda sinusoidale passata in input al clipper e dove si possono aggiustare i vari parametri, come le caratteristiche dei vari componenti o il valore di L\\
		\textbf{process.m}			&dove vengono processati i singoli campioni del segnale e dove viene richiamato anche il metodo numerico\\
		\textbf{fixed\_point.m}		&dove è implementato il metodo numerico a punto fisso geometrico\\
		\textbf{generator.m}			&dove che contiene la funzione che genera le onde dei segnali d'ingresso (non riportato in questo documento)
	\end{tabularx}\\
	
	\matlabcode{clipper.m}{clipper.m}
	\matlabcode{process.m}{process.m}
	\matlabcode{fixed_point.m}{fixed\_point.m}