\chapter{Simulazione}
	Partendo dalle formule ricavate dalla soluzione analitica, in particolare da (\ref{eq:discretizzazione}), (\ref{eq:punto_fisso2}), (\ref{eq:jacobiana}) e (\ref{eq:tensione_finale}), è stata realizzata una simulazione in linguaggio Matlab.
	
	\section{Parametri}
		Per questa simulazione si è scelto di assegnare caratteristiche uguali a tutti i diodi. In particolare il parametro $\alpha = \frac{1}{nV_{E}}$ dove $n \approx 2$ per i diodi al silicio. In questo caso quindi è stato arrotondato $n = 2$.
	
		I valori dei vari componenti scelti per la simulazione sono riassunti di seguito:
	
		\[
			R_{in} = 1\,k\Omega
		\]
		\[
			V_{out} = 100\,nF
		\]
		\[
			V_{E} = 2,23\,mV
		\]
		\[
			\beta = 2,52\,nA
		\]
		quindi
		\[
			\alpha = \frac{1}{nV_{E}} = \frac{1}{2 \cdot 2,23\,mV} = \frac{1}{4,46\,mV}
		\]
	
		Inoltre è stato scelto
		\[
			F_{s} = \frac{1}{T} = 44100\,Hz \; \Rightarrow \; T = \frac{1}{44100}\,s
		\]
		come frequenza di campionamento e step temporale e
		\[
			|V_{B}^{(\lambda+1)} - V_{B}^{(\lambda)}| < 0.1\,mV
		\]
		come criterio d'arresto per il metodo numerico.
	
		Come input è stata scelta un'onda sinusoidale caratterizzata dalla classica equazione $y(x) = \sin (2 \pi f x + \phi)$ e dai seguenti parametri (tranne dove specificato diversamente):
		\[
			F = 100\,Hz
		\]
		\[
			A = 1\,V
		\]
		\[
			\phi = 0
		\]
		
		\vspace{10px}
		L'arco temporale preso in esame è di 0.2\,s.
	\pagebreak
	
	\section{Grafico ingresso-uscita}
		