\chapter{Simulazione}
	Partendo dalle formule ricavate dalla soluzione analitica, in particolare da (\ref{eq:discretizzazione}), (\ref{eq:punto_fisso2}), (\ref{eq:jacobiano}) e (\ref{eq:tensione_finale}), è stata realizzata una simulazione in linguaggio Matlab \cite{matlab, matlab_book}. Matlab mette a disposizione un ambiente per il calcolo numerico che offre del programmatore delle funzionalità molto potenti, richiedendo di scrivere un codice di dimensioni abbastanza contenute. Il sorgente è disponibile nell'appendice (\ref{code:matlab}).
	
	\section{Parametri}
		Per questa simulazione si è scelto di assegnare caratteristiche uguali a tutti i diodi. In particolare vale $\alpha = \frac{1}{nV_{E}}$ dove $n \approx 2$ per i diodi al silicio. In questo caso quindi è stato arrotondato $n = 2$.
	
		I valori dei vari componenti scelti per la simulazione sono riassunti di seguito:
	
		\[
			R_{in} = 1k\Omega
		\]
		\[
			V_{out} = 100nF
		\]
		\[
			V_{E} = 2,23mV
		\]
		\[
			\beta = 2,52nA
		\]
		quindi
		\[
			\alpha = \frac{1}{nV_{E}} = \frac{1}{2 \cdot 2,23mV} = \frac{1}{4,46mV}
		\]
	
		Inoltre è stato scelto
		\[
			Fs = 48000kHz
		\]
		come frequenza di campionamento e di conseguenza
		\[
			T = \frac{1}{Fs} = \frac{1}{48000}
		\]
		come periodo di campionamento e step temporale.
		
		Come criterio d'arresto per il metodo numerico è invece stato scelto
		\[
			|V_{B}^{(\lambda+1)} - V_{B}^{(\lambda)}| < 0,1mV
		\]
	
		Come input è stata scelta un'onda sinusoidale caratterizzata dalla classica equazione $y(x) = \sin (2 \pi f x + \phi)$ e dai seguenti parametri (tranne dove specificato diversamente):
		\[
			f = 100Hz
		\]
		\[
			A = 1V
		\]
		\[
			\phi = 0
		\]
		
		\vspace{10px}
		L'arco temporale preso in esame è di 0.02s.\\
		
		Inoltre, per prevenire cicli infiniti dovuti alla non convergenza del metodo numerico è stato impostato un limite massimo di iterazioni nella risoluzione dell'equazione differenziale discreta (\ref{eq:discretizzazione}): in particolare questo limite è pari a $250$ iterazioni, dopo le quali si esce dal ciclo e si tiene il valore prodotto dalla $250$-esima iterazione.
	\pagebreak
	
	\section{Grafici}
		\subsection{Grafici input-output}
			\label{graphs:input-output}
			Nei seguenti grafici verrà mostrata con una linea blu continua
			\begin{tikzpicture}
				\draw[thick,color=matlab_blue] (0,0) -- (1,0);
				\draw[thick,color=white] (1,0) -- (1,-0.07);				%per allineamento verticale
			\end{tikzpicture}
			il segnale in \textcolor{matlab_blue}{input}, mentre con una linea tratteggiata arancione
			\begin{tikzpicture}
				\draw[thick,dashed,color=matlab_orange] (0,0) -- (1,0);
				\draw[thick,color=white] (1,0) -- (1,-0.07);				%per allineamento verticale
			\end{tikzpicture}
			sarà mostrato il segnale in \textcolor{matlab_orange}{output} generato dal clipper. Il parametro $L$ è fissato a $L = \infty$, in quanto l'output, al variare di $L$, non cambia, ma la convergenza è sempre garantita per tale valore di $L$. La frequenza di campionamento invece è $Fs = 48000kHz$, come già detto in precedenza.\\
		
			\subsubsection{Onda sinusoidale}
				Per prima cosa si è analizzato il comportamento del clipper al variare della frequenza di un'onda sinusoidale in ingresso, fissando $A = 1V$.
				\grafico{in-out/frequenze/50Hz}{Grafico input-output con $f = 50Hz$}
				\graficospace
				\grafico{in-out/frequenze/100Hz}{Grafico input-output con $f = 100Hz$}
				\graficospace
				\grafico{in-out/frequenze/250Hz}{Grafico input-output con $f = 250Hz$}
				\graficospace
				\grafico{in-out/frequenze/500Hz}{Grafico input-output con $f = 500Hz$}
				\graficospace
				\grafico{in-out/frequenze/1000Hz}{Grafico input-output con $f = 1000Hz$}
				\graficospace
				\grafico{in-out/frequenze/1500Hz}{Grafico input-output con $f = 1500Hz$}
				
				All'aumentare della frequenza notiamo uno sfasamento tra le onde di input e di output. Questa latenza, dovuta al fenomeno di carica e scarica di un condensatore, non varia mai, ma è più visibile ad alte frequenze del segnale di input perché l'intervallo di tempo nel quale si va ad analizzare i vari segnali si riduce con l'aumentare della frequenza.
				
				Inoltre l'ampiezza del segnale in uscita si riduce con l'aumentare della frequenza, poiché l'alta velocità di variazione del segnale richiede al condensatore di cambiare stato molto frequentemente e questo impedisce proprio al condensatore di caricarsi del tutto e anzi fa in modo che, all'aumentare della frequenza, il condensatore si carichi sempre di meno per ogni picco del segnale.
				\pagebreak
				
				Successivamente si è analizzato il comportamento del clipper al variare dell'ampiezza dell'onda in ingresso.
				Si è prima considerata un'onda con frequenza $f = 100Hz$.
				\grafico{in-out/ampiezze/100Hz/0,5V}{Grafico input-output con $f = 100Hz$ e $A = 0,5V$}
				\graficospace
				\grafico{in-out/ampiezze/100Hz/1,5V}{Grafico input-output con $f = 100Hz$ e $A = 1,5V$}
				\graficospace
				\grafico{in-out/ampiezze/100Hz/2V}{Grafico input-output con $f = 100Hz$ e $A = 2V$}
				\graficospace
				\grafico{in-out/ampiezze/100Hz/3V}{Grafico input-output con $f = 100Hz$ e $A = 3V$}
				\graficospace
				\grafico{in-out/ampiezze/100Hz/5V}{Grafico input-output con $f = 100Hz$ e $A = 5V$}
				\graficospace
				\grafico{in-out/ampiezze/100Hz/10V}{Grafico input-output con $f = 100Hz$ e $A = 10V$}
				\pagebreak
				
				Si è poi considerata un'onda con frequenza $f = 1000Hz$.
				\grafico{in-out/ampiezze/1000Hz/0,5V}{Grafico input-output con $f = 1000Hz$ e $A = 0,5V$}
				\graficospace
				\grafico{in-out/ampiezze/1000Hz/1,5V}{Grafico input-output con $f = 1000Hz$ e $A = 1,5V$}
				\graficospace
				\grafico{in-out/ampiezze/1000Hz/2V}{Grafico input-output con $f = 1000Hz$ e $A = 2V$}
				\graficospace
				\grafico{in-out/ampiezze/1000Hz/3V}{Grafico input-output con $f = 1000Hz$ e $A = 3V$}
				\graficospace
				\grafico{in-out/ampiezze/1000Hz/5V}{Grafico input-output con $f = 1000Hz$ e $A = 5V$}
				\graficospace
				\grafico{in-out/ampiezze/1000Hz/10V}{Grafico input-output con $f = 1000Hz$ e $A = 10V$}
				\pagebreak
				
				Dai vari grafici si può vedere come, per ogni ampiezza del segnale in ingresso, all'aumentare della frequenza, vengano "tagliate" le parti del segnale che superano una certa soglia di ampiezza, che si riduce all'aumentare dell'ampiezza del segnale in ingresso.
				
				Il valore del parametro $L$ come già detto equivale a $L = \infty$, dato che, per valori minori, non si riesce ad ottenere una buona convergenza dell'algoritmo per segnali sinusoidali. Ricordiamo che se $L = \infty$ allora $\sum_{l=0}^{L} \left(J_{\widehat V_{B}}(V_{B}^{(\lambda)})\right)^{l} = (\mathbf{I}-\mathbf{J_{f}(x)})^{-1}$ per (\ref{eq:fixed-point-sum}) e (\ref{eq:punto_fisso2}).
				
				Per valori di $L \in [0;50]$ l'algoritmo converge su segnali di input sinusoidali solo per valori dell'ampiezza $A < 1,5V$. Per $A = 1,5V$ invece si inizino ad intravedere dei segni di instabilità numerica nelle creste dell'onda, quando si raggiunge il picco dell'ampiezza, in particolare per $f = 100Hz$. Per segnali con questa frequenza e $L = 50$ l'algoritmo non converge, mentre per $L = 0$ si ottiene il seguente grafico.
				\grafico{in-out/ampiezze/100Hz_1,5V_L=0}{Grafico input-output con $f = 100Hz$, $A = 1,5V$ e $L = 0$}
				\pagebreak
				
				Se non ci fosse il limite sulle iterazioni, già per questo valore dell'ampiezza il metodo numerico non terminerebbe mai, come si può notare dal seguente grafico.
				\grafico{in-out/ampiezze/100Hz_1,5V_L=0-iterations}{Grafico delle iterazioni con $f = 100Hz$, $A = 1,5V$ e $L = 0$}
				
				Per $A \ge 1,6V$ (\ref{fig:plots/in-out/ampiezze/100Hz_1,6V_L=0}) l'algoritmo non converge più e da un certo momento in poi ($t \approx 2ms$) viene restituito infinito come valore di output.
				\grafico{in-out/ampiezze/100Hz_1,6V_L=0}{Grafico input-output con $f = 100Hz$, $A = 1,6V$ e $L = 0$}
				\pagebreak
				
				%Da notare come con $A = 1,7V$ (\ref{fig:plots/in-out/ampiezze/1,7V_L0}) si inizino ad intravedere dei segni di instabilità numerica nelle creste dell'onda, quindi quando si raggiunge il picco dell'ampiezza. Se non ci fosse il limite sulle iterazioni già per questo valore dell'ampiezza il metodo numerico non terminerebbe mai. A partire da circa $A = 1,9V$ (\ref{fig:plots/in-out/ampiezze/1,9V_L0}) l'algoritmo non converge più e da un certo momento in poi ($t \approx 2ms$) viene restituito infinito come valore di output.
				%\pagebreak
				
				%Vediamo che aumentando il valore di $L$ l'algoritmo adesso converge anche per valori che prima portavano ad una divergenza (\ref{fig:plots/in-out/ampiezze/1,9V_L50}), nonostante sia presente un'evidente instabilità numerica. Comunque per $L = 50$ il metodo numerico non converge più a partire da circa $A = 2,2V$ (\ref{fig:plots/in-out/ampiezze/2,2V_L50}).
				%\pagebreak
			
			\subsubsection{Altri tipi di onda}
				Di seguito sono riportati alcuni grafici con onde di diverso tipo, per dare un'idea di quale sia la risposta dell'algoritmo alle varie forme d'onda in input. Per ogni forma d'onda viene riportato un grafico con frequenza $f = 100Hz$ e  $f = 1000Hz$ con ampiezza $A = 1,5V$ in entrambi i casi.
				
				\grafico{in-out/altre/triangle/100Hz}{Grafico input-output di un'onda triangolare con $f = 100Hz$}
				\graficospace
				\grafico{in-out/altre/triangle/1000Hz}{Grafico input-output di un'onda triangolare con $f = 1000Hz$}
				\graficospace
				\grafico{in-out/altre/saw/100Hz}{Grafico input-output di un'onda a dente di sega con $f = 100Hz$}
				\graficospace
				\grafico{in-out/altre/saw/1000Hz}{Grafico input-output di un'onda a dente di sega con $f = 1000Hz$}
				\graficospace
				\grafico{in-out/altre/square/100Hz}{Grafico input-output di un'onda quadra con $f = 100Hz$}
				\graficospace
				\grafico{in-out/altre/square/1000Hz}{Grafico input-output di un'onda quadra con $f = 1000Hz$}
				\pagebreak
				
				Nel caso di un rumore in input, non potendo parlare di frequenza, si sono considerati due intervalli di tempo diversi, dando prima una visione d'insieme del comportamento del clipper per poter poi entrare più nel dettaglio
				
				\grafico{in-out/altre/noise/20ms}{Grafico input-output di un rumore per $t = 20ms$}
				\graficospace
				\grafico{in-out/altre/noise/2ms}{Grafico input-output di un rumore per $t = 2ms$}
				\pagebreak
				
		
		\subsection{Grafico iterazioni-tempo}
			\label{graphs:iterazioni-tempo}
			Nei seguenti grafici viene visualizzato il numero medio di iterazioni necessarie a calcolare la corrente $V_{out}$ ai capi del condensatore rispetto all'andamento della tensione in entrata $V_{in}$. Questo numero è ottenuto per ogni campione, dalla media delle iterazioni richieste per l'esecuzione del metodo numerico, che è stato testato per tutti i valori di $L \in [0; 50]$, mentre il segnale considerato è sempre sinusoidale.

			\grafico{iterations/media/0,5V}{Grafico iterazioni-tempo e $A = 0,5V$}
			\graficospace
			\grafico{iterations/media/1,0V}{Grafico iterazioni-tempo e $A = 1V$}
			\graficospace
			\grafico{iterations/media/1,45V}{Grafico iterazioni-tempo e $A = 1,45V$}
			\graficospace
			\grafico{iterations/media/1,5V}{Grafico iterazioni-tempo e $A = 1,5V$}
			\pagebreak

			In particolare per $L = 0$ possiamo notare come il numero di iterazioni richieste per arrivare ad una soluzione aumenta in modo esponenziale rispetto all'ampiezza massima del segnale.
			\grafico{iterations/L=0/1,4V}{Grafico iterazioni-tempo con $L = 0$ e $A = 1,4V$}
			\graficospace
			\grafico{iterations/L=0/1,45V}{Grafico iterazioni-tempo con $L = 0$ e $A = 1,45V$}
			\graficospace
			\grafico{iterations/L=0/1,5V}{Grafico iterazioni-tempo con $L = 0$ e $A = 1,5V$}
			\graficospace

			Per $L = 50$ notiamo invece che come il numero di iterazioni richieste aumenta in modo diverso rispetto al caso precedente. Rispetto ai singoli campioni si ha un esplosione esponenziale quando l'ampiezza del campione raggiunge il picco dell'onda.
			\grafico{iterations/L=50/1,4V}{Grafico iterazioni-tempo con $L = 50$ e $A = 1,4V$}
			\graficospace
			\grafico{iterations/L=50/1,45V}{Grafico iterazioni-tempo con $L = 50$ e $A = 1,45V$}
			\graficospace
			\grafico{iterations/L=50/1,5V}{Grafico iterazioni-tempo con $L = 50$ e $A = 1,5V$}
			
			Dai grafici si può notare che in base all'ampiezza dell'onda sinusoidale in ingresso e quindi in base al valore massimo della $V_{in}$ e anche in base al valore del parametro $L$, le iterazioni richieste dal metodo per convergere variano. In particolare più è alto il modulo della tensione in ingresso, più iterazioni sono richieste al metodo per convergere: avremo quindi dei picchi di iterazioni in corrispondenza dei picchi, sia positivi che negativi, della tensione in ingresso.
			
			Nel caso in cui $A = 1,45V$ si può addirittura notare che per alcuni campioni le iterazioni richieste sono uguali al limite massimo, ovvero $250$ iterazioni (in prossimità dei picchi del segnale). Se non ci fosse tale limite l'algoritmo iterativo non terminerebbe mai.
			\pagebreak
			
		\subsection{Grafico della matrice jacobiana}
			Approfondiamo adesso il motivo dell'instabilità numerica quando si supera una certa soglia dell'ampiezza del segnale in ingresso. Come detto nella sezione (\ref{sec:metodo_numerico}) per ottenere la convergenza del metodo bisogna che valga $\lVert\mathbf{J_{f}(x)}\rVert < 1$. Questo significa che la derivata (\ref{eq:jacobiano}), che corrisponde alla matrice jacobiana nel caso monodimensionale, deve essere sempre minore di $1$.
			
			Analizziamo quindi i valori massimi assunti da tale derivata per ogni campione. Noteremo che quando la derivata supera in modulo il valore $1$ allora l'algoritmo smetterà di convergere.
			
			\grafico{jacobiano/100Hz/1,4V}{Grafico della matrice jacobiana di un segnale sinusoidale con $f = 100\;Hz$ e $A = 1,4V$}
			\graficospace
			\grafico{jacobiano/100Hz/1,45V}{Grafico della matrice jacobiana di un segnale sinusoidale con $f = 100\;Hz$ e $A = 1,45V$}
			\graficospace
			\grafico{jacobiano/100Hz/1,48V}{Grafico della matrice jacobiana di un segnale sinusoidale con $f = 100\;Hz$ e $A = 1,48V$}
			\graficospace
			
			\grafico{jacobiano/1000Hz/1,4V}{Grafico della matrice jacobiana di un segnale sinusoidale con $f = 1000\;Hz$ e $A = 1,4V$}
			\graficospace
			\grafico{jacobiano/1000Hz/1,5V}{Grafico della matrice jacobiana di un segnale sinusoidale con $f = 1000\;Hz$ e $A = 1,5V$}
			\graficospace
			\grafico{jacobiano/1000Hz/1,6V}{Grafico della matrice jacobiana di un segnale sinusoidale con $f = 1000\;Hz$ e $A = 1,6V$}
			\graficospace
			
			Dai grafici si può anche osservare che la derivata per i valori di L più piccoli tende a decrescere meno velocemente, fermandosi vicino al valore $-1$, rimanendo poi più stabile nel tempo.
			\pagebreak
			
		\subsection{Grafico iterazioni-valore di L}
			Da come si può già osservare dai grafici del sottoparagrafo precedente c'è una relazione tra il numero di iterazioni richieste per trovare la soluzione dell'equazione differenziale (\ref{eq:discretizzazione}) e il valore del parametro $L$. Si è deciso quindi di indagare la questa relazione tra i due parametri, che viene dunque rappresentata nei seguenti grafici (i valori di $L$ per cui l'algoritmo non converge sono stati rimossi).
			
			\grafico{L-iterations/100Hz/1,0V}{Grafico iterazioni-valore di L di un segnale sinusoidale con $f = 100\;Hz$ e $A = 1,0V$}
			\graficospace
			\grafico{L-iterations/100Hz/1,45V}{Grafico iterazioni-valore di L di un segnale sinusoidale con $f = 100\;Hz$ e $A = 1,45V$}
			\graficospace
			\grafico{L-iterations/100Hz/1,5V}{Grafico iterazioni-valore di L di un segnale sinusoidale con $f = 100\;Hz$ e $A = 1,5V$}
			\graficospace
			\grafico{L-iterations/1000Hz/1,0V}{Grafico iterazioni-valore di L di un segnale sinusoidale con $f = 1000\;Hz$ e $A = 1,0V$}
			\graficospace
			\grafico{L-iterations/1000Hz/1,45V}{Grafico iterazioni-valore di L di un segnale sinusoidale con $f = 1000\;Hz$ e $A = 1,45V$}
			\graficospace
			\grafico{L-iterations/1000Hz/1,5V}{Grafico iterazioni-valore di L di un segnale sinusoidale con $f = 1000\;Hz$ e $A = 1,5V$}
			\graficospace
			
			Dai grafici osserviamo che più il valore di $L$ aumenta, più il numero di iterazioni per quel preciso valore di $L$ tende al numero di iterazioni richiesto dall'algoritmo di Newton-Raphson (ovvero quando $L = \infty$). L'andamento generale prevede dunque un decremento del numero di iterazioni richieste all'aumentare del valore di $L$.
			
			Da notare come, per il segnale sinusoidale con $f = 100Hz$, se $A \ge 1,45V$ alcuni valori a variare in modo anomalo, discostandosi significativamente dalla tendenza al valore di Newton-Raphson. Questo è legato alla convergenza dell'algoritmo: valori di iterazioni molto alti rispetto alla tendenza indicano che l'algoritmo richiede molte iterazioni per convergere, ma comunque, anche grazie al limite massimo delle iterazioni, poi arriva ad un risultato. Valori invece molto bassi, ovvero più bassi del numero di iterazioni richieste per $L$ infinito (che sono stati omessi), indicano che, da un certo campione in poi, l'algoritmo non convergerà più. Questo farà sì che il numero di iterazioni richieste dai campioni successivi sia nullo.
			
			\grafico{L-iterations/noise/3,0V}{Grafico iterazioni-valore di L di un rumore con $A = 3,0V$}
			\graficospace
			\grafico{L-iterations/noise/3,5V}{Grafico iterazioni-valore di L di un rumore con $A = 3,5V$}
			\graficospace
			\grafico{L-iterations/noise/4,0V}{Grafico iterazioni-valore di L di un rumore con $A = 4,0V$}
			\graficospace
			
			Se il segnale in input è un rumore possiamo osservare gli stessi fenomeni che si verificano nel caso del segnale sinusoidale. Un'importante differenza è però che l'ampiezza del segnale per la quale l'algoritmo smette di convergere è significativamente maggiore per il rumore rispetto alla sinusoide. Si tratta di più del doppio: circa $3,5V$ contro circa $1,5V$.
			\pagebreak
			
		\subsection{Grafico tempo d'esecuzione-Valore di L}
			\label{subsec:tempo_esecuzione}
			Nei seguenti grafici si può notare la durata dell'esecuzione della simulazione Matlab in base al variare del parametro $L$. Il tempo d'esecuzione mostrato nel grafico è una media dei valori di varie esecuzioni ripetute della simulazione.
			
			\grafico{L-time/sine/1,0V_no_optimization}{Grafico tempo d'esecuzione-Valore di L con $A = 1V$ con calcolo dello jacobiano per $L = 0$}
			\graficospace
			
			Da questo grafico possiamo osservare come l'andamento del tempo d'esecuzione riflette il variare del numero di iterazioni richieste per ogni valore di $L$. In particolare per valori piccoli di $L$, quindi per $L < 4$, il tempo d'esecuzione tenderà a diminuire con l'aumentare del valore di $L$, poiché il numero di iterazioni richieste per arrivare ad un risultato tende a diminuire (vedi \ref{fig:plots/L-iterations/1000Hz/1,0V}). Per $L \ge 4$ invece il tempo aumenta linearmente, seppur di poco, con l'aumentare di $L$. Questo perché si vanno ad aggiungere termini alla sommatoria descritta in (\ref{eq:fixed-point-sum}).
			
			Il grafico appena presentato è stato generato calcolando il valore dello jacobiano anche nel caso base, quindi quando $L = 0$: in questo caso la sommatoria (\ref{eq:fixed-point-sum}), contenuta nella (\ref{eq:punto_fisso2}), vale $1$, poiché
			\[
				\sum_{l=0}^{0} \left(J_{\widehat V_{B}}(V_{B}^{(\lambda)})\right)^{l} = \left(J_{\widehat V_{B}}(V_{B}^{(\lambda)})\right)^{0} = 1
			\]
			
			Sapendo questo è possibile evitare il calcolo di $J_{\widehat V_{B}}(V_{B}^{(\lambda)})$ per $L = 0$ e questo permette di risparmiare molto tempo, come si può osservare dai grafici seguenti.
			
			\grafico{L-time/sine/1,0V}{Grafico tempo d'esecuzione-valore di L di un segnale sinusoidale con $A = 1,0V$}
			\graficospace
			\grafico{L-time/sine/1,2V}{Grafico tempo d'esecuzione-valore di L di un segnale sinusoidale con $A = 1,2V$}
			\graficospace
			\grafico{L-time/sine/1,4V}{Grafico tempo d'esecuzione-valore di L di un segnale sinusoidale con $A = 1,4V$}
			\graficospace
			\grafico{L-time/sine/1,45V}{Grafico tempo d'esecuzione-valore di L di un segnale sinusoidale con $A = 1,45V$}
			\graficospace
			\grafico{L-time/sine/1,5V}{Grafico tempo d'esecuzione-valore di L di un segnale sinusoidale con $A = 1,5V$}
			\graficospace
			
			Anche per valori di $L$ piccoli (i.e. $L \in [1;3]$) il tempo richiesto dal calcolo della soluzione tramite il metodo numerico è molto basso. Il valore massimo del tempo d'esecuzione, ovvero $\approx 2,7ms$, per $L = 1$ all'ampiezza di $1,5V$, permette di poter utilizzare il clipper in applicazioni audio real-time senza avere latenze sensibili.
			
			Si osserva inoltre che, coerentemente ai valori visualizzati nei grafici della sottosezione precedente (\ref{graphs:iterazioni-tempo}), all'aumentare dell'ampiezza del segnale in ingresso, e quindi di $V_{in}$, aumenta il tempo richiesto per ottenere una convergenza del metodo. Questo aumento di tempo è dovuto al fatto che le iterazioni richieste per ottenere la convergenza del metodo aumentano con l'aumentare dell'ampiezza del segnale in ingresso.
			\pagebreak
			
			Nel caso in cui l'entrata sia un rumore, l'andamento del tempo d'esecuzione è lo stesso. Per questo specifico rumore l'algoritmo converge fino ad un valore $A = 4,5V$.
			
			\grafico{L-time/noise/3V}{Grafico tempo d'esecuzione-valore di L di un rumore con $A = 3,0V$}
			\graficospace
			\grafico{L-time/noise/3,5V}{Grafico tempo d'esecuzione-valore di L di un rumore con $A = 3,5V$}
			\graficospace
			\grafico{L-time/noise/4V}{Grafico tempo d'esecuzione-valore di L di un rumore con $A = 4,0V$}
			\graficospace
			\grafico{L-time/noise/4,5V}{Grafico tempo d'esecuzione-valore di L di un rumore con $A = 4,5V$}
			\graficospace