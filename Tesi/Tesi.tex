\documentclass[12pt,a4paper,twoside,english,italian]{book}

%%%%%%%%%%%%%%%%%%%%%%%%%%%%%%%%%%%%%%% Header %%%%%%%%%%%%%%%%%%%%%%%%%%%%%%%%%%%%%%%%
%%%%%%%%%%%%%%%%%%%%%%%%%%%%%%%%%%%%%% Packages %%%%%%%%%%%%%%%%%%%%%%%%%%%%%%%%%%%%%%%

\usepackage[italian]{babel}
\usepackage[utf8]{inputenc}


\usepackage{uniudtesi}
\usepackage{siunitx}

\usepackage[nottoc]{tocbibind}
\usepackage{indentfirst}
\usepackage{fancyhdr}
\usepackage{emptypage}

\usepackage{amsmath}
\usepackage{esdiff}
\usepackage{cancel}
\usepackage{circuitikz}
\usepackage{comment}
\usepackage{bigints}

\usepackage{hyperref}

\usepackage{xcolor}
\usepackage{graphicx}
\usepackage{subcaption}
\usepackage{rotating}
\usepackage{float}
\usepackage{wrapfig}
\usepackage{titlesec}
\usepackage{setspace}

\usepackage{listings}
\usepackage{gensymb}
\usepackage{adjustbox}

\usepackage[backend=biber,style=alphabetic,sorting=ynt]{biblatex}


%%%%%%%%%%%%%%%%%%%%%%%%%%%%%%%%%%%%%%% Configs %%%%%%%%%%%%%%%%%%%%%%%%%%%%%%%%%%%%%%%
%\title{Tesi}
%\author{Francesco Magoga}
%\date{\today}

\titolo{Realizzazione a tempo reale\\di un clipper per il suono\\attraverso schemi numerici\\innovativi}
\laureando{Magoga Francesco}
\annoaccademico{2020-2021}

%\setcounter{secnumdepth}{4}

\titleformat{\paragraph}
{\normalfont\normalsize\bfseries}{\theparagraph}{1em}{}
\titlespacing*{\paragraph}
{0pt}{3.25ex plus 1ex minus .2ex}{1.5ex plus .2ex}

\def\arraystretch{1.5}

\addbibresource{bibliography.bib}

\setstretch{1.2}

\definecolor{matlab_blue}{RGB}{0,114,189}
\definecolor{matlab_orange}{RGB}{217,83,25}

\definecolor{codebg}{gray}{0.95}
\colorlet{darkred}{red!90!black}

\newsavebox\mypostbreak
\savebox\mypostbreak{\mbox{\ensuremath{\color{red!85!black}\hookrightarrow}\space}}

\lstdefinestyle{code}
{
	%basicstyle=\ttfamily,
	basicstyle=\ttfamily\normalsize,
	backgroundcolor=\color{codebg},
	sensitive=false,
	alsoletter={.},
	xleftmargin=0.5cm,
	belowskip=2px,
	frame=single,
	framesep=5pt,
	framerule=1px,
	gobble=5,
	tabsize=1,
	columns=fullflexible,
	showstringspaces=false,
	numbers=left,
	firstnumber=1,
	numberfirstline=false,
	stepnumber=2,
	numberstyle=\scriptsize\color{black!80},
	numbersep=10pt,
	tabsize=4,
	keepspaces=true,
	breaklines=true,
	%postbreak={\hbox{\textcolor{red}{$\hookrightarrow$}\space}}
	postbreak=\usebox\mypostbreak
}

\lstdefinestyle{matlab}
{
	keywordstyle=\color{blue},
	stringstyle=\color{red},
	commentstyle=\color{green!80!black},
}

\lstdefinestyle{cpp}
{
	%basicstyle=\color{teal}\ttfamily,
	keywordstyle=\color{magenta!75!black},
	directivestyle=\color{brown!80!black},
	stringstyle=\color{red!85!black},
	commentstyle=\color{green!70!black},
}

\lstdefinelanguage{cppl}
{
	language=C++,
	morekeywords={override},
	moredirectives={once, \#ifndef, \#endif, \#if, \#else, \#endif},
	%moredelim=[s][\color{red!85!black}\ttfamily]{<}{>},
}

\graphicspath{{./imgs/}}

\pagestyle{fancy}
\renewcommand{\chaptermark}[1]{\markboth{#1}{}}
\renewcommand{\sectionmark}[1]{\markright{\thesection\ #1}}
\fancyhf{}
\fancyhead[LE,RO]{\bfseries\thepage}
\fancyhead[LO]{\bfseries\rightmark}
\fancyhead[RE]{\bfseries\leftmark}
\renewcommand{\headrulewidth}{0.5pt}
\renewcommand{\footrulewidth}{0pt}
\setlength{\headheight}{14.5pt}

\setlength{\intextsep}{2pt}

%\facolta{Scienze Matematiche, Fisiche e Naturali} % (default)
\corsodilaureamagistralein{Informatica}
\relatore[Prof.]{Federico Fontana}
%\relatoreDue[Prof.]{Secondo relatore}
%\correlatore{Talaltro dei Tali}
%\correlatoreDue{Secondo Correlatore}
%\dedica{Ai miei genitori\\
%    per non avermi tagliato i viveri} % (facoltativo)

%%%%%%%%%%%%%%%%%%%%%%%%%%%%%%%%%%% Custom commands %%%%%%%%%%%%%%%%%%%%%%%%%%%%%%%%%%%
\newcommand{\sourcecode}[5]{
	%\label{list:#3}
	\lstset{caption={#4},label=code:#3,style=code,#5}
	\lstinputlisting[language=#1]{#2}
}

\newcommand{\matlabcode}[2]{
	\sourcecode{Matlab}{../Matlab/#1}{#1}{#2}{style=matlab}
}

\newcommand{\cppcode}[1]{
	\sourcecode{cppl}{../Juce/Source/#1}{#1}{#1}{style=cpp}
}

\newcommand{\screenshot}[3][]
{
	\begin{figure}[H]
		\centering
		\includegraphics[#1]{#2}
		\vspace*{-25px}
		\caption{#3}
		\label{fig:#2}
	\end{figure}
}

\newcommand{\grafico}[2]
{
	\screenshot[width=\textwidth]{imgs/plots/#1}{#2}
}
\newcommand{\graficospace}{\vspace{15px}}

\DeclareMathOperator{\arcsinh}{arcsinh}

\newcommand{\Int}[2]{\bigint \hspace{-6px} #1 \mathrm{d}#2}

\makeatletter
\newcommand{\reallybig}{\bBigg@{4}}
\newcommand{\ReallyBig}{\bBigg@{5}}
\makeatother
%%%%%%%%%%%%%%%%%%%%%%%%%%%%%%%%%%%%%%% Header %%%%%%%%%%%%%%%%%%%%%%%%%%%%%%%%%%%%%%%%


%%%%%%%%%%%%%%%%%%%%%%%%%%%%%%%%%%%%%% Document %%%%%%%%%%%%%%%%%%%%%%%%%%%%%%%%%%%%%%%
\begin{document}
	\pagestyle{fancy} 
	\frontmatter
	\maketitle	

	\enlargethispage{-1.5\baselineskip}
	\tableofcontents
	%\listoffigures
%	\begingroup
%		\let\clearpage\relax
%		\listoffigures
%		\begingroup
%			\let\clearpage\relax
%			\let\cleardoublepage\relax
%			\par\vspace{2\baselineskip}
			%\lstlistoflistings
%		\endgroup
%		\thispagestyle{empty}
%	\endgroup

	\mainmatter

	\chapter*{Introduzione}
		Il seguente documento tratta dell'implementazione e dell'efficienza di un metodo numerico a punto fisso geometrico per calcolare la soluzione di circuiti non lineari di filtri digitali.
		
		Il metodo numerico proposto permette di passare da un risolutore a punto fisso ad un risolutore Newton-Raphson variando un parametro \texttt{L}.
		
		Per analizzare questo metodo numerico si è deciso di digitalizzare un circuito clipper analogico formato da 4 diodi.
		
	\chapter{Metodo mumerico}
	\label{sec:metodo_numerico}
	Definiamo un vettore di $N$ funzioni non lineari $\mathbf{f(x)} = [f_{1}(\mathbf{x}), \dots , f_{N}(\mathbf{x})]^{T}$ nel vettore sconosciuto $\mathbf{x} = [\mathbf{x}_{1}, \dots , \mathbf{x}_{N}]^{T}$, dove $^T$ indica la trasposizione. Christoffersen \cite{christoffersen} propone una famiglia di risolutori numerici per la computazione a punto fisso di una soluzione $\mathbf{x}^{*} = \mathbf{f(x^{*})}$ del vettore sconosciuto $\mathbf{x}$ attraverso il seguente schema:
	\begin{equation}
		\label{eq:fixed-point}
		\mathbf{x}^{(\lambda+1)} = \mathbf{x}^{(\lambda)}-\mathbf{K}(\mathbf{x}^{(\lambda)})(\mathbf{x}^{(\lambda)}-\mathbf{f}(\mathbf{x}^{(\lambda)}))
	\end{equation}
	dove $\mathbf{x}^{(\lambda)}$ è la $\lambda$-esima iterazione dello schema e $\mathbf{K(x)}$ è una matrice quadrata di dimensione $N \times N$, i quali elementi dipendono da $\mathbf{x}$. Se $\mathbf{K}(\mathbf{x}) = \mathbf{I}$, la matrice identità, allora il metodo si riduce ad un risolutore di punto fisso standard \cite{atkinson}. Se $\mathbf{K}(\mathbf{x}) = (\mathbf{I}-\mathbf{J_{f}(x)})^{-1}$, dove $\mathbf{J}_{f}$ è la matrice Jacobiana di $f$, allora il metodo in particolare diventa un risolutore Newton-Raphson \cite{atkinson}.
		
	Una caratteristica notevole è che il metodo di punto fisso standard e il metodo Newton-Raphson possono essere visti come due casi limite che racchiudono una sotto-famiglia di (\ref{eq:fixed-point}). Consideriamo la somma
	\begin{equation}
		\label{eq:fixed-point-sum}
		\mathbf{K}^{(L)}(\mathbf{x}) = \sum_{l=0}^{L} \left(\mathbf{J_{f}(x)}\right)^{l}
	\end{equation}
	allora $\mathbf{K}^{(0)}(\mathbf{x}) = \mathbf{I}$, per definizione di potenza di zero di una matrice non-nulla; d'altro canto dato $\lVert\mathbf{J_{f}(x)}\rVert < 1$, dove l'operatore $\lVert\,\cdot\,\rVert$ indica la norma euclidea, allora (\ref{eq:fixed-point-sum}) converge a $\mathbf{K}^{(\infty)}(\mathbf{x}) = (\mathbf{I}-\mathbf{J_{f}(x)})^{-1}$. Questo risultato generalizza la convergenza della somma geometrica scalare quando il loro rapporto comune ha una norma più piccola di uno. In altre parole, (\ref{eq:fixed-point}) definisce un risolutore di punto fisso standard se la sommatoria si ferma immediatamente (i.e. a $L = 0$), al contrario essa definisce un risolutore Newton-Raphson se la stessa sommatoria non si ferma mai. Successivamente considereremo i casi $0 < L < \infty$ sotto il vincolo $\lVert\mathbf{J_{f}(x)}\rVert < 1$: chiameremo la corrispondente famiglia risolutori di punto fisso, così come chiameremo il parametro $L$ l'ordine dello specifico risolutore.
	
	\chapter{Clipper audio}
	Per poter analizzare il metodo numerico appena descritto si è deciso di scegliere un circuito analogico da digitalizzare per creare una rete di filtri digitali e poter così eseguire le computazioni necessarie alla risoluzione di questa rete utilizzando proprio il metodo numerico proposto.
	
	\section{Descrizione di un clipper}
		I clipper, detti anche limitatori, sono dei circuiti che tagliano la parte di un'onda di un segnale che supera una certa ampiezza. Le onde sinusoidali che superano di molto questa soglia tenderanno ad assomigliare a delle onde quadre.
		
		\begin{comment}
		\begin{wrapfigure}{r}{0.615\textwidth}
			\begin{circuitikz}[american voltages, scale=0.9, transform shape]
				\draw
					% Maglia esterna
					(0,3) to[sinusoidal voltage source,l_=$V_{in}$] (0,0)		% segnale d'ingresso
					(0,3) to[resistor] (3,3)										% resistenza
					-- (7,3)														% filo in alto
					to[C, l^=$V_{out}$] (7,0)									% condensatore
					-- (0,0)														% filo in basso
					(3,0) node[ground]{}											% messa a terra
			
					% Diodi
					(3, 0) to[full diode, *-*] (3,3)								% diodo B di sinistra
					(5, 3) to[full diode, *-*] (5,0)								% diodo B di destra
					
					% Etichette
					(0,0.75) to[open, v_<=$$, outer sep = 2mm] (0,2.25)			% segno di Vin
					(6.5,2.75) to[open, v^=$$, outer sep = 6mm] (6.5,0.25)		% tensione Vout
					;
			\end{circuitikz}
			\caption{Circuito elettronico di un clipper audio semplice}
			\label{fig:easy_clipper}
		\end{wrapfigure}
		\end{comment}
		\begin{wrapfigure}{r}{0.615\textwidth}
			\begin{circuitikz}[american voltages, scale=0.9, transform shape]
				\draw
					% Maglia esterna
					(0,3) to[sinusoidal voltage source,l_=$V_{in}$] (0,0)		% segnale d'ingresso
					(0,3) to[resistor] (3,3)										% resistenza
					-- (7,3)														% filo in alto
					to[C, l^=$V_{out}$] (7,0)									% condensatore
					-- (0,0)														% filo in basso
					(3,0) node[ground]{}											% messa a terra
			
					% Diodi
					(3, 0) to[full diode, *-*] (3,3)								% diodo B di sinistra
					(5, 3) to[full diode, *-*] (5,0)								% diodo B di destra
					
					% Etichette
					(-0.3,0.65) to[open, v_<=$$, outer sep = 2mm] (-0.3,2.40)			% segno di Vin
					(6.5,2.75) to[open, v^=$$, outer sep = 6mm] (6.5,0.25)		% tensione Vout
					;
			\end{circuitikz}
			\caption{Circuito elettronico di un clipper audio semplice}
			\label{fig:easy_clipper}
		\end{wrapfigure}
		
		Il clipper più semplice, illustrato in figura \ref{fig:easy_clipper}, è composto da un generatore di tensione i serie con una resistenza, un condensatore e due diodi, tutto in parallelo.
		
		La scelta di utilizzare un clipper analogico per verificare l'efficacia del metodo numerico è dovuta al fatto che il clipper è un circuito semplice e molto utilizzato per la ricerca in ambito di effetti audio.
	\pagebreak
	
	\section{Circuito del clipper analogico utilizzato}
		Il clipper analogico che è stato scelto per analizzare l'efficacia del metodo numerico proposto differisce dal clipper classico in quanto in questo nuovo circuito sono presenti due ulteriori diodi.
		
		\vspace{15px}
		\begin{comment}
		\begin{figure}[H]
			\centering
			\begin{circuitikz}[american voltages]
				\draw
				% Maglia esterna
				(-1,7) to[sinusoidal voltage source,l_=$V_{in}$] (-1,0)		% segnale d'ingresso
				(-1,7) to[resistor=$R_{in}$] (3,7)							% resistenza
				to[short, i^=$i_{in}$] (4,7)									% iin
				to[short, i^=$i_{out}$] (5,7)								% iout
				-- (8,7)														% filo in alto
				to[C, l_=$C$] (8,0)											% condensatore
				-- (-1,0)													% filo in basso
				(4,0) node[ground]{}											% messa a terra
				
				% Diodi
				(4,7) to[short, *-*] (4,6)			% connessione tra la maglia esterna e i diodi A
				(4,7) to[short, i^=$i_{D}$] (4,6)	% iD
				(5,6) -- (3,6)						% connessione in alto tra i diodi A
				to[empty diode, l=$D_{A}$] (3,4)		% diodo A di sinistra
				-- (5,4)								% connessione in basso tra i diodi A
				to[empty diode, l_=$D_{A}$] (5,6)	% diodo A di destra
				(4,4) to[short, *-*] (4,3)			% connessione i diodi A e i diodi B
				(5,3) -- (3,3)						% connessione in alto tra i diodi B
				to[full diode, l=$D_{B}$] (3,1)		% diodo B di sinistra
				-- (5,1)								% connessione in basso tra i diodi B
				to[full diode, l_=$D_{B}$] (5,3)		% diodo B di destra
				(4,1) to[short, *-*] (4,0)			% connessione i diodi B e la maglia esterna
				
				% Etichette
				(-1,2.5) to[open, v_<=$$, outer sep = 2mm] (-1,4.5)			% segno di Vin
				(3,6) to[open, v_=$V_{A}$, outer sep = 5mm] (3,4)			% tensione dei diodi A
				(3,3) to[open, v_=$V_{B}$, outer sep = 5mm] (3,1)			% tensione dei diodi B
				(8,7) to[open, v^=$V_{out}$, outer sep = 6mm] (8,0)			% tensione Vout
				;
			\end{circuitikz}
			\caption{Circuito elettronico del clipper audio utilizzato}
			\label{fig:clipper}
		\end{figure}
		\end{comment}
		\begin{figure}[H]
			\centering
			\begin{circuitikz}[american voltages]
				\draw
				% Maglia esterna
				(-1,7) to[sinusoidal voltage source,l_=$V_{in}$] (-1,0)		% segnale d'ingresso
				(-1,7) to[resistor=$R_{in}$] (3,7)							% resistenza
				to[short, i^=$i_{in}$] (4,7)									% iin
				to[short, i^=$i_{out}$] (5,7)								% iout
				-- (8,7)														% filo in alto
				to[C, l_=$C$] (8,0)											% condensatore
				-- (-1,0)													% filo in basso
				(4,0) node[ground]{}											% messa a terra
				
				% Diodi
				(4,7) to[short, *-*] (4,6)			% connessione tra la maglia esterna e i diodi A
				(4,7) to[short, i^=$i_{D}$] (4,6)	% iD
				(5,6) -- (3,6)						% connessione in alto tra i diodi A
				to[empty diode, l=$D_{A}$] (3,4)		% diodo A di sinistra
				-- (5,4)								% connessione in basso tra i diodi A
				to[empty diode, l_=$D_{A}$] (5,6)	% diodo A di destra
				(4,4) to[short, *-*] (4,3)			% connessione i diodi A e i diodi B
				(5,3) -- (3,3)						% connessione in alto tra i diodi B
				to[full diode, l=$D_{B}$] (3,1)		% diodo B di sinistra
				-- (5,1)								% connessione in basso tra i diodi B
				to[full diode, l_=$D_{B}$] (5,3)		% diodo B di destra
				(4,1) to[short, *-*] (4,0)			% connessione i diodi B e la maglia esterna
				
				% Etichette
				(-1.25,2.5) to[open, v_<=$$] (-1.25,4.5)		% segno di Vin
				(2.3,6) to[open, v_=$V_{A}$] (2.3,4)			% tensione dei diodi A
				(2.3,3) to[open, v_=$V_{B}$] (2.3,1)			% tensione dei diodi B
				(9,7) to[open, v^=$V_{out}$] (9,0)			% tensione Vout
				;
			\end{circuitikz}
			\caption{Circuito elettronico del clipper audio utilizzato}
			\label{fig:clipper}
		\end{figure}
		\vspace{10px}
		
		Come possiamo notare dallo schema i diodi sono tra loro a due a due in parallelo e queste due coppie di diodi sono poi collegate in serie tra loro. In particolare la prima coppia di diodi ha caratteristiche diverse dalla seconda coppia. Le intensità dei vari diodi sono descritte dall'equazione di Shockley come segue:
		\[
			i_{D_{A}} = \beta_{A}\left(e^{\alpha_{A}V_{A}(t)}-1\right)
		\]
		\[
			i_{D_{B}} = \beta_{B}\left(e^{\alpha_{B}V_{B}(t)}-1\right)
		\]
		
		Le intensità delle due coppie sono quindi
		\[
			i_{D_{A'}} = \beta_{A}\left(e^{\alpha_{A}V_{A}(t)}-1\right)-\beta_{A}\left(e^{-\alpha_{A}V_{A}(t)}-1\right)
		\]
		per la prima coppia e
		\[
			i_{D_{B'}} = \beta_{B}\left(e^{\alpha_{B}V_{B}(t)}-1\right)-\beta_{B}\left(e^{-\alpha_{B}V_{B}(t)}-1\right)
		\]
		per la seconda.
		
		Essendo le coppie in serie sappiamo che $i_{D} = i_{D_{A'}} = i_{D_{B'}}$ dunque
		\[
			i_{D} = \beta_{A}\left(e^{\alpha_{A}V_{A}(t)}-1\right)-\beta_{A}\left(e^{-\alpha_{A}V_{A}(t)}-1\right) = \beta_{B}\left(e^{\alpha_{B}V_{B}(t)}-1\right)-\beta_{B}\left(e^{-\alpha_{B}V_{B}(t)}-1\right)
		\]
		
		Siamo quindi interessati a trovare il valore di $V_{out}$ fissati i vari parametri e dato il valore di $V_{in}$.
		
		\vspace{30px}
		In figura (\ref{fig:analog/example}) si può vedere un esempio di come un onda risulti prima e dopo l'applicazione del clipper.
		
		\screenshot[width=\textwidth]{analog/example}{Esempio di onda prima e dopo del clipper}
	
	\chapter{Clipper analogico}
	Per poter analizzare il metodo numerico appena descritto si è deciso di scegliere un circuito analogico da digitalizzare per creare una rete di filtri digitali e poter così eseguire le computazioni necessarie alla risoluzione di questa rete utilizzando proprio il metodo numerico proposto.
	
	\section{Descrizione di un clipper analogico}
		I clipper, detti anche limitatori, sono dei circuiti che tagliano la parte di un'onda che supera una certa ampiezza. Le onde sinusoidali che superano di molto questa soglia tenderanno ad assomigliare a delle onde quadre.
		%TODO aggiungere un esempio di grafico input-output di un clipper semplice
		
		\begin{wrapfigure}{r}{0.615\textwidth}
			\begin{circuitikz}[american voltages, scale=0.9, transform shape]
				\draw
					% Maglia esterna
					(0,3) to[sinusoidal voltage source,l_=$V_{in}$] (0,0)		% segnale d'ingresso
					(0,3) to[resistor] (3,3)										% resistenza
					-- (7,3)														% filo in alto
					to[C, l^=$V_{out}$] (7,0)									% condensatore
					-- (0,0)														% filo in basso
					(3,0) node[ground]{}											% messa a terra
			
					% Diodi
					(3, 0) to[full diode, *-*] (3,3)								% diodo B di sinistra
					(5, 3) to[full diode, *-*] (5,0)								% diodo B di destra
					
					% Etichette
					(0,0.75) to[open, v_<=$$, outer sep = 2mm] (0,2.25)			% segno di Vin
					(6.5,2.75) to[open, v^=$$, outer sep = 6mm] (6.5,0.25)		% tensione Vout
					;
			\end{circuitikz}
			\caption{Circuito elettronico di un clipper audio semplice}
			\label{fig:easy_clipper}
		\end{wrapfigure}
		
		Il clipper più semplice, illustrato in figura \ref{fig:easy_clipper}, è composto da un generatore di tensione i serie con una resistenza, un condensatore e due diodi, tutto in parallelo.
		
		La scelta di utilizzare un clipper analogico per verificare l'efficacia del metodo numerico è dovuta al fatto che il clipper è un circuito semplice e molto utilizzato per la ricerca in ambito di effetti audio.
	\pagebreak
	
	\section{Circuito del clipper analogico utilizzato}
		Il clipper analogico che è stato scelto per analizzare l'efficacia del metodo numerico proposto differisce dal clipper classico in quanto in questo nuovo circuito sono presenti due ulteriori diodi.
		
		\vspace{15px}
		\begin{figure}[H]
			\centering
			\begin{circuitikz}[american voltages]
				\draw
				% Maglia esterna
				(-1,7) to[sinusoidal voltage source,l_=$V_{in}$] (-1,0)		% segnale d'ingresso
				(-1,7) to[resistor=$R_{in}$] (3,7)							% resistenza
				to[short, i^=$i_{in}$] (4,7)									% iin
				to[short, i^=$i_{out}$] (5,7)								% iout
				-- (8,7)														% filo in alto
				to[C, l_=$C$] (8,0)											% condensatore
				-- (-1,0)													% filo in basso
				(4,0) node[ground]{}											% messa a terra
				
				% Diodi
				(4,7) to[short, *-*] (4,6)			% connessione tra la maglia esterna e i diodi A
				(4,7) to[short, i^=$i_{D}$] (4,6)	% iD
				(5,6) -- (3,6)						% connessione in alto tra i diodi A
				to[empty diode, l=$D_{A}$] (3,4)		% diodo A di sinistra
				-- (5,4)								% connessione in basso tra i diodi A
				to[empty diode, l_=$D_{A}$] (5,6)	% diodo A di destra
				(4,4) to[short, *-*] (4,3)			% connessione i diodi A e i diodi B
				(5,3) -- (3,3)						% connessione in alto tra i diodi B
				to[full diode, l=$D_{B}$] (3,1)		% diodo B di sinistra
				-- (5,1)								% connessione in basso tra i diodi B
				to[full diode, l_=$D_{B}$] (5,3)		% diodo B di destra
				(4,1) to[short, *-*] (4,0)			% connessione i diodi B e la maglia esterna
				
				% Etichette
				(-1,2.5) to[open, v_<=$$, outer sep = 2mm] (-1,4.5)			% segno di Vin
				(3,6) to[open, v_=$V_{A}$, outer sep = 5mm] (3,4)			% tensione dei diodi A
				(3,3) to[open, v_=$V_{B}$, outer sep = 5mm] (3,1)			% tensione dei diodi B
				(8,7) to[open, v^=$V_{out}$, outer sep = 6mm] (8,0)			% tensione Vout
				;
			\end{circuitikz}
			\caption{Circuito elettronico del clipper audio utilizzato}
			\label{fig:clipper}
		\end{figure}
		\vspace{10px}
		
		Come possiamo notare dallo schema i diodi sono tra loro a due a due in parallelo e queste due coppie di diodi sono poi collegate in serie tra loro. In particolare la prima coppia di diodi ha caratteristiche diverse dalla seconda coppia. Le intensità dei vari diodi sono descritte dall'equazione di Shockley come segue:
		\[
			i_{D_{A}} = \beta_{A}\left(e^{\alpha_{A}V_{A}(t)}-1\right)
		\]
		\[
			i_{D_{B}} = \beta_{B}\left(e^{\alpha_{B}V_{B}(t)}-1\right)
		\]
		
		Le intensità delle due coppie sono quindi
		\[
			i_{D_{A'}} = \beta_{A}\left(e^{\alpha_{A}V_{A}(t)}-1\right)-\beta_{A}\left(e^{-\alpha_{A}V_{A}(t)}-1\right)
		\]
		per la prima coppia e
		\[
			i_{D_{B'}} = \beta_{B}\left(e^{\alpha_{B}V_{B}(t)}-1\right)-\beta_{B}\left(e^{-\alpha_{B}V_{B}(t)}-1\right)
		\]
		per la seconda.
		
		Essendo le coppie in serie sappiamo che $i_{D} = i_{D_{A'}} = i_{D_{B'}}$ dunque
		\[
			i_{D} = \beta_{A}\left(e^{\alpha_{A}V_{A}(t)}-1\right)-\beta_{A}\left(e^{-\alpha_{A}V_{A}(t)}-1\right) = \beta_{B}\left(e^{\alpha_{B}V_{B}(t)}-1\right)-\beta_{B}\left(e^{-\alpha_{B}V_{B}(t)}-1\right)
		\]
		
		Siamo quindi interessati a trovare il valore di $V_{out}$ fissati i vari parametri e dato il valore di $V_{in}$.
	
	\raggedbottom
\chapter{Soluzione analitica}
	Il primo passo per trovare il valore di $V_{out}$ ad un istante di tempo $t$ è stato di trovare l'equazione differenziale che rappresenta la correlazione tra il valore della tensione in entrata $V_{in}$ e il valore delle tensioni delle due coppie di diodi $V_{A}$ e $V_{B}$.
	
	Per fare ciò si è partiti dalla seconda legge di Kirchhoff applicata al circuito preso in considerazione (figura \ref{fig:clipper}).
	\[
	\begin{cases}
		V_{in}-V_{R} = V_{out}\\
		V_{in}-V_{R} = V_{A}+V_{B}\\
		V_{A}+V_{B} = V_{out}\\
	\end{cases}
	\]
	
	considerando la definizione di capacità elettrica
	\[
		C = \frac{Q}{\Delta V} \Rightarrow Q = C \cdot \Delta V
	\]
	e che nell'intervallo di tempo infinitesimo $\mathrm{d}t$ l'intensità di corrente del condensatore vale 
	\[
		I = \diff{Q}{t}
	\]
	otteniamo che
	\[
		I = \diff{}{t}(C \cdot \Delta V) = C\diff{\Delta V}{t}
	\]
	%TODO i_{in} = i_{out}+i_{D}
	\pagebreak
	
	Da quanto appena osservato otteniamo
	\begin{equation}
		\label{eq:differenziale}
		\begin{split}
			V_{out}(t) &= V_{in}(t)-V_{R}(t)\\
			V_{out}(t) &= V_{in}(t)-R_{in}i_{in}\\
			V_{out}(t) &= V_{in}(t)-R_{in}\left[i_{out}+i_{D}\right]\\
			V_{out}(t) &= V_{in}(t)-R_{in}\left[C\diff{V_{out}(t)}{t}+i_{D}\right]\\
			V_{out}(t) &= V_{in}(t)-R_{in}\left[C\diff{V_{A}(t)}{t}+C\diff{V_{B}(t)}{t}+i_{D}\right]\\
			V_{A}(t)+V_{B}(t) &= V_{in}(t)-R_{in}\left[C\diff{V_{A}(t)}{t}+C\diff{V_{B}(t)}{t}+\beta_{B}\left(e^{\alpha_{B}V_{B}(t)}-e^{-\alpha_{B}V_{B}(t)}\right)\right]\\
			V_{A}(t)+V_{B}(t) &= V_{in}(t)-R_{in}\left[C\diff{V_{A}(t)}{t}+C\diff{V_{B}(t)}{t}+2\beta_{B}\sinh(\alpha_{B}V_{B}(t))\right]\\
			\frac{V_{A}(t)+V_{B}(t)}{R_{in}} &= \frac{V_{in}(t)}{R_{in}}-C\diff{V_{A}(t)}{t}-C\diff{V_{B}(t)}{t}-2\beta_{B}\sinh(\alpha_{B}V_{B}(t))\\
			C\diff{V_{A}(t)}{t}+C\diff{V_{B}(t)}{t} &= \frac{V_{in}(t)-V_{A}(t)-V_{B}(t)}{R_{in}}-2\beta_{B}\sinh(\alpha_{B}V_{B}(t))\\
			\diff{V_{A}(t)}{t}+\diff{V_{B}(t)}{t} &= \frac{1}{C}\left(\frac{V_{in}(t)-V_{A}(t)-V_{B}(t)}{R_{in}}-2\beta_{B}\sinh(\alpha_{B}V_{B}(t))\right)\\
		\end{split}
	\end{equation}
	\pagebreak
	
	Per poter risolvere l'equazione tramite il metodo numerico dobbiamo esprimere l'equazione differenziale (\ref{eq:differenziale}) in funzione di una delle due variabili $V_{A}$ o $V_{B}$, variabile che comparirà sia nel primo, che nel secondo termine della nostra equazione. In questo caso è stato scelto di esprimere tutto in funzione di $V_{B}$, perciò il passo successivo è stato di trovare una formula che mettesse in relazione $V_{A}$ con $V_{B}$.
	\begin{equation}
		\label{eq:tensione_A_B}
		\begin{split}
			\beta_{A}\left(e^{\alpha_{A}V_{A}(t)}-1\right)-\beta_{A}\left(e^{-\alpha_{A}V_{A}(t)}-1\right) &= \beta_{B}\left(e^{\alpha_{B}V_{B}(t)}-1\right)-\beta_{B}\left(e^{-\alpha_{B}V_{B}(t)}-1\right)\\
			\beta_{A}\left(e^{\alpha_{A}V_{A}(t)}-e^{-\alpha_{A}V_{A}(t)}\right) &= \beta_{B}\left(e^{\alpha_{B}V_{B}(t)}-e^{-\alpha_{B}V_{B}(t)}\right)\\
			\cancel{2}\beta_{A}\left(\frac{e^{\alpha_{A}V_{A}(t)}-e^{-\alpha_{A}V_{A}(t)}}{2}\right) &= \cancel{2}\beta_{B}\left(\frac{e^{\alpha_{B}V_{B}(t)}-e^{-\alpha_{B}V_{B}(t)}}{2}\right)\\
			\beta_{A} \sinh(\alpha_{A}V_{A}(t)) &= \beta_{B} \sinh(\alpha_{B}V_{B}(t))\\
			\sinh(\alpha_{A}V_{A}(t)) &= \frac{\beta_{B}}{\beta_{A}} \sinh(\alpha_{B}V_{B}(t))\\
			\alpha_{A}V_{A}(t) &= \arcsinh \left(\frac{\beta_{B}}{\beta_{A}} \sinh(\alpha_{B}V_{B}(t))\right)\\
			V_{A}(t) &= \frac{1}{\alpha_{A}} \arcsinh \left(\frac{\beta_{B}}{\beta_{A}} \sinh(\alpha_{B}V_{B}(t))\right)
		\end{split}
	\end{equation}
	\pagebreak
	
	Prima di sostituire le occorrenze di $V_{A}$ nell'equazione differenziale (\ref{eq:differenziale}) con la formula appena trovata (\ref{eq:tensione_A_B}) notiamo che nella (\ref{eq:differenziale}) è richiesto di derivare $V_{A}$.
	\begin{equation}
		\label{eq:derivata}
		\begin{split}
			\diff{V_{a}(t)}{t} &= \diff{}{t}\left(\frac{1}{\alpha_{A}} \arcsinh \left(\frac{\beta_{B}}{\beta_{A}} \sinh(\alpha_{B}V_{B}(t))\right)\hspace{-5px}\right)\\
			&= \frac{1}{\alpha_{A}}\cdot\diff{}{t}\left(\arcsin\left(\frac{\beta_{B}}{\beta_{A}}\sinh(\alpha_{B}V_{B}(t))\right)\hspace{-5px}\right)\\
			&= \frac{\diff{}{t}\left(\frac{\beta_{B}}{\beta_{A}}\sinh(\alpha_{B}V_{B}(t))\right)}{\alpha_{A}\sqrt{1+\left(\frac{\beta_{B}}{\beta_{A}}\sinh(\alpha_{B}V_{B}(t))\right)^{2}}}\\
			&= \frac{\beta_{B}}{\beta_{A}}\cdot\frac{\diff{}{t}(\sinh(\alpha_{B}V_{B}(t)))}{\alpha_{A}\sqrt{1+\left(\sinh(\alpha_{B}V_{B}(t))\right)^{2}}}\\
			&= \frac{\beta_{B}\cosh(\alpha_{B}V_{B}(t))\diff{}{t}(\alpha_{B}V_{B}(t))}{\alpha_{A}\beta_{A}\sqrt{1+\left(\frac{\beta_{B}}{\beta_{A}}\sinh(\alpha_{B}V_{B}(t))\right)^{2}}}\\
			&= \frac{\alpha_{B}\beta_{B}\cosh(\alpha_{B}V_{B}(t))}{\alpha_{A}\beta_{A}\sqrt{1+\left(\frac{\beta_{B}}{\beta_{A}}\sinh(\alpha_{B}V_{B}(t))\right)^{2}}}\cdot\diff{V_{B}(t)}{t}
		\end{split}
	\end{equation}
	
	\vspace{100px}
	
	Procediamo quindi con la sostituzione di $V_{A}$ e $\diff{V_{A}}{t}$ nell'equazione differenziale (\ref{eq:differenziale}) con quanto nelle equazioni (\ref{eq:tensione_A_B}) e (\ref{eq:derivata}).
	\pagebreak
	
	\begin{equation}
		\label{eq:sostituzione}
		\begin{split}
			&\frac{\alpha_{B}\beta_{B}\cosh(\alpha_{B}V_{B}(t))}{\alpha_{A}\beta_{A}\sqrt{1+\left(\frac{\beta_{B}}{\beta_{A}}\sinh(\alpha_{B}V_{B}(t))\right)^{2}}}\cdot\diff{V_{B}(t)}{t}+\diff{V_{B}(t)}{t} =\\&= \frac{1}{C}\left(\frac{1}{R_{in}}\left(V_{in}(t)-\frac{1}{\alpha_{A}} \arcsinh \left(\frac{\beta_{B}}{\beta_{A}} \sinh(\alpha_{B}V_{B}(t))\right)-V_{B}(t)\right)-2\beta_{B}\sinh(\alpha_{B}V_{B}(t))\right)\Rightarrow\\
			&\Rightarrow \diff{V_{B}(t)}{t}\left(\frac{\alpha_{B}\beta_{B}\cosh(\alpha_{B}V_{B}(t))}{\alpha_{A}\beta_{A}\sqrt{1+\left(\frac{\beta_{B}}{\beta_{A}}\sinh(\alpha_{B}V_{B}(t))\right)^{2}}}+1\right) =\\&= \frac{1}{C}\left(\frac{1}{R_{in}}\left(V_{in}(t)-\frac{1}{\alpha_{A}} \arcsinh \left(\frac{\beta_{B}}{\beta_{A}} \sinh(\alpha_{B}V_{B}(t))\right)-V_{B}(t)\right)-2\beta_{B}\sinh(\alpha_{B}V_{B}(t))\right)\Rightarrow\\
			&\Rightarrow \diff{V_{B}(t)}{t}\left(\frac{\alpha_{B}\beta_{B}\cosh(\alpha_{B}V_{B}(t))+\alpha_{A}\beta_{A}\sqrt{1+\left(\frac{\beta_{B}}{\beta_{A}}\sinh(\alpha_{B}V_{B}(t))\right)^{2}}}{\alpha_{A}\beta_{A}\sqrt{1+\left(\frac{\beta_{B}}{\beta_{A}}\sinh(\alpha_{B}V_{B}(t))\right)^{2}}}\right) =\\&= \frac{1}{C}\left(\frac{1}{R_{in}}\left(V_{in}(t)-\frac{1}{\alpha_{A}} \arcsinh \left(\frac{\beta_{B}}{\beta_{A}} \sinh(\alpha_{B}V_{B}(t))\right)-V_{B}(t)\right)-2\beta_{B}\sinh(\alpha_{B}V_{B}(t))\right)\Rightarrow\\
			&\Rightarrow \diff{V_{B}(t)}{t} = \frac{\alpha_{A}\beta_{A}\sqrt{1+\left(\frac{\beta_{B}}{\beta_{A}}\sinh(\alpha_{B}V_{B}(t))\right)^{2}}}{\alpha_{B}\beta_{B}\cosh(\alpha_{B}V_{B}(t))+\alpha_{A}\beta_{A}\sqrt{1+\left(\frac{\beta_{B}}{\beta_{A}}\sinh(\alpha_{B}V_{B}(t))\right)^{2}}}\cdot\\&\cdot\frac{1}{C}\left(\frac{1}{R_{in}}\left(V_{in}(t)-\frac{1}{\alpha_{A}} \arcsinh \left(\frac{\beta_{B}}{\beta_{A}} \sinh(\alpha_{B}V_{B}(t))\right)-V_{B}(t)\right)-2\beta_{B}\sinh(\alpha_{B}V_{B}(t))\right)\Rightarrow\\
			&\Rightarrow V_{B}(t) = \frac{\alpha_{A}\beta_{A}}{C}\Int{\frac{\sqrt{1+\left(\frac{\beta_{B}}{\beta_{A}}\sinh(\alpha_{B}V_{B}(t))\right)^{2}}}{\alpha_{B}\beta_{B}\cosh(\alpha_{B}V_{B}(t))+\alpha_{A}\beta_{A}\sqrt{1+\left(\frac{\beta_{B}}{\beta_{A}}\sinh(\alpha_{B}V_{B}(t))\right)^{2}}}\cdot\\&\cdot\left(\frac{1}{R_{in}}\left(V_{in}(t)-\frac{1}{\alpha_{A}} \arcsinh \left(\frac{\beta_{B}}{\beta_{A}} \sinh(\alpha_{B}V_{B}(t))\right)-V_{B}(t)\right)-2\beta_{B}\sinh(\alpha_{B}V_{B}(t))\right)}{t}\\
		\end{split}
	\end{equation}
	
	% Risolvo la differenziale
	Da notare che la precedente equazione (\ref{eq:sostituzione}) è in forma implicita, poiché il termine $V_{B}(t)$ compare sia nel lhs che nel rhs dell'equazione. Per tale motivo l'unico modo per trovare il valore di questa incognita è utilizzare un metodo numerico.
	
	Arrivati a questo punto per risolvere l'equazione differenziale appena ottenuta (\ref{eq:sostituzione}) si è utilizzato il metodo di Eulero implicito (o all'indietro). Si è scelto questo metodo per discretizzare l'equazione da risolvere perché ha una buona stabilità ed è semplice da applicare.
	
	Il metodo di Eulero all'indietro viene ricavato dall'approssimazione della derivata con le differenze finite all'indietro
	\[
		\diff{y}{x} \approx \frac{y_{n}-y_{n-1}}{h}
	\]
	da cui si ottiene
	\[
		\frac{y_{n}-y_{n-1}}{h} = f(x_{n},y_{n}) \implies y_{n} = y_{n-1}+hf(x_{n},y_{n})
	\]
	
	Sia $T = h$ allora l'equazione discretizzata risulta essere
	\begin{equation}
		\label{eq:discretizzazione}
		\begin{split}
			&\widehat V_{B}[n] = \frac{T}{C}\cdot\frac{\alpha_{A}\beta_{A}\sqrt{1+\left(\frac{\beta_{B}}{\beta_{A}}\sinh(\alpha_{B}\widehat V_{B}[n])\right)^{2}}}{\alpha_{B}\beta_{B}\cosh(\alpha_{B}\widehat V_{B}[n])+\alpha_{A}\beta_{A}\sqrt{1+\left(\frac{\beta_{B}}{\beta_{A}}\sinh(\alpha_{B}\widehat V_{B}[n])\right)^{2}}}\cdot\\&\cdot\left(\frac{1}{R_{in}}\left(V_{in}(t)-\frac{1}{\alpha_{A}} \arcsinh \left(\frac{\beta_{B}}{\beta_{A}} \sinh(\alpha_{B}\widehat V_{B}[n])\right)-\widehat V_{B}[n]\right)-2\beta_{B}\sinh(\alpha_{B}\widehat V_{B}[n])\right)+\\&+\widehat V_{B}[n-1]\\
		\end{split}
	\end{equation}
	\pagebreak
	
	% Risolvo la funzione ottenuta dalla risoluzione della differenziale
	Richiamiamo la formula del metodo numerico da applicare (\ref{eq:fixed-point})
	\begin{equation}
		\label{eq:punto_fisso1}
		\begin{split}
			\mathbf{x}^{(\lambda+1)} &= \mathbf{x}^{(\lambda)}-\mathbf{K}(\mathbf{x}^{(\lambda)})(\mathbf{x}^{(\lambda)}-\mathbf{f}(\mathbf{x}^{(\lambda)}))\\
			&= \mathbf{x}^{(\lambda)}-\sum_{l=0}^{L} \left(\mathbf{J_{f}(x)}\right)^{l}(\mathbf{x}^{(\lambda)}-\mathbf{f}(\mathbf{x}^{(\lambda)}))\\
		\end{split}
	\end{equation}
	poiché per (\ref{eq:fixed-point-sum}) abbiamo $\mathbf{K}^{(L)}(\mathbf{x}) = \sum_{l=0}^{L} \left(\mathbf{J_{f}(x)}\right)^{l}$\\
	
	%Applichiamo quindi il metodo numerico al nostro problema, ovvero trovare il valore numerico di $V_{B}$.
	Nel nostro caso la formula del metodo numerico verrà usata per trovare $V_{B}$, quindi la (\ref{eq:punto_fisso1}) diventa
	\begin{equation}
		\label{eq:punto_fisso2}
		\begin{split}
			%\widetilde V_{B}^{(\lambda+1)} = \widetilde V_{B}^{(\lambda)}-\sum_{l=0}^{L} \left(J_{V_{B}}(\widetilde V_{B}^{(\lambda)})\right)^{l}(\widetilde V_{B}^{(\lambda)}-\widehat V_{B}[\widetilde V_{B}^{(\lambda)}])
			V_{B}^{(\lambda+1)} = V_{B}^{(\lambda)}-\sum_{l=0}^{L} \left(J_{V_{B}}(V_{B}^{(\lambda)})\right)^{l}(V_{B}^{(\lambda)}-\widehat V_{B}[V_{B}^{(\lambda)}])
		\end{split}
	\end{equation}
	
	Il metodo richiede il calcolo del termine $J_{V_{B}}$, ovvero la matrice jacobiana della funzione $V_{B}$. La matrice jacobiana nel caso di vettori con un singolo elemento, come nel nostro caso, si riduce alla derivata della funzione a cui è associata. In poche parole, in questo caso $J_{V_{B}}(t) = \diff{V_{B}(t)}{t}$.
	\begin{equation}
		\label{eq:jacobiana}
		\begin{split}
			&J_{V_{B}}(t) = \diff{V_{B}(t)}{t} = \frac{1}{C}\cdot\frac{\alpha_{A}\beta_{A}\sqrt{1+\left(\frac{\beta_{B}}{\beta_{A}}\sinh(\alpha_{B}V_{B}(t))\right)^{2}}}{\alpha_{B}\beta_{B}\cosh(\alpha_{B}V_{B}(t))+\alpha_{A}\beta_{A}\sqrt{1+\left(\frac{\beta_{B}}{\beta_{A}}\sinh(\alpha_{B}V_{B}(t))\right)^{2}}}\cdot\\&\cdot\left(\frac{1}{R_{in}}\left(V_{in}(t)-\frac{1}{\alpha_{A}} \arcsinh \left(\frac{\beta_{B}}{\beta_{A}} \sinh(\alpha_{B}V_{B}(t))\right)-V_{B}(t)\right)-2\beta_{B}\sinh(\alpha_{B}V_{B}(t))\right)\hspace{-7px}
		\end{split}
	\end{equation}
	%\pagebreak
	
	Dopo aver calcolato numericamente il valore di $V_{B}$, si può calcolare la corrente ai capi del condensatore, ovvero $V_{out}$, utilizzando sempre la seconda legge di Kirchhoff:
	\begin{equation}
		\label{eq:tensione_finale}
		\begin{split}
			V_{out}(t) &= V_{A}(t)+V_{B}(t)\\
			&= \frac{1}{\alpha_{A}} \arcsinh \left(\frac{\beta_{B}}{\beta_{A}} \sinh(\alpha_{B}V_{B}(t))\right)+V_{B}(t)
		\end{split}
	\end{equation}
%\flushbottom
	
	\chapter{Simulazione}
	Partendo dalle formule ricavate dalla soluzione analitica, in particolare da (\ref{eq:discretizzazione}), (\ref{eq:punto_fisso2}), (\ref{eq:jacobiana}) e (\ref{eq:tensione_finale}), è stata realizzata una simulazione in linguaggio Matlab. Il sorgente è disponibile nell'appendice (\ref{code:matlab}).
	
	\section{Parametri}
		Per questa simulazione si è scelto di assegnare caratteristiche uguali a tutti i diodi. In particolare il parametro $\alpha = \frac{1}{nV_{E}}$ dove $n \approx 2$ per i diodi al silicio. In questo caso quindi è stato arrotondato $n = 2$.
	
		I valori dei vari componenti scelti per la simulazione sono riassunti di seguito:
	
		\[
			R_{in} = 1\,k\Omega
		\]
		\[
			V_{out} = 100\,nF
		\]
		\[
			V_{E} = 2,23\,mV
		\]
		\[
			\beta = 2,52\,nA
		\]
		quindi
		\[
			\alpha = \frac{1}{nV_{E}} = \frac{1}{2 \cdot 2,23\,mV} = \frac{1}{4,46\,mV}
		\]
	
		Inoltre è stato scelto
		\[
			F_{s} = \frac{1}{T} = 44100\,Hz \; \Rightarrow \; T = \frac{1}{44100}\,s
		\]
		come frequenza di campionamento e step temporale e
		\[
			|V_{B}^{(\lambda+1)} - V_{B}^{(\lambda)}| < 0.1\,mV
		\]
		come criterio d'arresto per il metodo numerico.
	
		Come input è stata scelta un'onda sinusoidale caratterizzata dalla classica equazione $y(x) = \sin (2 \pi f x + \phi)$ e dai seguenti parametri (tranne dove specificato diversamente):
		\[
			f = 100\,Hz
		\]
		\[
			A = 1\,V
		\]
		\[
			\phi = 0
		\]
		
		\vspace{10px}
		L'arco temporale preso in esame è di 0.2\,s.
		
		Per una questione di tempo è stato impostato un limite di iterazioni per il metodo numerico per prevenire cicli infiniti dovuti alla non convergenza del metodo: in particolare questo limite è pari a $250$ iterazioni, dopo le quali si esce dal ciclo e si tiene il valore prodotto dalla $250$-esima iterazione.
	\pagebreak
	
	\section{Grafici}
		\subsection{Grafici input-output}
			\label{graphs:input-output}
			Nei seguenti grafici verrà mostrata con una linea blu continua
			\begin{tikzpicture}
				\draw[thick,color=matlab_blue] (0,0) -- (1,0);
				\draw[thick,color=white] (1,0) -- (1,-0.07);				%per allineamento verticale
			\end{tikzpicture}
			il segnale in \textcolor{matlab_blue}{input}, mentre con una linea tratteggiata arancione
			\begin{tikzpicture}
				\draw[thick,dashed,color=matlab_orange] (0,0) -- (1,0);
				\draw[thick,color=white] (1,0) -- (1,-0.07);				%per allineamento verticale
			\end{tikzpicture}
			sarà mostrato il segnale in \textcolor{matlab_orange}{output} generato dal clipper. Il parametro $L$ è fissato a $L = 0$, in quanto l'output, al variare di $L$, non cambia in modo sensibile.
		
			Il trasferimento ingresso-uscita dell'onda con i parametri sopra riportati è
			\grafico{in-out/default}{Grafico input-output con $f = 100\,Hz$ e $A = 1\,V$}
			\pagebreak
		
			Per prima cosa si è analizzato il comportamento del clipper al variare della frequenza dell'onda in ingresso.
			\grafico{in-out/frequenze/50Hz}{Grafico input-output con $f = 50\,Hz$}
			\graficospace
			\grafico{in-out/frequenze/200Hz}{Grafico input-output con $f = 200\,Hz$}
			\graficospace
			\grafico{in-out/frequenze/400Hz}{Grafico input-output con $f = 400\,Hz$}
			\pagebreak
		
			Successivamente si è analizzato il comportamento del clipper al variare dell'ampiezza dell'onda in ingresso.
			\grafico{in-out/ampiezze/0,5V}{Grafico input-output con $A = 0,5\,V$}
			\graficospace
			\grafico{in-out/ampiezze/1,4V}{Grafico input-output con $A = 1,4\,V$}
			\graficospace
			\grafico{in-out/ampiezze/1,45V}{Grafico input-output con $A = 1,45\,V$}
			
			Da notare come nell'ultimo grafico (\ref{fig:in-out/ampiezze/1,45V}) si inizino ad intravedere dei segni di instabilità numerica nelle creste dell'onda, quindi quando si raggiunge il picco dell'ampiezza. Già da $A = 1.5\,V$ l'algoritmo non converge più e quindi ovviamente i grafici non sono stati riportati.
			\pagebreak
		
		\subsection{Grafico iterazioni-tempo}
			\label{graphs:iterazioni-tempo}
			Nei seguenti grafici viene visualizzato il numero medio di iterazioni necessarie a calcolare la corrente $V_{out}$ ai capi del condensatore rispetto all'andamento della tensione in entrata $V_{in}$. Questo numero è ottenuto per ogni sample dalla media delle iterazioni richieste per l'esecuzione del metodo numerico, che è stato testato per tutti i valori di $L \in [0; 1000]$. 

			\grafico{iterations/0,5V}{Grafico iterazioni-tempo con $A = 0,5\,V$}
			\graficospace
			\grafico{iterations/1,0V}{Grafico iterazioni-tempo con $A = 1,0\,V$}
			\graficospace
			\grafico{iterations/1,2V}{Grafico iterazioni-tempo con $A = 1,2\,V$}
			\graficospace
			\grafico{iterations/1,3V}{Grafico iterazioni-tempo con $A = 1,3\,V$}
			\graficospace
			\grafico{iterations/1,4V}{Grafico iterazioni-tempo con $A = 1,4\,V$}
			
			Dai grafici si può notare che in base all'ampiezza dell'onda sinusoidale in ingresso e quindi in base al valore massimo della $V_{in}$ le iterazioni richieste dal metodo per convergere variano. In particolare più è alto il modulo della tensione in ingresso, più iterazioni sono richieste al metodo per convergere: avremo quindi dei picchi di iterazioni in corrispondenza dei picchi, sia positivi che negativi, della tensione in ingresso.
			
			Nel caso in cui $A = 1,4\,V$ si può addirittura notare che le iterazioni richieste sono uguali o superano il limite delle $250$ massime in prossimità dei picchi, sia positivi che negativi, della tensione in ingresso, quindi quando $V_{in} \approx 1,4\,V$.
			\pagebreak
		
		\subsection{Grafico tempo d'esecuzione-Valore di L}
			Nei seguenti grafici si può notare la durata dell'esecuzione della simulazione Matlab in base al variare del parametro $L$.
			\grafico{L-time/0,5V}{Grafico tempo d'esecuzione-Valore di L con $A = 0,5\,V$}
			\graficospace
			\grafico{L-time/1,0V}{Grafico tempo d'esecuzione-Valore di L con $A = 1,0\,V$}
			\graficospace
			\grafico{L-time/1,4V}{Grafico tempo d'esecuzione-Valore di L con $A = 1,4\,V$}
			
			Dai grafici possiamo notare che il valore del tempo d'esecuzione aumenta linearmente al variare di $L$. Anche per valori di $L$ relativamente grandi (i.e. $L = 1000$) il tempo richiesto dal calcolo della soluzione tramite il metodo numerico se $A = 1\,V$ è abbastanza basso. Il valore massimo del tempo d'esecuzione, ovvero $\approx 16\,ms$, in questa condizione rientra comunque nei limiti di tolleranza per poter utilizzare il clipper audio anche in applicazioni audio real-time.
			
			Si osserva inoltre che, coerentemente ai valori visualizzati nei grafici della sotto-sezione precedente (\ref{graphs:iterazioni-tempo}), all'aumentare dell'ampiezza del segnale in ingresso, e quindi di $V_{in}$, aumenta il tempo richiesto per ottenere una convergenza del metodo. Questo sensibile aumento di tempo è dovuto al fatto che le iterazioni richieste per ottenere la convergenza del metodo aumenta con l'aumentare dell'ampiezza del segnale in ingresso.
			
			Il tempo d'esecuzione mostrato nel grafico è una media dei valori di varie esecuzioni ripetute della simulazione.
			
		\subsection{Grafico iterazioni-valore di L}
			\grafico{iterazioni-L_1,4V}{Grafico iterazioni-valore di L con $A = 1,4\,V$}

	\chapter{Implementazione}
	Partendo dal codice della simulazione scritto in Matlab è stata implementata una versione dell'algoritmo in C++. L'implementazione è stata scritta come plugin \textit{VST3}, uno fra gli standard più diffusi per la creazione i plugin audio.
	
	Il codice utilizza il framework JUCE ed è disponibile nella appendice (\ref{code:cpp}). Rispetto alla simulazione non è stata cambiata alcuna logica di funzionamento: il codice è solo stato riadattato per essere utilizzato col paradigma di programmazione orientato agli oggetti. L'unica cosa che il VST presenta in più è una semplicissima interfaccia grafica.
	\pagebreak
	
	\section{Grafica}
		L'interfaccia grafica del plugin VST è composta da due widget: una checkbox, chiamata toggle button in JUCE e uno slider orizzontale.
		
		\screenshot[scale=0.8]{vst/gui/L_0}{Interfaccia grafica del VST}
		
		Il primo widget serve per attivare o meno il plugin: quando è selezionato l'audio in entrata non viene modificato, quindi il plugin viene bypassato e non viene eseguito alcun calcolo, mentre se è selezionato l'effetto viene attivato e il segnale in ingresso viene processato come mostrato nei grafici input-output (\ref{graphs:input-output}) della simulazione.
		
		\screenshot[scale=0.8]{vst/gui/L_1_3}{Interfaccia grafica del VST con parametro $L$ a circa 1/3}
		\screenshot[scale=0.8]{vst/gui/L_2_3}{Interfaccia grafica del VST con parametro $L$ a circa 2/3}
		
		Lo slider invece serve ad aggiustare il parametro L: dal valore $0$ (default) si può arrivare fino al valore $500$. Il valore corrente viene mostrato in una tendina che si apre quando si clicca sul cursore dello slider e si chiude quando si rilascia il tasto del mouse.
		
		La scritta visualizzata può risultare non molto intuitiva, ma JUCE permette di appendere del testo solo dopo il valore numerico del parametro.
	
	\section{Prestazioni}
		Mentre la simulazione si presta molto a valutazioni quantitative l'implementazione si presta molto bene a misurazioni qualitative, sia sulla qualità dell'audio, sia sul tempo di risposta del metodo numerico. Se per esempio i calcoli eseguiti sul segnale audio in ingresso dovessero essere troppo pesanti, e quindi richiedere troppo tempo, l'audio in uscita ne risentirebbe, risultando frammentato oppure presentando dei click.
		
		Questa situazione si potrebbe presentare particolarmente in contesti real-time, dove il tempo di calcolo deve stare entro determinati limiti per eliminare difetti nel segnale in uscita. Inoltre più è breve il tempo di calcolo, più si può ridurre la latenza audio, ovvero il tempo trascorso dall'acquisizione del segnale e il tempo in cui il segnale viene riprodotto.
		
		È importante segnalare che per ridurre i tempi di calcolo è necessaria una buona ottimizzazione del codice. Nel codice sorgente presente nelle appendici (\ref{code:matlab}) e (\ref{code:cpp}) sono state apportate alcune semplici ottimizzazioni che però hanno influito notevolmente sulle prestazioni finali. Inizialmente infatti già per valori del parametro $L$ che si avvicinavano a $100$ il singolo core utilizzato dal VST veniva utilizzato al massimo, ma adesso è possibile utilizzare anche valori di $L \le 500$, richiedendo circa il $75\%$ di utilizzo del core singolo nel caso in cui $L = 500$, con un i7-4770S a $3,10\,$GHz.
		
		Le seguenti schermate possono dare un'idea qualitativa delle risorse richieste dall'implementazione in C++ del clipper. Dalle schermate si può riconoscere la Digital Audio Workstation (DAW) Live di Ableton, nel quale è stato utilizzato il VST.
		
		\screenshot[scale=0.75]{vst/performances/zero}{Risorse richieste dal clipper quando $L = 0$}
		\pagebreak
		\vspace*{-35px}
		\screenshot[scale=0.75]{vst/performances/half}{Risorse richieste dal clipper quando $L = 250$}
		\vspace*{-10px}
		\screenshot[scale=0.75]{vst/performances/full}{Risorse richieste dal clipper quando $L = 500$}
		\pagebreak
		
		Un'ulteriore misura qualitativa delle prestazioni consiste nel capire quante istanze del VST possono essere eseguite contemporaneamente, richieste per esempio in un contesto di produzione musicale o di missaggio del suono. Nella seguente schermata si può vedere che 4 istanze contemporanee del VST richiedono poco meno del $20\%$ di CPU. Difficilmente si supererà tale numero di istanze in un progetto musicale.
		
		\screenshot[width=\textwidth]{vst/four_instances}{Istanze multiple del VST}
		
		In questo caso è stato impostato il parametro $L = 0$, ma da come si può dedurre da quanto visto precedentemente più il parametro $L$ è alto, più risorse saranno richieste.
		
		

	\chapter{Conclusioni}
	
	
	\appendix
	
	\chapter{Codice Matlab}
	\label{code:matlab}
	Il codice Matlab si compone di 4U file:
	
	\noindent\hspace{5px}\begin{tabularx}{\textwidth}{r X}
		\textbf{clipper.m}			&dove viene generata l'onda sinusoidale passata in input al clipper e dove si possono aggiustare i vari parametri, come le caratteristiche dei vari componenti o il valore di L\\
		\textbf{process.m}			&dove vengono processati i singoli campioni del segnale e dove viene richiamato anche il metodo numerico\\
		\textbf{fixed\_point.m}		&dove è implementato il metodo numerico a punto fisso geometrico\\
		\textbf{generator.m}			&dove che contiene la funzione che genera le onde dei segnali d'ingresso (non riportato in questo documento)
	\end{tabularx}\\
	
	\matlabcode{clipper.m}{clipper.m}
	\matlabcode{process.m}{process.m}
	\matlabcode{fixed_point.m}{fixed\_point.m}
	
	\chapter{Codice C++}
	\label{code:cpp}
	%\cppcode{PluginEditor.h}
	%\cppcode{PluginEditor.cpp}
	%\cppcode{PluginProcessor.h}
	%\cppcode{PluginProcessor.cpp}
	\cppcode{Clipper.hpp}
	\cppcode{Clipper.cpp}
	
	\backmatter
	
	\printbibliography
\end{document}