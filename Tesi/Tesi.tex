\documentclass[12pt,a4paper,twoside,english,italian]{book}

%%%%%%%%%%%%%%%%%%%%%%%%%%%%%%%%%%%%%%% Header %%%%%%%%%%%%%%%%%%%%%%%%%%%%%%%%%%%%%%%%
%%%%%%%%%%%%%%%%%%%%%%%%%%%%%%%%%%%%%% Packages %%%%%%%%%%%%%%%%%%%%%%%%%%%%%%%%%%%%%%%

\usepackage[italian]{babel}
\usepackage[utf8]{inputenc}

\usepackage[a-1b]{pdfx}

\usepackage{uniudtesi}
\usepackage{siunitx}

\usepackage[nottoc]{tocbibind}
\usepackage{indentfirst}
\usepackage{fancyhdr}
\usepackage{emptypage}

\usepackage{amsmath}
\usepackage{esdiff}
\usepackage{cancel}
\usepackage{circuitikz}
\usepackage{comment}
\usepackage{bigints}

\usepackage{xcolor}
\usepackage{graphicx}
\usepackage{subcaption}
\usepackage{rotating}
\usepackage{float}
\usepackage{wrapfig}
\usepackage{titlesec}
\usepackage{setspace}

\usepackage{listings}
\usepackage{gensymb}
\usepackage{adjustbox}
\usepackage{tabularx}
\usepackage{makecell}
\renewcommand{\cellalign}{tr}

\usepackage[backend=biber,style=numeric,sorting=none]{biblatex}
\bibliography{bibliography}


%%%%%%%%%%%%%%%%%%%%%%%%%%%%%%%%%%%%%%% Configs %%%%%%%%%%%%%%%%%%%%%%%%%%%%%%%%%%%%%%%
%\title{Tesi}
%\author{Francesco Magoga}
%\date{\today}

\titolo{Realizzazione a tempo reale\\di un clipper per il suono\\attraverso schemi numerici\\innovativi}
\laureando{Magoga Francesco}
\annoaccademico{2020-2021}

%\setcounter{secnumdepth}{4}

\titleformat{\paragraph}
{\normalfont\normalsize\bfseries}{\theparagraph}{1em}{}
\titlespacing*{\paragraph}
{0pt}{3.25ex plus 1ex minus .2ex}{1.5ex plus .2ex}

\def\arraystretch{1.5}

\setstretch{1.2}

\definecolor{matlab_blue}{RGB}{0,114,189}
\definecolor{matlab_orange}{RGB}{217,83,25}

\definecolor{codebg}{gray}{0.95}
\colorlet{darkred}{red!90!black}

\newsavebox\mypostbreak
\savebox\mypostbreak{\mbox{\ensuremath{\color{red!85!black}\hookrightarrow}\space}}

\lstdefinestyle{code}
{
	%basicstyle=\ttfamily,
	basicstyle=\ttfamily\normalsize,
	backgroundcolor=\color{codebg},
	sensitive=false,
	alsoletter={.},
	xleftmargin=0.5cm,
	belowskip=2px,
	frame=single,
	framesep=5pt,
	framerule=1px,
	gobble=5,
	tabsize=1,
	columns=fullflexible,
	showstringspaces=false,
	numbers=left,
	firstnumber=1,
	numberfirstline=false,
	stepnumber=2,
	numberstyle=\scriptsize\color{black!80},
	numbersep=10pt,
	tabsize=4,
	keepspaces=true,
	breaklines=true,
	%postbreak={\hbox{\textcolor{red}{$\hookrightarrow$}\space}}
	postbreak=\usebox\mypostbreak,
	captionpos=b,
	literate=
    {á}{{\'a}}1 {é}{{\'e}}1 {í}{{\'i}}1 {ó}{{\'o}}1 {ú}{{\'u}}1
    {Á}{{\'A}}1 {É}{{\'E}}1 {Í}{{\'I}}1 {Ó}{{\'O}}1 {Ú}{{\'U}}1
    {à}{{\`a}}1 {è}{{\`e}}1 {ì}{{\`i}}1 {ò}{{\`o}}1 {ù}{{\`u}}1
    {À}{{\`A}}1 {È}{{\'E}}1 {Ì}{{\`I}}1 {Ò}{{\`O}}1 {Ù}{{\`U}}1
    {ä}{{\"a}}1 {ë}{{\"e}}1 {ï}{{\"i}}1 {ö}{{\"o}}1 {ü}{{\"u}}1
    {Ä}{{\"A}}1 {Ë}{{\"E}}1 {Ï}{{\"I}}1 {Ö}{{\"O}}1 {Ü}{{\"U}}1
    {â}{{\^a}}1 {ê}{{\^e}}1 {î}{{\^i}}1 {ô}{{\^o}}1 {û}{{\^u}}1
    {Â}{{\^A}}1 {Ê}{{\^E}}1 {Î}{{\^I}}1 {Ô}{{\^O}}1 {Û}{{\^U}}1
    {œ}{{\oe}}1 {Œ}{{\OE}}1 {æ}{{\ae}}1 {Æ}{{\AE}}1 {ß}{{\ss}}1
    {ç}{{\c c}}1 {Ç}{{\c C}}1 {ø}{{\o}}1 {å}{{\r a}}1 {Å}{{\r A}}1
    {€}{{\EUR}}1 {£}{{\pounds}}1 {~}{{\fontfamily{ptm}\selectfont \textasciitilde}}1
}

\lstdefinestyle{matlab}
{
	keywordstyle=\color{blue},
	stringstyle=\color{red},
	commentstyle=\color{green!80!black},
}

\lstdefinestyle{cpp}
{
	%basicstyle=\color{teal}\ttfamily,
	keywordstyle=\color{magenta!75!black},
	%directivestyle=\color{brown!80!black},
	stringstyle=\color{red!85!black},
	commentstyle=\color{green!70!black},
}

\lstdefinelanguage{cppl}
{
	language=C++,
	morekeywords={override},
	moredirectives={once, \#ifndef, \#endif, \#if, \#else, \#endif},
	%moredelim=[s][\color{red!85!black}\ttfamily]{<}{>},
}

\graphicspath{{./imgs/}}

\pagestyle{fancy}
\renewcommand{\chaptermark}[1]{\markboth{#1}{}}
\renewcommand{\sectionmark}[1]{\markright{\thesection\ #1}}
\fancyhf{}
\fancyhead[LE,RO]{\bfseries\thepage}
\fancyhead[LO]{\bfseries\rightmark}
\fancyhead[RE]{\bfseries\leftmark}
\renewcommand{\headrulewidth}{0.5pt}
\renewcommand{\footrulewidth}{0pt}
\setlength{\headheight}{14.5pt}

\setlength{\intextsep}{2pt}

%\facolta{Scienze Matematiche, Fisiche e Naturali} % (default)
\corsodilaureamagistralein{Informatica}
\relatore[Prof.]{Fontana Federico}
%\relatoreDue[Prof.]{Secondo relatore}
%\correlatore{Bozzo Enrico}
%\correlatoreDue{Secondo Correlatore}
\dedica{Ai miei genitori e ai miei nonni\\
    per avermi sempre supportato in ogni modo}

%%%%%%%%%%%%%%%%%%%%%%%%%%%%%%%%%%% Custom commands %%%%%%%%%%%%%%%%%%%%%%%%%%%%%%%%%%%
\newcommand{\sourcecode}[5]{
	%\label{list:#3}
	\lstset{caption={#4},label=code:#3,style=code,#5}
	\lstinputlisting[language=#1]{#2}
	\vspace{10px}
}

\newcommand{\matlabcode}[2]{
	\sourcecode{Matlab}{../Matlab/#1}{#1}{#2}{style=matlab}
}

\newcommand{\cppcode}[1]{
	\sourcecode{cppl}{../Juce/Source/#1}{#1}{#1}{style=cpp}
}

\newcommand{\screenshot}[3][]
{
	\begin{figure}[H]
		\centering
		\includegraphics[#1]{#2}
		\vspace*{-25px}
		\caption{#3}
		\label{fig:#2}
	\end{figure}
}

\newcommand{\grafico}[2]
{
	\screenshot[width=\textwidth]{plots/#1}{#2}
}
\newcommand{\graficospace}{\vspace{15px}}

\newcommand{\subscreenshot}[3][]
{
	\begin{subfigure}{0.5\textwidth}
		\centering
		\includegraphics[width=0.8\linewidth,#1]{#2}
		\caption{#3}
		\label{fig:#2}
	\end{subfigure}
}

\DeclareMathOperator{\arcsinh}{arcsinh}

\newcommand{\Int}[2]{\bigint \hspace{-6px} #1 \mathrm{d}#2}

\makeatletter
\newcommand{\reallybig}{\bBigg@{4}}
\newcommand{\ReallyBig}{\bBigg@{5}}
\makeatother
%%%%%%%%%%%%%%%%%%%%%%%%%%%%%%%%%%%%%%% Header %%%%%%%%%%%%%%%%%%%%%%%%%%%%%%%%%%%%%%%%


%%%%%%%%%%%%%%%%%%%%%%%%%%%%%%%%%%%%%% Document %%%%%%%%%%%%%%%%%%%%%%%%%%%%%%%%%%%%%%%
\begin{document}
	\pagestyle{fancy} 
	\frontmatter
	\maketitle	

	\enlargethispage{-1.5\baselineskip}
	\tableofcontents
	%\listoffigures
	
	\chapter*{Introduzione}
		\markboth{Introduzione}{}
		Spesso nella creazione di effetti audio digitali si parte da un modello analogico che viene opportunamente modificato per poter sottostare ai limiti imposti dalle caratteristiche hardware del computer. Questo si traduce in un passaggio di digitalizzazione del modello di partenza, poiché quando si lavora con i computer, che presentano una memoria limitata, non è possibile lavorare con segnali continui: bisogna utilizzare quindi dei segnali discreti.
		
		Un altro problema che può presentarsi quando si affronta il tema degli effetti audio digitali è la modellazione di \textit{delay-free loop}, ovvero circuiti che presentano un ciclo che collega l'uscita con l'entrata, detto feedback e non vi sono componenti che creano dei ritardi in tale ciclo. Questo significa che ad ogni instante il segnale d'uscita fa parte anche del segnale d'ingresso. Questa situazione crea non pochi problemi nella risoluzione delle equazioni associate alla rete di filtri che descrive il circuito analogico, poiché l'unico modo per calcolare il risultato è utilizzare dei metodi numerici.
		
		Un'ulteriore complicazione è causata dalle componenti non lineari che spesso caratterizzano queste delay-free loop: spesso nel percorso del feedback sono presenti dei componenti elettronici che presentano caratteristiche non lineari che devono essere opportunamente tenute in considerazione.
			
		Il seguente documento tratta quindi dell'implementazione e dell'efficienza di un metodo numerico a punto fisso geometrico, utile per calcolare la soluzione di circuiti delay-free non lineari di filtri digitali. Il metodo numerico proposto permette di passare da un risolutore a punto fisso ad un risolutore Newton-Raphson variando un parametro \texttt{L}.
		
		Per analizzare questo metodo numerico si è deciso di digitalizzare un circuito clipper analogico formato da 4 diodi, poiché tale circuito presenta entrambe le caratteristiche di non linearità e di assenza di ritardo nel feedback.
		
		Si è partiti quindi dalla realizzazione fisica del circuito analogico, per poi passare ad una simulazione della rete di filtri digitali ricavata dallo schema elettrico del circuito creato ed infine, basandosi sui risultati e sul codice della simulazione, è stato poi implementato un effetto audio digitale utilizzabile in tempo reale, anche in contesti molto pratici, tra i quali per esempio la produzione musicale.
	
	\mainmatter

	\raggedbottom
	
	\chapter{Metodo mumerico}
	\label{sec:metodo_numerico}
	Definiamo un vettore di $N$ funzioni non lineari $\mathbf{f(x)} = [f_{1}(\mathbf{x}), \dots , f_{N}(\mathbf{x})]^{T}$ nel vettore sconosciuto $\mathbf{x} = [\mathbf{x}_{1}, \dots , \mathbf{x}_{N}]^{T}$, dove $^T$ indica la trasposizione. Christoffersen \cite{christoffersen} propone una famiglia di risolutori numerici per la computazione a punto fisso di una soluzione $\mathbf{x}^{*} = \mathbf{f(x^{*})}$ del vettore sconosciuto $\mathbf{x}$ attraverso lo schema seguente:
	\begin{equation}
		\label{eq:fixed-point}
		\mathbf{x}^{(\lambda+1)} = \mathbf{x}^{(\lambda)}-\mathbf{K}(\mathbf{x}^{(\lambda)})(\mathbf{x}^{(\lambda)}-\mathbf{f}(\mathbf{x}^{(\lambda)}))
	\end{equation}
	dove $\mathbf{x}^{(\lambda)}$ è la $\lambda$-esima iterazione dello schema e $\mathbf{K(x)}$ è una matrice quadrata di dimensione $N \times N$, i quali elementi dipendono da $\mathbf{x}$. Se $\mathbf{K}(\mathbf{x}) = \mathbf{I}$, la matrice identità, allora il metodo si riduce ad un risolutore di punto fisso standard \cite{atkinson}. Se $\mathbf{K}(\mathbf{x}) = (\mathbf{I}-\mathbf{J_{f}(x)})^{-1}$, dove $\mathbf{J}_{f}$ è la matrice Jacobiana di $f$, allora il metodo in particolare diventa un risolutore Newton-Raphson \cite{atkinson}.
		
	Una caratteristica notevole è che il metodo di punto fisso standard e il metodo Newton-Raphson possono essere visti come due casi limite che racchiudono una sotto-famiglia di (\ref{eq:fixed-point}). Consideriamo la somma
	\begin{equation}
		\label{eq:fixed-point-sum}
		\mathbf{K}^{(L)}(\mathbf{x}) = \sum_{l=0}^{L} \left(\mathbf{J_{f}(x)}\right)^{l}
	\end{equation}
	allora $\mathbf{K}^{(0)}(\mathbf{x}) = \mathbf{I}$, per definizione di potenza di zero di una matrice non-nulla; d'altro canto dato $\lVert\mathbf{J_{f}(x)}\rVert < 1$, dove l'operatore $\lVert\,\cdot\,\rVert$ indica la norma euclidea, allora (\ref{eq:fixed-point-sum}) converge a $\mathbf{K}^{(\infty)}(\mathbf{x}) = (\mathbf{I}-\mathbf{J_{f}(x)})^{-1}$. Questo risultato generalizza la convergenza della somma geometrica scalare quando il loro rapporto comune ha una norma più piccola di uno. In altre parole, (\ref{eq:fixed-point}) definisce un risolutore di punto fisso standard se la sommatoria si ferma immediatamente (i.e. a $L = 0$), al contrario essa definisce un risolutore Newton-Raphson se la stessa sommatoria non si ferma mai. Successivamente considereremo i casi $0 < L < \infty$ sotto il vincolo $\lVert\mathbf{J_{f}(x)}\rVert < 1$: chiameremo la corrispondente famiglia risolutori di punto fisso, così come chiameremo il parametro $L$ l'ordine dello specifico risolutore.
	
	\chapter{Clipper audio}
	Per poter analizzare il metodo numerico appena descritto si è deciso di scegliere un circuito analogico da digitalizzare per creare una rete di filtri digitali e poter così eseguire le computazioni necessarie alla risoluzione di questa rete, utilizzando proprio il metodo numerico proposto.
	
	\section{Descrizione di un clipper}
		I clipper, detti anche limitatori, sono dei circuiti che tagliano la parte di un'onda di un segnale che supera una certa ampiezza. Le onde sinusoidali che superano di molto questa soglia tenderanno ad assomigliare a delle onde quadre.
		
		\begin{wrapfigure}{r}{0.615\textwidth}
			\begin{circuitikz}[american voltages, scale=0.9, transform shape]
				\draw
					% Maglia esterna
					(0,3) to[sinusoidal voltage source,l_=$V_{in}$] (0,0)		% segnale d'ingresso
					(0,3) to[resistor] (3,3)										% resistenza
					-- (7,3)														% filo in alto
					to[C, l^=$V_{out}$] (7,0)									% condensatore
					-- (0,0)														% filo in basso
					(3,0) node[ground]{}											% messa a terra
			
					% Diodi
					(3, 0) to[full diode, *-*] (3,3)								% diodo B di sinistra
					(5, 3) to[full diode, *-*] (5,0)								% diodo B di destra
					
					% Etichette
					(-0.3,0.65) to[open, v_<=$$, outer sep = 2mm] (-0.3,2.40)			% segno di Vin
					(6.5,2.75) to[open, v^=$$, outer sep = 6mm] (6.5,0.25)		% tensione Vout
					;
			\end{circuitikz}
			\caption{Circuito elettronico di un clipper audio semplice}
			\label{fig:easy_clipper}
		\end{wrapfigure}
		
		Il clipper più semplice, illustrato in figura \ref{fig:easy_clipper}, è composto da un generatore di tensione in serie con una resistenza, un condensatore e due diodi in parallelo.
		
		La scelta di utilizzare un clipper analogico per verificare l'efficacia del metodo numerico è dovuta al fatto che il clipper è un circuito semplice e molto utilizzato per la ricerca in ambito di effetti audio.
	\pagebreak
	
	\section{Circuito del clipper analogico utilizzato}
		Il clipper analogico che è stato scelto per analizzare l'efficacia del metodo numerico proposto differisce dal clipper classico in quanto, in questo nuovo circuito, sono presenti due ulteriori diodi.
		
		\vspace{15px}
		\begin{figure}[H]
			\centering
			\begin{circuitikz}[american voltages]
				\draw
				% Maglia esterna
				(-1,7) to[sinusoidal voltage source,l_=$V_{in}$] (-1,0)		% segnale d'ingresso
				(-1,7) to[resistor=$R_{in}$] (3,7)							% resistenza
				to[short, i^=$i_{in}$] (4,7)									% iin
				to[short, i^=$i_{out}$] (5,7)								% iout
				-- (8,7)														% filo in alto
				to[C, l_=$C$] (8,0)											% condensatore
				-- (-1,0)													% filo in basso
				(4,0) node[ground]{}											% messa a terra
				
				% Diodi
				(4,7) to[short, *-*] (4,6)			% connessione tra la maglia esterna e i diodi A
				(4,7) to[short, i^=$i_{D}$] (4,6)	% iD
				(5,6) -- (3,6)						% connessione in alto tra i diodi A
				to[empty diode, l=$D_{A}$] (3,4)		% diodo A di sinistra
				-- (5,4)								% connessione in basso tra i diodi A
				to[empty diode, l_=$D_{A}$] (5,6)	% diodo A di destra
				(4,4) to[short, *-*] (4,3)			% connessione i diodi A e i diodi B
				(5,3) -- (3,3)						% connessione in alto tra i diodi B
				to[full diode, l=$D_{B}$] (3,1)		% diodo B di sinistra
				-- (5,1)								% connessione in basso tra i diodi B
				to[full diode, l_=$D_{B}$] (5,3)		% diodo B di destra
				(4,1) to[short, *-*] (4,0)			% connessione i diodi B e la maglia esterna
				
				% Etichette
				(-1.25,2.5) to[open, v_<=$$] (-1.25,4.5)		% segno di Vin
				(2.3,6) to[open, v_=$V_{A}$] (2.3,4)			% tensione dei diodi A
				(2.3,3) to[open, v_=$V_{B}$] (2.3,1)			% tensione dei diodi B
				(9,7) to[open, v^=$V_{out}$] (9,0)			% tensione Vout
				;
			\end{circuitikz}
			\caption{Circuito elettronico del clipper audio utilizzato}
			\label{fig:clipper}
		\end{figure}
		\vspace{10px}
		
		Come possiamo notare dallo schema i diodi sono tra loro a due a due in parallelo e queste due coppie di diodi sono poi collegate in serie tra loro. In particolare la prima coppia di diodi ha caratteristiche diverse dalla seconda coppia. Le intensità dei vari diodi sono descritte dall'equazione di Shockley come segue:
		\[
			i_{D_{A}} = \beta_{A}\left(e^{\alpha_{A}V_{A}(t)}-1\right)
		\]
		\[
			i_{D_{B}} = \beta_{B}\left(e^{\alpha_{B}V_{B}(t)}-1\right)
		\]
		
		Le intensità delle due coppie sono quindi
		\[
			i_{D_{A'}} = \beta_{A}\left(e^{\alpha_{A}V_{A}(t)}-1\right)-\beta_{A}\left(e^{-\alpha_{A}V_{A}(t)}-1\right)
		\]
		per la prima coppia e
		\[
			i_{D_{B'}} = \beta_{B}\left(e^{\alpha_{B}V_{B}(t)}-1\right)-\beta_{B}\left(e^{-\alpha_{B}V_{B}(t)}-1\right)
		\]
		per la seconda.
		
		Essendo le coppie in serie sappiamo che $i_{D} = i_{D_{A'}} = i_{D_{B'}}$ dunque
		\[
			i_{D} = \beta_{A}\left(e^{\alpha_{A}V_{A}(t)}-1\right)-\beta_{A}\left(e^{-\alpha_{A}V_{A}(t)}-1\right) = \beta_{B}\left(e^{\alpha_{B}V_{B}(t)}-1\right)-\beta_{B}\left(e^{-\alpha_{B}V_{B}(t)}-1\right)
		\]
		
		Siamo quindi interessati a trovare il valore di $V_{out}$ fissati i vari parametri e dato il valore di $V_{in}$.
		
		\vspace{30px}
		In figura (\ref{fig:analog/example}) si può vedere un esempio di come un onda risulti prima e dopo l'applicazione del clipper.
		
		\screenshot[width=\textwidth]{analog/example}{Esempio di onda prima e dopo del clipper}
	
	\chapter{Soluzione analitica}
	Il primo passo per trovare il valore di $V_{out}$ ad un istante di tempo $t$ è stato di trovare l'equazione differenziale che rappresenta la correlazione tra il valore della tensione in entrata $V_{in}$ e il valore delle tensioni delle due coppie di diodi $V_{A}$ e $V_{B}$.
	\begin{equation}
		\label{eq:differenziale}
		\begin{split}
			V_{out}(t) &= V_{in}(t)-R_{in}\left[i_{out}+i_{D}\right]\\
			V_{out}(t) &= V_{in}(t)-R_{in}\left[C\diff{V_{out}(t)}{t}+i_{D}\right]\\
			V_{out}(t) &= V_{in}(t)-R_{in}\left[C\diff{V_{A}(t)}{t}+C\diff{V_{B}(t)}{t}+i_{D}\right]\\
			V_{A}(t)+V_{B}(t) &= V_{in}(t)-R_{in}\left[C\diff{V_{A}(t)}{t}+C\diff{V_{B}(t)}{t}+\beta_{B}\left(e^{\alpha_{B}V_{B}(t)}-e^{-\alpha_{B}V_{B}(t)}\right)\right]\\
			V_{A}(t)+V_{B}(t) &= V_{in}(t)-R_{in}\left[C\diff{V_{A}(t)}{t}+C\diff{V_{B}(t)}{t}+2\beta_{B}\sinh(\alpha_{B}V_{B}(t))\right]\\
			\frac{V_{A}(t)+V_{B}(t)}{R_{in}} &= \frac{V_{in}(t)}{R_{in}}-C\diff{V_{A}(t)}{t}-C\diff{V_{B}(t)}{t}-2\beta_{B}\sinh(\alpha_{B}V_{B}(t))\\
			%C\diff{V_{B}(t)}{t} &= \frac{V_{in}(t)-V_{A}(t)-V_{B}(t)}{R_{in}}-C\diff{V_{A}(t)}{t}-2\beta_{B}\sinh(\alpha_{B}V_{B}(t))\\
			%\diff{V_{B}(t)}{t} &= \frac{1}{C}\left(\frac{V_{in}(t)-V_{A}(t)-V_{B}(t)}{R_{in}}-C\diff{V_{A}(t)}{t}-2\beta_{B}\sinh(\alpha_{B}V_{B}(t))\right)\\
			%V_{B}(t) &= \frac{1}{C}\Int{\left(\frac{V_{in}(t)-V_{A}(t)-V_{B}(t)}{R_{in}}-C\diff{V_{A}(t)}{t}-2\beta_{B}\sinh(\alpha_{B}V_{B}(t))\right)}{t}\\
			C\diff{V_{A}(t)}{t}+C\diff{V_{B}(t)}{t} &= \frac{V_{in}(t)-V_{A}(t)-V_{B}(t)}{R_{in}}-2\beta_{B}\sinh(\alpha_{B}V_{B}(t))\\
			\diff{V_{A}(t)}{t}+\diff{V_{B}(t)}{t} &= \frac{1}{C}\left(\frac{V_{in}(t)-V_{A}(t)-V_{B}(t)}{R_{in}}-2\beta_{B}\sinh(\alpha_{B}V_{B}(t))\right)\\
		\end{split}
	\end{equation}
	\pagebreak
	
	Per poter risolvere l'equazione tramite il metodo numerico dobbiamo esprimere l'equazione differenziale (\ref{eq:differenziale}) in funzione di una delle due variabili $V_{A}$ o $V_{B}$, variabile che comparirà sia nel primo, che nel secondo termine della nostra equazione. In questo caso è stato scelto di esprimere tutto in funzione di $V_{B}$, perciò il passo successivo è stato di trovare una formula che mettesse in relazione $V_{A}$ con $V_{B}$.
	%TODO aggiungere le Rightarrrow
	\begin{equation}
		\label{eq:tensione_A_B}
		\begin{split}
			\beta_{A}\left(e^{\alpha_{A}V_{A}(t)}-1\right)-\beta_{A}\left(e^{-\alpha_{A}V_{A}(t)}-1\right) &= \beta_{B}\left(e^{\alpha_{B}V_{B}(t)}-1\right)-\beta_{B}\left(e^{-\alpha_{B}V_{B}(t)}-1\right)\\
			\beta_{A}\left(e^{\alpha_{A}V_{A}(t)}-e^{-\alpha_{A}V_{A}(t)}\right) &= \beta_{B}\left(e^{\alpha_{B}V_{B}(t)}-e^{-\alpha_{B}V_{B}(t)}\right)\\
			\cancel{2}\beta_{A}\left(\frac{e^{\alpha_{A}V_{A}(t)}-e^{-\alpha_{A}V_{A}(t)}}{2}\right) &= \cancel{2}\beta_{B}\left(\frac{e^{\alpha_{B}V_{B}(t)}-e^{-\alpha_{B}V_{B}(t)}}{2}\right)\\
			\beta_{A} \sinh(\alpha_{A}V_{A}(t)) &= \beta_{B} \sinh(\alpha_{B}V_{B}(t))\\
			\sinh(\alpha_{A}V_{A}(t)) &= \frac{\beta_{B}}{\beta_{A}} \sinh(\alpha_{B}V_{B}(t))\\
			\alpha_{A}V_{A}(t) &= \arcsinh \left(\frac{\beta_{B}}{\beta_{A}} \sinh(\alpha_{B}V_{B}(t))\right)\\
			V_{A}(t) &= \frac{1}{\alpha_{A}} \arcsinh \left(\frac{\beta_{B}}{\beta_{A}} \sinh(\alpha_{B}V_{B}(t))\right)
		\end{split}
	\end{equation}
	\pagebreak
	
	Prima di sostituire le occorrenze di $V_{A}$ nell'equazione differenziale (\ref{eq:differenziale}) con la formula appena trovata (\ref{eq:tensione_A_B}) notiamo che nella (\ref{eq:differenziale}) è richiesto di derivare $V_{A}$.
	\begin{equation}
		\label{eq:derivata}
		\begin{split}
			\diff{V_{a}(t)}{t} &= \diff{}{t}\left(\frac{1}{\alpha_{A}} \arcsinh \left(\frac{\beta_{B}}{\beta_{A}} \sinh(\alpha_{B}V_{B}(t))\right)\hspace{-5px}\right)\\
			&= \frac{1}{\alpha_{A}}\cdot\diff{}{t}\left(\arcsin\left(\frac{\beta_{B}}{\beta_{A}}\sinh(\alpha_{B}V_{B}(t))\right)\hspace{-5px}\right)\\
			&= \frac{\diff{}{t}\left(\frac{\beta_{B}}{\beta_{A}}\sinh(\alpha_{B}V_{B}(t))\right)}{\alpha_{A}\sqrt{1+\left(\frac{\beta_{B}}{\beta_{A}}\sinh(\alpha_{B}V_{B}(t))\right)^{2}}}\\
			&= \frac{\beta_{B}}{\beta_{A}}\cdot\frac{\diff{}{t}(\sinh(\alpha_{B}V_{B}(t)))}{\alpha_{A}\sqrt{1+\left(\sinh(\alpha_{B}V_{B}(t))\right)^{2}}}\\
			&= \frac{\beta_{B}\cosh(\alpha_{B}V_{B}(t))\diff{}{t}(\alpha_{B}V_{B}(t))}{\alpha_{A}\beta_{A}\sqrt{1+\left(\frac{\beta_{B}}{\beta_{A}}\sinh(\alpha_{B}V_{B}(t))\right)^{2}}}\\
			&= \frac{\alpha_{B}\beta_{B}\cosh(\alpha_{B}V_{B}(t))}{\alpha_{A}\beta_{A}\sqrt{1+\left(\frac{\beta_{B}}{\beta_{A}}\sinh(\alpha_{B}V_{B}(t))\right)^{2}}}\cdot\diff{V_{B}(t)}{t}
		\end{split}
	\end{equation}
	
	Procediamo quindi con la sostituzione di $V_{A}$ e $\diff{V_{A}}{t}$ nell'equazione differenziale (\ref{eq:differenziale}) con quanto nelle equazioni (\ref{eq:tensione_A_B}) e (\ref{eq:derivata}).
	\pagebreak
	
	%TODO aggiungere le Rightarrrow
	\begin{equation}
		\label{eq:sostituzione}
		\begin{split}
			&\frac{\alpha_{B}\beta_{B}\cosh(\alpha_{B}V_{B}(t))}{\alpha_{A}\beta_{A}\sqrt{1+\left(\frac{\beta_{B}}{\beta_{A}}\sinh(\alpha_{B}V_{B}(t))\right)^{2}}}\cdot\diff{V_{B}(t)}{t}+\diff{V_{B}(t)}{t} =\\&= \frac{1}{C}\left(\frac{1}{R_{in}}\left(V_{in}(t)-\frac{1}{\alpha_{A}} \arcsinh \left(\frac{\beta_{B}}{\beta_{A}} \sinh(\alpha_{B}V_{B}(t))\right)-V_{B}(t)\right)-2\beta_{B}\sinh(\alpha_{B}V_{B}(t))\right)\\
			&\diff{V_{B}(t)}{t}\left(\frac{\alpha_{B}\beta_{B}\cosh(\alpha_{B}V_{B}(t))}{\alpha_{A}\beta_{A}\sqrt{1+\left(\frac{\beta_{B}}{\beta_{A}}\sinh(\alpha_{B}V_{B}(t))\right)^{2}}}+1\right) =\\&= \frac{1}{C}\left(\frac{1}{R_{in}}\left(V_{in}(t)-\frac{1}{\alpha_{A}} \arcsinh \left(\frac{\beta_{B}}{\beta_{A}} \sinh(\alpha_{B}V_{B}(t))\right)-V_{B}(t)\right)-2\beta_{B}\sinh(\alpha_{B}V_{B}(t))\right)\\
			&\diff{V_{B}(t)}{t}\left(\frac{\alpha_{B}\beta_{B}\cosh(\alpha_{B}V_{B}(t))+\alpha_{A}\beta_{A}\sqrt{1+\left(\frac{\beta_{B}}{\beta_{A}}\sinh(\alpha_{B}V_{B}(t))\right)^{2}}}{\alpha_{A}\beta_{A}\sqrt{1+\left(\frac{\beta_{B}}{\beta_{A}}\sinh(\alpha_{B}V_{B}(t))\right)^{2}}}\right) =\\&= \frac{1}{C}\left(\frac{1}{R_{in}}\left(V_{in}(t)-\frac{1}{\alpha_{A}} \arcsinh \left(\frac{\beta_{B}}{\beta_{A}} \sinh(\alpha_{B}V_{B}(t))\right)-V_{B}(t)\right)-2\beta_{B}\sinh(\alpha_{B}V_{B}(t))\right)\\
			&\diff{V_{B}(t)}{t} = \frac{\alpha_{A}\beta_{A}\sqrt{1+\left(\frac{\beta_{B}}{\beta_{A}}\sinh(\alpha_{B}V_{B}(t))\right)^{2}}}{\alpha_{B}\beta_{B}\cosh(\alpha_{B}V_{B}(t))+\alpha_{A}\beta_{A}\sqrt{1+\left(\frac{\beta_{B}}{\beta_{A}}\sinh(\alpha_{B}V_{B}(t))\right)^{2}}}\cdot\\&\cdot\frac{1}{C}\left(\frac{1}{R_{in}}\left(V_{in}(t)-\frac{1}{\alpha_{A}} \arcsinh \left(\frac{\beta_{B}}{\beta_{A}} \sinh(\alpha_{B}V_{B}(t))\right)-V_{B}(t)\right)-2\beta_{B}\sinh(\alpha_{B}V_{B}(t))\right)\\
			&V_{B}(t) = \frac{\alpha_{A}\beta_{A}}{C}\Int{\frac{\sqrt{1+\left(\frac{\beta_{B}}{\beta_{A}}\sinh(\alpha_{B}V_{B}(t))\right)^{2}}}{\alpha_{B}\beta_{B}\cosh(\alpha_{B}V_{B}(t))+\alpha_{A}\beta_{A}\sqrt{1+\left(\frac{\beta_{B}}{\beta_{A}}\sinh(\alpha_{B}V_{B}(t))\right)^{2}}}\cdot\\&\cdot\left(\frac{1}{R_{in}}\left(V_{in}(t)-\frac{1}{\alpha_{A}} \arcsinh \left(\frac{\beta_{B}}{\beta_{A}} \sinh(\alpha_{B}V_{B}(t))\right)-V_{B}(t)\right)-2\beta_{B}\sinh(\alpha_{B}V_{B}(t))\right)}{t}\\
		\end{split}
	\end{equation}
	
	% Risolvo la differenziale
	Discretizzo la (\ref{eq:sostituzione})\\
	Metodo di Eulero all'indietro: $\diff{y}{x} \sim \frac{y_{n}-y_{n-1}}{h}$ quindi $\frac{y_{n}-y_{n-1}}{h} = f(x_{n},y_{n})$ e si ottiene $y_{n} = y_{n-1}+hf(x_{n},y_{n})$. Sia $T = h$ allora
	\begin{equation}
		\label{eq:discretizzazione}
		\begin{split}
			&\widehat V_{B}[n] = \frac{T}{C}\cdot\frac{\alpha_{A}\beta_{A}\sqrt{1+\left(\frac{\beta_{B}}{\beta_{A}}\sinh(\alpha_{B}\widehat V_{B}[n])\right)^{2}}}{\alpha_{B}\beta_{B}\cosh(\alpha_{B}\widehat V_{B}[n])+\alpha_{A}\beta_{A}\sqrt{1+\left(\frac{\beta_{B}}{\beta_{A}}\sinh(\alpha_{B}\widehat V_{B}[n])\right)^{2}}}\cdot\\&\cdot\left(\frac{1}{R_{in}}\left(V_{in}(t)-\frac{1}{\alpha_{A}} \arcsinh \left(\frac{\beta_{B}}{\beta_{A}} \sinh(\alpha_{B}\widehat V_{B}[n])\right)-\widehat V_{B}[n]\right)-2\beta_{B}\sinh(\alpha_{B}\widehat V_{B}[n])\right)+\\&+\widehat V_{B}[n-1]\\
		\end{split}
	\end{equation}
	
	% Risolvo la funzione ottenuta dalla risoluzione della differenziale
	Algoritmo di punto fisso
	\begin{equation}
		\label{eq:punto_fisso1}
		\begin{split}
			x^{(\lambda+1)} &= x^{(\lambda)}-K(x^{(\lambda)})(x^{(\lambda)}-g(x^{(\lambda)}))\\
			&= x^{(\lambda)}-\sum_{l=0}^{L} \left(J_{f}(x^{(\lambda)})\right)^{l}(x^{(\lambda)}-f(x^{(\lambda)}))\\
		\end{split}
	\end{equation}
	poiché $K^{(L)}(x) = \sum_{l=0}^{L} \left(J_{f}(x)\right)^{l}$\\
	
	Applico l'algoritmo di punto fisso
	\begin{equation}
		\label{eq:punto_fisso2}
		\begin{split}
			x^{(\lambda+1)} = x^{(\lambda)}-\sum_{l=0}^{L} \left(J_{f}(x^{(\lambda)})\right)^{l}(x^{(\lambda)}-f(x^{(\lambda)}))\Rightarrow\\
			\Rightarrow \widetilde V_{B}^{(\lambda+1)} = \widetilde V_{B}^{(\lambda)}-\sum_{l=0}^{L} \left(J_{V_{B}}(\widetilde V_{B}^{(\lambda)})\right)^{l}(\widetilde V_{B}^{(\lambda)}-\widehat V_{B}[\widetilde V_{B}^{(\lambda)}])
		\end{split}
	\end{equation}
	
	Calcolo la matrice jacobiana della funzione $V_{B}$
	\begin{equation}
		\label{eq:jacobiana}
		\begin{split}
			&J_{V_{B}}(t) = \diff{V_{B}(t)}{t} = \frac{\alpha_{A}\beta_{A}}{C}\reallybig(\frac{\sqrt{1+\left(\frac{\beta_{B}}{\beta_{A}}\sinh(\alpha_{B}V_{B}(t))\right)^{2}}}{\alpha_{B}\beta_{B}\cosh(\alpha_{B}V_{B}(t))+\alpha_{A}\beta_{A}\sqrt{1+\left(\frac{\beta_{B}}{\beta_{A}}\sinh(\alpha_{B}V_{B}(t))\right)^{2}}}\cdot\\&\cdot\left(\frac{1}{R_{in}}\left(V_{in}(t)-\frac{1}{\alpha_{A}} \arcsinh \left(\frac{\beta_{B}}{\beta_{A}} \sinh(\alpha_{B}V_{B}(t))\right)-V_{B}(t)\right)-2\beta_{B}\sinh(\alpha_{B}V_{B}(t))\right)\hspace{-7px}\reallybig)
		\end{split}
	\end{equation}
	
	Adesso posso calcolare la corrente ai capi del condensatore:
	\begin{equation}
		\label{eq:punto_fisso2}
		\begin{split}
			V_{out}(t) &= V_{A}(t)+V_{B}(t)\\
			&= \frac{1}{\alpha_{A}} \arcsinh \left(\frac{\beta_{B}}{\beta_{A}} \sinh(\alpha_{B}V_{B}(t))\right)+V_{B}(t)
		\end{split}
	\end{equation}
	
	\chapter{Simulazione}
	Partendo dalle formule ricavate dalla soluzione analitica, in particolare da (\ref{eq:discretizzazione}), (\ref{eq:punto_fisso2}), (\ref{eq:jacobiana}) e (\ref{eq:tensione_finale}), è stata realizzata una simulazione in linguaggio Matlab.
	
	\section{Parametri}
		Per questa simulazione si è scelto di assegnare caratteristiche uguali a tutti i diodi. In particolare il parametro $\alpha = \frac{1}{nV_{E}}$ dove $n \approx 2$ per i diodi al silicio. In questo caso quindi è stato arrotondato $n = 2$.
	
		I valori dei vari componenti scelti per la simulazione sono riassunti di seguito:
	
		\[
			R_{in} = 1\,k\Omega
		\]
		\[
			V_{out} = 100\,nF
		\]
		\[
			V_{E} = 2,23\,mV
		\]
		\[
			\beta = 2,52\,nA
		\]
		quindi
		\[
			\alpha = \frac{1}{nV_{E}} = \frac{1}{2 \cdot 2,23\,mV} = \frac{1}{4,46\,mV}
		\]
	
		Inoltre è stato scelto
		\[
			F_{s} = \frac{1}{T} = 44100\,Hz \; \Rightarrow \; T = \frac{1}{44100}\,s
		\]
		come frequenza di campionamento e step temporale e
		\[
			|V_{B}^{(\lambda+1)} - V_{B}^{(\lambda)}| < 0.1\,mV
		\]
		come criterio d'arresto per il metodo numerico.
	
		Come input è stata scelta un'onda sinusoidale caratterizzata dalla classica equazione $y(x) = \sin (2 \pi f x + \phi)$ e dai seguenti parametri (tranne dove specificato diversamente):
		\[
			F = 100\,Hz
		\]
		\[
			A = 1\,V
		\]
		\[
			\phi = 0
		\]
		
		\vspace{10px}
		L'arco temporale preso in esame è di 0.2\,s.
	\pagebreak
	
	\section{Grafico ingresso-uscita}
		
	
	\chapter{Implementazione}
	\label{sec:implementazione}
	Partendo dal codice della simulazione scritto in Matlab è stata implementata una versione dell'algoritmo in C++. L'implementazione è stata scritta come plugin \textit{VST3}, uno fra gli standard più diffusi per la creazione i plugin audio.
	
	Il codice, che si basa su JUCE \cite{juce}, è disponibile nella appendice (\ref{code:cpp}). JUCE è framework cross-platform e open-source particolarmente focalizzato sulle applicazioni audio. La sua popolarità negli ultimi anni è molto cresciuta poiché permette di scrivere delle applicazioni molto avanzate con pochissimo codice. Le prestazioni delle applicazioni sviluppate risultano eccellenti, visto che sia il framework, sia la logica di controllo delle applicazioni sono scritte in C++.
	
	Rispetto alla simulazione non è stata cambiata alcuna logica di funzionamento: il codice è solo stato riadattato per essere utilizzato col paradigma di programmazione orientato agli oggetti. L'unica cosa che il VST presenta in più è una semplicissima interfaccia grafica.
	\pagebreak
	
	\section{Grafica}
		L'interfaccia grafica del plugin VST è composta da tre widget: due checkbox, chiamata toggle button in JUCE e uno slider orizzontale.
		
		\screenshot[scale=0.8]{vst/gui/L_0}{Interfaccia grafica del VST}
		
		Il primo widget serve per attivare o meno il plugin: quando è selezionato l'audio in entrata non viene modificato, quindi il plugin viene bypassato e non viene eseguito alcun calcolo, mentre se è selezionato l'effetto viene attivato e il segnale in ingresso viene processato come mostrato nei grafici input-output (\ref{graphs:input-output}) della simulazione.
		
		Il secondo permette di attivare il metodo Newton-Raphson: se selezionato imposterà il valore della sommatoria (\ref{eq:fixed-point-sum}) a $(\mathbf{I}-\mathbf{J_{f}(x)})^{-1}$ e quindi il valore di $L$ selezionabile tramite lo slider non verrà considerato.
		
		\screenshot[scale=0.8]{vst/gui/L_1_3}{Interfaccia grafica del VST con parametro $L$ a circa 1/3}
		\screenshot[scale=0.8]{vst/gui/L_2_3}{Interfaccia grafica del VST con parametro $L$ a circa 2/3}
		
		Lo slider invece serve ad aggiustare il parametro L: dal valore $0$ (default) si può arrivare fino al valore $50$. Il valore corrente viene mostrato in una tendina che si apre quando si clicca sul cursore dello slider e si chiude quando si rilascia il tasto del mouse.
		
		La scritta visualizzata può risultare non molto intuitiva, ma JUCE permette di appendere del testo solo dopo il valore numerico del parametro.
	
	\section{Prestazioni}
		Mentre la simulazione si presta molto a valutazioni quantitative, l'implementazione si presta molto bene a misurazioni qualitative, sia sulla qualità dell'audio, sia sul tempo di risposta del metodo numerico. Se per esempio i calcoli eseguiti sul segnale audio in ingresso dovessero essere troppo pesanti, e quindi richiedere troppo tempo, l'audio in uscita ne risentirebbe, risultando frammentato oppure presentando dei click.
		
		Questa situazione si potrebbe presentare in modo particolare in contesti real-time, dove il tempo di calcolo deve stare entro determinati limiti: più è breve il tempo di calcolo, più si può ridurre la latenza audio, ovvero il tempo trascorso dall'acquisizione del segnale al momento in cui il segnale viene riprodotto.
		
		È importante dire che per ridurre i tempi di calcolo è necessaria una buona ottimizzazione del codice. Nel codice sorgente presente nelle appendici (\ref{code:matlab}) e (\ref{code:cpp}) sono state apportate alcune semplici ottimizzazioni che però hanno influito notevolmente sulle prestazioni finali. Una di queste è già stata introdotta durante la discussione dei grafici riguardanti il tempo d'esecuzione della simulazione (\ref{subsec:tempo_esecuzione}). Inizialmente infatti per valori relativamente alti del parametro $L$ il singolo core utilizzato dal VST veniva utilizzato al massimo, ma adesso viene richiesto circa il $15\%$ di utilizzo del core singolo, nel caso in cui $L = 50$, con un i7-4770S a $3,10GHz$, a fronte di un $9\%$ di utilizzo se il VST viene disattivato.
		
		Le seguenti schermate possono dare un'idea qualitativa delle risorse richieste dall'implementazione in C++ del clipper. Dalle schermate si può riconoscere la Digital Audio Workstation (DAW) Live di Ableton, nella quale è stato utilizzato il VST.
		
		\begin{figure}
			\subscreenshot{vst/performances/disabled}{CPU utilizzata dal DAW quando il clipper non viene utilizzato}
			\subscreenshot{vst/performances/not_playing}{CPU utilizzata dal clipper quando non c'è alcun segnale in input}
			\graficospace
			
			\subscreenshot{vst/performances/L=0}{CPU utilizzata dal clipper quando\\$L = 0$}
			\subscreenshot{vst/performances/L=1}{CPU utilizzata dal clipper quando\\$L = 1$}
		\end{figure}
			
		\begin{figure}\ContinuedFloat
			\subscreenshot{vst/performances/L=25}{CPU utilizzata dal clipper quando\\$L = 25$}
			\subscreenshot{vst/performances/L=50}{CPU utilizzata dal clipper quando\\$L = 50$}
			\graficospace
			
			\subscreenshot{vst/performances/newton_raphson}{CPU utilizzata dal clipper quando è attivato il metodo Newton-Raphson}
			
			\caption{Performance del clipper}
			\label{fig:performance}
		\end{figure}
		\pagebreak
		
		Le immagini evidenziano una leggera differenza rispetto ai grafici della sezione (\ref{subsec:tempo_esecuzione}). Infatti ci si aspettava un valore di utilizzo della CPU maggiore per $L = 1$, che però risulta uguale al caso in cui si utilizzi il metodo Newton-Raphson. Questo probabilmente perché in un contesto real-time, anche se diminuiscono le iterazioni eseguite dall'algoritmo, il tempo richiesto dagli altri calcoli, come per esempio il calcolo dello jacobiano, diventa predominante.
		
		Un'ulteriore misura qualitativa delle prestazioni consiste nel capire quante istanze del VST possono essere eseguite contemporaneamente, richieste per esempio in un contesto di produzione musicale o di missaggio del suono. Nella seguente schermata si può vedere che 4 istanze contemporanee del VST richiedono poco meno del $20\%$ di CPU. Difficilmente si supererà tale numero di istanze in un progetto musicale.
		
		\screenshot[width=\textwidth]{vst/four_instances}{Istanze multiple del VST}
		
		In questo caso è stato impostato il parametro $L = 0$, ma da come si può dedurre da quanto visto precedentemente più il parametro $L$ è alto, più risorse saranno richieste.
		
		
	
	\chapter{Circuito analogico}
	Per confermare i risultati ottenuti si è deciso poi di realizzare il circuito analogico del clipper analizzato. Il circuito, che segue lo schema (\ref{fig:clipper}), è mostrato nelle figure (\ref{fig:analog/circuito}) e (\ref{fig:analog/circuito2}).
	
	\screenshot[width=\textwidth]{analog/circuito}{Circuito analogico del clipper}
	\screenshot[width=\textwidth]{analog/circuito2}{Circuito analogico del clipper}
	
	In particolare nel circuito sono stati utilizzati $4$ diodi $1N4148$ e un condensatore a disco non polarizzato.
	
	Per generare il segnale è stato utilizzato un generatore di funzioni, mentre per visualizzare il segnale in uscita è stato usato un oscilloscopio digitale. Il cavo di bypass serve per visualizzare sull'oscilloscopio l'onda originale. Nell'immagine (\ref{fig:analog/banco}) si può vedere la strumentazione coinvolta e i collegamenti tra i vari strumenti.
	
	In figura (\ref{fig:analog/generatore_funzioni}) è mostrato il generatore di funzioni utilizzato. La frequenza in questo caso è impostata a $100Hz$, mentre l'ampiezza è pari a $1,5V$ e l'onda è di tipo sinusoidale. Da notare che l'ampiezza visualizzata nello schermo del generatore di funzioni è l'ampiezza picco-picco, che indica quindi la differenza di tensione tra un picco positivo dell'onda e un picco negativo. Nel nostro caso, visto che l'onda è verticalmente centrata, per passare dall'ampiezza picco-picco all'ampiezza di picco, quindi quella usata finora, basta semplicemente dividere per $2$.
	
	\screenshot[width=\textwidth]{analog/banco}{Banco di lavoro}
	\screenshot[width=\textwidth]{analog/generatore_funzioni}{Generatore di funzioni}
	
	\section{Risultati}
		Il circuito è stato provato con delle onde sinusoidali, tutte a frequenza di $1000Hz$ e con ampiezza variabile. Di seguito sono riportate delle schermate dell'oscilloscopio: il segnale di \textcolor{red!90!black}{input}, quindi ciò che viene generato dal generatore di funzioni, è riportato in \textcolor{red!90!black}{rosso}, mentre l'\textcolor{yellow!90!black}{output}, quindi il segnale processato dal clipper, è visualizzato in \textcolor{yellow!90!black}{giallo}.
				
		\screenshot[width=\textwidth]{analog/screens/1,0V}{Input e output del clipper analogico con $V = 1V$}
		\pagebreak
		\vspace*{-30px}
		\screenshot[width=\textwidth]{analog/screens/1,5V}{Input e output del clipper analogico con $V = 1,5V$}
		\screenshot[width=\textwidth]{analog/screens/2,0V}{Input e output del clipper analogico con $V = 2V$}
		\pagebreak


	\chapter{Conclusioni}
	Il comportamento del clipper è già stato messo in evidenza nei grafici della sezione (\ref{sec:implementazione}). Questi grafici ci danno un'idea del segnale uscente generato dal clipper in una situazione ideale.
	
	\section{Segnale di rumore}
		Per avere un'idea più realistica si è poi deciso di analizzare il comportamento del clipper quando in entrata viene fornito un segnale di rumore bianco. Questo segnale simula una composizione di più onde, così come un suono è composto da più armoniche e una canzone è composta da più suoni.
	
		Il rumore è stato generato casualmente. Dato che la forma dell'onda in input influisce sia sul segnale di output e che sulla convergenza dell'algoritmo è possibile che si ottengano risultati leggermente differenti in base al segnale generato. Nelle prove effettuate viene cambiata solo l'ampiezza del segnale, ma non la forma d'onda. %f = 100
	
		\grafico{conclusioni/noise/noise}{Grafico del rumore bianco con $A = 1V$}
		\graficospace
	
		Quando il segnale di rumore ha un ampiezza tale da garantire la convergenza del metodo, come nel nostro caso in cui $A = 2,5V$, le iterazioni medie richieste per calcolare il risultato su un singolo campione diminuiscono con l'aumentare del valore del parametro $L$. Nel seguente grafico si può vedere come varia il numero medio di iterazioni per campione in base al valore assunto dal parametro $L$.
	
		\grafico{conclusioni/noise/2,5V}{Grafico delle iterazioni per il rumore bianco con $A = 2,5V$}
		\pagebreak
	
		Nel caso in cui invece il rumore abbia un'ampiezza per cui l'algoritmo non converge per alcuni valori di $L$, come per esempio $A = 4,5V$, allora il numero medio di iterazioni si riduce drasticamente, poiché da un determinato istante di tempo, ovvero quando l'algoritmo inizia a divergere, il numero di iterazioni richieste per calcolare il risultato sui vari campioni sarà pari a zero. Nel seguente grafico sono stati rimossi i valori di $L$ per cui l'algoritmo non converge.
	
		\grafico{conclusioni/noise/4,5V}{Grafico delle iterazioni per il rumore bianco con $A = 4,5V$}
		\pagebreak
	
	\section{Segnale sinusoidale}
		Per essere certi che il segnale di output calcolato dall'algoritmo sia corretto si sono messi a confronto i risultati riscontrati nell'implementazione digitale con quelli ottenuti dall'implementazione analogica. Per semplicità è stato scelto di eseguire questo confronto su segnali sinusoidali di frequenza pari a $f = 1000Hz$.
	
		Il primo confronto è stato effettuato su un'onda sinusoidale con un'ampiezza tale per cui la convergenza dell'algoritmo è sempre garantita: è stato quindi scelto $A = 1,3V$.
	
		\grafico{conclusioni/sine/1,3V_iterations}{Grafico delle iterazioni per ogni campione del segnale sinusoidale con $A = 1,3V$}
		\pagebreak
	
		Possiamo vedere che il segnale di uscita del circuito analogico corrisponde a quanto calcolato dal codice Matlab.
	
		\grafico{conclusioni/analog/1,3V}{Grafico del segnale sinusoidale analogico con $A = 1,3V$}
			
		Notiamo inoltre che, anche nel caso di segnali sinusoidali, se l'input ha un ampiezza che garantisce la convergenza, il numero di iterazioni richieste diminuisce con l'aumentare del valore del parametro $L$, esattamente come per il caso del rumore bianco.
	
		\grafico{conclusioni/sine/1,3V}{Grafico delle iterazioni per il segnale sinusoidale con $A = 1,3V$}
		\pagebreak
	
		Un ulteriore confronto è stato effettuato su un'onda sinusoidale per la quale l'algoritmo non converge per alcuni valori di $L$ da un certo istante di tempo: in questo caso quindi $A = 1,6V$.
	
		\grafico{conclusioni/sine/1,6V_iterations}{Grafico delle iterazioni per ogni campione del segnale sinusoidale con $A = 1,6V$}
			
		Possiamo vedere che in questo caso l'output calcolato dal codice Matlab presenta alcune differenze rispetto al segnale prodotto dal circuito.
	
		\grafico{conclusioni/analog/1,6V}{Grafico del segnale sinusoidale analogico con $A = 1,6V$}
		\pagebreak
		
		In particolare vicino ai picchi l'output calcolato dalla simulazione presenta delle imprecisioni, dovute alla limitazione delle iterazioni dell'algoritmo. Queste limitazioni come già detto impediscono che il metodo entri in loop infinito.
	
		Anche nel caso del segnale sinusoidale per cui l'algoritmo non converge per alcuni valori di $L$ il numero medio di iterazioni si riduce drasticamente, esattamente come per il caso del segnale costituito da un rumore. Anche in questo grafico sono stati rimossi i valori di $L$ per cui l'algoritmo non converge.
	
		\grafico{conclusioni/sine/1,6V}{Grafico delle iterazioni per ogni campione del segnale sinusoidale con $A = 1,6V$}
		\graficospace
	
		Possiamo affermare quindi che quando il segnale in ingresso ha un'ampiezza che supera una certa soglia, determinata anche in base alla tipologia dell'onda che rappresenta il segnale, l'algoritmo non converge più per alcuni o tutti i valori di $L$, a partire da un determinato campione, solitamente vicino ad un picco dell'onda.
	
	\appendix
	
	\chapter{Codice Matlab}
	\label{code:matlab}
	Il codice Matlab si compone di 4 file:
	
	\noindent\hspace{5px}\begin{tabularx}{\textwidth}{r X}
		\textbf{clipper.m}			&dove viene generata l'onda sinusoidale passata in input al clipper e dove si possono aggiustare i vari parametri, come le caratteristiche dei vari componenti o il valore di L\\
		\textbf{process.m}			&dove vengono processati i singoli campioni del segnale e dove viene richiamato anche il metodo numerico\\
		\textbf{fixed\_point.m}		&dove è implementato il metodo numerico a punto fisso geometrico\\
		\textbf{generator.m}			&dove che contiene la funzione che genera le onde dei segnali d'ingresso (non riportato in questo documento)
	\end{tabularx}\\
	
	\matlabcode{clipper.m}{clipper.m}
	\matlabcode{process.m}{process.m}
	\matlabcode{fixed_point.m}{fixed\_point.m}
	
	\chapter{Codice C++}
	\label{code:cpp}
	Il codice C++ si compone di 3 file \texttt{.cpp} con i 3 relativi header \texttt{.h}:
	
	\noindent\hspace{5px}\begin{tabularx}{\textwidth}{r X}
		\makecell{\textbf{PluginEditor.h}\\\textbf{PluginEditor.cpp}}			&che implementano tutti i metodi relativi all'interfaccia grafica del plugin\\
		\makecell{\textbf{PluginProcessor.h}\\\textbf{PluginProcessor.cpp}}	&che implementano tutti i metodi legati alla tecnologia VST, in particolare qui verrà richiamata la funzione che processerà i singoli sample\\
		\makecell{\textbf{Clipper.h}\\\textbf{Clipper.cpp}}					&dove è implementa tutta la logica di funzionamento del clipper, compreso il metodo numerico a punto fisso geometrico
	\end{tabularx}\\
	
	%\cppcode{PluginEditor.h}
	%\cppcode{PluginEditor.cpp}
	%\cppcode{PluginProcessor.h}
	%\cppcode{PluginProcessor.cpp}
	Nel seguente listato, che rappresenta parte del file \texttt{PluginProcessor.cpp}, parte del codice è stato omesso a causa della sua eccessiva lunghezza. Sono state incluse solo le funzioni strettamente necessarie alla comprensione del funzionamento dell'intero progetto.
	\vspace{10px}
	\begin{lstlisting}[caption={PluginProcessor.cpp},label=code:PluginProcessor.cpp,style=code,style=cpp,language=cppl,gobble=8]
		ClipperAudioProcessor::ClipperAudioProcessor()
			#ifndef JucePlugin_PreferredChannelConfigurations
				: AudioProcessor (BusesProperties()
					#if ! JucePlugin_IsMidiEffect
						#if ! JucePlugin_IsSynth
							.withInput  ("Input",  juce::AudioChannelSet::stereo(), true)
						#endif
						.withOutput ("Output", juce::AudioChannelSet::stereo(), true)
					#endif
				)
		    #endif
			{
				sampleCallback = [](unsigned int, float) {};
				
				auto totalNumInputChannels = getTotalNumInputChannels();
				for(int channel=0;channel<totalNumInputChannels;channel++)
					clippers.push_back(Clipper());
			}

		ClipperAudioProcessor::~ClipperAudioProcessor() {}
		
		.
		.
		.
		
		void ClipperAudioProcessor::processBlock(juce::AudioBuffer<float>& buffer, juce::MidiBuffer& midiMessages)
		{
			juce::ScopedNoDenormals noDenormals;
			auto totalNumInputChannels  = getTotalNumInputChannels();
			auto totalNumOutputChannels = getTotalNumOutputChannels();

			for(auto i=totalNumInputChannels;i<totalNumOutputChannels;i++)
				buffer.clear(i, 0, buffer.getNumSamples());

			if(!bypass)
			{
				for(int channel=0;channel<totalNumInputChannels;channel++)
				{
					auto *channelData = buffer.getWritePointer(channel);
					
					clippers[channel].process(channelData, buffer.getNumSamples());
					
					for(int i=0;i<buffer.getNumSamples();i++)
						sampleCallback(channel, channelData[i]);
				}
			}
		}

		void ClipperAudioProcessor::setSampleCallback(std::function<void(unsigned int, float)> sampleCallback)
		{
			this->sampleCallback = sampleCallback;
		}

		void ClipperAudioProcessor::setBypass(bool bypass)
		{
			this->bypass = bypass;
		}

		void ClipperAudioProcessor::setL(unsigned int L)
		{
			for(Clipper &clipper : clippers)
				clipper.setL(L);
		}

		void ClipperAudioProcessor::useNewtonRaphson(bool newtonRaphson)
		{
			for(Clipper &clipper : clippers)
				clipper.useNewtonRaphson(newtonRaphson);
		}
		
		.
		.
		.
	\end{lstlisting}
	\vspace{10px}
	
	\cppcode{Clipper.h}
	\pagebreak
	\cppcode{Clipper.cpp}
	
	\backmatter
	
	\printbibliography

\end{document}