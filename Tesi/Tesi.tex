\documentclass[12pt,a4paper,twoside,english,italian]{book}

%%%%%%%%%%%%%%%%%%%%%%%%%%%%%%%%%%%%%%% Header %%%%%%%%%%%%%%%%%%%%%%%%%%%%%%%%%%%%%%%%
%%%%%%%%%%%%%%%%%%%%%%%%%%%%%%%%%%%%%% Packages %%%%%%%%%%%%%%%%%%%%%%%%%%%%%%%%%%%%%%%

\usepackage[italian]{babel}
\usepackage[utf8]{inputenc}


\usepackage{uniudtesi}
\usepackage{siunitx}

\usepackage[nottoc]{tocbibind}
\usepackage{indentfirst}
\usepackage{fancyhdr}
\usepackage{emptypage}

\usepackage{amsmath}
\usepackage{esdiff}
\usepackage{cancel}
\usepackage{circuitikz}
\usepackage{comment}
\usepackage{bigints}

\usepackage{hyperref}

\usepackage{xcolor}
\usepackage{graphicx}
\usepackage{subcaption}
\usepackage{rotating}
\usepackage{float}
\usepackage{wrapfig}
\usepackage{titlesec}
\usepackage{setspace}

\usepackage{listings}

\usepackage[backend=biber,style=alphabetic,sorting=ynt]{biblatex}


%%%%%%%%%%%%%%%%%%%%%%%%%%%%%%%%%%%%%%% Configs %%%%%%%%%%%%%%%%%%%%%%%%%%%%%%%%%%%%%%%
%\title{Tesi}
%\author{Francesco Magoga}
%\date{\today}

\titolo{Clipper audio digitale}
\laureando{Magoga Francesco}
\annoaccademico{2020-2021}

%\setcounter{secnumdepth}{4}

\titleformat{\paragraph}
{\normalfont\normalsize\bfseries}{\theparagraph}{1em}{}
\titlespacing*{\paragraph}
{0pt}{3.25ex plus 1ex minus .2ex}{1.5ex plus .2ex}

\def\arraystretch{1.5}

\addbibresource{bibliography.bib}

\setstretch{1.2}

\definecolor{codebg}{gray}{0.95}
\colorlet{darkred}{red!90!black}

\lstdefinestyle{code}
{
	%basicstyle=\ttfamily,
	basicstyle=\normalsize,
	backgroundcolor=\color{codebg},
	sensitive=false,
	alsoletter={.},
	xleftmargin=0.5cm,
	belowskip=2px,
	frame=tlbr,
	framesep=4pt,
	framerule=0pt,
	gobble=5,
	tabsize=1,
	columns=fullflexible,
	showstringspaces=false,
	numbers=left,
	stepnumber=1,
	numbersep=10pt,
	tabsize=4,
}

%\lstnewenvironment{bash}[1][]
%{\lstset{language=bash,style=code,#1}}
%{}

\newcommand{\sourcecode}[4]{
	%\label{list:#3}
	\lstset{caption={#4},label=list:#3,style=code}
	\lstinputlisting[language=#1]{#2}
}

\newcommand{\matlabcode}[3]{
	\sourcecode{Matlab}{../Matlab/#1}{#1}{#1}
}

\newcommand{\cppcode}[4]{
	\sourcecode{C++}{../Juce/Source/#1}{#1}{#1}
}

%%%%%%%%%%%%%%%%%%%%%%%%%%%%%%%%%%% Custom commands %%%%%%%%%%%%%%%%%%%%%%%%%%%%%%%%%%%
\DeclareMathOperator{\arcsinh}{arcsinh}

\newcommand{\includesketch}[2][1=]{\setlength{\fboxsep}{0pt}\setlength{\fboxrule}{1pt}\fbox{\includegraphics[#1]{#2}}}
\newcommand{\Int}[2]{\bigint \hspace{-6px} #1 \mathrm{d}#2}

\makeatletter
\newcommand{\reallybig}{\bBigg@{4}}
\newcommand{\ReallyBig}{\bBigg@{5}}
\makeatother

\graphicspath{{./imgs/}}

\pagestyle{fancy}
\renewcommand{\chaptermark}[1]{\markboth{#1}{}}
\renewcommand{\sectionmark}[1]{\markright{\thesection\ #1}}
\fancyhf{}
\fancyhead[LE,RO]{\bfseries\thepage}
\fancyhead[LO]{\bfseries\rightmark}
\fancyhead[RE]{\bfseries\leftmark}
\renewcommand{\headrulewidth}{0.5pt}
\renewcommand{\footrulewidth}{0pt}
\setlength{\headheight}{14.5pt}

\setlength{\intextsep}{2pt}

%\facolta{Scienze Matematiche, Fisiche e Naturali} % (default)
\corsodilaureamagistralein{Informatica}
\relatore[Prof.]{Federico Fontana}
%\relatoreDue[Prof.]{Secondo relatore}
%\correlatore{Talaltro dei Tali}
%\correlatoreDue{Secondo Correlatore}
%\dedica{Ai miei genitori\\
%    per non avermi tagliato i viveri} % (facoltativo)
%%%%%%%%%%%%%%%%%%%%%%%%%%%%%%%%%%%%%%% Header %%%%%%%%%%%%%%%%%%%%%%%%%%%%%%%%%%%%%%%%


%%%%%%%%%%%%%%%%%%%%%%%%%%%%%%%%%%%%%% Document %%%%%%%%%%%%%%%%%%%%%%%%%%%%%%%%%%%%%%%
\begin{document}
	\pagestyle{fancy} 
	\frontmatter
	\maketitle	

	\enlargethispage{-1.5\baselineskip}
	\tableofcontents
	%\listoffigures
%	\begingroup
%		\let\clearpage\relax
%		\listoffigures
%		\begingroup
%			\let\clearpage\relax
%			\let\cleardoublepage\relax
%			\par\vspace{2\baselineskip}
			%\lstlistoflistings
%		\endgroup
%		\thispagestyle{empty}
%	\endgroup

	\mainmatter

	\chapter*{Introduzione}
		Il seguente documento tratta dell'implementazione e dell'efficienza di un metodo numerico a punto fisso per calcolare la soluzione di circuiti non lineari di filtri digitali.
		
		Il metodo numerico utilizzato è quello proposto in \cite{bib:fixed-point}. Questo permette di passare da un risolutore a punto fisso ad un risolutore Newton-Raphson variando un parametro \texttt{L}.
		
		Per analizzare questo metodo numerico si è deciso di digitalizzare un circuito clipper analogico formato da 4 diodi.
		
	\chapter{Metodo mumerico}
	\label{sec:metodo_numerico}
	Definiamo un vettore di $N$ funzioni non lineari $\mathbf{f(x)} = [f_{1}(\mathbf{x}), \dots , f_{N}(\mathbf{x})]^{T}$ nel vettore sconosciuto $\mathbf{x} = [\mathbf{x}_{1}, \dots , \mathbf{x}_{N}]^{T}$, dove $^T$ indica la trasposizione. Christoffersen \cite{christoffersen} propone una famiglia di risolutori numerici per la computazione a punto fisso di una soluzione $\mathbf{x}^{*} = \mathbf{f(x^{*})}$ del vettore sconosciuto $\mathbf{x}$ attraverso lo schema seguente:
	\begin{equation}
		\label{eq:fixed-point}
		\mathbf{x}^{(\lambda+1)} = \mathbf{x}^{(\lambda)}-\mathbf{K}(\mathbf{x}^{(\lambda)})(\mathbf{x}^{(\lambda)}-\mathbf{f}(\mathbf{x}^{(\lambda)}))
	\end{equation}
	dove $\mathbf{x}^{(\lambda)}$ è la $\lambda$-esima iterazione dello schema e $\mathbf{K(x)}$ è una matrice quadrata di dimensione $N \times N$, i quali elementi dipendono da $\mathbf{x}$. Se $\mathbf{K}(\mathbf{x}) = \mathbf{I}$, la matrice identità, allora il metodo si riduce ad un risolutore di punto fisso standard \cite{atkinson}. Se $\mathbf{K}(\mathbf{x}) = (\mathbf{I}-\mathbf{J_{f}(x)})^{-1}$, dove $\mathbf{J}_{f}$ è la matrice Jacobiana di $f$, allora il metodo in particolare diventa un risolutore Newton-Raphson \cite{atkinson}.
		
	Una caratteristica notevole è che il metodo di punto fisso standard e il metodo Newton-Raphson possono essere visti come due casi limite che racchiudono una sotto-famiglia di (\ref{eq:fixed-point}). Consideriamo la somma
	\begin{equation}
		\label{eq:fixed-point-sum}
		\mathbf{K}^{(L)}(\mathbf{x}) = \sum_{l=0}^{L} \left(\mathbf{J_{f}(x)}\right)^{l}
	\end{equation}
	allora $\mathbf{K}^{(0)}(\mathbf{x}) = \mathbf{I}$, per definizione di potenza di zero di una matrice non-nulla; d'altro canto dato $\lVert\mathbf{J_{f}(x)}\rVert < 1$, dove l'operatore $\lVert\,\cdot\,\rVert$ indica la norma euclidea, allora (\ref{eq:fixed-point-sum}) converge a $\mathbf{K}^{(\infty)}(\mathbf{x}) = (\mathbf{I}-\mathbf{J_{f}(x)})^{-1}$. Questo risultato generalizza la convergenza della somma geometrica scalare quando il loro rapporto comune ha una norma più piccola di uno. In altre parole, (\ref{eq:fixed-point}) definisce un risolutore di punto fisso standard se la sommatoria si ferma immediatamente (i.e. a $L = 0$), al contrario essa definisce un risolutore Newton-Raphson se la stessa sommatoria non si ferma mai. Successivamente considereremo i casi $0 < L < \infty$ sotto il vincolo $\lVert\mathbf{J_{f}(x)}\rVert < 1$: chiameremo la corrispondente famiglia risolutori di punto fisso, così come chiameremo il parametro $L$ l'ordine dello specifico risolutore.
	
	\chapter{Clipper audio}
	Per poter analizzare il metodo numerico appena descritto si è deciso di scegliere un circuito analogico da digitalizzare per creare una rete di filtri digitali e poter così eseguire le computazioni necessarie alla risoluzione di questa rete, utilizzando proprio il metodo numerico proposto.
	
	\section{Descrizione di un clipper}
		I clipper, detti anche limitatori, sono dei circuiti che tagliano la parte di un'onda di un segnale che supera una certa ampiezza. Le onde sinusoidali che superano di molto questa soglia tenderanno ad assomigliare a delle onde quadre.
		
		\begin{wrapfigure}{r}{0.615\textwidth}
			\begin{circuitikz}[american voltages, scale=0.9, transform shape]
				\draw
					% Maglia esterna
					(0,3) to[sinusoidal voltage source,l_=$V_{in}$] (0,0)		% segnale d'ingresso
					(0,3) to[resistor] (3,3)										% resistenza
					-- (7,3)														% filo in alto
					to[C, l^=$V_{out}$] (7,0)									% condensatore
					-- (0,0)														% filo in basso
					(3,0) node[ground]{}											% messa a terra
			
					% Diodi
					(3, 0) to[full diode, *-*] (3,3)								% diodo B di sinistra
					(5, 3) to[full diode, *-*] (5,0)								% diodo B di destra
					
					% Etichette
					(-0.3,0.65) to[open, v_<=$$, outer sep = 2mm] (-0.3,2.40)			% segno di Vin
					(6.5,2.75) to[open, v^=$$, outer sep = 6mm] (6.5,0.25)		% tensione Vout
					;
			\end{circuitikz}
			\caption{Circuito elettronico di un clipper audio semplice}
			\label{fig:easy_clipper}
		\end{wrapfigure}
		
		Il clipper più semplice, illustrato in figura \ref{fig:easy_clipper}, è composto da un generatore di tensione in serie con una resistenza, un condensatore e due diodi in parallelo.
		
		La scelta di utilizzare un clipper analogico per verificare l'efficacia del metodo numerico è dovuta al fatto che il clipper è un circuito semplice e molto utilizzato per la ricerca in ambito di effetti audio.
	\pagebreak
	
	\section{Circuito del clipper analogico utilizzato}
		Il clipper analogico che è stato scelto per analizzare l'efficacia del metodo numerico proposto differisce dal clipper classico in quanto, in questo nuovo circuito, sono presenti due ulteriori diodi.
		
		\vspace{15px}
		\begin{figure}[H]
			\centering
			\begin{circuitikz}[american voltages]
				\draw
				% Maglia esterna
				(-1,7) to[sinusoidal voltage source,l_=$V_{in}$] (-1,0)		% segnale d'ingresso
				(-1,7) to[resistor=$R_{in}$] (3,7)							% resistenza
				to[short, i^=$i_{in}$] (4,7)									% iin
				to[short, i^=$i_{out}$] (5,7)								% iout
				-- (8,7)														% filo in alto
				to[C, l_=$C$] (8,0)											% condensatore
				-- (-1,0)													% filo in basso
				(4,0) node[ground]{}											% messa a terra
				
				% Diodi
				(4,7) to[short, *-*] (4,6)			% connessione tra la maglia esterna e i diodi A
				(4,7) to[short, i^=$i_{D}$] (4,6)	% iD
				(5,6) -- (3,6)						% connessione in alto tra i diodi A
				to[empty diode, l=$D_{A}$] (3,4)		% diodo A di sinistra
				-- (5,4)								% connessione in basso tra i diodi A
				to[empty diode, l_=$D_{A}$] (5,6)	% diodo A di destra
				(4,4) to[short, *-*] (4,3)			% connessione i diodi A e i diodi B
				(5,3) -- (3,3)						% connessione in alto tra i diodi B
				to[full diode, l=$D_{B}$] (3,1)		% diodo B di sinistra
				-- (5,1)								% connessione in basso tra i diodi B
				to[full diode, l_=$D_{B}$] (5,3)		% diodo B di destra
				(4,1) to[short, *-*] (4,0)			% connessione i diodi B e la maglia esterna
				
				% Etichette
				(-1.25,2.5) to[open, v_<=$$] (-1.25,4.5)		% segno di Vin
				(2.3,6) to[open, v_=$V_{A}$] (2.3,4)			% tensione dei diodi A
				(2.3,3) to[open, v_=$V_{B}$] (2.3,1)			% tensione dei diodi B
				(9,7) to[open, v^=$V_{out}$] (9,0)			% tensione Vout
				;
			\end{circuitikz}
			\caption{Circuito elettronico del clipper audio utilizzato}
			\label{fig:clipper}
		\end{figure}
		\vspace{10px}
		
		Come possiamo notare dallo schema i diodi sono tra loro a due a due in parallelo e queste due coppie di diodi sono poi collegate in serie tra loro. In particolare la prima coppia di diodi ha caratteristiche diverse dalla seconda coppia. Le intensità dei vari diodi sono descritte dall'equazione di Shockley come segue:
		\[
			i_{D_{A}} = \beta_{A}\left(e^{\alpha_{A}V_{A}(t)}-1\right)
		\]
		\[
			i_{D_{B}} = \beta_{B}\left(e^{\alpha_{B}V_{B}(t)}-1\right)
		\]
		
		Le intensità delle due coppie sono quindi
		\[
			i_{D_{A'}} = \beta_{A}\left(e^{\alpha_{A}V_{A}(t)}-1\right)-\beta_{A}\left(e^{-\alpha_{A}V_{A}(t)}-1\right)
		\]
		per la prima coppia e
		\[
			i_{D_{B'}} = \beta_{B}\left(e^{\alpha_{B}V_{B}(t)}-1\right)-\beta_{B}\left(e^{-\alpha_{B}V_{B}(t)}-1\right)
		\]
		per la seconda.
		
		Essendo le coppie in serie sappiamo che $i_{D} = i_{D_{A'}} = i_{D_{B'}}$ dunque
		\[
			i_{D} = \beta_{A}\left(e^{\alpha_{A}V_{A}(t)}-1\right)-\beta_{A}\left(e^{-\alpha_{A}V_{A}(t)}-1\right) = \beta_{B}\left(e^{\alpha_{B}V_{B}(t)}-1\right)-\beta_{B}\left(e^{-\alpha_{B}V_{B}(t)}-1\right)
		\]
		
		Siamo quindi interessati a trovare il valore di $V_{out}$ fissati i vari parametri e dato il valore di $V_{in}$.
		
		\vspace{30px}
		In figura (\ref{fig:analog/example}) si può vedere un esempio di come un onda risulti prima e dopo l'applicazione del clipper.
		
		\screenshot[width=\textwidth]{analog/example}{Esempio di onda prima e dopo del clipper}
	
	\chapter{Formule}
		Tensione $V_{A}$
		\begin{equation}
			\label{eq:corrente}
			\begin{split}
				\beta_{A}\left(e^{\alpha_{A}V_{A}(t)}-1\right)-\beta_{A}\left(e^{-\alpha_{A}V_{A}(t)}-1\right) &= \beta_{B}\left(e^{\alpha_{B}V_{B}(t)}-1\right)-\beta_{B}\left(e^{-\alpha_{B}V_{B}(t)}-1\right)\\
				\beta_{A}\left(e^{\alpha_{A}V_{A}(t)}-e^{-\alpha_{A}V_{A}(t)}\right) &= \beta_{B}\left(e^{\alpha_{B}V_{B}(t)}-e^{-\alpha_{B}V_{B}(t)}\right)\\
				\cancel{2}\beta_{A}\left(\frac{e^{\alpha_{A}V_{A}(t)}-e^{-\alpha_{A}V_{A}(t)}}{2}\right) &= \cancel{2}\beta_{B}\left(\frac{e^{\alpha_{B}V_{B}(t)}-e^{-\alpha_{B}V_{B}(t)}}{2}\right)\\
				\beta_{A} \sinh(\alpha_{A}V_{A}(t)) &= \beta_{B} \sinh(\alpha_{B}V_{B}(t))\\
				\sinh(\alpha_{A}V_{A}(t)) &= \frac{\beta_{B}}{\beta_{A}} \sinh(\alpha_{B}V_{B}(t))\\
				\alpha_{A}V_{A}(t) &= \arcsinh \left(\frac{\beta_{B}}{\beta_{A}} \sinh(\alpha_{B}V_{B}(t))\right)\\
				V_{A}(t) &= \frac{1}{\alpha_{A}} \arcsinh \left(\frac{\beta_{B}}{\beta_{A}} \sinh(\alpha_{B}V_{B}(t))\right)
			\end{split}
		\end{equation}
		
		Equazione differenziale
		\begin{equation}
			\label{eq:eqdiff}
			\begin{split}
				V_{out}(t) &= V_{in}(t)-R_{in}\left[i_{out}+i_{D}\right]\\
				V_{out}(t) &= V_{in}(t)-R_{in}\left[C\diff{V_{out}(t)}{t}+i_{D}\right]\\
				V_{out}(t) &= V_{in}(t)-R_{in}\left[C\diff{V_{A}(t)}{t}+C\diff{V_{B}(t)}{t}+i_{D}\right]\\
				V_{A}(t)+V_{B}(t) &= V_{in}(t)-R_{in}\left[C\diff{V_{A}(t)}{t}+C\diff{V_{B}(t)}{t}+\beta_{B}\left(e^{\alpha_{B}V_{B}(t)}-e^{-\alpha_{B}V_{B}(t)}\right)\right]\\
				V_{A}(t)+V_{B}(t) &= V_{in}(t)-R_{in}\left[C\diff{V_{A}(t)}{t}+C\diff{V_{B}(t)}{t}+2\beta_{B}\sinh(\alpha_{B}V_{B}(t))\right]\\
				\frac{V_{A}(t)+V_{B}(t)}{R_{in}} &= \frac{V_{in}(t)}{R_{in}}-C\diff{V_{A}(t)}{t}-C\diff{V_{B}(t)}{t}-2\beta_{B}\sinh(\alpha_{B}V_{B}(t))\\
				%C\diff{V_{B}(t)}{t} &= \frac{V_{in}(t)-V_{A}(t)-V_{B}(t)}{R_{in}}-C\diff{V_{A}(t)}{t}-2\beta_{B}\sinh(\alpha_{B}V_{B}(t))\\
				%\diff{V_{B}(t)}{t} &= \frac{1}{C}\left(\frac{V_{in}(t)-V_{A}(t)-V_{B}(t)}{R_{in}}-C\diff{V_{A}(t)}{t}-2\beta_{B}\sinh(\alpha_{B}V_{B}(t))\right)\\
				%V_{B}(t) &= \frac{1}{C}\Int{\left(\frac{V_{in}(t)-V_{A}(t)-V_{B}(t)}{R_{in}}-C\diff{V_{A}(t)}{t}-2\beta_{B}\sinh(\alpha_{B}V_{B}(t))\right)}{t}\\
				C\diff{V_{A}(t)}{t}+C\diff{V_{B}(t)}{t} &= \frac{V_{in}(t)-V_{A}(t)-V_{B}(t)}{R_{in}}-2\beta_{B}\sinh(\alpha_{B}V_{B}(t))\\
				\diff{V_{A}(t)}{t}+\diff{V_{B}(t)}{t} &= \frac{1}{C}\left(\frac{V_{in}(t)-V_{A}(t)-V_{B}(t)}{R_{in}}-2\beta_{B}\sinh(\alpha_{B}V_{B}(t))\right)\\
			\end{split}
		\end{equation}
		
		Derivo $V_{a}$
		\begin{equation}
			\label{eq:derivata}
			\begin{split}
				\diff{V_{a}(t)}{t} &= \diff{}{t}\left(\frac{1}{\alpha_{A}} \arcsinh \left(\frac{\beta_{B}}{\beta_{A}} \sinh(\alpha_{B}V_{B}(t))\right)\hspace{-5px}\right)\\
				&= \frac{1}{\alpha_{A}}\cdot\diff{}{t}\left(\arcsin\left(\frac{\beta_{B}}{\beta_{A}}\sinh(\alpha_{B}V_{B}(t))\right)\hspace{-5px}\right)\\
				&= \frac{\diff{}{t}\left(\frac{\beta_{B}}{\beta_{A}}\sinh(\alpha_{B}V_{B}(t))\right)}{\alpha_{A}\sqrt{1+\left(\frac{\beta_{B}}{\beta_{A}}\sinh(\alpha_{B}V_{B}(t))\right)^{2}}}\\
				&= \frac{\beta_{B}}{\beta_{A}}\cdot\frac{\diff{}{t}(\sinh(\alpha_{B}V_{B}(t)))}{\alpha_{A}\sqrt{1+\left(\sinh(\alpha_{B}V_{B}(t))\right)^{2}}}\\
				&= \frac{\beta_{B}\cosh(\alpha_{B}V_{B}(t))\diff{}{t}(\alpha_{B}V_{B}(t))}{\alpha_{A}\beta_{A}\sqrt{1+\left(\frac{\beta_{B}}{\beta_{A}}\sinh(\alpha_{B}V_{B}(t))\right)^{2}}}\\
				&= \frac{\alpha_{B}\beta_{B}\cosh(\alpha_{B}V_{B}(t))}{\alpha_{A}\beta_{A}\sqrt{1+\left(\frac{\beta_{B}}{\beta_{A}}\sinh(\alpha_{B}V_{B}(t))\right)^{2}}}\cdot\diff{V_{B}(t)}{t}
			\end{split}
		\end{equation}
		
		Sostituisco $V_{B}$ e $\diff{V_{B}}{t}$ nella (\ref{eq:eqdiff})
		\begin{comment}
		\begin{equation}
			\label{eq:sostituzione}
			\begin{split}
				V_{B}(t) &= \frac{1}{C}\Int{\left(\frac{V_{in}(t)-V_{A}(t)-V_{B}(t)}{R_{in}}-C\diff{V_{A}(t)}{t}-2\beta_{B}\sinh(\alpha_{B}V_{B}(t))\right)}{t}\\
				&= \frac{1}{C}\Int{\bigg(\frac{1}{R_{in}}\left(V_{in}(t)-\frac{1}{\alpha_{A}} \arcsinh \left(\frac{\beta_{B}}{\beta_{A}} \sinh(\alpha_{B}V_{B}(t))\right)-V_{B}(t)\right)-\\&-C\frac{\alpha_{B}\beta_{B}\cosh(\alpha_{B}V_{B}(t))\diff{}{t}(V_{B}(t))}{\alpha_{A}\beta_{A}\sqrt{1+\left(\frac{\beta_{B}}{\beta_{A}}\sinh(\alpha_{B}V_{B}(t))\right)^{2}}}-2\beta_{B}\sinh(\alpha_{B}V_{B}(t))\bigg)}{t}\\
			\end{split}
		\end{equation}
		\end{comment}
		\begin{equation}
			\label{eq:sostituzione}
			\begin{split}
				&\frac{\alpha_{B}\beta_{B}\cosh(\alpha_{B}V_{B}(t))}{\alpha_{A}\beta_{A}\sqrt{1+\left(\frac{\beta_{B}}{\beta_{A}}\sinh(\alpha_{B}V_{B}(t))\right)^{2}}}\cdot\diff{V_{B}(t)}{t}+\diff{V_{B}(t)}{t} =\\&= \frac{1}{C}\left(\frac{1}{R_{in}}\left(V_{in}(t)-\frac{1}{\alpha_{A}} \arcsinh \left(\frac{\beta_{B}}{\beta_{A}} \sinh(\alpha_{B}V_{B}(t))\right)-V_{B}(t)\right)-2\beta_{B}\sinh(\alpha_{B}V_{B}(t))\right)\\
				&\diff{V_{B}(t)}{t}\left(\frac{\alpha_{B}\beta_{B}\cosh(\alpha_{B}V_{B}(t))}{\alpha_{A}\beta_{A}\sqrt{1+\left(\frac{\beta_{B}}{\beta_{A}}\sinh(\alpha_{B}V_{B}(t))\right)^{2}}}+1\right) =\\&= \frac{1}{C}\left(\frac{1}{R_{in}}\left(V_{in}(t)-\frac{1}{\alpha_{A}} \arcsinh \left(\frac{\beta_{B}}{\beta_{A}} \sinh(\alpha_{B}V_{B}(t))\right)-V_{B}(t)\right)-2\beta_{B}\sinh(\alpha_{B}V_{B}(t))\right)\\
				&\diff{V_{B}(t)}{t}\left(\frac{\alpha_{B}\beta_{B}\cosh(\alpha_{B}V_{B}(t))+\alpha_{A}\beta_{A}\sqrt{1+\left(\frac{\beta_{B}}{\beta_{A}}\sinh(\alpha_{B}V_{B}(t))\right)^{2}}}{\alpha_{A}\beta_{A}\sqrt{1+\left(\frac{\beta_{B}}{\beta_{A}}\sinh(\alpha_{B}V_{B}(t))\right)^{2}}}\right) =\\&= \frac{1}{C}\left(\frac{1}{R_{in}}\left(V_{in}(t)-\frac{1}{\alpha_{A}} \arcsinh \left(\frac{\beta_{B}}{\beta_{A}} \sinh(\alpha_{B}V_{B}(t))\right)-V_{B}(t)\right)-2\beta_{B}\sinh(\alpha_{B}V_{B}(t))\right)\\
				&\diff{V_{B}(t)}{t} = \frac{\alpha_{A}\beta_{A}\sqrt{1+\left(\frac{\beta_{B}}{\beta_{A}}\sinh(\alpha_{B}V_{B}(t))\right)^{2}}}{\alpha_{B}\beta_{B}\cosh(\alpha_{B}V_{B}(t))+\alpha_{A}\beta_{A}\sqrt{1+\left(\frac{\beta_{B}}{\beta_{A}}\sinh(\alpha_{B}V_{B}(t))\right)^{2}}}\cdot\\&\cdot\frac{1}{C}\left(\frac{1}{R_{in}}\left(V_{in}(t)-\frac{1}{\alpha_{A}} \arcsinh \left(\frac{\beta_{B}}{\beta_{A}} \sinh(\alpha_{B}V_{B}(t))\right)-V_{B}(t)\right)-2\beta_{B}\sinh(\alpha_{B}V_{B}(t))\right)\\
				&V_{B}(t) = \frac{\alpha_{A}\beta_{A}}{C}\Int{\frac{\sqrt{1+\left(\frac{\beta_{B}}{\beta_{A}}\sinh(\alpha_{B}V_{B}(t))\right)^{2}}}{\alpha_{B}\beta_{B}\cosh(\alpha_{B}V_{B}(t))+\alpha_{A}\beta_{A}\sqrt{1+\left(\frac{\beta_{B}}{\beta_{A}}\sinh(\alpha_{B}V_{B}(t))\right)^{2}}}\cdot\\&\cdot\left(\frac{1}{R_{in}}\left(V_{in}(t)-\frac{1}{\alpha_{A}} \arcsinh \left(\frac{\beta_{B}}{\beta_{A}} \sinh(\alpha_{B}V_{B}(t))\right)-V_{B}(t)\right)-2\beta_{B}\sinh(\alpha_{B}V_{B}(t))\right)}{t}\\
			\end{split}
		\end{equation}
		
		% Risolvo la differenziale
		Discretizzo la (\ref{eq:sostituzione})\\
		Metodo di Eulero all'indietro: $\diff{y}{x} \sim \frac{y_{n}-y_{n-1}}{h}$ quindi $\frac{y_{n}-y_{n-1}}{h} = f(x_{n},y_{n})$ e si ottiene $y_{n} = y_{n-1}+hf(x_{n},y_{n})$. Sia $T = h$ allora
		\begin{equation}
			\label{eq:discretizzazione}
			\begin{split}
				&\widehat V_{B}[n] = \frac{T}{C}\cdot\frac{\alpha_{A}\beta_{A}\sqrt{1+\left(\frac{\beta_{B}}{\beta_{A}}\sinh(\alpha_{B}\widehat V_{B}[n])\right)^{2}}}{\alpha_{B}\beta_{B}\cosh(\alpha_{B}\widehat V_{B}[n])+\alpha_{A}\beta_{A}\sqrt{1+\left(\frac{\beta_{B}}{\beta_{A}}\sinh(\alpha_{B}\widehat V_{B}[n])\right)^{2}}}\cdot\\&\cdot\left(\frac{1}{R_{in}}\left(V_{in}(t)-\frac{1}{\alpha_{A}} \arcsinh \left(\frac{\beta_{B}}{\beta_{A}} \sinh(\alpha_{B}\widehat V_{B}[n])\right)-\widehat V_{B}[n]\right)-2\beta_{B}\sinh(\alpha_{B}\widehat V_{B}[n])\right)+\\&+\widehat V_{B}[n-1]\\
			\end{split}
		\end{equation}
		
		% Risolvo la funzione ottenuta dalla risoluzione della differenziale
		Algoritmo di punto fisso
		\begin{equation}
			\label{eq:punto_fisso1}
			\begin{split}
				x^{(\lambda+1)} &= x^{(\lambda)}-K(x^{(\lambda)})(x^{(\lambda)}-g(x^{(\lambda)}))\\
				&= x^{(\lambda)}-\sum_{l=0}^{L} \left(J_{f}(x^{(\lambda)})\right)^{l}(x^{(\lambda)}-f(x^{(\lambda)}))\\
			\end{split}
		\end{equation}
		poiché $K^{(L)}(x) = \sum_{l=0}^{L} \left(J_{f}(x)\right)^{l}$\\
		
		Applico l'algoritmo di punto fisso
		\begin{equation}
			\label{eq:punto_fisso2}
			\begin{split}
				x^{(\lambda+1)} = x^{(\lambda)}-\sum_{l=0}^{L} \left(J_{f}(x^{(\lambda)})\right)^{l}(x^{(\lambda)}-f(x^{(\lambda)}))\Rightarrow\\
				\Rightarrow \widetilde V_{B}^{(\lambda+1)} = \widetilde V_{B}^{(\lambda)}-\sum_{l=0}^{L} \left(J_{V_{B}}(\widetilde V_{B}^{(\lambda)})\right)^{l}(\widetilde V_{B}^{(\lambda)}-\widehat V_{B}[\widetilde V_{B}^{(\lambda)}])
			\end{split}
		\end{equation}
		
		Calcolo la matrice jacobiana della funzione $V_{B}$
		\begin{equation}
			\label{eq:jacobiana}
			\begin{split}
				&J_{V_{B}}(t) = \diff{V_{B}(t)}{t} = \frac{\alpha_{A}\beta_{A}}{C}\reallybig(\frac{\sqrt{1+\left(\frac{\beta_{B}}{\beta_{A}}\sinh(\alpha_{B}V_{B}(t))\right)^{2}}}{\alpha_{B}\beta_{B}\cosh(\alpha_{B}V_{B}(t))+\alpha_{A}\beta_{A}\sqrt{1+\left(\frac{\beta_{B}}{\beta_{A}}\sinh(\alpha_{B}V_{B}(t))\right)^{2}}}\cdot\\&\cdot\left(\frac{1}{R_{in}}\left(V_{in}(t)-\frac{1}{\alpha_{A}} \arcsinh \left(\frac{\beta_{B}}{\beta_{A}} \sinh(\alpha_{B}V_{B}(t))\right)-V_{B}(t)\right)-2\beta_{B}\sinh(\alpha_{B}V_{B}(t))\right)\hspace{-7px}\reallybig)
			\end{split}
		\end{equation}
		
		Adesso posso calcolare la corrente ai capi del condensatore:
		\begin{equation}
			\label{eq:punto_fisso2}
			\begin{split}
				V_{out}(t) &= V_{A}(t)+V_{B}(t)\\
				&= \frac{1}{\alpha_{A}} \arcsinh \left(\frac{\beta_{B}}{\beta_{A}} \sinh(\alpha_{B}V_{B}(t))\right)+V_{B}(t)
			\end{split}
		\end{equation}
	\pagebreak
	
	\appendix
	
	\chapter{Codice Matlab}
	\matlabcode{tesi.m}{tesi.m}
	\matlabcode{fixed_point.m}{fixed\_point.m}
	
	\chapter{Codice C++}
	\cppcode{PluginEditor.h}
	\cppcode{PluginEditor.cpp}
	\cppcode{PluginProcessor.h}
	\cppcode{PluginProcessor.cpp}
	\cppcode{Clipper.hpp}
	\cppcode{Clipper.cpp}
	
	\backmatter
	
	\printbibliography
\end{document}