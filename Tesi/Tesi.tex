\documentclass[12pt,a4paper,twoside,english,italian]{book}

%%%%%%%%%%%%%%%%%%%%%%%%%%%%%%%%%%%%%%% Header %%%%%%%%%%%%%%%%%%%%%%%%%%%%%%%%%%%%%%%%
%%%%%%%%%%%%%%%%%%%%%%%%%%%%%%%%%%%%%% Packages %%%%%%%%%%%%%%%%%%%%%%%%%%%%%%%%%%%%%%%

\usepackage[italian]{babel}
\usepackage[utf8]{inputenc}

\usepackage[a-1b]{pdfx}

\usepackage{uniudtesi}
\usepackage{siunitx}

\usepackage[nottoc]{tocbibind}
\usepackage{indentfirst}
\usepackage{fancyhdr}
\usepackage{emptypage}

\usepackage{amsmath}
\usepackage{esdiff}
\usepackage{cancel}
\usepackage{circuitikz}
\usepackage{comment}
\usepackage{bigints}

\usepackage{xcolor}
\usepackage{graphicx}
\usepackage{subcaption}
\usepackage{rotating}
\usepackage{float}
\usepackage{wrapfig}
\usepackage{titlesec}
\usepackage{setspace}

\usepackage{listings}
\usepackage{gensymb}
\usepackage{adjustbox}
\usepackage{tabularx}
\usepackage{makecell}
\renewcommand{\cellalign}{tr}

\usepackage[backend=biber,style=numeric,sorting=none]{biblatex}
\bibliography{bibliography}


%%%%%%%%%%%%%%%%%%%%%%%%%%%%%%%%%%%%%%% Configs %%%%%%%%%%%%%%%%%%%%%%%%%%%%%%%%%%%%%%%
%\title{Tesi}
%\author{Francesco Magoga}
%\date{\today}

\titolo{Realizzazione a tempo reale\\di un clipper per il suono\\attraverso schemi numerici\\innovativi}
\laureando{Magoga Francesco}
\annoaccademico{2020-2021}

%\setcounter{secnumdepth}{4}

\titleformat{\paragraph}
{\normalfont\normalsize\bfseries}{\theparagraph}{1em}{}
\titlespacing*{\paragraph}
{0pt}{3.25ex plus 1ex minus .2ex}{1.5ex plus .2ex}

\def\arraystretch{1.5}

\setstretch{1.2}

\definecolor{matlab_blue}{RGB}{0,114,189}
\definecolor{matlab_orange}{RGB}{217,83,25}

\definecolor{codebg}{gray}{0.95}
\colorlet{darkred}{red!90!black}

\newsavebox\mypostbreak
\savebox\mypostbreak{\mbox{\ensuremath{\color{red!85!black}\hookrightarrow}\space}}

\lstdefinestyle{code}
{
	%basicstyle=\ttfamily,
	basicstyle=\ttfamily\normalsize,
	backgroundcolor=\color{codebg},
	sensitive=false,
	alsoletter={.},
	xleftmargin=0.5cm,
	belowskip=2px,
	frame=single,
	framesep=5pt,
	framerule=1px,
	gobble=5,
	tabsize=1,
	columns=fullflexible,
	showstringspaces=false,
	numbers=left,
	firstnumber=1,
	numberfirstline=false,
	stepnumber=2,
	numberstyle=\scriptsize\color{black!80},
	numbersep=10pt,
	tabsize=4,
	keepspaces=true,
	breaklines=true,
	%postbreak={\hbox{\textcolor{red}{$\hookrightarrow$}\space}}
	postbreak=\usebox\mypostbreak,
	captionpos=b,
	literate=
    {á}{{\'a}}1 {é}{{\'e}}1 {í}{{\'i}}1 {ó}{{\'o}}1 {ú}{{\'u}}1
    {Á}{{\'A}}1 {É}{{\'E}}1 {Í}{{\'I}}1 {Ó}{{\'O}}1 {Ú}{{\'U}}1
    {à}{{\`a}}1 {è}{{\`e}}1 {ì}{{\`i}}1 {ò}{{\`o}}1 {ù}{{\`u}}1
    {À}{{\`A}}1 {È}{{\'E}}1 {Ì}{{\`I}}1 {Ò}{{\`O}}1 {Ù}{{\`U}}1
    {ä}{{\"a}}1 {ë}{{\"e}}1 {ï}{{\"i}}1 {ö}{{\"o}}1 {ü}{{\"u}}1
    {Ä}{{\"A}}1 {Ë}{{\"E}}1 {Ï}{{\"I}}1 {Ö}{{\"O}}1 {Ü}{{\"U}}1
    {â}{{\^a}}1 {ê}{{\^e}}1 {î}{{\^i}}1 {ô}{{\^o}}1 {û}{{\^u}}1
    {Â}{{\^A}}1 {Ê}{{\^E}}1 {Î}{{\^I}}1 {Ô}{{\^O}}1 {Û}{{\^U}}1
    {œ}{{\oe}}1 {Œ}{{\OE}}1 {æ}{{\ae}}1 {Æ}{{\AE}}1 {ß}{{\ss}}1
    {ç}{{\c c}}1 {Ç}{{\c C}}1 {ø}{{\o}}1 {å}{{\r a}}1 {Å}{{\r A}}1
    {€}{{\EUR}}1 {£}{{\pounds}}1 {~}{{\fontfamily{ptm}\selectfont \textasciitilde}}1
}

\lstdefinestyle{matlab}
{
	keywordstyle=\color{blue},
	stringstyle=\color{red},
	commentstyle=\color{green!80!black},
}

\lstdefinestyle{cpp}
{
	%basicstyle=\color{teal}\ttfamily,
	keywordstyle=\color{magenta!75!black},
	%directivestyle=\color{brown!80!black},
	stringstyle=\color{red!85!black},
	commentstyle=\color{green!70!black},
}

\lstdefinelanguage{cppl}
{
	language=C++,
	morekeywords={override},
	moredirectives={once, \#ifndef, \#endif, \#if, \#else, \#endif},
	%moredelim=[s][\color{red!85!black}\ttfamily]{<}{>},
}

\graphicspath{{./imgs/}}

\pagestyle{fancy}
\renewcommand{\chaptermark}[1]{\markboth{#1}{}}
\renewcommand{\sectionmark}[1]{\markright{\thesection\ #1}}
\fancyhf{}
\fancyhead[LE,RO]{\bfseries\thepage}
\fancyhead[LO]{\bfseries\rightmark}
\fancyhead[RE]{\bfseries\leftmark}
\renewcommand{\headrulewidth}{0.5pt}
\renewcommand{\footrulewidth}{0pt}
\setlength{\headheight}{14.5pt}

\setlength{\intextsep}{2pt}

%\facolta{Scienze Matematiche, Fisiche e Naturali} % (default)
\corsodilaureamagistralein{Informatica}
\relatore[Prof.]{Fontana Federico}
%\relatoreDue[Prof.]{Secondo relatore}
%\correlatore{Bozzo Enrico}
%\correlatoreDue{Secondo Correlatore}
\dedica{Ai miei genitori e ai miei nonni\\
    per avermi sempre supportato in ogni modo}

%%%%%%%%%%%%%%%%%%%%%%%%%%%%%%%%%%% Custom commands %%%%%%%%%%%%%%%%%%%%%%%%%%%%%%%%%%%
\newcommand{\sourcecode}[5]{
	%\label{list:#3}
	\lstset{caption={#4},label=code:#3,style=code,#5}
	\lstinputlisting[language=#1]{#2}
	\vspace{10px}
}

\newcommand{\matlabcode}[2]{
	\sourcecode{Matlab}{../Matlab/#1}{#1}{#2}{style=matlab}
}

\newcommand{\cppcode}[1]{
	\sourcecode{cppl}{../Juce/Source/#1}{#1}{#1}{style=cpp}
}

\newcommand{\screenshot}[3][]
{
	\begin{figure}[H]
		\centering
		\includegraphics[#1]{#2}
		\vspace*{-25px}
		\caption{#3}
		\label{fig:#2}
	\end{figure}
}

\newcommand{\grafico}[2]
{
	\screenshot[width=\textwidth]{plots/#1}{#2}
}
\newcommand{\graficospace}{\vspace{15px}}

\newcommand{\subscreenshot}[3][]
{
	\begin{subfigure}{0.5\textwidth}
		\centering
		\includegraphics[width=0.8\linewidth,#1]{#2}
		\caption{#3}
		\label{fig:#2}
	\end{subfigure}
}

\DeclareMathOperator{\arcsinh}{arcsinh}

\newcommand{\Int}[2]{\bigint \hspace{-6px} #1 \mathrm{d}#2}

\makeatletter
\newcommand{\reallybig}{\bBigg@{4}}
\newcommand{\ReallyBig}{\bBigg@{5}}
\makeatother
%%%%%%%%%%%%%%%%%%%%%%%%%%%%%%%%%%%%%%% Header %%%%%%%%%%%%%%%%%%%%%%%%%%%%%%%%%%%%%%%%


%%%%%%%%%%%%%%%%%%%%%%%%%%%%%%%%%%%%%% Document %%%%%%%%%%%%%%%%%%%%%%%%%%%%%%%%%%%%%%%
\begin{document}
	\pagestyle{fancy} 
	\frontmatter
	\maketitle	

	\enlargethispage{-1.5\baselineskip}
	\tableofcontents
	%\listoffigures
	
	\chapter*{Introduzione}
		\markboth{Introduzione}{}
		Spesso nella creazione di effetti audio digitali si parte da un modello analogico che viene opportunamente modificato per poter sottostare ai limiti imposti dalle caratteristiche hardware del computer. Questo si traduce in un passaggio di digitalizzazione del modello di partenza, poiché quando si lavora con i computer, che presentano una memoria limitata, non è possibile lavorare con segnali continui: bisogna utilizzare quindi dei segnali discreti.
		
		Un altro problema che può presentarsi quando si affronta il tema degli effetti audio digitali è la modellazione di \textit{delay-free loop}, ovvero circuiti che presentano un ciclo che collega l'uscita con l'entrata, detto feedback e non vi sono componenti che creano dei ritardi in tale ciclo. Questo significa che ad ogni instante il segnale d'uscita fa parte anche del segnale d'ingresso. Questa situazione crea non pochi problemi nella risoluzione delle equazioni associate alla rete di filtri che descrive il circuito analogico, poiché l'unico modo per calcolare il risultato è utilizzare dei metodi numerici.
		
		Un'ulteriore complicazione è causata dalle componenti non lineari che spesso caratterizzano queste delay-free loop: spesso nel percorso del feedback sono presenti dei componenti elettronici che presentano caratteristiche non lineari che devono essere opportunamente tenute in considerazione.
			
		Il seguente documento tratta quindi dell'implementazione e dell'efficienza di un metodo numerico a punto fisso geometrico, utile per calcolare la soluzione di circuiti delay-free non lineari di filtri digitali. Il metodo numerico proposto permette di passare da un risolutore a punto fisso ad un risolutore Newton-Raphson variando un parametro \texttt{L}.
		
		Per analizzare questo metodo numerico si è deciso di digitalizzare un circuito clipper analogico formato da 4 diodi, poiché tale circuito presenta entrambe le caratteristiche di non linearità e di assenza di ritardo nel feedback.
		
		Si è partiti quindi dalla realizzazione fisica del circuito analogico, per poi passare ad una simulazione della rete di filtri digitali ricavata dallo schema elettrico del circuito creato ed infine, basandosi sui risultati e sul codice della simulazione, è stato poi implementato un effetto audio digitale utilizzabile in tempo reale, anche in contesti molto pratici, tra i quali per esempio la produzione musicale.
	
	\mainmatter

	\raggedbottom
	
	\chapter{Metodo mumerico}
	\label{sec:metodo_numerico}
	Definiamo un vettore di $N$ funzioni non lineari $\mathbf{f(x)} = [f_{1}(\mathbf{x}), \dots , f_{N}(\mathbf{x})]^{T}$ nel vettore sconosciuto $\mathbf{x} = [\mathbf{x}_{1}, \dots , \mathbf{x}_{N}]^{T}$, dove $^T$ indica la trasposizione. Christoffersen \cite{christoffersen} propone una famiglia di risolutori numerici per la computazione a punto fisso di una soluzione $\mathbf{x}^{*} = \mathbf{f(x^{*})}$ del vettore sconosciuto $\mathbf{x}$ attraverso il seguente schema:
	\begin{equation}
		\label{eq:fixed-point}
		\mathbf{x}^{(\lambda+1)} = \mathbf{x}^{(\lambda)}-\mathbf{K}(\mathbf{x}^{(\lambda)})(\mathbf{x}^{(\lambda)}-\mathbf{f}(\mathbf{x}^{(\lambda)}))
	\end{equation}
	dove $\mathbf{x}^{(\lambda)}$ è la $\lambda$-esima iterazione dello schema e $\mathbf{K(x)}$ è una matrice quadrata di dimensione $N \times N$, i quali elementi dipendono da $\mathbf{x}$. Se $\mathbf{K}(\mathbf{x}) = \mathbf{I}$, la matrice identità, allora il metodo si riduce ad un risolutore di punto fisso standard \cite{atkinson}. Se $\mathbf{K}(\mathbf{x}) = (\mathbf{I}-\mathbf{J_{f}(x)})^{-1}$, dove $\mathbf{J}_{f}$ è la matrice Jacobiana di $f$, allora il metodo in particolare diventa un risolutore Newton-Raphson \cite{atkinson}.
		
	Una caratteristica notevole è che il metodo di punto fisso standard e il metodo Newton-Raphson possono essere visti come due casi limite che racchiudono una sotto-famiglia di (\ref{eq:fixed-point}). Consideriamo la somma
	\begin{equation}
		\label{eq:fixed-point-sum}
		\mathbf{K}^{(L)}(\mathbf{x}) = \sum_{l=0}^{L} \left(\mathbf{J_{f}(x)}\right)^{l}
	\end{equation}
	allora $\mathbf{K}^{(0)}(\mathbf{x}) = \mathbf{I}$, per definizione di potenza di zero di una matrice non-nulla; d'altro canto dato $\lVert\mathbf{J_{f}(x)}\rVert < 1$, dove l'operatore $\lVert\,\cdot\,\rVert$ indica la norma euclidea, allora (\ref{eq:fixed-point-sum}) converge a $\mathbf{K}^{(\infty)}(\mathbf{x}) = (\mathbf{I}-\mathbf{J_{f}(x)})^{-1}$. Questo risultato generalizza la convergenza della somma geometrica scalare quando il loro rapporto comune ha una norma più piccola di uno. In altre parole, (\ref{eq:fixed-point}) definisce un risolutore di punto fisso standard se la sommatoria si ferma immediatamente (i.e. a $L = 0$), al contrario essa definisce un risolutore Newton-Raphson se la stessa sommatoria non si ferma mai. Successivamente considereremo i casi $0 < L < \infty$ sotto il vincolo $\lVert\mathbf{J_{f}(x)}\rVert < 1$: chiameremo la corrispondente famiglia risolutori di punto fisso, così come chiameremo il parametro $L$ l'ordine dello specifico risolutore.
	
	\chapter{Clipper audio}
	Per poter analizzare il metodo numerico appena descritto si è deciso di scegliere un circuito analogico da digitalizzare per creare una rete di filtri digitali e poter così eseguire le computazioni necessarie alla risoluzione di questa rete utilizzando proprio il metodo numerico proposto.
	
	\section{Descrizione di un clipper}
		I clipper, detti anche limitatori, sono dei circuiti che tagliano la parte di un'onda di un segnale che supera una certa ampiezza. Le onde sinusoidali che superano di molto questa soglia tenderanno ad assomigliare a delle onde quadre.
		
		\begin{comment}
		\begin{wrapfigure}{r}{0.615\textwidth}
			\begin{circuitikz}[american voltages, scale=0.9, transform shape]
				\draw
					% Maglia esterna
					(0,3) to[sinusoidal voltage source,l_=$V_{in}$] (0,0)		% segnale d'ingresso
					(0,3) to[resistor] (3,3)										% resistenza
					-- (7,3)														% filo in alto
					to[C, l^=$V_{out}$] (7,0)									% condensatore
					-- (0,0)														% filo in basso
					(3,0) node[ground]{}											% messa a terra
			
					% Diodi
					(3, 0) to[full diode, *-*] (3,3)								% diodo B di sinistra
					(5, 3) to[full diode, *-*] (5,0)								% diodo B di destra
					
					% Etichette
					(0,0.75) to[open, v_<=$$, outer sep = 2mm] (0,2.25)			% segno di Vin
					(6.5,2.75) to[open, v^=$$, outer sep = 6mm] (6.5,0.25)		% tensione Vout
					;
			\end{circuitikz}
			\caption{Circuito elettronico di un clipper audio semplice}
			\label{fig:easy_clipper}
		\end{wrapfigure}
		\end{comment}
		\begin{wrapfigure}{r}{0.615\textwidth}
			\begin{circuitikz}[american voltages, scale=0.9, transform shape]
				\draw
					% Maglia esterna
					(0,3) to[sinusoidal voltage source,l_=$V_{in}$] (0,0)		% segnale d'ingresso
					(0,3) to[resistor] (3,3)										% resistenza
					-- (7,3)														% filo in alto
					to[C, l^=$V_{out}$] (7,0)									% condensatore
					-- (0,0)														% filo in basso
					(3,0) node[ground]{}											% messa a terra
			
					% Diodi
					(3, 0) to[full diode, *-*] (3,3)								% diodo B di sinistra
					(5, 3) to[full diode, *-*] (5,0)								% diodo B di destra
					
					% Etichette
					(-0.3,0.65) to[open, v_<=$$, outer sep = 2mm] (-0.3,2.40)			% segno di Vin
					(6.5,2.75) to[open, v^=$$, outer sep = 6mm] (6.5,0.25)		% tensione Vout
					;
			\end{circuitikz}
			\caption{Circuito elettronico di un clipper audio semplice}
			\label{fig:easy_clipper}
		\end{wrapfigure}
		
		Il clipper più semplice, illustrato in figura \ref{fig:easy_clipper}, è composto da un generatore di tensione i serie con una resistenza, un condensatore e due diodi, tutto in parallelo.
		
		La scelta di utilizzare un clipper analogico per verificare l'efficacia del metodo numerico è dovuta al fatto che il clipper è un circuito semplice e molto utilizzato per la ricerca in ambito di effetti audio.
	\pagebreak
	
	\section{Circuito del clipper analogico utilizzato}
		Il clipper analogico che è stato scelto per analizzare l'efficacia del metodo numerico proposto differisce dal clipper classico in quanto in questo nuovo circuito sono presenti due ulteriori diodi.
		
		\vspace{15px}
		\begin{comment}
		\begin{figure}[H]
			\centering
			\begin{circuitikz}[american voltages]
				\draw
				% Maglia esterna
				(-1,7) to[sinusoidal voltage source,l_=$V_{in}$] (-1,0)		% segnale d'ingresso
				(-1,7) to[resistor=$R_{in}$] (3,7)							% resistenza
				to[short, i^=$i_{in}$] (4,7)									% iin
				to[short, i^=$i_{out}$] (5,7)								% iout
				-- (8,7)														% filo in alto
				to[C, l_=$C$] (8,0)											% condensatore
				-- (-1,0)													% filo in basso
				(4,0) node[ground]{}											% messa a terra
				
				% Diodi
				(4,7) to[short, *-*] (4,6)			% connessione tra la maglia esterna e i diodi A
				(4,7) to[short, i^=$i_{D}$] (4,6)	% iD
				(5,6) -- (3,6)						% connessione in alto tra i diodi A
				to[empty diode, l=$D_{A}$] (3,4)		% diodo A di sinistra
				-- (5,4)								% connessione in basso tra i diodi A
				to[empty diode, l_=$D_{A}$] (5,6)	% diodo A di destra
				(4,4) to[short, *-*] (4,3)			% connessione i diodi A e i diodi B
				(5,3) -- (3,3)						% connessione in alto tra i diodi B
				to[full diode, l=$D_{B}$] (3,1)		% diodo B di sinistra
				-- (5,1)								% connessione in basso tra i diodi B
				to[full diode, l_=$D_{B}$] (5,3)		% diodo B di destra
				(4,1) to[short, *-*] (4,0)			% connessione i diodi B e la maglia esterna
				
				% Etichette
				(-1,2.5) to[open, v_<=$$, outer sep = 2mm] (-1,4.5)			% segno di Vin
				(3,6) to[open, v_=$V_{A}$, outer sep = 5mm] (3,4)			% tensione dei diodi A
				(3,3) to[open, v_=$V_{B}$, outer sep = 5mm] (3,1)			% tensione dei diodi B
				(8,7) to[open, v^=$V_{out}$, outer sep = 6mm] (8,0)			% tensione Vout
				;
			\end{circuitikz}
			\caption{Circuito elettronico del clipper audio utilizzato}
			\label{fig:clipper}
		\end{figure}
		\end{comment}
		\begin{figure}[H]
			\centering
			\begin{circuitikz}[american voltages]
				\draw
				% Maglia esterna
				(-1,7) to[sinusoidal voltage source,l_=$V_{in}$] (-1,0)		% segnale d'ingresso
				(-1,7) to[resistor=$R_{in}$] (3,7)							% resistenza
				to[short, i^=$i_{in}$] (4,7)									% iin
				to[short, i^=$i_{out}$] (5,7)								% iout
				-- (8,7)														% filo in alto
				to[C, l_=$C$] (8,0)											% condensatore
				-- (-1,0)													% filo in basso
				(4,0) node[ground]{}											% messa a terra
				
				% Diodi
				(4,7) to[short, *-*] (4,6)			% connessione tra la maglia esterna e i diodi A
				(4,7) to[short, i^=$i_{D}$] (4,6)	% iD
				(5,6) -- (3,6)						% connessione in alto tra i diodi A
				to[empty diode, l=$D_{A}$] (3,4)		% diodo A di sinistra
				-- (5,4)								% connessione in basso tra i diodi A
				to[empty diode, l_=$D_{A}$] (5,6)	% diodo A di destra
				(4,4) to[short, *-*] (4,3)			% connessione i diodi A e i diodi B
				(5,3) -- (3,3)						% connessione in alto tra i diodi B
				to[full diode, l=$D_{B}$] (3,1)		% diodo B di sinistra
				-- (5,1)								% connessione in basso tra i diodi B
				to[full diode, l_=$D_{B}$] (5,3)		% diodo B di destra
				(4,1) to[short, *-*] (4,0)			% connessione i diodi B e la maglia esterna
				
				% Etichette
				(-1.25,2.5) to[open, v_<=$$] (-1.25,4.5)		% segno di Vin
				(2.3,6) to[open, v_=$V_{A}$] (2.3,4)			% tensione dei diodi A
				(2.3,3) to[open, v_=$V_{B}$] (2.3,1)			% tensione dei diodi B
				(9,7) to[open, v^=$V_{out}$] (9,0)			% tensione Vout
				;
			\end{circuitikz}
			\caption{Circuito elettronico del clipper audio utilizzato}
			\label{fig:clipper}
		\end{figure}
		\vspace{10px}
		
		Come possiamo notare dallo schema i diodi sono tra loro a due a due in parallelo e queste due coppie di diodi sono poi collegate in serie tra loro. In particolare la prima coppia di diodi ha caratteristiche diverse dalla seconda coppia. Le intensità dei vari diodi sono descritte dall'equazione di Shockley come segue:
		\[
			i_{D_{A}} = \beta_{A}\left(e^{\alpha_{A}V_{A}(t)}-1\right)
		\]
		\[
			i_{D_{B}} = \beta_{B}\left(e^{\alpha_{B}V_{B}(t)}-1\right)
		\]
		
		Le intensità delle due coppie sono quindi
		\[
			i_{D_{A'}} = \beta_{A}\left(e^{\alpha_{A}V_{A}(t)}-1\right)-\beta_{A}\left(e^{-\alpha_{A}V_{A}(t)}-1\right)
		\]
		per la prima coppia e
		\[
			i_{D_{B'}} = \beta_{B}\left(e^{\alpha_{B}V_{B}(t)}-1\right)-\beta_{B}\left(e^{-\alpha_{B}V_{B}(t)}-1\right)
		\]
		per la seconda.
		
		Essendo le coppie in serie sappiamo che $i_{D} = i_{D_{A'}} = i_{D_{B'}}$ dunque
		\[
			i_{D} = \beta_{A}\left(e^{\alpha_{A}V_{A}(t)}-1\right)-\beta_{A}\left(e^{-\alpha_{A}V_{A}(t)}-1\right) = \beta_{B}\left(e^{\alpha_{B}V_{B}(t)}-1\right)-\beta_{B}\left(e^{-\alpha_{B}V_{B}(t)}-1\right)
		\]
		
		Siamo quindi interessati a trovare il valore di $V_{out}$ fissati i vari parametri e dato il valore di $V_{in}$.
		
		\vspace{30px}
		In figura (\ref{fig:analog/example}) si può vedere un esempio di come un onda risulti prima e dopo l'applicazione del clipper.
		
		\screenshot[width=\textwidth]{analog/example}{Esempio di onda prima e dopo del clipper}
	
	\chapter{Soluzione analitica}
	Il primo passo per trovare il valore di $V_{out}$ ad un istante di tempo $t$ è stato trovare l'equazione differenziale che rappresenta la correlazione tra il valore della tensione in entrata $V_{in}$ e il valore delle tensioni delle due coppie di diodi $V_{A}$ e $V_{B}$.
	
	Per fare ciò si è partiti dalla seconda legge di Kirchhoff applicata al circuito preso in considerazione (figura \ref{fig:clipper}).
	\[
	\begin{cases}
		V_{in}-V_{R} = V_{out}\\
		V_{in}-V_{R} = V_{A}+V_{B}\\
		V_{A}+V_{B} = V_{out}\\
	\end{cases}
	\]
	
	Inoltre applicando la prima legge di Kirchhoff abbiamo
	\[
		i_{in} = i_{out}+i_{D}
	\]
	
	Considerando la definizione di capacità elettrica
	\[
		C = \frac{Q}{\Delta V} \Rightarrow Q = C \cdot \Delta V
	\]
	e che nell'intervallo di tempo infinitesimo $\mathrm{d}t$ l'intensità di corrente del condensatore vale 
	\[
		I = \diff{Q}{t}
	\]
	otteniamo che
	\[
		I = \diff{}{t}(C \cdot \Delta V) = C\diff{\Delta V}{t}
	\]
	\pagebreak
		
	\section{Equazione differenziale}
		Da quanto appena osservato otteniamo
		\begin{equation}
			\label{eq:differenziale}
			\begin{split}
				V_{out}(t) &= V_{in}(t)-V_{R}(t)\\
				V_{out}(t) &= V_{in}(t)-R_{in}i_{in}\\
				V_{out}(t) &= V_{in}(t)-R_{in}\left[i_{out}+i_{D}\right]\\
				V_{out}(t) &= V_{in}(t)-R_{in}\left[C\diff{V_{out}(t)}{t}+i_{D}\right]\\
				V_{out}(t) &= V_{in}(t)-R_{in}\left[C\diff{V_{A}(t)}{t}+C\diff{V_{B}(t)}{t}+i_{D}\right]\\
				V_{A}(t)+V_{B}(t) &= V_{in}(t)-R_{in}\left[C\diff{V_{A}(t)}{t}+C\diff{V_{B}(t)}{t}+\beta_{B}\left(e^{\alpha_{B}V_{B}(t)}-e^{-\alpha_{B}V_{B}(t)}\right)\right]\\
				V_{A}(t)+V_{B}(t) &= V_{in}(t)-R_{in}\left[C\diff{V_{A}(t)}{t}+C\diff{V_{B}(t)}{t}+2\beta_{B}\sinh(\alpha_{B}V_{B}(t))\right]\\
				\frac{V_{A}(t)+V_{B}(t)}{R_{in}} &= \frac{V_{in}(t)}{R_{in}}-C\diff{V_{A}(t)}{t}-C\diff{V_{B}(t)}{t}-2\beta_{B}\sinh(\alpha_{B}V_{B}(t))\\
				C\diff{V_{A}(t)}{t}+C\diff{V_{B}(t)}{t} &= \frac{V_{in}(t)-V_{A}(t)-V_{B}(t)}{R_{in}}-2\beta_{B}\sinh(\alpha_{B}V_{B}(t))\\
				\diff{V_{A}(t)}{t}+\diff{V_{B}(t)}{t} &= \frac{1}{C}\left(\frac{V_{in}(t)-V_{A}(t)-V_{B}(t)}{R_{in}}-2\beta_{B}\sinh(\alpha_{B}V_{B}(t))\right)\\
			\end{split}
		\end{equation}
		che è l'equazione che mette in relazione le tensioni sui vari componenti del circuito.
		\pagebreak
		
		Per poter risolvere l'equazione tramite il metodo numerico dobbiamo esprimere l'equazione differenziale (\ref{eq:differenziale}) appena trovata in funzione di una delle due variabili $V_{A}$ o $V_{B}$, variabile che comparirà sia nel primo, che nel secondo termine della nostra equazione. In questo caso è stato scelto di esprimere tutto in funzione di $V_{B}$, perciò il passo successivo è stato di trovare una formula che esprimesse $V_{A}$ in funzione di $V_{B}$.
		\begin{equation}
			\label{eq:tensione_A_B}
			\begin{split}
				\beta_{A}\left(e^{\alpha_{A}V_{A}(t)}-1\right)-\beta_{A}\left(e^{-\alpha_{A}V_{A}(t)}-1\right) &= \beta_{B}\left(e^{\alpha_{B}V_{B}(t)}-1\right)-\beta_{B}\left(e^{-\alpha_{B}V_{B}(t)}-1\right)\\
				\beta_{A}\left(e^{\alpha_{A}V_{A}(t)}-e^{-\alpha_{A}V_{A}(t)}\right) &= \beta_{B}\left(e^{\alpha_{B}V_{B}(t)}-e^{-\alpha_{B}V_{B}(t)}\right)\\
				\cancel{2}\beta_{A}\left(\frac{e^{\alpha_{A}V_{A}(t)}-e^{-\alpha_{A}V_{A}(t)}}{2}\right) &= \cancel{2}\beta_{B}\left(\frac{e^{\alpha_{B}V_{B}(t)}-e^{-\alpha_{B}V_{B}(t)}}{2}\right)\\
				\beta_{A} \sinh(\alpha_{A}V_{A}(t)) &= \beta_{B} \sinh(\alpha_{B}V_{B}(t))\\
				\sinh(\alpha_{A}V_{A}(t)) &= \frac{\beta_{B}}{\beta_{A}} \sinh(\alpha_{B}V_{B}(t))\\
				\alpha_{A}V_{A}(t) &= \arcsinh \left(\frac{\beta_{B}}{\beta_{A}} \sinh(\alpha_{B}V_{B}(t))\right)\\
				V_{A}(t) &= \frac{1}{\alpha_{A}} \arcsinh \left(\frac{\beta_{B}}{\beta_{A}} \sinh(\alpha_{B}V_{B}(t))\right)
			\end{split}
		\end{equation}
		\pagebreak
		
		Prima di sostituire le occorrenze di $V_{A}$ nell'equazione differenziale (\ref{eq:differenziale}) con la formula appena trovata (\ref{eq:tensione_A_B}) notiamo che nella (\ref{eq:differenziale}) è richiesto di derivare $V_{A}$.
		\begin{equation}
			\label{eq:derivata_va}
			\begin{split}
				\diff{V_{a}(t)}{t} &= \diff{}{t}\left(\frac{1}{\alpha_{A}} \arcsinh \left(\frac{\beta_{B}}{\beta_{A}} \sinh(\alpha_{B}V_{B}(t))\right)\hspace{-5px}\right)\\
				&= \frac{1}{\alpha_{A}}\cdot\diff{}{t}\left(\arcsin\left(\frac{\beta_{B}}{\beta_{A}}\sinh(\alpha_{B}V_{B}(t))\right)\hspace{-5px}\right)\\
				&= \frac{\diff{}{t}\left(\frac{\beta_{B}}{\beta_{A}}\sinh(\alpha_{B}V_{B}(t))\right)}{\alpha_{A}\sqrt{1+\left(\frac{\beta_{B}}{\beta_{A}}\sinh(\alpha_{B}V_{B}(t))\right)^{2}}}\\
				&= \frac{\beta_{B}}{\beta_{A}}\cdot\frac{\diff{}{t}(\sinh(\alpha_{B}V_{B}(t)))}{\alpha_{A}\sqrt{1+\left(\sinh(\alpha_{B}V_{B}(t))\right)^{2}}}\\
				&= \frac{\beta_{B}\cosh(\alpha_{B}V_{B}(t))\diff{}{t}(\alpha_{B}V_{B}(t))}{\alpha_{A}\beta_{A}\sqrt{1+\left(\frac{\beta_{B}}{\beta_{A}}\sinh(\alpha_{B}V_{B}(t))\right)^{2}}}\\
				&= \frac{\alpha_{B}\beta_{B}\cosh(\alpha_{B}V_{B}(t))}{\alpha_{A}\beta_{A}\sqrt{1+\left(\frac{\beta_{B}}{\beta_{A}}\sinh(\alpha_{B}V_{B}(t))\right)^{2}}}\cdot\diff{V_{B}(t)}{t}
			\end{split}
		\end{equation}
		\pagebreak

		Procediamo quindi con la sostituzione di $V_{A}$ e $\diff{V_{A}}{t}$ nell'equazione differenziale (\ref{eq:differenziale}) con quanto trovato nelle equazioni (\ref{eq:tensione_A_B}) e (\ref{eq:derivata_va}).
		\begin{equation}
			\label{eq:integrale}
			\begin{split}
				&\frac{\alpha_{B}\beta_{B}\cosh(\alpha_{B}V_{B}(t))}{\alpha_{A}\beta_{A}\sqrt{1+\left(\frac{\beta_{B}}{\beta_{A}}\sinh(\alpha_{B}V_{B}(t))\right)^{2}}}\cdot\diff{V_{B}(t)}{t}+\diff{V_{B}(t)}{t} =\\&= \frac{1}{C}\left(\frac{1}{R_{in}}\left(V_{in}(t)-\frac{1}{\alpha_{A}} \arcsinh \left(\frac{\beta_{B}}{\beta_{A}} \sinh(\alpha_{B}V_{B}(t))\right)-V_{B}(t)\right)-2\beta_{B}\sinh(\alpha_{B}V_{B}(t))\right)\Rightarrow\\
				&\Rightarrow \diff{V_{B}(t)}{t}\left(\frac{\alpha_{B}\beta_{B}\cosh(\alpha_{B}V_{B}(t))}{\alpha_{A}\beta_{A}\sqrt{1+\left(\frac{\beta_{B}}{\beta_{A}}\sinh(\alpha_{B}V_{B}(t))\right)^{2}}}+1\right) =\\&= \frac{1}{C}\left(\frac{1}{R_{in}}\left(V_{in}(t)-\frac{1}{\alpha_{A}} \arcsinh \left(\frac{\beta_{B}}{\beta_{A}} \sinh(\alpha_{B}V_{B}(t))\right)-V_{B}(t)\right)-2\beta_{B}\sinh(\alpha_{B}V_{B}(t))\right)\Rightarrow\\
				&\Rightarrow \diff{V_{B}(t)}{t} = \left(\frac{\alpha_{B}\beta_{B}\cosh(\alpha_{B}V_{B}(t))}{\alpha_{A}\beta_{A}\sqrt{1+\left(\frac{\beta_{B}}{\beta_{A}}\sinh(\alpha_{B}V_{B}(t))\right)^{2}}}+1\right)^{-1}\cdot\\&\cdot\frac{1}{C}\left(\frac{1}{R_{in}}\left(V_{in}(t)-\frac{1}{\alpha_{A}} \arcsinh \left(\frac{\beta_{B}}{\beta_{A}} \sinh(\alpha_{B}V_{B}(t))\right)-V_{B}(t)\right)-2\beta_{B}\sinh(\alpha_{B}V_{B}(t))\right)\Rightarrow\\
				&\Rightarrow V_{B}(t) = \Int{\Bigg(\left(\frac{\alpha_{B}\beta_{B}\cosh(\alpha_{B}V_{B}(t))}{\alpha_{A}\beta_{A}\sqrt{1+\left(\frac{\beta_{B}}{\beta_{A}}\sinh(\alpha_{B}V_{B}(t))\right)^{2}}}+1\right)^{-1}\cdot\\&\cdot\frac{1}{C}\left(\frac{1}{R_{in}}\left(V_{in}(t)-\frac{1}{\alpha_{A}} \arcsinh \left(\frac{\beta_{B}}{\beta_{A}} \sinh(\alpha_{B}V_{B}(t))\right)-V_{B}(t)\right)-2\beta_{B}\sinh(\alpha_{B}V_{B}(t))\right)\Bigg)}{t}\\
			\end{split}
		\end{equation}
		\pagebreak
		
	\section{Calcolo numerico dell'equazione differenziale}
		Da notare che la precedente equazione (\ref{eq:sostituzione}) è in forma implicita, poiché il termine $V_{B}(t)$ compare sia nel lhs che nel rhs dell'equazione. Per tale motivo l'unico modo per trovare il valore di questa incognita è utilizzare un metodo numerico.
		
		Arrivati a questo punto per risolvere l'equazione differenziale appena ottenuta (\ref{eq:sostituzione}) si è utilizzato il metodo di Eulero implicito (o all'indietro). Si è scelto questo metodo per discretizzare l'equazione da risolvere perché ha una buona stabilità ed è semplice da applicare.
		
		Il metodo di Eulero all'indietro viene ricavato dall'approssimazione della derivata con le differenze finite all'indietro
		\[
			\diff{y}{x} \approx \frac{y_{n}-y_{n-1}}{h}
		\]
		da cui si ottiene
		\[
			\frac{y_{n}-y_{n-1}}{h} = f(x_{n},y_{n}) \implies y_{n} = y_{n-1}+hf(x_{n},y_{n})
		\]
		
		Sappiamo che 
		\begin{equation}
			\label{eq:sostituzione}
			\begin{split}
				&\left(\frac{\alpha_{B}\beta_{B}\cosh(\alpha_{B}V_{B}(t))}{\alpha_{A}\beta_{A}\sqrt{1+\left(\frac{\beta_{B}}{\beta_{A}}\sinh(\alpha_{B}V_{B}(t))\right)^{2}}}+1\right)^{-1} =\\
				&= \left(\frac{\alpha_{B}\beta_{B}\cosh(\alpha_{B}V_{B}(t))+\alpha_{A}\beta_{A}\sqrt{1+\left(\frac{\beta_{B}}{\beta_{A}}\sinh(\alpha_{B}V_{B}(t))\right)^{2}}}{\alpha_{A}\beta_{A}\sqrt{1+\left(\frac{\beta_{B}}{\beta_{A}}\sinh(\alpha_{B}V_{B}(t))\right)^{2}}}\right)^{-1}\\
				&= \frac{\sqrt{1+\left(\frac{\beta_{B}}{\beta_{A}}\sinh(\alpha_{B}V_{B}(t))\right)^{2}}}{\alpha_{B}\beta_{B}\cosh(\alpha_{B}V_{B}(t))+\alpha_{A}\beta_{A}\sqrt{1+\left(\frac{\beta_{B}}{\beta_{A}}\sinh(\alpha_{B}V_{B}(t))\right)^{2}}}
			\end{split}
		\end{equation}
		
		Impostiamo il parametro $h$ in modo che sia equivalente al periodo di campionamento
		\begin{equation}
			\label{eq:valore_h}
			\begin{split}
				h = T = \frac{1}{Fs}
			\end{split}
		\end{equation}
		
		L'equazione discretizzata risulta quindi essere
		\begin{equation}
			\label{eq:discretizzazione}
			\begin{split}
				&\widehat V_{B}[n] = h \; \widehat V_{B}[n]+\widehat V_{B}[n-1] =\\
				&= \frac{T}{C}\cdot\frac{\sqrt{1+\left(\frac{\beta_{B}}{\beta_{A}}\sinh(\alpha_{B}V_{B}[n])\right)^{2}}}{\alpha_{B}\beta_{B}\cosh(\alpha_{B}V_{B}[n])+\alpha_{A}\beta_{A}\sqrt{1+\left(\frac{\beta_{B}}{\beta_{A}}\sinh(\alpha_{B}V_{B}[n])\right)^{2}}}\cdot\\&\cdot\left(\frac{1}{R_{in}}\left(V_{in}(n)-\frac{1}{\alpha_{A}} \arcsinh \left(\frac{\beta_{B}}{\beta_{A}} \sinh(\alpha_{B}\widehat V_{B}[n])\right)-\widehat V_{B}[n]\right)-2\beta_{B}\sinh(\alpha_{B}\widehat V_{B}[n])\right)+\\&+\widehat V_{B}[n-1]\\
			\end{split}
		\end{equation}
		\pagebreak
	
	\section{Applicazione del metodo numerico}
		Richiamiamo ora la formula (\ref{eq:fixed-point}) del metodo numerico da applicare
		\begin{equation}
			\label{eq:punto_fisso1}
			\begin{split}
				\mathbf{x}^{(\lambda+1)} &= \mathbf{x}^{(\lambda)}-\mathbf{K}(\mathbf{x}^{(\lambda)})(\mathbf{x}^{(\lambda)}-\mathbf{f}(\mathbf{x}^{(\lambda)}))\\
				&= \mathbf{x}^{(\lambda)}-\sum_{l=0}^{L} \left(\mathbf{J_{f}(x)}\right)^{l}(\mathbf{x}^{(\lambda)}-\mathbf{f}(\mathbf{x}^{(\lambda)}))\\
			\end{split}
		\end{equation}
		poiché per (\ref{eq:fixed-point-sum}) abbiamo $\mathbf{K}^{(L)}(\mathbf{x}) = \sum_{l=0}^{L} \left(\mathbf{J_{f}(x)}\right)^{l}$\\
		
		Nel nostro caso la formula del metodo numerico verrà usata per calcolare il valore $V_{B}$, utilizzando la funzione discretizzata $f = \widehat V_{B}$. La (\ref{eq:punto_fisso1}) dunque diventa
		\begin{equation}
			\label{eq:punto_fisso2}
			\begin{split}
				V_{B}^{(\lambda+1)} = V_{B}^{(\lambda)}-\sum_{l=0}^{L} \left(J_{\widehat V_{B}}(V_{B}^{(\lambda)})\right)^{l}(V_{B}^{(\lambda)}-\widehat V_{B}[V_{B}^{(\lambda)}])
			\end{split}
		\end{equation}
		
		Il metodo richiede il calcolo del termine $J_{\widehat V_{B}}$, ovvero la matrice jacobiana della funzione $\widehat V_{B}$. La matrice jacobiana nel caso di vettori con un singolo elemento, come nel nostro caso, equivale alla derivata della funzione a cui è associata. In poche parole, in questo caso $J_{\widehat V_{B}}(n) = \diff{\widehat V_{B}[n]}{n}$.\\
		
		Per calcolare tale derivata bisognerà applicare la regola del prodotto, che richiederà di scomporre la funzione $\widehat V_{B}[n] = \frac{h}{C}\cdot f[n]\cdot g[n]$.
		Le due funzioni $f[n]$ e $g[n]$ risulteranno essere
		\begin{equation}
			\begin{split}
				f[n] = \left(\frac{\alpha_{B}\beta_{B}\cosh(\alpha_{B}\widehat V_{B}[n])}{\alpha_{A}\beta_{A}\sqrt{1+\left(\frac{\beta_{B}}{\beta_{A}}\sinh(\alpha_{B}\widehat V_{B}[n])\right)^{2}}}+1\right)^{-1} = \frac{1}{\varphi[n]}
			\end{split}
		\end{equation}
		\begin{equation}
			\begin{split}
				g[n] = \frac{1}{R_{in}}\left(V_{in}(n)-V_{A}[n]-\widehat V_{B}[n]\right)-2\beta_{B}\sinh(\alpha_{B}\widehat V_{B}[n])
			\end{split}
		\end{equation}
		dove si è sostituito il termine della sottrazione con $V_{A}[n]$ seguendo la (\ref{eq:tensione_A_B}).

		La derivata sarà
		\begin{equation}
			\label{eq:jacobiano}
			\begin{split}
				\diff{\widehat V_{B}[n]}{n} &= \frac{h}{C}\left(\diff{f[n]}{n}\cdot g[n]+f[n]\cdot \diff{g[n]}{n}\right)\\
				&= \frac{h}{C}\left(-\diff{\varphi[n]}{n}\cdot\frac{1}{\varphi[n]^{2}}\cdot g[n]+\frac{1}{\varphi[n]}\cdot \diff{g[n]}{n}\right)
			\end{split}
		\end{equation}

		Iniziamo con la derivata della funzione $f[n]$ che abbiamo detto equivalere a
		\begin{equation}
			\label{eq:jacobiano}
			\begin{split}
				\diff{f[n]}{n} = \diff{\frac{1}{\varphi[n]}}{n} = -\diff{\varphi[n]}{n}\cdot\frac{1}{\varphi[n]^{2}}
			\end{split}
		\end{equation}

		
		A questo punto bisogna calcolare la derivata della funzione $\varphi[n]$.
		\begin{equation}
			\begin{split}
				\diff{\varphi[n]}{n} &= \diff{}{n}\left(\frac{\alpha_{B}\beta_{B}\cosh(\alpha_{B}\widehat V_{B}[n])}{\alpha_{A}\beta_{A}\sqrt{1+\left(\frac{\beta_{B}}{\beta_{A}}\sinh(\alpha_{B}\widehat V_{B}[n])\right)^{2}}}+1\right)\\
				&= \frac{\alpha_{B}\beta_{B}}{\alpha_{A}\beta_{A}}\cdot\diff{}{n}\left(\frac{\cosh(\alpha_{B}\widehat V_{B}[n])}{\sqrt{1+\left(\frac{\beta_{B}}{\beta_{A}}\sinh(\alpha_{B}\widehat V_{B}[n])\right)^{2}}}\right)\\
				&= \frac{\alpha_{B}\beta_{B}}{\alpha_{A}\beta_{A}}\cdot\frac{\psi}{1+\left(\frac{\beta_{B}}{\beta_{A}}\sinh(\alpha_{B}\widehat V_{B}[n])\right)^{2}}
			\end{split}
		\end{equation}
		
		dove $\psi[n]$ è definita come segue
		\begin{equation}
			\begin{split}
				\psi[n] = \alpha_{B}\sinh(\alpha_{B}\widehat V_{B}[n])\cdot\sqrt{1+\left(\frac{\beta_{B}}{\beta_{A}}\sinh(\alpha_{B}\widehat V_{B}[n])\right)^{2}}-\cosh(\alpha_{B}\widehat V_{B}[n])\,\cdot\\
				\cdot\,\diff{}{n}\left(\sqrt{1+\left(\frac{\beta_{B}}{\beta_{A}}\sinh(\alpha_{B}\widehat V_{B}[n])\right)^{2}}\right)
			\end{split}
		\end{equation}
		\pagebreak
		
		Calcoliamo infine l'ultima derivata
		\vspace{-10px}
		\begin{equation}
			\begin{split}
				\diff{}{n}\left(\sqrt{1+\left(\frac{\beta_{B}}{\beta_{A}}\sinh(\alpha_{B}\widehat V_{B}[n])\right)^{2}}\right) &= \frac{\diff{}{n}\left(1+\left(\frac{\beta_{B}}{\beta_{A}}\sinh(\alpha_{B}\widehat V_{B}[n])\right)^{2}\right)}{2\sqrt{1+\left(\frac{\beta_{B}}{\beta_{A}}\sinh(\alpha_{B}\widehat V_{B}[n])\right)^{2}}}\\
				&= \frac{2\left(\frac{\beta_{B}}{\beta_{A}}\sinh(\alpha_{B}\widehat V_{B}[n])\right)\frac{\beta_{B}}{\beta_{A}}\diff{}{n}\left(\sinh(\alpha_{B}\widehat V_{B}[n])\right)}{2\sqrt{1+\left(\frac{\beta_{B}}{\beta_{A}}\sinh(\alpha_{B}\widehat V_{B}[n])\right)^{2}}}\\
				&= \frac{\alpha_{B}\beta_{B}^{2}}{\beta_{A}^{2}}\cdot\frac{\sinh(\alpha_{B}\widehat V_{B}[n])\cosh(\alpha_{B}\widehat V_{B}[n])}{\sqrt{1+\left(\frac{\beta_{B}}{\beta_{A}}\sinh(\alpha_{B}\widehat V_{B}[n])\right)^{2}}}
			\end{split}
		\end{equation}

		Passiamo ora alla derivata della funzione $g[n]$, che equivale a
		\begin{equation}
			\begin{split}
				\diff{g[n]}{n} &= \frac{1}{R_{in}}\cdot\diff{\left(V_{in}(n)-V_{A}[n]-V_{B}[n]\right)}{n}-\diff{\left(2\beta_{B}\sinh(\alpha_{B}\widehat V_{B}[n])\right)}{n} =\\
				&= \frac{1}{R_{in}}\left(\diff{V_{in}(n)}{n}-\diff{V_{A}[n]}{n}-\diff{V_{B}[n]}{n}\right)-2\alpha_{B}\beta_{B}\cosh(\alpha_{B}\widehat V_{B}[n]) =\\
				&= \frac{1}{R_{in}}\left(-\frac{\alpha_{B}\beta_{B}\cosh(\alpha_{B}\widehat V_{B}[n])}{\alpha_{A}\beta_{A}\sqrt{1+\left(\frac{\beta_{B}}{\beta_{A}}\sinh(\alpha_{B}\widehat V_{B}[n])\right)^{2}}}-1\right)-2\alpha_{B}\beta_{B}\cosh(\alpha_{B}\widehat V_{B}[n])
			\end{split}
		\end{equation}
		\vspace{-8px}
		
		\noindent dove $\diff{V_{A}[n]}{n}$ è data dalla (\ref{eq:derivata_va}).\\
		
		Dopo aver calcolato numericamente il valore di $V_{B}$ tramite la formula (\ref{eq:punto_fisso2}), si può calcolare la corrente ai capi del condensatore, ovvero $V_{out}$, utilizzando sempre la seconda legge di Kirchhoff:
		\begin{equation}
			\label{eq:tensione_finale}
			\begin{split}
				V_{out}[n] &= V_{A}[n]+V_{B}[n]\\
				&= \frac{1}{\alpha_{A}} \arcsinh \left(\frac{\beta_{B}}{\beta_{A}} \sinh(\alpha_{B}V_{B}[n])\right)+V_{B}[n]
			\end{split}
		\end{equation}
%\flushbottom
	
	\chapter{Simulazione}
	Partendo dalle formule ricavate dalla soluzione analitica, in particolare da (\ref{eq:discretizzazione}), (\ref{eq:punto_fisso2}), (\ref{eq:jacobiano}) e (\ref{eq:tensione_finale}), è stata realizzata una simulazione in linguaggio Matlab \cite{matlab, matlab_book}. Matlab mette a disposizione del programmatore un ambiente per il calcolo numerico che offre delle funzionalità molto potenti, richiedendo di scrivere un codice di dimensioni abbastanza contenute. Il sorgente è disponibile nell'appendice (\ref{code:matlab}).
	
	\section{Parametri}
		Per questa simulazione si è scelto di assegnare caratteristiche uguali a tutti i diodi. In particolare vale $\alpha = \frac{1}{nV_{E}}$ dove $n \approx 2$ per i diodi al silicio. In questo caso quindi è stato arrotondato $n = 2$.
	
		I valori dei vari componenti scelti per la simulazione sono riassunti di seguito:
		\[
			R_{in} = 1k\Omega
		\]
		\[
			V_{out} = 100nF
		\]
		\[
			V_{E} = 2,23mV
		\]
		\[
			\beta = 2,52nA
		\]
		quindi
		\[
			\alpha = \frac{1}{nV_{E}} = \frac{1}{2 \cdot 2,23mV} = \frac{1}{4,46mV}
		\]
	
		Inoltre è stato scelto
		\[
			Fs = 48kHz
		\]
		come frequenza di campionamento e di conseguenza
		\[
			T = \frac{1}{Fs} = \frac{1}{48000}s
		\]
		come periodo di campionamento e step temporale.
		
		Come criterio d'arresto per il metodo numerico è invece stato scelto
		\[
			|V_{B}^{(\lambda+1)} - V_{B}^{(\lambda)}| < 0,1mV
		\]
	
		Come input è stata scelta un'onda sinusoidale caratterizzata dalla classica equazione $y(x) = A\sin (2 \pi f x + \phi)$ e dai seguenti parametri (tranne dove specificato diversamente):
		\[
			f = 100Hz
		\]
		\[
			A = 1V
		\]
		\[
			\phi = 0
		\]
		
		\vspace{10px}Inoltre, per prevenire cicli infiniti dovuti alla non convergenza del metodo numerico è stato impostato un limite massimo di iterazioni nella risoluzione dell'equazione differenziale discreta (\ref{eq:discretizzazione}): in particolare questo limite è pari a $250$ iterazioni, dopo le quali si esce dal ciclo e si tiene il valore prodotto dalla $250$-esima iterazione.
	\pagebreak
	
	\section{Grafici}
		Vediamo ora alcuni grafici risultanti dalla simulazione, tramite i quali potremo capire il comportamento del circuito e del metodo numerico.
		
		\subsection{Grafici input-output}
			\label{graphs:input-output}
			Nei seguenti grafici verrà mostrata con una linea blu continua
			\begin{tikzpicture}
				\draw[thick,color=matlab_blue] (0,0) -- (1,0);
				\draw[thick,color=white] (1,0) -- (1,-0.07);				%per allineamento verticale
			\end{tikzpicture}
			il segnale in \textcolor{matlab_blue}{input}, mentre con una linea tratteggiata arancione
			\begin{tikzpicture}
				\draw[thick,dashed,color=matlab_orange] (0,0) -- (1,0);
				\draw[thick,color=white] (1,0) -- (1,-0.07);				%per allineamento verticale
			\end{tikzpicture}
			sarà mostrato il segnale in \textcolor{matlab_orange}{output} generato dal clipper. Il parametro $L$ è fissato a $L = \infty$, in quanto l'output, al variare di $L$, non cambia, ma la convergenza è sempre garantita per tale valore di $L$. La frequenza di campionamento invece è $Fs = 48kHz$, come già detto in precedenza.
		
			\subsubsection{Onda sinusoidale}
				Per prima cosa si è analizzato il comportamento del clipper al variare della frequenza di un'onda sinusoidale in ingresso, fissando $A = 1V$.
				\grafico{in-out/frequenze/50Hz}{Grafico input-output con $f = 50Hz$}
				\graficospace
				\grafico{in-out/frequenze/100Hz}{Grafico input-output con $f = 100Hz$}
				\graficospace
				\grafico{in-out/frequenze/250Hz}{Grafico input-output con $f = 250Hz$}
				\graficospace
				\grafico{in-out/frequenze/500Hz}{Grafico input-output con $f = 500Hz$}
				\graficospace
				\grafico{in-out/frequenze/1000Hz}{Grafico input-output con $f = 1000Hz$}
				\graficospace
				\grafico{in-out/frequenze/1500Hz}{Grafico input-output con $f = 1500Hz$}
				
				All'aumentare della frequenza notiamo uno sfasamento tra le onde di input e di output. Questa latenza, dovuta al fenomeno di carica e scarica di un condensatore, non varia mai, ma è più visibile ad alte frequenze del segnale di input perché l'intervallo di tempo nel quale si va ad analizzare i vari segnali si riduce con l'aumentare della frequenza.
				
				Inoltre l'ampiezza del segnale in uscita si riduce con l'aumentare della frequenza, poiché l'alta velocità di variazione del segnale richiede al condensatore di cambiare stato molto frequentemente e questo impedisce proprio al condensatore di caricarsi del tutto e anzi fa in modo che, all'aumentare della frequenza, il condensatore si carichi sempre di meno per ogni picco del segnale.
				\pagebreak
				
				Successivamente si è analizzato il comportamento del clipper al variare dell'ampiezza dell'onda in ingresso.
				Si è prima considerata un'onda con frequenza $f = 100Hz$.
				\grafico{in-out/ampiezze/100Hz/0,5V}{Grafico input-output con $f = 100Hz$ e $A = 0,5V$}
				\graficospace
				\grafico{in-out/ampiezze/100Hz/1,5V}{Grafico input-output con $f = 100Hz$ e $A = 1,5V$}
				\graficospace
				\grafico{in-out/ampiezze/100Hz/2V}{Grafico input-output con $f = 100Hz$ e $A = 2V$}
				\graficospace
				\grafico{in-out/ampiezze/100Hz/3V}{Grafico input-output con $f = 100Hz$ e $A = 3V$}
				\graficospace
				\grafico{in-out/ampiezze/100Hz/5V}{Grafico input-output con $f = 100Hz$ e $A = 5V$}
				\graficospace
				\grafico{in-out/ampiezze/100Hz/10V}{Grafico input-output con $f = 100Hz$ e $A = 10V$}
				\pagebreak
				
				Si è poi considerata un'onda con frequenza $f = 1000Hz$.
				\grafico{in-out/ampiezze/1000Hz/0,5V}{Grafico input-output con $f = 1000Hz$ e $A = 0,5V$}
				\graficospace
				\grafico{in-out/ampiezze/1000Hz/1,5V}{Grafico input-output con $f = 1000Hz$ e $A = 1,5V$}
				\graficospace
				\grafico{in-out/ampiezze/1000Hz/2V}{Grafico input-output con $f = 1000Hz$ e $A = 2V$}
				\graficospace
				\grafico{in-out/ampiezze/1000Hz/3V}{Grafico input-output con $f = 1000Hz$ e $A = 3V$}
				\graficospace
				\grafico{in-out/ampiezze/1000Hz/5V}{Grafico input-output con $f = 1000Hz$ e $A = 5V$}
				\graficospace
				\grafico{in-out/ampiezze/1000Hz/10V}{Grafico input-output con $f = 1000Hz$ e $A = 10V$}
				\pagebreak
				
				Dai vari grafici si può vedere come, per ogni ampiezza del segnale in ingresso, all'aumentare della frequenza, vengano "tagliate" le parti del segnale che superano una certa soglia di ampiezza, che si riduce all'aumentare dell'ampiezza del segnale in ingresso.
				
				Il valore del parametro $L$ come già detto equivale a $L = \infty$, dato che, per valori minori, non si riesce ad ottenere una buona convergenza dell'algoritmo per segnali sinusoidali. Ricordiamo che se $L = \infty$ allora $\sum_{l=0}^{L} \left(J_{\widehat V_{B}}(V_{B}^{(\lambda)})\right)^{l} = (\mathbf{I}-\mathbf{J_{f}(x)})^{-1}$ per (\ref{eq:fixed-point-sum}) e (\ref{eq:punto_fisso2}).
				
				Per valori di $L \in [0;50]$ l'algoritmo converge su segnali di input sinusoidali solo per valori dell'ampiezza $A < 1,5V$. Per $A = 1,5V$ invece si inizino ad intravedere dei segni di instabilità numerica nelle creste dell'onda, quando si raggiunge il picco dell'ampiezza, in particolare per $f = 100Hz$. Per segnali con questa frequenza e $L = 50$ l'algoritmo non converge, mentre per $L = 0$ si ottiene il seguente grafico.
				\grafico{in-out/ampiezze/100Hz_1,5V_L=0}{Grafico input-output con $f = 100Hz$, $A = 1,5V$ e $L = 0$}
				\pagebreak
				
				Se non ci fosse il limite sulle iterazioni, già per questo valore dell'ampiezza il metodo numerico non terminerebbe mai, come si può notare dal seguente grafico.
				\grafico{in-out/ampiezze/100Hz_1,5V_L=0-iterations}{Grafico delle iterazioni con $f = 100Hz$, $A = 1,5V$ e $L = 0$}
				\graficospace
				
				Per $A \ge 1,6V$ (\ref{fig:plots/in-out/ampiezze/100Hz_1,6V_L=0}) l'algoritmo non converge più e da un certo momento in poi ($t \approx 2ms$) e il valore di output risulta infinito.
				\grafico{in-out/ampiezze/100Hz_1,6V_L=0}{Grafico input-output con $f = 100Hz$, $A = 1,6V$ e $L = 0$}
				\pagebreak
				
				%Da notare come con $A = 1,7V$ (\ref{fig:plots/in-out/ampiezze/1,7V_L0}) si inizino ad intravedere dei segni di instabilità numerica nelle creste dell'onda, quindi quando si raggiunge il picco dell'ampiezza. Se non ci fosse il limite sulle iterazioni già per questo valore dell'ampiezza il metodo numerico non terminerebbe mai. A partire da circa $A = 1,9V$ (\ref{fig:plots/in-out/ampiezze/1,9V_L0}) l'algoritmo non converge più e da un certo momento in poi ($t \approx 2ms$) viene restituito infinito come valore di output.
				%\pagebreak
				
				%Vediamo che aumentando il valore di $L$ l'algoritmo adesso converge anche per valori che prima portavano ad una divergenza (\ref{fig:plots/in-out/ampiezze/1,9V_L50}), nonostante sia presente un'evidente instabilità numerica. Comunque per $L = 50$ il metodo numerico non converge più a partire da circa $A = 2,2V$ (\ref{fig:plots/in-out/ampiezze/2,2V_L50}).
				%\pagebreak
			
			\subsubsection{Altri tipi di onda}
				Di seguito sono riportati alcuni grafici con onde di diverso tipo, per dare un'idea di quale sia la risposta dell'algoritmo alle varie forme d'onda in input. Per ogni forma d'onda viene riportato un grafico con frequenza $f = 100Hz$ e  $f = 1000Hz$ con ampiezza $A = 1,5V$ in entrambi i casi.
				
				\grafico{in-out/altre/triangle/100Hz}{Grafico input-output di un'onda triangolare con $f = 100Hz$}
				\graficospace
				\grafico{in-out/altre/triangle/1000Hz}{Grafico input-output di un'onda triangolare con $f = 1000Hz$}
				\graficospace
				\grafico{in-out/altre/saw/100Hz}{Grafico input-output di un'onda a dente di sega con $f = 100Hz$}
				\graficospace
				\grafico{in-out/altre/saw/1000Hz}{Grafico input-output di un'onda a dente di sega con $f = 1000Hz$}
				\graficospace
				\grafico{in-out/altre/square/100Hz}{Grafico input-output di un'onda quadra con $f = 100Hz$}
				\graficospace
				\grafico{in-out/altre/square/1000Hz}{Grafico input-output di un'onda quadra con $f = 1000Hz$}
				\pagebreak
				
				Nel caso di un rumore in input, non potendo parlare di frequenza, si sono considerati due intervalli di tempo diversi, dando prima una visione d'insieme del comportamento del clipper per poter poi entrare più nel dettaglio.
				
				\grafico{in-out/altre/noise/20ms}{Grafico input-output di un rumore per $t = 20ms$}
				\graficospace
				\grafico{in-out/altre/noise/2ms}{Grafico input-output di un rumore per $t = 2ms$}
				\pagebreak
				
		
		\subsection{Grafico iterazioni-tempo}
			\label{graphs:iterazioni-tempo}
			Nei seguenti grafici viene visualizzato il numero medio di iterazioni necessarie a calcolare la corrente $V_{out}$ ai capi del condensatore rispetto all'andamento della tensione in entrata $V_{in}$. Questo numero è ottenuto per ogni campione, dalla media delle iterazioni richieste per l'esecuzione del metodo numerico, che è stato testato per tutti i valori di $L \in [0; 50]$, mentre il segnale considerato è sempre sinusoidale.

			\grafico{iterations/media/0,5V}{Grafico iterazioni-tempo con $f = 100Hz$ e $A = 0,5V$}
			\graficospace
			\grafico{iterations/media/1,0V}{Grafico iterazioni-tempo con $f = 100Hz$ e $A = 1V$}
			\graficospace
			\grafico{iterations/media/1,45V}{Grafico iterazioni-tempo con $f = 100Hz$ e $A = 1,45V$}
			\graficospace
			\grafico{iterations/media/1,5V}{Grafico iterazioni-tempo con $f = 100Hz$ e $A = 1,5V$}
			\pagebreak

			In particolare per $L = 0$ possiamo notare come il numero di iterazioni richieste per arrivare ad una soluzione aumenti in modo esponenziale rispetto all'ampiezza massima del segnale.
			\grafico{iterations/L=0/1,4V}{Grafico iterazioni-tempo con $L = 0$ e $A = 1,4V$}
			\graficospace
			\grafico{iterations/L=0/1,45V}{Grafico iterazioni-tempo con $L = 0$ e $A = 1,45V$}
			\graficospace
			\grafico{iterations/L=0/1,5V}{Grafico iterazioni-tempo con $L = 0$ e $A = 1,5V$}
			\graficospace

			Per $L = 50$ notiamo invece che il numero di iterazioni richieste aumenta in modo diverso rispetto al caso precedente. Rispetto ai singoli campioni si ha un esplosione esponenziale quando l'ampiezza del campione raggiunge il picco dell'onda.
			\grafico{iterations/L=50/1,4V}{Grafico iterazioni-tempo con $L = 50$ e $A = 1,4V$}
			\graficospace
			\grafico{iterations/L=50/1,45V}{Grafico iterazioni-tempo con $L = 50$ e $A = 1,45V$}
			\graficospace
			\grafico{iterations/L=50/1,5V}{Grafico iterazioni-tempo con $L = 50$ e $A = 1,5V$}
			
			Dai grafici si può notare che in base all'ampiezza dell'onda sinusoidale in ingresso e quindi in base al valore massimo della $V_{in}$ e anche in base al valore del parametro $L$, le iterazioni richieste dal metodo per convergere variano. In particolare più è alto il modulo della tensione in ingresso, più iterazioni sono richieste dal metodo per convergere: avremo quindi dei picchi di iterazioni in corrispondenza di picchi, sia positivi che negativi, della tensione in ingresso.
			
			Nel caso in cui $A = 1,45V$ si può addirittura notare che per alcuni campioni le iterazioni richieste sono uguali al limite massimo, ovvero $250$ (in prossimità dei picchi del segnale). Se non ci fosse tale limite l'algoritmo iterativo non terminerebbe mai.
			\pagebreak
			
		\subsection{Grafico della matrice jacobiana}
			Approfondiamo adesso il motivo dell'instabilità numerica quando viene superata una certa soglia dell'ampiezza del segnale in ingresso. Come detto nella sezione (\ref{sec:metodo_numerico}) per ottenere la convergenza del metodo bisogna che valga $\lVert\mathbf{J_{f}(x)}\rVert < 1$. Questo significa che la derivata (\ref{eq:jacobiano}), che corrisponde alla matrice jacobiana nel caso monodimensionale, deve valere sempre meno di $1$.
			
			Analizziamo quindi i valori massimi assunti da tale derivata per ogni campione. Noteremo che quando la derivata supera in modulo il valore $1$ allora l'algoritmo smetterà di convergere.
			
			\grafico{jacobiano/100Hz/1,4V}{Grafico della matrice jacobiana di un segnale sinusoidale con $f = 100\;Hz$ e $A = 1,4V$}
			\graficospace
			\grafico{jacobiano/100Hz/1,45V}{Grafico della matrice jacobiana di un segnale sinusoidale con $f = 100\;Hz$ e $A = 1,45V$}
			\graficospace
			\grafico{jacobiano/100Hz/1,48V}{Grafico della matrice jacobiana di un segnale sinusoidale con $f = 100\;Hz$ e $A = 1,48V$}
			\graficospace
			
			\grafico{jacobiano/1000Hz/1,4V}{Grafico della matrice jacobiana di un segnale sinusoidale con $f = 1000\;Hz$ e $A = 1,4V$}
			\graficospace
			\grafico{jacobiano/1000Hz/1,5V}{Grafico della matrice jacobiana di un segnale sinusoidale con $f = 1000\;Hz$ e $A = 1,5V$}
			\graficospace
			\grafico{jacobiano/1000Hz/1,6V}{Grafico della matrice jacobiana di un segnale sinusoidale con $f = 1000\;Hz$ e $A = 1,6V$}
			\graficospace
			
			Dai grafici si può anche osservare che la derivata per i valori di L più piccoli tende a decrescere meno velocemente, fermandosi vicino al valore $-1$, rimanendo poi più stabile nel tempo.
			\pagebreak
			
		\subsection{Grafico iterazioni-valore di L}
			Da come si può già osservare dai grafici del sottoparagrafo precedente c'è una relazione tra il numero di iterazioni richieste per trovare la soluzione dell'equazione differenziale (\ref{eq:discretizzazione}) e il valore del parametro $L$. Si è deciso quindi di indagare questa relazione tra i due parametri, che viene dunque rappresentata nei seguenti grafici (i valori di $L$ per cui l'algoritmo non converge sono stati rimossi).
			
			\grafico{L-iterations/100Hz/1,0V}{Grafico iterazioni-valore di L di un segnale sinusoidale con $f = 100Hz$ e $A = 1,0V$}
			\graficospace
			\grafico{L-iterations/100Hz/1,45V}{Grafico iterazioni-valore di L di un segnale sinusoidale con $f = 100Hz$ e $A = 1,45V$}
			\graficospace
			\grafico{L-iterations/100Hz/1,5V}{Grafico iterazioni-valore di L di un segnale sinusoidale con $f = 100Hz$ e $A = 1,5V$}
			\graficospace
			\grafico{L-iterations/1000Hz/1,0V}{Grafico iterazioni-valore di L di un segnale sinusoidale con $f = 1000Hz$ e $A = 1,0V$}
			\graficospace
			\grafico{L-iterations/1000Hz/1,45V}{Grafico iterazioni-valore di L di un segnale sinusoidale con $f = 1000Hz$ e $A = 1,45V$}
			\graficospace
			\grafico{L-iterations/1000Hz/1,5V}{Grafico iterazioni-valore di L di un segnale sinusoidale con $f = 1000Hz$ e $A = 1,5V$}
			\graficospace
			
			Dai grafici osserviamo che più il valore di $L$ aumenta, più il numero di iterazioni per quel preciso valore di $L$ tende al numero di iterazioni richiesto dall'algoritmo di Newton-Raphson (ovvero quando $L = \infty$). L'andamento generale prevede dunque un decremento del numero di iterazioni richieste all'aumentare del valore di $L$.
			
			Da notare come, per il segnale sinusoidale con $f = 100Hz$, se $A \ge 1,45V$ alcuni valori inizino a variare in modo anomalo, discostandosi significativamente dalla tendenza al valore di Newton-Raphson. Questo è legato alla convergenza dell'algoritmo: valori di iterazioni molto alti rispetto alla tendenza indicano che l'algoritmo richiede molte iterazioni per convergere, ma comunque, anche grazie al limite massimo delle iterazioni, poi arriva ad un risultato. Valori invece molto bassi, ovvero più bassi del numero di iterazioni richieste per $L$ infinito (che sono stati omessi), indicano che, da un certo campione in poi, l'algoritmo non convergerà più. Questo farà sì che il numero di iterazioni richieste dai campioni successivi sia nullo.
			
			\grafico{L-iterations/noise/3,0V}{Grafico iterazioni-valore di L di un rumore con $A = 3,0V$}
			\graficospace
			\grafico{L-iterations/noise/3,5V}{Grafico iterazioni-valore di L di un rumore con $A = 3,5V$}
			\graficospace
			\grafico{L-iterations/noise/4,0V}{Grafico iterazioni-valore di L di un rumore con $A = 4,0V$}
			\graficospace
			
			Se il segnale in input è un rumore possiamo osservare gli stessi fenomeni che si verificano nel caso del segnale sinusoidale. Un'importante differenza è però che l'ampiezza del segnale per la quale l'algoritmo smette di convergere è significativamente maggiore per il rumore rispetto alla sinusoide. Si tratta di più del doppio: circa $3,5V$ contro circa $1,5V$.
			\pagebreak
			
		\subsection{Grafico tempo d'esecuzione-Valore di L}
			\label{subsec:tempo_esecuzione}
			Nei seguenti grafici si può notare la durata dell'esecuzione della simulazione Matlab in base al variare del parametro $L$. Il tempo d'esecuzione mostrato nel grafico è una media dei valori di varie esecuzioni ripetute della simulazione.
			
			\grafico{L-time/sine/1,0V_no_optimization}{Grafico tempo d'esecuzione-Valore di L con $A = 1V$ con calcolo dello jacobiano per $L = 0$}
			\graficospace
			
			Da questo grafico possiamo osservare come l'andamento del tempo d'esecuzione riflette il variare del numero di iterazioni richieste per ogni valore di $L$. In particolare per valori piccoli di $L$, quindi per $L < 4$, il tempo d'esecuzione tenderà a diminuire con l'aumentare del valore di $L$, poiché il numero di iterazioni richieste per arrivare ad un risultato tende a diminuire (vedi \ref{fig:plots/L-iterations/1000Hz/1,0V}). Per $L \ge 4$ invece il tempo aumenta linearmente, seppur di poco, con l'aumentare di $L$. Questo perché si vanno ad aggiungere termini alla sommatoria descritta in (\ref{eq:fixed-point-sum}).
			
			Il grafico appena presentato è stato generato calcolando il valore dello jacobiano anche nel caso base, quindi quando $L = 0$: in questo caso la sommatoria (\ref{eq:fixed-point-sum}), contenuta nella (\ref{eq:punto_fisso2}), vale $1$, poiché
			\[
				\sum_{l=0}^{0} \left(J_{\widehat V_{B}}(V_{B}^{(\lambda)})\right)^{l} = \left(J_{\widehat V_{B}}(V_{B}^{(\lambda)})\right)^{0} = 1
			\]
			
			Sapendo questo è possibile evitare il calcolo di $J_{\widehat V_{B}}(V_{B}^{(\lambda)})$ per $L = 0$ e questo permette di risparmiare molto tempo, come si può osservare dai grafici seguenti.
			
			\grafico{L-time/sine/1,0V}{Grafico tempo d'esecuzione-valore di L di un segnale sinusoidale con $A = 1,0V$}
			\graficospace
			\grafico{L-time/sine/1,2V}{Grafico tempo d'esecuzione-valore di L di un segnale sinusoidale con $A = 1,2V$}
			\graficospace
			\grafico{L-time/sine/1,4V}{Grafico tempo d'esecuzione-valore di L di un segnale sinusoidale con $A = 1,4V$}
			\graficospace
			\grafico{L-time/sine/1,45V}{Grafico tempo d'esecuzione-valore di L di un segnale sinusoidale con $A = 1,45V$}
			\graficospace
			\grafico{L-time/sine/1,5V}{Grafico tempo d'esecuzione-valore di L di un segnale sinusoidale con $A = 1,5V$}
			\graficospace
			
			Anche per valori di $L$ piccoli (i.e. $L \in [1;3]$) il tempo richiesto dal calcolo della soluzione tramite il metodo numerico è molto basso. Il valore massimo del tempo d'esecuzione, ovvero $\approx 2,7ms$, per $L = 1$ all'ampiezza di $1,5V$, permette di poter utilizzare il clipper in applicazioni audio real-time senza avere latenze sensibili.
			
			Si osserva inoltre che, coerentemente ai valori visualizzati nei grafici della sottosezione precedente (\ref{graphs:iterazioni-tempo}), all'aumentare dell'ampiezza del segnale in ingresso, e quindi di $V_{in}$, aumenta il tempo richiesto per ottenere una convergenza del metodo. Questo aumento di tempo è dovuto al fatto che le iterazioni richieste per ottenere la convergenza del metodo aumentano con l'aumentare dell'ampiezza del segnale in ingresso.
			\pagebreak
			
			Nel caso in cui l'entrata sia un rumore, l'andamento del tempo d'esecuzione è lo stesso. Per questo specifico rumore l'algoritmo converge fino ad un valore $A = 4,5V$.
			
			\grafico{L-time/noise/3V}{Grafico tempo d'esecuzione-valore di L di un rumore con $A = 3,0V$}
			\graficospace
			\grafico{L-time/noise/3,5V}{Grafico tempo d'esecuzione-valore di L di un rumore con $A = 3,5V$}
			\graficospace
			\grafico{L-time/noise/4V}{Grafico tempo d'esecuzione-valore di L di un rumore con $A = 4,0V$}
			\graficospace
			\grafico{L-time/noise/4,5V}{Grafico tempo d'esecuzione-valore di L di un rumore con $A = 4,5V$}
			\graficospace
	
	\chapter{Implementazione}
	\label{sec:implementazione}
	Partendo dal codice della simulazione scritto in Matlab è stata implementata una versione dell'algoritmo in C++. L'implementazione è stata scritta come plugin \textit{VST3}, uno fra gli standard più diffusi per la creazione i plugin audio.
	
	Il codice, che si basa su JUCE \cite{juce}, è disponibile nella appendice (\ref{code:cpp}). JUCE è framework cross-platform e open-source particolarmente focalizzato sulle applicazioni audio. La sua popolarità negli ultimi anni è molto cresciuta poiché permette di scrivere delle applicazioni molto avanzate con pochissimo codice. Le prestazioni delle applicazioni sviluppate risultano eccellenti, visto che sia il framework, sia la logica di controllo delle applicazioni sono scritte in C++.
	
	Rispetto alla simulazione non è stata cambiata alcuna logica di funzionamento: il codice è solo stato riadattato per essere utilizzato col paradigma di programmazione orientato agli oggetti. L'unica cosa che il VST presenta in più è una semplicissima interfaccia grafica.
	\pagebreak
	
	\section{Grafica}
		L'interfaccia grafica del plugin VST è composta da tre widget: due checkbox, chiamata toggle button in JUCE e uno slider orizzontale.
		
		\screenshot[scale=0.8]{vst/gui/L_0}{Interfaccia grafica del VST}
		
		Il primo widget serve per attivare o meno il plugin: quando è selezionato l'audio in entrata non viene modificato, quindi il plugin viene bypassato e non viene eseguito alcun calcolo, mentre se è selezionato l'effetto viene attivato e il segnale in ingresso viene processato come mostrato nei grafici input-output (\ref{graphs:input-output}) della simulazione.
		
		Il secondo permette di attivare il metodo Newton-Raphson: se selezionato imposterà il valore della sommatoria (\ref{eq:fixed-point-sum}) a $(\mathbf{I}-\mathbf{J_{f}(x)})^{-1}$ e quindi il valore di $L$ selezionabile tramite lo slider non verrà considerato.
		
		\screenshot[scale=0.8]{vst/gui/L_1_3}{Interfaccia grafica del VST con parametro $L$ a circa 1/3}
		\screenshot[scale=0.8]{vst/gui/L_2_3}{Interfaccia grafica del VST con parametro $L$ a circa 2/3}
		
		Lo slider invece serve ad aggiustare il parametro L: dal valore $0$ (default) si può arrivare fino al valore $50$. Il valore corrente viene mostrato in una tendina che si apre quando si clicca sul cursore dello slider e si chiude quando si rilascia il tasto del mouse.
		
		La scritta visualizzata può risultare non molto intuitiva, ma JUCE permette di appendere del testo solo dopo il valore numerico del parametro.
	
	\section{Prestazioni}
		Mentre la simulazione si presta molto a valutazioni quantitative, l'implementazione si presta molto bene a misurazioni qualitative, sia sulla qualità dell'audio, sia sul tempo di risposta del metodo numerico. Se per esempio i calcoli eseguiti sul segnale audio in ingresso dovessero essere troppo pesanti, e quindi richiedere troppo tempo, l'audio in uscita ne risentirebbe, risultando frammentato oppure presentando dei click.
		
		Questa situazione si potrebbe presentare in modo particolare in contesti real-time, dove il tempo di calcolo deve stare entro determinati limiti: più è breve il tempo di calcolo, più si può ridurre la latenza audio, ovvero il tempo trascorso dall'acquisizione del segnale al momento in cui il segnale viene riprodotto.
		
		È importante dire che per ridurre i tempi di calcolo è necessaria una buona ottimizzazione del codice. Nel codice sorgente presente nelle appendici (\ref{code:matlab}) e (\ref{code:cpp}) sono state apportate alcune semplici ottimizzazioni che però hanno influito notevolmente sulle prestazioni finali. Una di queste è già stata introdotta durante la discussione dei grafici riguardanti il tempo d'esecuzione della simulazione (\ref{subsec:tempo_esecuzione}). Inizialmente infatti per valori relativamente alti del parametro $L$ il singolo core utilizzato dal VST veniva utilizzato al massimo, ma adesso viene richiesto circa il $15\%$ di utilizzo del core singolo, nel caso in cui $L = 50$, con un i7-4770S a $3,10GHz$, a fronte di un $9\%$ di utilizzo se il VST viene disattivato.
		
		Le seguenti schermate possono dare un'idea qualitativa delle risorse richieste dall'implementazione in C++ del clipper. Dalle schermate si può riconoscere la Digital Audio Workstation (DAW) Live di Ableton, nella quale è stato utilizzato il VST.
		
		\begin{figure}
			\subscreenshot{vst/performances/disabled}{CPU utilizzata dal DAW quando il clipper non viene utilizzato}
			\subscreenshot{vst/performances/not_playing}{CPU utilizzata dal clipper quando non c'è alcun segnale in input}
			\graficospace
			
			\subscreenshot{vst/performances/L=0}{CPU utilizzata dal clipper quando\\$L = 0$}
			\subscreenshot{vst/performances/L=1}{CPU utilizzata dal clipper quando\\$L = 1$}
		\end{figure}
			
		\begin{figure}\ContinuedFloat
			\subscreenshot{vst/performances/L=25}{CPU utilizzata dal clipper quando\\$L = 25$}
			\subscreenshot{vst/performances/L=50}{CPU utilizzata dal clipper quando\\$L = 50$}
			\graficospace
			
			\subscreenshot{vst/performances/newton_raphson}{CPU utilizzata dal clipper quando è attivato il metodo Newton-Raphson}
			
			\caption{Performance del clipper}
			\label{fig:performance}
		\end{figure}
		\pagebreak
		
		Le immagini evidenziano una leggera differenza rispetto ai grafici della sezione (\ref{subsec:tempo_esecuzione}). Infatti ci si aspettava un valore di utilizzo della CPU maggiore per $L = 1$, che però risulta uguale al caso in cui si utilizzi il metodo Newton-Raphson. Questo probabilmente perché in un contesto real-time, anche se diminuiscono le iterazioni eseguite dall'algoritmo, il tempo richiesto dagli altri calcoli, come per esempio il calcolo dello jacobiano, diventa predominante.
		
		Un'ulteriore misura qualitativa delle prestazioni consiste nel capire quante istanze del VST possono essere eseguite contemporaneamente, richieste per esempio in un contesto di produzione musicale o di missaggio del suono. Nella seguente schermata si può vedere che 4 istanze contemporanee del VST richiedono poco meno del $20\%$ di CPU. Difficilmente si supererà tale numero di istanze in un progetto musicale.
		
		\screenshot[width=\textwidth]{vst/four_instances}{Istanze multiple del VST}
		
		In questo caso è stato impostato il parametro $L = 0$, ma da come si può dedurre da quanto visto precedentemente più il parametro $L$ è alto, più risorse saranno richieste.
		
		
	
	\chapter{Circuito analogico}
	Per confermare i risultati ottenuti si è deciso poi di realizzare il circuito analogico del clipper analizzato. Il circuito, che segue lo schema (\ref{fig:clipper}), è mostrato nelle figure (\ref{fig:analog/circuito}) e (\ref{fig:analog/circuito2}).
	
	\screenshot[width=\textwidth]{analog/circuito}{Circuito analogico del clipper}
	\screenshot[width=\textwidth]{analog/circuito2}{Circuito analogico del clipper}
	
	In particolare nel circuito sono stati utilizzati $4$ diodi $1N4148$ e un condensatore a disco non polarizzato.
	
	Per generare il segnale è stato utilizzato un generatore di funzioni, mentre per visualizzare il segnale in uscita è stato usato un oscilloscopio digitale. Il cavo di bypass serve per visualizzare sull'oscilloscopio l'onda originale. Nell'immagine (\ref{fig:analog/banco}) si può vedere la strumentazione coinvolta e i collegamenti tra i vari strumenti.
	
	In figura (\ref{fig:analog/generatore_funzioni}) è mostrato il generatore di funzioni utilizzato. La frequenza in questo caso è impostata a $100Hz$, mentre l'ampiezza è pari a $1,5V$ e l'onda è di tipo sinusoidale. Da notare che l'ampiezza visualizzata nello schermo del generatore di funzioni è l'ampiezza picco-picco, che indica quindi la differenza di tensione tra un picco positivo dell'onda e un picco negativo. Nel nostro caso, visto che l'onda è verticalmente centrata, per passare dall'ampiezza picco-picco all'ampiezza di picco, quindi quella usata finora, basta semplicemente dividere per $2$.
	
	\screenshot[width=\textwidth]{analog/banco}{Banco di lavoro}
	\screenshot[width=\textwidth]{analog/generatore_funzioni}{Generatore di funzioni}
	
	\section{Risultati}
		Il circuito è stato provato con delle onde sinusoidali, tutte a frequenza di $1000Hz$ e con ampiezza variabile. Di seguito sono riportate delle schermate dell'oscilloscopio: il segnale di \textcolor{red!90!black}{input}, quindi ciò che viene generato dal generatore di funzioni, è riportato in \textcolor{red!90!black}{rosso}, mentre l'\textcolor{yellow!90!black}{output}, quindi il segnale processato dal clipper, è visualizzato in \textcolor{yellow!90!black}{giallo}.
				
		\screenshot[width=\textwidth]{analog/screens/1,0V}{Input e output del clipper analogico con $V = 1V$}
		\pagebreak
		\vspace*{-30px}
		\screenshot[width=\textwidth]{analog/screens/1,5V}{Input e output del clipper analogico con $V = 1,5V$}
		\screenshot[width=\textwidth]{analog/screens/2,0V}{Input e output del clipper analogico con $V = 2V$}
		\pagebreak


	\chapter{Conclusioni}
	
	
	\appendix
	
	\chapter{Codice Matlab}
	\label{code:matlab}
	Il codice Matlab si compone di 4U file:
	
	\noindent\hspace{5px}\begin{tabularx}{\textwidth}{r X}
		\textbf{clipper.m}			&dove viene generata l'onda sinusoidale passata in input al clipper e dove si possono aggiustare i vari parametri, come le caratteristiche dei vari componenti o il valore di L\\
		\textbf{process.m}			&dove vengono processati i singoli campioni del segnale e dove viene richiamato anche il metodo numerico\\
		\textbf{fixed\_point.m}		&dove è implementato il metodo numerico a punto fisso geometrico\\
		\textbf{generator.m}			&dove che contiene la funzione che genera le onde dei segnali d'ingresso (non riportato in questo documento)
	\end{tabularx}\\
	
	\matlabcode{clipper.m}{clipper.m}
	\matlabcode{process.m}{process.m}
	\matlabcode{fixed_point.m}{fixed\_point.m}
	
	\chapter{Codice C++}
	\label{code:cpp}
	%\cppcode{PluginEditor.h}
	%\cppcode{PluginEditor.cpp}
	%\cppcode{PluginProcessor.h}
	%\cppcode{PluginProcessor.cpp}
	\cppcode{Clipper.hpp}
	\cppcode{Clipper.cpp}
	
	\backmatter
	
	\printbibliography

\end{document}