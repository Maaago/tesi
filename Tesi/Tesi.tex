\documentclass[12pt]{article}

%%%%%%%%%%%%%%%%%%%%%%%%%%%%%%%%%%%%%%% Header %%%%%%%%%%%%%%%%%%%%%%%%%%%%%%%%%%%%%%%%
%%%%%%%%%%%%%%%%%%%%%%%%%%%%%%%%%%%%%% Packages %%%%%%%%%%%%%%%%%%%%%%%%%%%%%%%%%%%%%%%
\usepackage[italian]{babel}
\usepackage[utf8]{inputenc}

%\usepackage{indentfirst}		%per indentare anche il primo paragrafo

\usepackage{amsmath}
\usepackage{esdiff}
\usepackage{cancel}
\usepackage{circuitikz}
\usepackage{comment}
\usepackage{bigints}

\usepackage{hyperref}

\usepackage{xcolor}
\usepackage{graphicx}
\usepackage{subcaption}
\usepackage{rotating}
\usepackage{float}
\usepackage{titlesec}

%%%%%%%%%%%%%%%%%%%%%%%%%%%%%%%%%%%%%%% Configs %%%%%%%%%%%%%%%%%%%%%%%%%%%%%%%%%%%%%%%
\title{Tesi}
\author{Francesco Magoga}
\date{\today}

\setcounter{secnumdepth}{4}

\titleformat{\paragraph}
{\normalfont\normalsize\bfseries}{\theparagraph}{1em}{}
\titlespacing*{\paragraph}
{0pt}{3.25ex plus 1ex minus .2ex}{1.5ex plus .2ex}

\def\arraystretch{1.5}

%%%%%%%%%%%%%%%%%%%%%%%%%%%%%%%%%%% Custom commands %%%%%%%%%%%%%%%%%%%%%%%%%%%%%%%%%%%
\DeclareMathOperator{\arcsinh}{arcsinh}

\newcommand{\includesketch}[2][1=]{\setlength{\fboxsep}{0pt}\setlength{\fboxrule}{1pt}\fbox{\includegraphics[#1]{#2}}}
\newcommand{\Int}[2]{\bigint \hspace{-6px} #1 \mathrm{d}#2}

\makeatletter
\newcommand{\reallybig}{\bBigg@{4}}
\newcommand{\ReallyBig}{\bBigg@{5}}
\makeatother
%%%%%%%%%%%%%%%%%%%%%%%%%%%%%%%%%%%%%%% Header %%%%%%%%%%%%%%%%%%%%%%%%%%%%%%%%%%%%%%%%


%%%%%%%%%%%%%%%%%%%%%%%%%%%%%%%%%%%%%% Document %%%%%%%%%%%%%%%%%%%%%%%%%%%%%%%%%%%%%%%
\begin{document}
	\maketitle
	\pagebreak
	
	\tableofcontents
	\pagebreak

	%\section{Introduzione}
		%Si spiega il problema da affrontare
	
	\section{}
		Circuito\\
		\begin{circuitikz}[american voltages]%[scale = 'some_scale_factor', transform shape]
			\draw
				% Maglia esterna
				(-1,7) to[sinusoidal voltage source,l_=$V_{in}$] (-1,0)	% segnale d'ingresso
				(-1,7) to[resistor=$R_{in}$] (3,7)						% resistenza
				to[short, i^=$i_{in}$] (4,7)								% iin
				to[short, i^=$i_{out}$] (5,7)							% iout
				-- (8,7)													% filo in alto
				to[C, l_=$C$] (8,0)										% condensatore
				-- (-1,0)												% filo in basso
				
				% Diodi
				(4,7) to[short, *-*] (4,6)			% connessione tra la maglia esterna e i diodi A
				(4,7) to[short, i^=$i_{D}$] (4,6)	% iD
				(5,6) -- (3,6)						% connessione in alto tra i diodi A
				to[empty diode, l=$D_{A}$] (3,4)		% diodo A di sinistra
				-- (5,4)								% connessione in basso tra i diodi A
				to[empty diode, l_=$D_{A}$] (5,6)	% diodo A di destra
				(4,4) to[short, *-*] (4,3)			% connessione i diodi A e i diodi B
				(5,3) -- (3,3)						% connessione in alto tra i diodi B
				to[full diode, l=$D_{B}$] (3,1)		% diodo B di sinistra
				-- (5,1)								% connessione in basso tra i diodi B
				to[full diode, l_=$D_{B}$] (5,3)		% diodo B di destra
				(4,1) to[short, *-*] (4,0)			% connessione i diodi B e la maglia esterna
				
				% Etichette
				(-1,2.5) to[open, v_<=$$, outer sep = 2mm] (-1,4.5)			% segno di Vin
				(3,6) to[open, v_=$V_{A}$, outer sep = 5mm] (3,4)			% tensione dei diodi A
				(3,3) to[open, v_=$V_{B}$, outer sep = 5mm] (3,1)			% tensione dei diodi B
				(8,7) to[open, v^=$V_{out}$, outer sep = 6mm] (8,0)			% tensione Vout
				;
		\end{circuitikz}\\
		
		$i_{D_{A}} = \beta_{A}\left(e^{\alpha_{A}V_{A}(t)}-1\right)$
		
		$i_{D_{B}} = \beta_{B}\left(e^{\alpha_{B}V_{B}(t)}-1\right)$
		
		$i_{D} = \beta_{A}\left(e^{\alpha_{A}V_{A}(t)}-1\right)-\beta_{A}\left(e^{-\alpha_{A}V_{A}(t)}-1\right) = \beta_{B}\left(e^{\alpha_{B}V_{B}(t)}-1\right)-\beta_{B}\left(e^{-\alpha_{B}V_{B}(t)}-1\right)$
		\pagebreak
		
		Tensione $V_{A}$
		\begin{equation}
			\label{eq:corrente}
			\begin{split}
				\beta_{A}\left(e^{\alpha_{A}V_{A}(t)}-1\right)-\beta_{A}\left(e^{-\alpha_{A}V_{A}(t)}-1\right) &= \beta_{B}\left(e^{\alpha_{B}V_{B}(t)}-1\right)-\beta_{B}\left(e^{-\alpha_{B}V_{B}(t)}-1\right)\\
				\beta_{A}\left(e^{\alpha_{A}V_{A}(t)}-e^{-\alpha_{A}V_{A}(t)}\right) &= \beta_{B}\left(e^{\alpha_{B}V_{B}(t)}-e^{-\alpha_{B}V_{B}(t)}\right)\\
				\cancel{2}\beta_{A}\left(\frac{e^{\alpha_{A}V_{A}(t)}-e^{-\alpha_{A}V_{A}(t)}}{2}\right) &= \cancel{2}\beta_{B}\left(\frac{e^{\alpha_{B}V_{B}(t)}-e^{-\alpha_{B}V_{B}(t)}}{2}\right)\\
				\beta_{A} \sinh(\alpha_{A}V_{A}(t)) &= \beta_{B} \sinh(\alpha_{B}V_{B}(t))\\
				\sinh(\alpha_{A}V_{A}(t)) &= \frac{\beta_{B}}{\beta_{A}} \sinh(\alpha_{B}V_{B}(t))\\
				\alpha_{A}V_{A}(t) &= \arcsinh \left(\frac{\beta_{B}}{\beta_{A}} \sinh(\alpha_{B}V_{B}(t))\right)\\
				V_{A}(t) &= \frac{1}{\alpha_{A}} \arcsinh \left(\frac{\beta_{B}}{\beta_{A}} \sinh(\alpha_{B}V_{B}(t))\right)
			\end{split}
		\end{equation}
		
		Equazione differenziale
		\begin{equation}
			\label{eq:eqdiff}
			\begin{split}
				V_{out}(t) &= V_{in}(t)-R_{in}\left[i_{out}+i_{D}\right]\\
				V_{out}(t) &= V_{in}(t)-R_{in}\left[C\diff{V_{out}(t)}{t}+i_{D}\right]\\
				V_{out}(t) &= V_{in}(t)-R_{in}\left[C\diff{V_{A}(t)}{t}+C\diff{V_{B}(t)}{t}+i_{D}\right]\\
				V_{A}(t)+V_{B}(t) &= V_{in}(t)-R_{in}\left[C\diff{V_{A}(t)}{t}+C\diff{V_{B}(t)}{t}+\beta_{B}\left(e^{\alpha_{B}V_{B}(t)}-e^{-\alpha_{B}V_{B}(t)}\right)\right]\\
				V_{A}(t)+V_{B}(t) &= V_{in}(t)-R_{in}\left[C\diff{V_{A}(t)}{t}+C\diff{V_{B}(t)}{t}+2\beta_{B}\sinh(\alpha_{B}V_{B}(t))\right]\\
				\frac{V_{A}(t)+V_{B}(t)}{R_{in}} &= \frac{V_{in}(t)}{R_{in}}-C\diff{V_{A}(t)}{t}-C\diff{V_{B}(t)}{t}-2\beta_{B}\sinh(\alpha_{B}V_{B}(t))\\
				%C\diff{V_{B}(t)}{t} &= \frac{V_{in}(t)-V_{A}(t)-V_{B}(t)}{R_{in}}-C\diff{V_{A}(t)}{t}-2\beta_{B}\sinh(\alpha_{B}V_{B}(t))\\
				%\diff{V_{B}(t)}{t} &= \frac{1}{C}\left(\frac{V_{in}(t)-V_{A}(t)-V_{B}(t)}{R_{in}}-C\diff{V_{A}(t)}{t}-2\beta_{B}\sinh(\alpha_{B}V_{B}(t))\right)\\
				%V_{B}(t) &= \frac{1}{C}\Int{\left(\frac{V_{in}(t)-V_{A}(t)-V_{B}(t)}{R_{in}}-C\diff{V_{A}(t)}{t}-2\beta_{B}\sinh(\alpha_{B}V_{B}(t))\right)}{t}\\
				C\diff{V_{A}(t)}{t}+C\diff{V_{B}(t)}{t} &= \frac{V_{in}(t)-V_{A}(t)-V_{B}(t)}{R_{in}}-2\beta_{B}\sinh(\alpha_{B}V_{B}(t))\\
				\diff{V_{A}(t)}{t}+\diff{V_{B}(t)}{t} &= \frac{1}{C}\left(\frac{V_{in}(t)-V_{A}(t)-V_{B}(t)}{R_{in}}-2\beta_{B}\sinh(\alpha_{B}V_{B}(t))\right)\\
			\end{split}
		\end{equation}
		
		Derivo $V_{a}$
		\begin{equation}
			\label{eq:derivata}
			\begin{split}
				\diff{V_{a}(t)}{t} &= \diff{}{t}\left(\frac{1}{\alpha_{A}} \arcsinh \left(\frac{\beta_{B}}{\beta_{A}} \sinh(\alpha_{B}V_{B}(t))\right)\hspace{-5px}\right)\\
				&= \frac{1}{\alpha_{A}}\cdot\diff{}{t}\left(\arcsin\left(\frac{\beta_{B}}{\beta_{A}}\sinh(\alpha_{B}V_{B}(t))\right)\hspace{-5px}\right)\\
				&= \frac{\diff{}{t}\left(\frac{\beta_{B}}{\beta_{A}}\sinh(\alpha_{B}V_{B}(t))\right)}{\alpha_{A}\sqrt{1+\left(\frac{\beta_{B}}{\beta_{A}}\sinh(\alpha_{B}V_{B}(t))\right)^{2}}}\\
				&= \frac{\beta_{B}}{\beta_{A}}\cdot\frac{\diff{}{t}(\sinh(\alpha_{B}V_{B}(t)))}{\alpha_{A}\sqrt{1+\left(\sinh(\alpha_{B}V_{B}(t))\right)^{2}}}\\
				&= \frac{\beta_{B}\cosh(\alpha_{B}V_{B}(t))\diff{}{t}(\alpha_{B}V_{B}(t))}{\alpha_{A}\beta_{A}\sqrt{1+\left(\frac{\beta_{B}}{\beta_{A}}\sinh(\alpha_{B}V_{B}(t))\right)^{2}}}\\
				&= \frac{\alpha_{B}\beta_{B}\cosh(\alpha_{B}V_{B}(t))}{\alpha_{A}\beta_{A}\sqrt{1+\left(\frac{\beta_{B}}{\beta_{A}}\sinh(\alpha_{B}V_{B}(t))\right)^{2}}}\cdot\diff{V_{B}(t)}{t}
			\end{split}
		\end{equation}
		
		Sostituisco $V_{B}$ e $\diff{V_{B}}{t}$ nella (\ref{eq:eqdiff})
		\begin{comment}
		\begin{equation}
			\label{eq:sostituzione}
			\begin{split}
				V_{B}(t) &= \frac{1}{C}\Int{\left(\frac{V_{in}(t)-V_{A}(t)-V_{B}(t)}{R_{in}}-C\diff{V_{A}(t)}{t}-2\beta_{B}\sinh(\alpha_{B}V_{B}(t))\right)}{t}\\
				&= \frac{1}{C}\Int{\bigg(\frac{1}{R_{in}}\left(V_{in}(t)-\frac{1}{\alpha_{A}} \arcsinh \left(\frac{\beta_{B}}{\beta_{A}} \sinh(\alpha_{B}V_{B}(t))\right)-V_{B}(t)\right)-\\&-C\frac{\alpha_{B}\beta_{B}\cosh(\alpha_{B}V_{B}(t))\diff{}{t}(V_{B}(t))}{\alpha_{A}\beta_{A}\sqrt{1+\left(\frac{\beta_{B}}{\beta_{A}}\sinh(\alpha_{B}V_{B}(t))\right)^{2}}}-2\beta_{B}\sinh(\alpha_{B}V_{B}(t))\bigg)}{t}\\
			\end{split}
		\end{equation}
		\end{comment}
		\begin{equation}
			\label{eq:sostituzione}
			\begin{split}
				&\frac{\alpha_{B}\beta_{B}\cosh(\alpha_{B}V_{B}(t))}{\alpha_{A}\beta_{A}\sqrt{1+\left(\frac{\beta_{B}}{\beta_{A}}\sinh(\alpha_{B}V_{B}(t))\right)^{2}}}\cdot\diff{V_{B}(t)}{t}+\diff{V_{B}(t)}{t} =\\&= \frac{1}{C}\left(\frac{1}{R_{in}}\left(V_{in}(t)-\frac{1}{\alpha_{A}} \arcsinh \left(\frac{\beta_{B}}{\beta_{A}} \sinh(\alpha_{B}V_{B}(t))\right)-V_{B}(t)\right)-2\beta_{B}\sinh(\alpha_{B}V_{B}(t))\right)\\
				&\diff{V_{B}(t)}{t}\left(\frac{\alpha_{B}\beta_{B}\cosh(\alpha_{B}V_{B}(t))}{\alpha_{A}\beta_{A}\sqrt{1+\left(\frac{\beta_{B}}{\beta_{A}}\sinh(\alpha_{B}V_{B}(t))\right)^{2}}}+1\right) =\\&= \frac{1}{C}\left(\frac{1}{R_{in}}\left(V_{in}(t)-\frac{1}{\alpha_{A}} \arcsinh \left(\frac{\beta_{B}}{\beta_{A}} \sinh(\alpha_{B}V_{B}(t))\right)-V_{B}(t)\right)-2\beta_{B}\sinh(\alpha_{B}V_{B}(t))\right)\\
				&\diff{V_{B}(t)}{t}\left(\frac{\alpha_{B}\beta_{B}\cosh(\alpha_{B}V_{B}(t))+\alpha_{A}\beta_{A}\sqrt{1+\left(\frac{\beta_{B}}{\beta_{A}}\sinh(\alpha_{B}V_{B}(t))\right)^{2}}}{\alpha_{A}\beta_{A}\sqrt{1+\left(\frac{\beta_{B}}{\beta_{A}}\sinh(\alpha_{B}V_{B}(t))\right)^{2}}}\right) =\\&= \frac{1}{C}\left(\frac{1}{R_{in}}\left(V_{in}(t)-\frac{1}{\alpha_{A}} \arcsinh \left(\frac{\beta_{B}}{\beta_{A}} \sinh(\alpha_{B}V_{B}(t))\right)-V_{B}(t)\right)-2\beta_{B}\sinh(\alpha_{B}V_{B}(t))\right)\\
				&\diff{V_{B}(t)}{t} = \frac{\alpha_{A}\beta_{A}\sqrt{1+\left(\frac{\beta_{B}}{\beta_{A}}\sinh(\alpha_{B}V_{B}(t))\right)^{2}}}{\alpha_{B}\beta_{B}\cosh(\alpha_{B}V_{B}(t))+\alpha_{A}\beta_{A}\sqrt{1+\left(\frac{\beta_{B}}{\beta_{A}}\sinh(\alpha_{B}V_{B}(t))\right)^{2}}}\cdot\\&\cdot\frac{1}{C}\left(\frac{1}{R_{in}}\left(V_{in}(t)-\frac{1}{\alpha_{A}} \arcsinh \left(\frac{\beta_{B}}{\beta_{A}} \sinh(\alpha_{B}V_{B}(t))\right)-V_{B}(t)\right)-2\beta_{B}\sinh(\alpha_{B}V_{B}(t))\right)\\
				&V_{B}(t) = \frac{\alpha_{A}\beta_{A}}{C}\Int{\frac{\sqrt{1+\left(\frac{\beta_{B}}{\beta_{A}}\sinh(\alpha_{B}V_{B}(t))\right)^{2}}}{\alpha_{B}\beta_{B}\cosh(\alpha_{B}V_{B}(t))+\alpha_{A}\beta_{A}\sqrt{1+\left(\frac{\beta_{B}}{\beta_{A}}\sinh(\alpha_{B}V_{B}(t))\right)^{2}}}\cdot\\&\cdot\left(\frac{1}{R_{in}}\left(V_{in}(t)-\frac{1}{\alpha_{A}} \arcsinh \left(\frac{\beta_{B}}{\beta_{A}} \sinh(\alpha_{B}V_{B}(t))\right)-V_{B}(t)\right)-2\beta_{B}\sinh(\alpha_{B}V_{B}(t))\right)}{t}\\
			\end{split}
		\end{equation}
		
		% Risolvo la differenziale
		Discretizzo la (\ref{eq:sostituzione})\\
		Metodo di Eulero all'indietro: $\diff{y}{x} \sim \frac{y_{n}-y_{n-1}}{h}$ quindi $\frac{y_{n}-y_{n-1}}{h} = f(x_{n},y_{n})$ e si ottiene $y_{n} = y_{n-1}+hf(x_{n},y_{n})$. Sia $T = h$ allora
		\begin{equation}
			\label{eq:discretizzazione}
			\begin{split}
				&\widehat V_{B}[n] = \frac{T}{C}\cdot\frac{\alpha_{A}\beta_{A}\sqrt{1+\left(\frac{\beta_{B}}{\beta_{A}}\sinh(\alpha_{B}\widehat V_{B}[n])\right)^{2}}}{\alpha_{B}\beta_{B}\cosh(\alpha_{B}\widehat V_{B}[n])+\alpha_{A}\beta_{A}\sqrt{1+\left(\frac{\beta_{B}}{\beta_{A}}\sinh(\alpha_{B}\widehat V_{B}[n])\right)^{2}}}\cdot\\&\cdot\left(\frac{1}{R_{in}}\left(V_{in}(t)-\frac{1}{\alpha_{A}} \arcsinh \left(\frac{\beta_{B}}{\beta_{A}} \sinh(\alpha_{B}\widehat V_{B}[n])\right)-\widehat V_{B}[n]\right)-2\beta_{B}\sinh(\alpha_{B}\widehat V_{B}[n])\right)+\\&+\widehat V_{B}[n-1]\\
			\end{split}
		\end{equation}
		
		% Risolvo la funzione ottenuta dalla risoluzione della differenziale
		Algoritmo di punto fisso
		\begin{equation}
			\label{eq:punto_fisso1}
			\begin{split}
				x^{(\lambda+1)} &= x^{(\lambda)}-K(x^{(\lambda)})(x^{(\lambda)}-g(x^{(\lambda)}))\\
				&= x^{(\lambda)}-\sum_{l=0}^{L} \left(J_{f}(x^{(\lambda)})\right)^{l}(x^{(\lambda)}-f(x^{(\lambda)}))\\
			\end{split}
		\end{equation}
		poiché $K^{(L)}(x) = \sum_{l=0}^{L} \left(J_{f}(x)\right)^{l}$\\
		
		Applico l'algoritmo di punto fisso
		\begin{equation}
			\label{eq:punto_fisso2}
			\begin{split}
				x^{(\lambda+1)} = x^{(\lambda)}-\sum_{l=0}^{L} \left(J_{f}(x^{(\lambda)})\right)^{l}(x^{(\lambda)}-f(x^{(\lambda)}))\Rightarrow\\
				\Rightarrow \widetilde V_{B}^{(\lambda+1)} = \widetilde V_{B}^{(\lambda)}-\sum_{l=0}^{L} \left(J_{V_{B}}(\widetilde V_{B}^{(\lambda)})\right)^{l}(\widetilde V_{B}^{(\lambda)}-\widehat V_{B}[\widetilde V_{B}^{(\lambda)}])
			\end{split}
		\end{equation}
		
		Calcolo la matrice jacobiana della funzione $V_{B}$
		\begin{equation}
			\label{eq:jacobiana}
			\begin{split}
				&J_{V_{B}}(t) = \diff{V_{B}(t)}{t} = \frac{\alpha_{A}\beta_{A}}{C}\reallybig(\frac{\sqrt{1+\left(\frac{\beta_{B}}{\beta_{A}}\sinh(\alpha_{B}V_{B}(t))\right)^{2}}}{\alpha_{B}\beta_{B}\cosh(\alpha_{B}V_{B}(t))+\alpha_{A}\beta_{A}\sqrt{1+\left(\frac{\beta_{B}}{\beta_{A}}\sinh(\alpha_{B}V_{B}(t))\right)^{2}}}\cdot\\&\cdot\left(\frac{1}{R_{in}}\left(V_{in}(t)-\frac{1}{\alpha_{A}} \arcsinh \left(\frac{\beta_{B}}{\beta_{A}} \sinh(\alpha_{B}V_{B}(t))\right)-V_{B}(t)\right)-2\beta_{B}\sinh(\alpha_{B}V_{B}(t))\right)\hspace{-7px}\reallybig)
			\end{split}
		\end{equation}
	\pagebreak
\end{document}