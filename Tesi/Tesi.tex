\documentclass[12pt,a4paper,twoside,english,italian]{book}

%%%%%%%%%%%%%%%%%%%%%%%%%%%%%%%%%%%%%%% Header %%%%%%%%%%%%%%%%%%%%%%%%%%%%%%%%%%%%%%%%
%%%%%%%%%%%%%%%%%%%%%%%%%%%%%%%%%%%%%% Packages %%%%%%%%%%%%%%%%%%%%%%%%%%%%%%%%%%%%%%%

\usepackage[italian]{babel}
\usepackage[utf8]{inputenc}


\usepackage{uniudtesi}
\usepackage{siunitx}

\usepackage[nottoc]{tocbibind}
\usepackage{indentfirst}
\usepackage{fancyhdr}
\usepackage{emptypage}

\usepackage{amsmath}
\usepackage{esdiff}
\usepackage{cancel}
\usepackage{circuitikz}
\usepackage{comment}
\usepackage{bigints}

\usepackage{hyperref}

\usepackage{xcolor}
\usepackage{graphicx}
\usepackage{subcaption}
\usepackage{rotating}
\usepackage{float}
\usepackage{wrapfig}
\usepackage{titlesec}
\usepackage{setspace}

\usepackage{listings}
\usepackage{gensymb}

\usepackage[backend=biber,style=alphabetic,sorting=ynt]{biblatex}


%%%%%%%%%%%%%%%%%%%%%%%%%%%%%%%%%%%%%%% Configs %%%%%%%%%%%%%%%%%%%%%%%%%%%%%%%%%%%%%%%
%\title{Tesi}
%\author{Francesco Magoga}
%\date{\today}

\titolo{Clipper audio digitale}
\laureando{Magoga Francesco}
\annoaccademico{2020-2021}

%\setcounter{secnumdepth}{4}

\titleformat{\paragraph}
{\normalfont\normalsize\bfseries}{\theparagraph}{1em}{}
\titlespacing*{\paragraph}
{0pt}{3.25ex plus 1ex minus .2ex}{1.5ex plus .2ex}

\def\arraystretch{1.5}

\addbibresource{bibliography.bib}

\setstretch{1.2}

\definecolor{codebg}{gray}{0.95}
\colorlet{darkred}{red!90!black}

\newsavebox\mypostbreak
\savebox\mypostbreak{\mbox{\ensuremath{\color{red!85!black}\hookrightarrow}\space}}

\lstdefinestyle{code}
{
	%basicstyle=\ttfamily,
	basicstyle=\ttfamily\normalsize,
	backgroundcolor=\color{codebg},
	sensitive=false,
	alsoletter={.},
	xleftmargin=0.5cm,
	belowskip=2px,
	frame=single,
	framesep=5pt,
	framerule=1px,
	gobble=5,
	tabsize=1,
	columns=fullflexible,
	showstringspaces=false,
	numbers=left,
	firstnumber=1,
	numberfirstline=false,
	stepnumber=2,
	numberstyle=\scriptsize\color{black!80},
	numbersep=10pt,
	tabsize=4,
	keepspaces=true,
	breaklines=true,
	%postbreak={\hbox{\textcolor{red}{$\hookrightarrow$}\space}}
	postbreak=\usebox\mypostbreak
}

\lstdefinestyle{matlab}
{
	keywordstyle=\color{blue},
	stringstyle=\color{red},
	commentstyle=\color{green!80!black},
}

\lstdefinestyle{cpp}
{
	%basicstyle=\color{teal}\ttfamily,
	keywordstyle=\color{magenta!75!black},
	directivestyle=\color{brown!80!black},
	stringstyle=\color{red!85!black},
	commentstyle=\color{green!70!black},
}

\lstdefinelanguage{cppl}
{
	language=C++,
	morekeywords={override},
	moredirectives={once, \#ifndef, \#endif, \#if, \#else, \#endif},
	%moredelim=[s][\color{red!85!black}\ttfamily]{<}{>},
}

\newcommand{\sourcecode}[5]{
	%\label{list:#3}
	\lstset{caption={#4},label=code:#3,style=code,#5}
	\lstinputlisting[language=#1]{#2}
}

\newcommand{\matlabcode}[2]{
	\sourcecode{Matlab}{../Matlab/#1}{#1}{#2}{style=matlab}
}

\newcommand{\cppcode}[1]{
	\sourcecode{cppl}{../Juce/Source/#1}{#1}{#1}{style=cpp}
}

%%%%%%%%%%%%%%%%%%%%%%%%%%%%%%%%%%% Custom commands %%%%%%%%%%%%%%%%%%%%%%%%%%%%%%%%%%%
\DeclareMathOperator{\arcsinh}{arcsinh}

\newcommand{\includesketch}[2][1=]{\setlength{\fboxsep}{0pt}\setlength{\fboxrule}{1pt}\fbox{\includegraphics[#1]{#2}}}
\newcommand{\Int}[2]{\bigint \hspace{-6px} #1 \mathrm{d}#2}

\makeatletter
\newcommand{\reallybig}{\bBigg@{4}}
\newcommand{\ReallyBig}{\bBigg@{5}}
\makeatother

\graphicspath{{./imgs/}}

\pagestyle{fancy}
\renewcommand{\chaptermark}[1]{\markboth{#1}{}}
\renewcommand{\sectionmark}[1]{\markright{\thesection\ #1}}
\fancyhf{}
\fancyhead[LE,RO]{\bfseries\thepage}
\fancyhead[LO]{\bfseries\rightmark}
\fancyhead[RE]{\bfseries\leftmark}
\renewcommand{\headrulewidth}{0.5pt}
\renewcommand{\footrulewidth}{0pt}
\setlength{\headheight}{14.5pt}

\setlength{\intextsep}{2pt}

%\facolta{Scienze Matematiche, Fisiche e Naturali} % (default)
\corsodilaureamagistralein{Informatica}
\relatore[Prof.]{Federico Fontana}
%\relatoreDue[Prof.]{Secondo relatore}
%\correlatore{Talaltro dei Tali}
%\correlatoreDue{Secondo Correlatore}
%\dedica{Ai miei genitori\\
%    per non avermi tagliato i viveri} % (facoltativo)
%%%%%%%%%%%%%%%%%%%%%%%%%%%%%%%%%%%%%%% Header %%%%%%%%%%%%%%%%%%%%%%%%%%%%%%%%%%%%%%%%


%%%%%%%%%%%%%%%%%%%%%%%%%%%%%%%%%%%%%% Document %%%%%%%%%%%%%%%%%%%%%%%%%%%%%%%%%%%%%%%
\begin{document}
	\pagestyle{fancy} 
	\frontmatter
	\maketitle	

	\enlargethispage{-1.5\baselineskip}
	\tableofcontents
	%\listoffigures
%	\begingroup
%		\let\clearpage\relax
%		\listoffigures
%		\begingroup
%			\let\clearpage\relax
%			\let\cleardoublepage\relax
%			\par\vspace{2\baselineskip}
			%\lstlistoflistings
%		\endgroup
%		\thispagestyle{empty}
%	\endgroup

	\mainmatter

	\chapter*{Introduzione}
		Il seguente documento tratta dell'implementazione e dell'efficienza di un metodo numerico a punto fisso geometrico per calcolare la soluzione di circuiti non lineari di filtri digitali.
		
		Il metodo numerico proposto permette di passare da un risolutore a punto fisso ad un risolutore Newton-Raphson variando un parametro \texttt{L}.
		
		Per analizzare questo metodo numerico si è deciso di digitalizzare un circuito clipper analogico formato da 4 diodi.
		
	\chapter{Metodo mumerico}
	\label{sec:metodo_numerico}
	Definiamo un vettore di $N$ funzioni non lineari $\mathbf{f(x)} = [f_{1}(\mathbf{x}), \dots , f_{N}(\mathbf{x})]^{T}$ nel vettore sconosciuto $\mathbf{x} = [\mathbf{x}_{1}, \dots , \mathbf{x}_{N}]^{T}$, dove $^T$ indica la trasposizione. Christoffersen \cite{christoffersen} propone una famiglia di risolutori numerici per la computazione a punto fisso di una soluzione $\mathbf{x}^{*} = \mathbf{f(x^{*})}$ del vettore sconosciuto $\mathbf{x}$ attraverso il seguente schema:
	\begin{equation}
		\label{eq:fixed-point}
		\mathbf{x}^{(\lambda+1)} = \mathbf{x}^{(\lambda)}-\mathbf{K}(\mathbf{x}^{(\lambda)})(\mathbf{x}^{(\lambda)}-\mathbf{f}(\mathbf{x}^{(\lambda)}))
	\end{equation}
	dove $\mathbf{x}^{(\lambda)}$ è la $\lambda$-esima iterazione dello schema e $\mathbf{K(x)}$ è una matrice quadrata di dimensione $N \times N$, i quali elementi dipendono da $\mathbf{x}$. Se $\mathbf{K}(\mathbf{x}) = \mathbf{I}$, la matrice identità, allora il metodo si riduce ad un risolutore di punto fisso standard \cite{atkinson}. Se $\mathbf{K}(\mathbf{x}) = (\mathbf{I}-\mathbf{J_{f}(x)})^{-1}$, dove $\mathbf{J}_{f}$ è la matrice Jacobiana di $f$, allora il metodo in particolare diventa un risolutore Newton-Raphson \cite{atkinson}.
		
	Una caratteristica notevole è che il metodo di punto fisso standard e il metodo Newton-Raphson possono essere visti come due casi limite che racchiudono una sotto-famiglia di (\ref{eq:fixed-point}). Consideriamo la somma
	\begin{equation}
		\label{eq:fixed-point-sum}
		\mathbf{K}^{(L)}(\mathbf{x}) = \sum_{l=0}^{L} \left(\mathbf{J_{f}(x)}\right)^{l}
	\end{equation}
	allora $\mathbf{K}^{(0)}(\mathbf{x}) = \mathbf{I}$, per definizione di potenza di zero di una matrice non-nulla; d'altro canto dato $\lVert\mathbf{J_{f}(x)}\rVert < 1$, dove l'operatore $\lVert\,\cdot\,\rVert$ indica la norma euclidea, allora (\ref{eq:fixed-point-sum}) converge a $\mathbf{K}^{(\infty)}(\mathbf{x}) = (\mathbf{I}-\mathbf{J_{f}(x)})^{-1}$. Questo risultato generalizza la convergenza della somma geometrica scalare quando il loro rapporto comune ha una norma più piccola di uno. In altre parole, (\ref{eq:fixed-point}) definisce un risolutore di punto fisso standard se la sommatoria si ferma immediatamente (i.e. a $L = 0$), al contrario essa definisce un risolutore Newton-Raphson se la stessa sommatoria non si ferma mai. Successivamente considereremo i casi $0 < L < \infty$ sotto il vincolo $\lVert\mathbf{J_{f}(x)}\rVert < 1$: chiameremo la corrispondente famiglia risolutori di punto fisso, così come chiameremo il parametro $L$ l'ordine dello specifico risolutore.
	
	\chapter{Clipper audio}
	Per poter analizzare il metodo numerico appena descritto si è deciso di scegliere un circuito analogico da digitalizzare per creare una rete di filtri digitali e poter così eseguire le computazioni necessarie alla risoluzione di questa rete utilizzando proprio il metodo numerico proposto.
	
	\section{Descrizione di un clipper}
		I clipper, detti anche limitatori, sono dei circuiti che tagliano la parte di un'onda di un segnale che supera una certa ampiezza. Le onde sinusoidali che superano di molto questa soglia tenderanno ad assomigliare a delle onde quadre.
		
		\begin{comment}
		\begin{wrapfigure}{r}{0.615\textwidth}
			\begin{circuitikz}[american voltages, scale=0.9, transform shape]
				\draw
					% Maglia esterna
					(0,3) to[sinusoidal voltage source,l_=$V_{in}$] (0,0)		% segnale d'ingresso
					(0,3) to[resistor] (3,3)										% resistenza
					-- (7,3)														% filo in alto
					to[C, l^=$V_{out}$] (7,0)									% condensatore
					-- (0,0)														% filo in basso
					(3,0) node[ground]{}											% messa a terra
			
					% Diodi
					(3, 0) to[full diode, *-*] (3,3)								% diodo B di sinistra
					(5, 3) to[full diode, *-*] (5,0)								% diodo B di destra
					
					% Etichette
					(0,0.75) to[open, v_<=$$, outer sep = 2mm] (0,2.25)			% segno di Vin
					(6.5,2.75) to[open, v^=$$, outer sep = 6mm] (6.5,0.25)		% tensione Vout
					;
			\end{circuitikz}
			\caption{Circuito elettronico di un clipper audio semplice}
			\label{fig:easy_clipper}
		\end{wrapfigure}
		\end{comment}
		\begin{wrapfigure}{r}{0.615\textwidth}
			\begin{circuitikz}[american voltages, scale=0.9, transform shape]
				\draw
					% Maglia esterna
					(0,3) to[sinusoidal voltage source,l_=$V_{in}$] (0,0)		% segnale d'ingresso
					(0,3) to[resistor] (3,3)										% resistenza
					-- (7,3)														% filo in alto
					to[C, l^=$V_{out}$] (7,0)									% condensatore
					-- (0,0)														% filo in basso
					(3,0) node[ground]{}											% messa a terra
			
					% Diodi
					(3, 0) to[full diode, *-*] (3,3)								% diodo B di sinistra
					(5, 3) to[full diode, *-*] (5,0)								% diodo B di destra
					
					% Etichette
					(-0.3,0.65) to[open, v_<=$$, outer sep = 2mm] (-0.3,2.40)			% segno di Vin
					(6.5,2.75) to[open, v^=$$, outer sep = 6mm] (6.5,0.25)		% tensione Vout
					;
			\end{circuitikz}
			\caption{Circuito elettronico di un clipper audio semplice}
			\label{fig:easy_clipper}
		\end{wrapfigure}
		
		Il clipper più semplice, illustrato in figura \ref{fig:easy_clipper}, è composto da un generatore di tensione i serie con una resistenza, un condensatore e due diodi, tutto in parallelo.
		
		La scelta di utilizzare un clipper analogico per verificare l'efficacia del metodo numerico è dovuta al fatto che il clipper è un circuito semplice e molto utilizzato per la ricerca in ambito di effetti audio.
	\pagebreak
	
	\section{Circuito del clipper analogico utilizzato}
		Il clipper analogico che è stato scelto per analizzare l'efficacia del metodo numerico proposto differisce dal clipper classico in quanto in questo nuovo circuito sono presenti due ulteriori diodi.
		
		\vspace{15px}
		\begin{comment}
		\begin{figure}[H]
			\centering
			\begin{circuitikz}[american voltages]
				\draw
				% Maglia esterna
				(-1,7) to[sinusoidal voltage source,l_=$V_{in}$] (-1,0)		% segnale d'ingresso
				(-1,7) to[resistor=$R_{in}$] (3,7)							% resistenza
				to[short, i^=$i_{in}$] (4,7)									% iin
				to[short, i^=$i_{out}$] (5,7)								% iout
				-- (8,7)														% filo in alto
				to[C, l_=$C$] (8,0)											% condensatore
				-- (-1,0)													% filo in basso
				(4,0) node[ground]{}											% messa a terra
				
				% Diodi
				(4,7) to[short, *-*] (4,6)			% connessione tra la maglia esterna e i diodi A
				(4,7) to[short, i^=$i_{D}$] (4,6)	% iD
				(5,6) -- (3,6)						% connessione in alto tra i diodi A
				to[empty diode, l=$D_{A}$] (3,4)		% diodo A di sinistra
				-- (5,4)								% connessione in basso tra i diodi A
				to[empty diode, l_=$D_{A}$] (5,6)	% diodo A di destra
				(4,4) to[short, *-*] (4,3)			% connessione i diodi A e i diodi B
				(5,3) -- (3,3)						% connessione in alto tra i diodi B
				to[full diode, l=$D_{B}$] (3,1)		% diodo B di sinistra
				-- (5,1)								% connessione in basso tra i diodi B
				to[full diode, l_=$D_{B}$] (5,3)		% diodo B di destra
				(4,1) to[short, *-*] (4,0)			% connessione i diodi B e la maglia esterna
				
				% Etichette
				(-1,2.5) to[open, v_<=$$, outer sep = 2mm] (-1,4.5)			% segno di Vin
				(3,6) to[open, v_=$V_{A}$, outer sep = 5mm] (3,4)			% tensione dei diodi A
				(3,3) to[open, v_=$V_{B}$, outer sep = 5mm] (3,1)			% tensione dei diodi B
				(8,7) to[open, v^=$V_{out}$, outer sep = 6mm] (8,0)			% tensione Vout
				;
			\end{circuitikz}
			\caption{Circuito elettronico del clipper audio utilizzato}
			\label{fig:clipper}
		\end{figure}
		\end{comment}
		\begin{figure}[H]
			\centering
			\begin{circuitikz}[american voltages]
				\draw
				% Maglia esterna
				(-1,7) to[sinusoidal voltage source,l_=$V_{in}$] (-1,0)		% segnale d'ingresso
				(-1,7) to[resistor=$R_{in}$] (3,7)							% resistenza
				to[short, i^=$i_{in}$] (4,7)									% iin
				to[short, i^=$i_{out}$] (5,7)								% iout
				-- (8,7)														% filo in alto
				to[C, l_=$C$] (8,0)											% condensatore
				-- (-1,0)													% filo in basso
				(4,0) node[ground]{}											% messa a terra
				
				% Diodi
				(4,7) to[short, *-*] (4,6)			% connessione tra la maglia esterna e i diodi A
				(4,7) to[short, i^=$i_{D}$] (4,6)	% iD
				(5,6) -- (3,6)						% connessione in alto tra i diodi A
				to[empty diode, l=$D_{A}$] (3,4)		% diodo A di sinistra
				-- (5,4)								% connessione in basso tra i diodi A
				to[empty diode, l_=$D_{A}$] (5,6)	% diodo A di destra
				(4,4) to[short, *-*] (4,3)			% connessione i diodi A e i diodi B
				(5,3) -- (3,3)						% connessione in alto tra i diodi B
				to[full diode, l=$D_{B}$] (3,1)		% diodo B di sinistra
				-- (5,1)								% connessione in basso tra i diodi B
				to[full diode, l_=$D_{B}$] (5,3)		% diodo B di destra
				(4,1) to[short, *-*] (4,0)			% connessione i diodi B e la maglia esterna
				
				% Etichette
				(-1.25,2.5) to[open, v_<=$$] (-1.25,4.5)		% segno di Vin
				(2.3,6) to[open, v_=$V_{A}$] (2.3,4)			% tensione dei diodi A
				(2.3,3) to[open, v_=$V_{B}$] (2.3,1)			% tensione dei diodi B
				(9,7) to[open, v^=$V_{out}$] (9,0)			% tensione Vout
				;
			\end{circuitikz}
			\caption{Circuito elettronico del clipper audio utilizzato}
			\label{fig:clipper}
		\end{figure}
		\vspace{10px}
		
		Come possiamo notare dallo schema i diodi sono tra loro a due a due in parallelo e queste due coppie di diodi sono poi collegate in serie tra loro. In particolare la prima coppia di diodi ha caratteristiche diverse dalla seconda coppia. Le intensità dei vari diodi sono descritte dall'equazione di Shockley come segue:
		\[
			i_{D_{A}} = \beta_{A}\left(e^{\alpha_{A}V_{A}(t)}-1\right)
		\]
		\[
			i_{D_{B}} = \beta_{B}\left(e^{\alpha_{B}V_{B}(t)}-1\right)
		\]
		
		Le intensità delle due coppie sono quindi
		\[
			i_{D_{A'}} = \beta_{A}\left(e^{\alpha_{A}V_{A}(t)}-1\right)-\beta_{A}\left(e^{-\alpha_{A}V_{A}(t)}-1\right)
		\]
		per la prima coppia e
		\[
			i_{D_{B'}} = \beta_{B}\left(e^{\alpha_{B}V_{B}(t)}-1\right)-\beta_{B}\left(e^{-\alpha_{B}V_{B}(t)}-1\right)
		\]
		per la seconda.
		
		Essendo le coppie in serie sappiamo che $i_{D} = i_{D_{A'}} = i_{D_{B'}}$ dunque
		\[
			i_{D} = \beta_{A}\left(e^{\alpha_{A}V_{A}(t)}-1\right)-\beta_{A}\left(e^{-\alpha_{A}V_{A}(t)}-1\right) = \beta_{B}\left(e^{\alpha_{B}V_{B}(t)}-1\right)-\beta_{B}\left(e^{-\alpha_{B}V_{B}(t)}-1\right)
		\]
		
		Siamo quindi interessati a trovare il valore di $V_{out}$ fissati i vari parametri e dato il valore di $V_{in}$.
		
		\vspace{30px}
		In figura (\ref{fig:analog/example}) si può vedere un esempio di come un onda risulti prima e dopo l'applicazione del clipper.
		
		\screenshot[width=\textwidth]{analog/example}{Esempio di onda prima e dopo del clipper}
	
	\chapter{Soluzione analitica}
	Il primo passo per trovare il valore di $V_{out}$ ad un istante di tempo $t$ è stato trovare l'equazione differenziale che rappresenta la correlazione tra il valore della tensione in entrata $V_{in}$ e il valore delle tensioni delle due coppie di diodi $V_{A}$ e $V_{B}$.
	
	Per fare ciò si è partiti dalla seconda legge di Kirchhoff applicata al circuito preso in considerazione (figura \ref{fig:clipper}).
	\[
	\begin{cases}
		V_{in}-V_{R} = V_{out}\\
		V_{in}-V_{R} = V_{A}+V_{B}\\
		V_{A}+V_{B} = V_{out}\\
	\end{cases}
	\]
	
	Inoltre applicando la prima legge di Kirchhoff abbiamo
	\[
		i_{in} = i_{out}+i_{D}
	\]
	
	Considerando la definizione di capacità elettrica
	\[
		C = \frac{Q}{\Delta V} \Rightarrow Q = C \cdot \Delta V
	\]
	e che nell'intervallo di tempo infinitesimo $\mathrm{d}t$ l'intensità di corrente del condensatore vale 
	\[
		I = \diff{Q}{t}
	\]
	otteniamo che
	\[
		I = \diff{}{t}(C \cdot \Delta V) = C\diff{\Delta V}{t}
	\]
	\pagebreak
		
	\section{Equazione differenziale}
		Da quanto appena osservato otteniamo
		\begin{equation}
			\label{eq:differenziale}
			\begin{split}
				V_{out}(t) &= V_{in}(t)-V_{R}(t)\\
				V_{out}(t) &= V_{in}(t)-R_{in}i_{in}\\
				V_{out}(t) &= V_{in}(t)-R_{in}\left[i_{out}+i_{D}\right]\\
				V_{out}(t) &= V_{in}(t)-R_{in}\left[C\diff{V_{out}(t)}{t}+i_{D}\right]\\
				V_{out}(t) &= V_{in}(t)-R_{in}\left[C\diff{V_{A}(t)}{t}+C\diff{V_{B}(t)}{t}+i_{D}\right]\\
				V_{A}(t)+V_{B}(t) &= V_{in}(t)-R_{in}\left[C\diff{V_{A}(t)}{t}+C\diff{V_{B}(t)}{t}+\beta_{B}\left(e^{\alpha_{B}V_{B}(t)}-e^{-\alpha_{B}V_{B}(t)}\right)\right]\\
				V_{A}(t)+V_{B}(t) &= V_{in}(t)-R_{in}\left[C\diff{V_{A}(t)}{t}+C\diff{V_{B}(t)}{t}+2\beta_{B}\sinh(\alpha_{B}V_{B}(t))\right]\\
				\frac{V_{A}(t)+V_{B}(t)}{R_{in}} &= \frac{V_{in}(t)}{R_{in}}-C\diff{V_{A}(t)}{t}-C\diff{V_{B}(t)}{t}-2\beta_{B}\sinh(\alpha_{B}V_{B}(t))\\
				C\diff{V_{A}(t)}{t}+C\diff{V_{B}(t)}{t} &= \frac{V_{in}(t)-V_{A}(t)-V_{B}(t)}{R_{in}}-2\beta_{B}\sinh(\alpha_{B}V_{B}(t))\\
				\diff{V_{A}(t)}{t}+\diff{V_{B}(t)}{t} &= \frac{1}{C}\left(\frac{V_{in}(t)-V_{A}(t)-V_{B}(t)}{R_{in}}-2\beta_{B}\sinh(\alpha_{B}V_{B}(t))\right)\\
			\end{split}
		\end{equation}
		che è l'equazione che mette in relazione le tensioni sui vari componenti del circuito.
		\pagebreak
		
		Per poter risolvere l'equazione tramite il metodo numerico dobbiamo esprimere l'equazione differenziale (\ref{eq:differenziale}) appena trovata in funzione di una delle due variabili $V_{A}$ o $V_{B}$, variabile che comparirà sia nel primo, che nel secondo termine della nostra equazione. In questo caso è stato scelto di esprimere tutto in funzione di $V_{B}$, perciò il passo successivo è stato di trovare una formula che esprimesse $V_{A}$ in funzione di $V_{B}$.
		\begin{equation}
			\label{eq:tensione_A_B}
			\begin{split}
				\beta_{A}\left(e^{\alpha_{A}V_{A}(t)}-1\right)-\beta_{A}\left(e^{-\alpha_{A}V_{A}(t)}-1\right) &= \beta_{B}\left(e^{\alpha_{B}V_{B}(t)}-1\right)-\beta_{B}\left(e^{-\alpha_{B}V_{B}(t)}-1\right)\\
				\beta_{A}\left(e^{\alpha_{A}V_{A}(t)}-e^{-\alpha_{A}V_{A}(t)}\right) &= \beta_{B}\left(e^{\alpha_{B}V_{B}(t)}-e^{-\alpha_{B}V_{B}(t)}\right)\\
				\cancel{2}\beta_{A}\left(\frac{e^{\alpha_{A}V_{A}(t)}-e^{-\alpha_{A}V_{A}(t)}}{2}\right) &= \cancel{2}\beta_{B}\left(\frac{e^{\alpha_{B}V_{B}(t)}-e^{-\alpha_{B}V_{B}(t)}}{2}\right)\\
				\beta_{A} \sinh(\alpha_{A}V_{A}(t)) &= \beta_{B} \sinh(\alpha_{B}V_{B}(t))\\
				\sinh(\alpha_{A}V_{A}(t)) &= \frac{\beta_{B}}{\beta_{A}} \sinh(\alpha_{B}V_{B}(t))\\
				\alpha_{A}V_{A}(t) &= \arcsinh \left(\frac{\beta_{B}}{\beta_{A}} \sinh(\alpha_{B}V_{B}(t))\right)\\
				V_{A}(t) &= \frac{1}{\alpha_{A}} \arcsinh \left(\frac{\beta_{B}}{\beta_{A}} \sinh(\alpha_{B}V_{B}(t))\right)
			\end{split}
		\end{equation}
		\pagebreak
		
		Prima di sostituire le occorrenze di $V_{A}$ nell'equazione differenziale (\ref{eq:differenziale}) con la formula appena trovata (\ref{eq:tensione_A_B}) notiamo che nella (\ref{eq:differenziale}) è richiesto di derivare $V_{A}$.
		\begin{equation}
			\label{eq:derivata_va}
			\begin{split}
				\diff{V_{a}(t)}{t} &= \diff{}{t}\left(\frac{1}{\alpha_{A}} \arcsinh \left(\frac{\beta_{B}}{\beta_{A}} \sinh(\alpha_{B}V_{B}(t))\right)\hspace{-5px}\right)\\
				&= \frac{1}{\alpha_{A}}\cdot\diff{}{t}\left(\arcsin\left(\frac{\beta_{B}}{\beta_{A}}\sinh(\alpha_{B}V_{B}(t))\right)\hspace{-5px}\right)\\
				&= \frac{\diff{}{t}\left(\frac{\beta_{B}}{\beta_{A}}\sinh(\alpha_{B}V_{B}(t))\right)}{\alpha_{A}\sqrt{1+\left(\frac{\beta_{B}}{\beta_{A}}\sinh(\alpha_{B}V_{B}(t))\right)^{2}}}\\
				&= \frac{\beta_{B}}{\beta_{A}}\cdot\frac{\diff{}{t}(\sinh(\alpha_{B}V_{B}(t)))}{\alpha_{A}\sqrt{1+\left(\sinh(\alpha_{B}V_{B}(t))\right)^{2}}}\\
				&= \frac{\beta_{B}\cosh(\alpha_{B}V_{B}(t))\diff{}{t}(\alpha_{B}V_{B}(t))}{\alpha_{A}\beta_{A}\sqrt{1+\left(\frac{\beta_{B}}{\beta_{A}}\sinh(\alpha_{B}V_{B}(t))\right)^{2}}}\\
				&= \frac{\alpha_{B}\beta_{B}\cosh(\alpha_{B}V_{B}(t))}{\alpha_{A}\beta_{A}\sqrt{1+\left(\frac{\beta_{B}}{\beta_{A}}\sinh(\alpha_{B}V_{B}(t))\right)^{2}}}\cdot\diff{V_{B}(t)}{t}
			\end{split}
		\end{equation}
		\pagebreak

		Procediamo quindi con la sostituzione di $V_{A}$ e $\diff{V_{A}}{t}$ nell'equazione differenziale (\ref{eq:differenziale}) con quanto trovato nelle equazioni (\ref{eq:tensione_A_B}) e (\ref{eq:derivata_va}).
		\begin{equation}
			\label{eq:integrale}
			\begin{split}
				&\frac{\alpha_{B}\beta_{B}\cosh(\alpha_{B}V_{B}(t))}{\alpha_{A}\beta_{A}\sqrt{1+\left(\frac{\beta_{B}}{\beta_{A}}\sinh(\alpha_{B}V_{B}(t))\right)^{2}}}\cdot\diff{V_{B}(t)}{t}+\diff{V_{B}(t)}{t} =\\&= \frac{1}{C}\left(\frac{1}{R_{in}}\left(V_{in}(t)-\frac{1}{\alpha_{A}} \arcsinh \left(\frac{\beta_{B}}{\beta_{A}} \sinh(\alpha_{B}V_{B}(t))\right)-V_{B}(t)\right)-2\beta_{B}\sinh(\alpha_{B}V_{B}(t))\right)\Rightarrow\\
				&\Rightarrow \diff{V_{B}(t)}{t}\left(\frac{\alpha_{B}\beta_{B}\cosh(\alpha_{B}V_{B}(t))}{\alpha_{A}\beta_{A}\sqrt{1+\left(\frac{\beta_{B}}{\beta_{A}}\sinh(\alpha_{B}V_{B}(t))\right)^{2}}}+1\right) =\\&= \frac{1}{C}\left(\frac{1}{R_{in}}\left(V_{in}(t)-\frac{1}{\alpha_{A}} \arcsinh \left(\frac{\beta_{B}}{\beta_{A}} \sinh(\alpha_{B}V_{B}(t))\right)-V_{B}(t)\right)-2\beta_{B}\sinh(\alpha_{B}V_{B}(t))\right)\Rightarrow\\
				&\Rightarrow \diff{V_{B}(t)}{t} = \left(\frac{\alpha_{B}\beta_{B}\cosh(\alpha_{B}V_{B}(t))}{\alpha_{A}\beta_{A}\sqrt{1+\left(\frac{\beta_{B}}{\beta_{A}}\sinh(\alpha_{B}V_{B}(t))\right)^{2}}}+1\right)^{-1}\cdot\\&\cdot\frac{1}{C}\left(\frac{1}{R_{in}}\left(V_{in}(t)-\frac{1}{\alpha_{A}} \arcsinh \left(\frac{\beta_{B}}{\beta_{A}} \sinh(\alpha_{B}V_{B}(t))\right)-V_{B}(t)\right)-2\beta_{B}\sinh(\alpha_{B}V_{B}(t))\right)\Rightarrow\\
				&\Rightarrow V_{B}(t) = \Int{\Bigg(\left(\frac{\alpha_{B}\beta_{B}\cosh(\alpha_{B}V_{B}(t))}{\alpha_{A}\beta_{A}\sqrt{1+\left(\frac{\beta_{B}}{\beta_{A}}\sinh(\alpha_{B}V_{B}(t))\right)^{2}}}+1\right)^{-1}\cdot\\&\cdot\frac{1}{C}\left(\frac{1}{R_{in}}\left(V_{in}(t)-\frac{1}{\alpha_{A}} \arcsinh \left(\frac{\beta_{B}}{\beta_{A}} \sinh(\alpha_{B}V_{B}(t))\right)-V_{B}(t)\right)-2\beta_{B}\sinh(\alpha_{B}V_{B}(t))\right)\Bigg)}{t}\\
			\end{split}
		\end{equation}
		\pagebreak
		
	\section{Calcolo numerico dell'equazione differenziale}
		Da notare che la precedente equazione (\ref{eq:sostituzione}) è in forma implicita, poiché il termine $V_{B}(t)$ compare sia nel lhs che nel rhs dell'equazione. Per tale motivo l'unico modo per trovare il valore di questa incognita è utilizzare un metodo numerico.
		
		Arrivati a questo punto per risolvere l'equazione differenziale appena ottenuta (\ref{eq:sostituzione}) si è utilizzato il metodo di Eulero implicito (o all'indietro). Si è scelto questo metodo per discretizzare l'equazione da risolvere perché ha una buona stabilità ed è semplice da applicare.
		
		Il metodo di Eulero all'indietro viene ricavato dall'approssimazione della derivata con le differenze finite all'indietro
		\[
			\diff{y}{x} \approx \frac{y_{n}-y_{n-1}}{h}
		\]
		da cui si ottiene
		\[
			\frac{y_{n}-y_{n-1}}{h} = f(x_{n},y_{n}) \implies y_{n} = y_{n-1}+hf(x_{n},y_{n})
		\]
		
		Sappiamo che 
		\begin{equation}
			\label{eq:sostituzione}
			\begin{split}
				&\left(\frac{\alpha_{B}\beta_{B}\cosh(\alpha_{B}V_{B}(t))}{\alpha_{A}\beta_{A}\sqrt{1+\left(\frac{\beta_{B}}{\beta_{A}}\sinh(\alpha_{B}V_{B}(t))\right)^{2}}}+1\right)^{-1} =\\
				&= \left(\frac{\alpha_{B}\beta_{B}\cosh(\alpha_{B}V_{B}(t))+\alpha_{A}\beta_{A}\sqrt{1+\left(\frac{\beta_{B}}{\beta_{A}}\sinh(\alpha_{B}V_{B}(t))\right)^{2}}}{\alpha_{A}\beta_{A}\sqrt{1+\left(\frac{\beta_{B}}{\beta_{A}}\sinh(\alpha_{B}V_{B}(t))\right)^{2}}}\right)^{-1}\\
				&= \frac{\sqrt{1+\left(\frac{\beta_{B}}{\beta_{A}}\sinh(\alpha_{B}V_{B}(t))\right)^{2}}}{\alpha_{B}\beta_{B}\cosh(\alpha_{B}V_{B}(t))+\alpha_{A}\beta_{A}\sqrt{1+\left(\frac{\beta_{B}}{\beta_{A}}\sinh(\alpha_{B}V_{B}(t))\right)^{2}}}
			\end{split}
		\end{equation}
		
		Impostiamo il parametro $h$ in modo che sia equivalente al periodo di campionamento
		\begin{equation}
			\label{eq:valore_h}
			\begin{split}
				h = T = \frac{1}{Fs}
			\end{split}
		\end{equation}
		
		L'equazione discretizzata risulta quindi essere
		\begin{equation}
			\label{eq:discretizzazione}
			\begin{split}
				&\widehat V_{B}[n] = h \; \widehat V_{B}[n]+\widehat V_{B}[n-1] =\\
				&= \frac{T}{C}\cdot\frac{\sqrt{1+\left(\frac{\beta_{B}}{\beta_{A}}\sinh(\alpha_{B}V_{B}[n])\right)^{2}}}{\alpha_{B}\beta_{B}\cosh(\alpha_{B}V_{B}[n])+\alpha_{A}\beta_{A}\sqrt{1+\left(\frac{\beta_{B}}{\beta_{A}}\sinh(\alpha_{B}V_{B}[n])\right)^{2}}}\cdot\\&\cdot\left(\frac{1}{R_{in}}\left(V_{in}(n)-\frac{1}{\alpha_{A}} \arcsinh \left(\frac{\beta_{B}}{\beta_{A}} \sinh(\alpha_{B}\widehat V_{B}[n])\right)-\widehat V_{B}[n]\right)-2\beta_{B}\sinh(\alpha_{B}\widehat V_{B}[n])\right)+\\&+\widehat V_{B}[n-1]\\
			\end{split}
		\end{equation}
		\pagebreak
	
	\section{Applicazione del metodo numerico}
		Richiamiamo ora la formula (\ref{eq:fixed-point}) del metodo numerico da applicare
		\begin{equation}
			\label{eq:punto_fisso1}
			\begin{split}
				\mathbf{x}^{(\lambda+1)} &= \mathbf{x}^{(\lambda)}-\mathbf{K}(\mathbf{x}^{(\lambda)})(\mathbf{x}^{(\lambda)}-\mathbf{f}(\mathbf{x}^{(\lambda)}))\\
				&= \mathbf{x}^{(\lambda)}-\sum_{l=0}^{L} \left(\mathbf{J_{f}(x)}\right)^{l}(\mathbf{x}^{(\lambda)}-\mathbf{f}(\mathbf{x}^{(\lambda)}))\\
			\end{split}
		\end{equation}
		poiché per (\ref{eq:fixed-point-sum}) abbiamo $\mathbf{K}^{(L)}(\mathbf{x}) = \sum_{l=0}^{L} \left(\mathbf{J_{f}(x)}\right)^{l}$\\
		
		Nel nostro caso la formula del metodo numerico verrà usata per calcolare il valore $V_{B}$, utilizzando la funzione discretizzata $f = \widehat V_{B}$. La (\ref{eq:punto_fisso1}) dunque diventa
		\begin{equation}
			\label{eq:punto_fisso2}
			\begin{split}
				V_{B}^{(\lambda+1)} = V_{B}^{(\lambda)}-\sum_{l=0}^{L} \left(J_{\widehat V_{B}}(V_{B}^{(\lambda)})\right)^{l}(V_{B}^{(\lambda)}-\widehat V_{B}[V_{B}^{(\lambda)}])
			\end{split}
		\end{equation}
		
		Il metodo richiede il calcolo del termine $J_{\widehat V_{B}}$, ovvero la matrice jacobiana della funzione $\widehat V_{B}$. La matrice jacobiana nel caso di vettori con un singolo elemento, come nel nostro caso, equivale alla derivata della funzione a cui è associata. In poche parole, in questo caso $J_{\widehat V_{B}}(n) = \diff{\widehat V_{B}[n]}{n}$.\\
		
		Per calcolare tale derivata bisognerà applicare la regola del prodotto, che richiederà di scomporre la funzione $\widehat V_{B}[n] = \frac{h}{C}\cdot f[n]\cdot g[n]$.
		Le due funzioni $f[n]$ e $g[n]$ risulteranno essere
		\begin{equation}
			\begin{split}
				f[n] = \left(\frac{\alpha_{B}\beta_{B}\cosh(\alpha_{B}\widehat V_{B}[n])}{\alpha_{A}\beta_{A}\sqrt{1+\left(\frac{\beta_{B}}{\beta_{A}}\sinh(\alpha_{B}\widehat V_{B}[n])\right)^{2}}}+1\right)^{-1} = \frac{1}{\varphi[n]}
			\end{split}
		\end{equation}
		\begin{equation}
			\begin{split}
				g[n] = \frac{1}{R_{in}}\left(V_{in}(n)-V_{A}[n]-\widehat V_{B}[n]\right)-2\beta_{B}\sinh(\alpha_{B}\widehat V_{B}[n])
			\end{split}
		\end{equation}
		dove si è sostituito il termine della sottrazione con $V_{A}[n]$ seguendo la (\ref{eq:tensione_A_B}).

		La derivata sarà
		\begin{equation}
			\label{eq:jacobiano}
			\begin{split}
				\diff{\widehat V_{B}[n]}{n} &= \frac{h}{C}\left(\diff{f[n]}{n}\cdot g[n]+f[n]\cdot \diff{g[n]}{n}\right)\\
				&= \frac{h}{C}\left(-\diff{\varphi[n]}{n}\cdot\frac{1}{\varphi[n]^{2}}\cdot g[n]+\frac{1}{\varphi[n]}\cdot \diff{g[n]}{n}\right)
			\end{split}
		\end{equation}

		Iniziamo con la derivata della funzione $f[n]$ che abbiamo detto equivalere a
		\begin{equation}
			\label{eq:jacobiano}
			\begin{split}
				\diff{f[n]}{n} = \diff{\frac{1}{\varphi[n]}}{n} = -\diff{\varphi[n]}{n}\cdot\frac{1}{\varphi[n]^{2}}
			\end{split}
		\end{equation}

		
		A questo punto bisogna calcolare la derivata della funzione $\varphi[n]$.
		\begin{equation}
			\begin{split}
				\diff{\varphi[n]}{n} &= \diff{}{n}\left(\frac{\alpha_{B}\beta_{B}\cosh(\alpha_{B}\widehat V_{B}[n])}{\alpha_{A}\beta_{A}\sqrt{1+\left(\frac{\beta_{B}}{\beta_{A}}\sinh(\alpha_{B}\widehat V_{B}[n])\right)^{2}}}+1\right)\\
				&= \frac{\alpha_{B}\beta_{B}}{\alpha_{A}\beta_{A}}\cdot\diff{}{n}\left(\frac{\cosh(\alpha_{B}\widehat V_{B}[n])}{\sqrt{1+\left(\frac{\beta_{B}}{\beta_{A}}\sinh(\alpha_{B}\widehat V_{B}[n])\right)^{2}}}\right)\\
				&= \frac{\alpha_{B}\beta_{B}}{\alpha_{A}\beta_{A}}\cdot\frac{\psi}{1+\left(\frac{\beta_{B}}{\beta_{A}}\sinh(\alpha_{B}\widehat V_{B}[n])\right)^{2}}
			\end{split}
		\end{equation}
		
		dove $\psi[n]$ è definita come segue
		\begin{equation}
			\begin{split}
				\psi[n] = \alpha_{B}\sinh(\alpha_{B}\widehat V_{B}[n])\cdot\sqrt{1+\left(\frac{\beta_{B}}{\beta_{A}}\sinh(\alpha_{B}\widehat V_{B}[n])\right)^{2}}-\cosh(\alpha_{B}\widehat V_{B}[n])\,\cdot\\
				\cdot\,\diff{}{n}\left(\sqrt{1+\left(\frac{\beta_{B}}{\beta_{A}}\sinh(\alpha_{B}\widehat V_{B}[n])\right)^{2}}\right)
			\end{split}
		\end{equation}
		\pagebreak
		
		Calcoliamo infine l'ultima derivata
		\vspace{-10px}
		\begin{equation}
			\begin{split}
				\diff{}{n}\left(\sqrt{1+\left(\frac{\beta_{B}}{\beta_{A}}\sinh(\alpha_{B}\widehat V_{B}[n])\right)^{2}}\right) &= \frac{\diff{}{n}\left(1+\left(\frac{\beta_{B}}{\beta_{A}}\sinh(\alpha_{B}\widehat V_{B}[n])\right)^{2}\right)}{2\sqrt{1+\left(\frac{\beta_{B}}{\beta_{A}}\sinh(\alpha_{B}\widehat V_{B}[n])\right)^{2}}}\\
				&= \frac{2\left(\frac{\beta_{B}}{\beta_{A}}\sinh(\alpha_{B}\widehat V_{B}[n])\right)\frac{\beta_{B}}{\beta_{A}}\diff{}{n}\left(\sinh(\alpha_{B}\widehat V_{B}[n])\right)}{2\sqrt{1+\left(\frac{\beta_{B}}{\beta_{A}}\sinh(\alpha_{B}\widehat V_{B}[n])\right)^{2}}}\\
				&= \frac{\alpha_{B}\beta_{B}^{2}}{\beta_{A}^{2}}\cdot\frac{\sinh(\alpha_{B}\widehat V_{B}[n])\cosh(\alpha_{B}\widehat V_{B}[n])}{\sqrt{1+\left(\frac{\beta_{B}}{\beta_{A}}\sinh(\alpha_{B}\widehat V_{B}[n])\right)^{2}}}
			\end{split}
		\end{equation}

		Passiamo ora alla derivata della funzione $g[n]$, che equivale a
		\begin{equation}
			\begin{split}
				\diff{g[n]}{n} &= \frac{1}{R_{in}}\cdot\diff{\left(V_{in}(n)-V_{A}[n]-V_{B}[n]\right)}{n}-\diff{\left(2\beta_{B}\sinh(\alpha_{B}\widehat V_{B}[n])\right)}{n} =\\
				&= \frac{1}{R_{in}}\left(\diff{V_{in}(n)}{n}-\diff{V_{A}[n]}{n}-\diff{V_{B}[n]}{n}\right)-2\alpha_{B}\beta_{B}\cosh(\alpha_{B}\widehat V_{B}[n]) =\\
				&= \frac{1}{R_{in}}\left(-\frac{\alpha_{B}\beta_{B}\cosh(\alpha_{B}\widehat V_{B}[n])}{\alpha_{A}\beta_{A}\sqrt{1+\left(\frac{\beta_{B}}{\beta_{A}}\sinh(\alpha_{B}\widehat V_{B}[n])\right)^{2}}}-1\right)-2\alpha_{B}\beta_{B}\cosh(\alpha_{B}\widehat V_{B}[n])
			\end{split}
		\end{equation}
		\vspace{-8px}
		
		\noindent dove $\diff{V_{A}[n]}{n}$ è data dalla (\ref{eq:derivata_va}).\\
		
		Dopo aver calcolato numericamente il valore di $V_{B}$ tramite la formula (\ref{eq:punto_fisso2}), si può calcolare la corrente ai capi del condensatore, ovvero $V_{out}$, utilizzando sempre la seconda legge di Kirchhoff:
		\begin{equation}
			\label{eq:tensione_finale}
			\begin{split}
				V_{out}[n] &= V_{A}[n]+V_{B}[n]\\
				&= \frac{1}{\alpha_{A}} \arcsinh \left(\frac{\beta_{B}}{\beta_{A}} \sinh(\alpha_{B}V_{B}[n])\right)+V_{B}[n]
			\end{split}
		\end{equation}
%\flushbottom
	
	\appendix
	
	%\chapter{Codice Matlab}
	\label{code:matlab}
	\matlabcode{tesi.m}{tesi.m}
	\matlabcode{fixed_point.m}{fixed\_point.m}
	
	%\chapter{Codice C++}
	\label{code:cpp}
	%\cppcode{PluginEditor.h}
	%\cppcode{PluginEditor.cpp}
	%\cppcode{PluginProcessor.h}
	%\cppcode{PluginProcessor.cpp}
	\cppcode{Clipper.hpp}
	\cppcode{Clipper.cpp}
	
	\backmatter
	
	\printbibliography
\end{document}