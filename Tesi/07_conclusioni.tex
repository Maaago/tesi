\chapter{Conclusioni}
	Il comportamento del clipper è già stato messo in evidenza nei grafici della sezione (\ref{sec:implementazione}). Questi grafici ci danno un'idea del segnale uscente generato dal clipper in una situazione ideale.
	
	\section{Segnale di rumore}
		Per avere un'idea più realistica si è poi deciso di analizzare il comportamento del clipper quando in entrata viene fornito un segnale di rumore bianco. Questo segnale simula una composizione di più onde, così come un suono è composto da più armoniche e una canzone è composta da più suoni.
	
		Il rumore è stato generato casualmente. Dato che la forma dell'onda in input influisce sia sul segnale di output e che sulla convergenza dell'algoritmo è possibile che si ottengano risultati leggermente differenti in base al segnale generato. Nelle prove effettuate viene cambiata solo l'ampiezza del segnale, ma non la forma d'onda. %f = 100
	
		\grafico{conclusioni/noise/noise}{Grafico del rumore bianco con $A = 1V$}
		\graficospace
	
		Quando il segnale di rumore ha un ampiezza tale da garantire la convergenza del metodo, come nel nostro caso in cui $A = 2,5V$, le iterazioni medie richieste per calcolare il risultato su un singolo campione diminuiscono con l'aumentare del valore del parametro $L$. Nel seguente grafico si può vedere come varia il numero medio di iterazioni per campione in base al valore assunto dal parametro $L$.
	
		\grafico{conclusioni/noise/2,5V}{Grafico delle iterazioni per il rumore bianco con $A = 2,5V$}
		\pagebreak
	
		Nel caso in cui invece il rumore abbia un'ampiezza per cui l'algoritmo non converge per alcuni valori di $L$, come per esempio $A = 4,5V$, allora il numero medio di iterazioni si riduce drasticamente, poiché da un determinato istante di tempo, ovvero quando l'algoritmo inizia a divergere, il numero di iterazioni richieste per calcolare il risultato sui vari campioni sarà pari a zero. Nel seguente grafico sono stati rimossi i valori di $L$ per cui l'algoritmo non converge.
	
		\grafico{conclusioni/noise/4,5V}{Grafico delle iterazioni per il rumore bianco con $A = 4,5V$}
		\pagebreak
	
	\section{Segnale sinusoidale}
		Per essere certi che il segnale di output calcolato dall'algoritmo sia corretto si sono messi a confronto i risultati riscontrati nell'implementazione digitale con quelli ottenuti dall'implementazione analogica. Per semplicità è stato scelto di eseguire questo confronto su segnali sinusoidali di frequenza pari a $f = 1000Hz$.
	
		Il primo confronto è stato effettuato su un'onda sinusoidale con un'ampiezza tale per cui la convergenza dell'algoritmo è sempre garantita: è stato quindi scelto $A = 1,3V$.
	
		\grafico{conclusioni/sine/1,3V_iterations}{Grafico delle iterazioni per ogni campione del segnale sinusoidale con $A = 1,3V$}
		\pagebreak
	
		Possiamo vedere che il segnale di uscita del circuito analogico corrisponde a quanto calcolato dal codice Matlab.
	
		\grafico{conclusioni/analog/1,3V}{Grafico del segnale sinusoidale analogico con $A = 1,3V$}
			
		Notiamo inoltre che, anche nel caso di segnali sinusoidali, se l'input ha un ampiezza che garantisce la convergenza, il numero di iterazioni richieste diminuisce con l'aumentare del valore del parametro $L$, esattamente come per il caso del rumore bianco.
	
		\grafico{conclusioni/sine/1,3V}{Grafico delle iterazioni per il segnale sinusoidale con $A = 1,3V$}
		\pagebreak
	
		Un ulteriore confronto è stato effettuato su un'onda sinusoidale per la quale l'algoritmo non converge per alcuni valori di $L$ da un certo istante di tempo: in questo caso quindi $A = 1,6V$.
	
		\grafico{conclusioni/sine/1,6V_iterations}{Grafico delle iterazioni per ogni campione del segnale sinusoidale con $A = 1,6V$}
			
		Possiamo vedere che in questo caso l'output calcolato dal codice Matlab presenta alcune differenze rispetto al segnale prodotto dal circuito.
	
		\grafico{conclusioni/analog/1,6V}{Grafico del segnale sinusoidale analogico con $A = 1,6V$}
		\pagebreak
		
		In particolare vicino ai picchi l'output calcolato dalla simulazione presenta delle imprecisioni, dovute alla limitazione delle iterazioni dell'algoritmo. Queste limitazioni come già detto impediscono che il metodo entri in loop infinito.
	
		Anche nel caso del segnale sinusoidale per cui l'algoritmo non converge per alcuni valori di $L$ il numero medio di iterazioni si riduce drasticamente, esattamente come per il caso del segnale costituito da un rumore. Anche in questo grafico sono stati rimossi i valori di $L$ per cui l'algoritmo non converge.
	
		\grafico{conclusioni/sine/1,6V}{Grafico delle iterazioni per ogni campione del segnale sinusoidale con $A = 1,6V$}
		\graficospace
	
		Possiamo affermare quindi che quando il segnale in ingresso ha un'ampiezza che supera una certa soglia, determinata anche in base alla tipologia dell'onda che rappresenta il segnale, l'algoritmo non converge più per alcuni o tutti i valori di $L$, a partire da un determinato campione, solitamente vicino ad un picco dell'onda.