\chapter{Codice C++}
	\label{code:cpp}
	Il codice C++ si compone di 3 file \texttt{.cpp} con i 3 relativi header \texttt{.h}:
	
	\noindent\hspace{5px}\begin{tabularx}{\textwidth}{r X}
		\makecell{\textbf{PluginEditor.h}\\\textbf{PluginEditor.cpp}}			&che implementano tutti i metodi relativi all'interfaccia grafica del plugin\\
		\makecell{\textbf{PluginProcessor.h}\\\textbf{PluginProcessor.cpp}}	&che implementano tutti i metodi legati alla tecnologia VST, in particolare qui verrà richiamata la funzione che processerà i singoli sample\\
		\makecell{\textbf{Clipper.h}\\\textbf{Clipper.cpp}}					&dove è implementa tutta la logica di funzionamento del clipper, compreso il metodo numerico a punto fisso geometrico
	\end{tabularx}\\
	
	%\cppcode{PluginEditor.h}
	%\cppcode{PluginEditor.cpp}
	%\cppcode{PluginProcessor.h}
	%\cppcode{PluginProcessor.cpp}
	Nel seguente listato, che rappresenta parte del file \texttt{PluginProcessor.cpp}, parte del codice è stato omesso a causa della sua eccessiva lunghezza. Sono state incluse solo le funzioni strettamente necessarie alla comprensione del funzionamento dell'intero progetto.
	\vspace{10px}
	\begin{lstlisting}[caption={PluginProcessor.cpp},label=code:PluginProcessor.cpp,style=code,style=cpp,language=cppl,gobble=8]
		ClipperAudioProcessor::ClipperAudioProcessor()
			#ifndef JucePlugin_PreferredChannelConfigurations
				: AudioProcessor (BusesProperties()
					#if ! JucePlugin_IsMidiEffect
						#if ! JucePlugin_IsSynth
							.withInput  ("Input",  juce::AudioChannelSet::stereo(), true)
						#endif
						.withOutput ("Output", juce::AudioChannelSet::stereo(), true)
					#endif
				)
		    #endif
			{
				sampleCallback = [](unsigned int, float) {};
				
				auto totalNumInputChannels = getTotalNumInputChannels();
				for(int channel=0;channel<totalNumInputChannels;channel++)
					clippers.push_back(Clipper());
			}

		ClipperAudioProcessor::~ClipperAudioProcessor() {}
		
		.
		.
		.
		
		void ClipperAudioProcessor::processBlock(juce::AudioBuffer<float>& buffer, juce::MidiBuffer& midiMessages)
		{
			juce::ScopedNoDenormals noDenormals;
			auto totalNumInputChannels  = getTotalNumInputChannels();
			auto totalNumOutputChannels = getTotalNumOutputChannels();

			for(auto i=totalNumInputChannels;i<totalNumOutputChannels;i++)
				buffer.clear(i, 0, buffer.getNumSamples());

			if(!bypass)
			{
				for(int channel=0;channel<totalNumInputChannels;channel++)
				{
					auto *channelData = buffer.getWritePointer(channel);
					
					clippers[channel].process(channelData, buffer.getNumSamples());
					
					for(int i=0;i<buffer.getNumSamples();i++)
						sampleCallback(channel, channelData[i]);
				}
			}
		}

		void ClipperAudioProcessor::setSampleCallback(std::function<void(unsigned int, float)> sampleCallback)
		{
			this->sampleCallback = sampleCallback;
		}

		void ClipperAudioProcessor::setBypass(bool bypass)
		{
			this->bypass = bypass;
		}

		void ClipperAudioProcessor::setL(unsigned int L)
		{
			for(Clipper &clipper : clippers)
				clipper.setL(L);
		}

		void ClipperAudioProcessor::useNewtonRaphson(bool newtonRaphson)
		{
			for(Clipper &clipper : clippers)
				clipper.useNewtonRaphson(newtonRaphson);
		}
		
		.
		.
		.
	\end{lstlisting}
	\vspace{10px}
	
	\cppcode{Clipper.h}
	\pagebreak
	\cppcode{Clipper.cpp}